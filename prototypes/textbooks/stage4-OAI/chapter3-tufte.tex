% Chapter 3 Standalone with Tufte Style

\documentclass[justified,notoc]{tufte-book}

% Essential packages
\usepackage[utf8]{inputenc}
\usepackage[T1]{fontenc}
\usepackage{graphicx}
\graphicspath{{./images/}}
\usepackage{amsmath,amssymb}
\usepackage{float} % For better float control
\usepackage{placeins} % For \FloatBarrier command
\usepackage{morefloats} % Increase maximum number of floats
\usepackage{xcolor} % For colored text
\usepackage{tcolorbox} % For colored boxes
\usepackage{enumitem} % For better lists

% Custom colors
\definecolor{primary}{RGB}{0, 73, 144} % Deep blue

% Float adjustment to reduce figure/table drift
\setcounter{topnumber}{9}
\setcounter{bottomnumber}{9}
\setcounter{totalnumber}{16}
\renewcommand{\topfraction}{0.9}
\renewcommand{\bottomfraction}{0.9}
\renewcommand{\textfraction}{0.05}
\renewcommand{\floatpagefraction}{0.5}

% Increase float storage capacity
\extrafloats{100}

\newenvironment{keyconcept}[1]{%
    \begin{tcolorbox}[colback=primary!5,colframe=primary,title=\textbf{Key Concept: #1}]
}{%
    \end{tcolorbox}
}

\newenvironment{investigation}[1]{%
    \begin{tcolorbox}[colback=primary!5,colframe=primary,title=\textbf{Investigation: #1}]
}{%
    \end{tcolorbox}
}

\newenvironment{tieredquestions}[1]{%
    \begin{tcolorbox}[colback=primary!5,colframe=primary,title=\textbf{Practice Questions - #1}]
}{%
    \end{tcolorbox}
}

\title{Chapter 3: Mixtures and Separation Techniques}
\author{Emergent Mind Press}
\date{\today}

\begin{document}

\maketitle

\chapter{Mixtures and Separation Techniques}

\section{Introduction to Mixtures}

Mixtures are a fundamental concept in chemistry. A mixture consists of two or more substances that are physically combined but not chemically bonded.

\begin{keyconcept}{Key Properties of Mixtures}
Unlike compounds, mixtures:
\begin{itemize}
    \item Can be separated by physical means
    \item Retain the properties of their components
    \item Can have variable composition
\end{itemize}
\end{keyconcept}

\FloatBarrier

\section{Types of Mixtures}

Mixtures can be classified into two main categories:

\subsection{Homogeneous Mixtures}

Homogeneous mixtures have a uniform composition throughout. The components are evenly distributed and not distinguishable by eye.

\begin{marginfigure}[0pt]
  %\includegraphics[width=\linewidth]{homogeneous_mixture.png}
  \caption{Example of a homogeneous mixture: salt dissolved in water.}
  \label{fig:homogeneous}
\end{marginfigure}

Examples of homogeneous mixtures include solutions like salt water, air, and alloys like brass or bronze. In these mixtures, the components are so thoroughly mixed that they appear uniform even under a microscope.

\subsection{Heterogeneous Mixtures}

Heterogeneous mixtures do not have a uniform composition. The components are unevenly distributed and can be visibly distinguished.

\begin{marginfigure}[0pt]
  %\includegraphics[width=\linewidth]{heterogeneous_mixture.png}
  \caption{Example of a heterogeneous mixture: soil with visible components.}
  \label{fig:heterogeneous}
\end{marginfigure}

Examples of heterogeneous mixtures include salad, soil, and concrete. In these mixtures, you can often see the different components with the naked eye.

\FloatBarrier

\section{Separation Techniques}

Because the components in mixtures retain their properties, mixtures can be separated physically. Some common separation techniques include:

\subsection{Filtration}
Used to separate an insoluble solid from a liquid.

\begin{figure}[h]
  %\includegraphics[width=0.7\linewidth]{filtration.png}
  \caption{Filtration process separating a solid from a liquid using filter paper.}
  \label{fig:filtration}
\end{figure}

Filtration works by passing a mixture through a filter that has pores small enough to retain the solid particles while allowing the liquid to pass through. This technique is commonly used to remove impurities from water or to collect a desired solid product from a reaction mixture.

\FloatBarrier

\subsection{Evaporation}
Used to separate a dissolved solid from a solution.

\begin{figure}[h]
  %\includegraphics[width=0.7\linewidth]{evaporation.png}
  \caption{Evaporation of salt water to recover salt crystals.}
  \label{fig:evaporation}
\end{figure}

When a solution is heated, the liquid component evaporates, leaving the dissolved solid behind. This method is often used to recover salt from seawater in salt production.

\FloatBarrier

\section{Applications of Separation Techniques}

Understanding and applying separation techniques has numerous practical applications in daily life and industry:

\begin{itemize}
    \item \textbf{Water purification}: Filtration is used to remove impurities from drinking water.
    \item \textbf{Food processing}: Separation techniques are used in processing foods like extracting oils from seeds.
    \item \textbf{Mining}: Various separation methods are used to extract valuable minerals from ores.
    \item \textbf{Medical testing}: Chromatography helps in analyzing blood and urine samples.
\end{itemize}

\begin{investigation}{Separating a Mixture of Sand and Salt}
\textbf{Aim:} To separate a mixture of sand and salt using appropriate separation techniques.

\textbf{Materials:} 
\begin{itemize}
    \item Mixture of sand and salt
    \item Water
    \item Filter paper and funnel
    \item Beaker
    \item Heat source
    \item Evaporating dish
\end{itemize}

\textbf{Procedure:}
\begin{enumerate}
    \item Add water to the sand-salt mixture and stir well.
    \item Set up the filter paper in the funnel over a beaker.
    \item Pour the mixture through the filter.
    \item Transfer the filtered solution (containing dissolved salt) to an evaporating dish.
    \item Gently heat the solution until the water evaporates.
    \item Observe the salt crystals that remain in the dish.
\end{enumerate}
\end{investigation}

\FloatBarrier

\section{Conclusion}

Understanding mixtures and how to separate them is essential for many scientific and industrial applications. The ability to identify different types of mixtures and select appropriate separation techniques is a fundamental skill in chemistry and environmental science.

\begin{tieredquestions}{Basic}
\begin{enumerate}
    \item Define the terms homogeneous mixture and heterogeneous mixture.
    \item List three examples of each type of mixture.
    \item Which separation technique would you use to separate:
    \begin{itemize}
        \item Salt from seawater?
        \item Sand from water?
        \item Alcohol from water?
    \end{itemize}
\end{enumerate}
\end{tieredquestions}

\FloatBarrier

\end{document}