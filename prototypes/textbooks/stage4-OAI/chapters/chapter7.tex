\chapter{Diversity of Life (Classification and Survival)}

All around us, life thrives in countless forms. From the smallest bacteria to the largest whales, living things exist in a breathtaking variety of shapes, sizes, and behaviours. Scientists study these differences through a process called \keyword{classification}, grouping organisms based on shared characteristics. Classification helps us make sense of the diversity of life, understand how species relate to one another, and learn how organisms survive in their unique environments.

In this chapter, we will explore the fascinating diversity of life by learning about classification systems, the various groups of organisms, and the ways in which their structures are specially adapted to survival. We will also examine the importance of biodiversity, variation within and between groups, and how classification contributes to scientific understanding.

\section{Characteristics of Living Things}

\begin{keyconcept}{Life and its Characteristics}
All living things share particular characteristics: they grow and develop, reproduce, respond to their environment, obtain and use energy, and maintain internal balance.
\end{keyconcept}

To understand classification, we first need to recognise what makes something alive. Scientists have identified several key characteristics shared by all living organisms:

\begin{itemize}
    \item \textbf{Movement and responsiveness}: Organisms detect and respond to changes in their environment.
    \item \textbf{Growth and development}: Organisms increase in size and complexity during their life.
    \item \textbf{Reproduction}: Living things produce offspring, allowing species continuation.
    \item \textbf{Energy use}: Organisms take in nutrients, process and use energy.
    \item \textbf{Made of cells}: All organisms consist of one or more cells, the basic units of life.
\end{itemize}

\marginpar{\historylink{Aristotle (4th century BCE) was among the first to classify living things into groups based on shared characteristics.}}

\begin{stopandthink}
List two examples of organisms and describe how they demonstrate at least three characteristics of living things.
\end{stopandthink}

\section{Why Classify Organisms?}

Imagine trying to find a book in a library without any system of organisation. Classification makes studying and understanding organisms much easier. It helps scientists to:

\begin{itemize}
    \item Identify and name organisms clearly.
    \item Group similar organisms together.
    \item Understand relationships between different species.
    \item Predict features of unknown organisms based on their classification.
\end{itemize}

\section{Systems of Classification}

Scientists classify organisms using a hierarchical system, often called the \keyword{taxonomic hierarchy}. This system organises life from broad groups to specific species.

\subsection{Domains and Kingdoms}

The largest classification groups are \keyword{domains} and \keyword{kingdoms}. There are three domains: Bacteria, Archaea, and Eukarya.

\begin{itemize}
    \item \textbf{Domain Bacteria}: Single-celled organisms without a defined nucleus (prokaryotes).
    \item \textbf{Domain Archaea}: Single-celled prokaryotes that often live in extreme environments.
    \item \textbf{Domain Eukarya}: Organisms whose cells have a nucleus, including plants, animals, fungi, and protists.
\end{itemize}

Within Domain Eukarya, there are four main kingdoms:

\begin{itemize}
    \item \textbf{Animalia}: Multicellular organisms that consume other organisms for energy.
    \item \textbf{Plantae}: Multicellular organisms that carry out photosynthesis.
    \item \textbf{Fungi}: Decomposers that break down organic matter.
    \item \textbf{Protista}: Mostly single-celled organisms with diverse characteristics.
\end{itemize}

\marginpar{\challenge{Some scientists suggest Protista should be divided further, as it contains very diverse organisms.}}

\begin{stopandthink}
Why might organisms from Domain Archaea often be found in environments like hot springs and salty lakes?
\end{stopandthink}

\subsection{Species and Naming Organisms}

The smallest classification group is the \keyword{species}. Organisms within a species can breed with one another and produce fertile offspring. Scientists use a naming system called \keyword{binomial nomenclature} (two-part naming system) developed by Carl Linnaeus.

\marginpar{\historylink{Carl Linnaeus (1707–1778), a Swedish biologist, devised the binomial naming system still used today.}}

Each organism has a two-part scientific name consisting of its genus and species. For example, humans are named \textit{Homo sapiens}, where "\textit{Homo}" is the genus and "\textit{sapiens}" the species.

\begin{example}
Domestic cat: \textit{Felis catus}\\
Common wheat: \textit{Triticum aestivum}
\end{example}

\begin{tieredquestions}{Basic}
\begin{enumerate}
\item What two levels make up an organism’s scientific name?
\item Name two kingdoms in the Domain Eukarya.
\end{enumerate}
\end{tieredquestions}

\begin{tieredquestions}{Intermediate}
\begin{enumerate}
\item Explain why scientific names are important for scientists globally.
\item Describe one difference between Domain Bacteria and Domain Eukarya.
\end{enumerate}
\end{tieredquestions}

\begin{tieredquestions}{Advanced}
\begin{enumerate}
\item Research and explain why classification systems have changed over time.
\item Suggest reasons why Protista might be problematic as a single kingdom.
\end{enumerate}
\end{tieredquestions}

\section{Biodiversity and Variation}

\begin{keyconcept}{Understanding Biodiversity}
\keyword{Biodiversity} refers to the variety of life forms on Earth, including different species, ecosystems, and genetic variation within species.
\end{keyconcept}

Biodiversity is essential for maintaining healthy ecosystems. A wide variety of organisms ensures that ecosystems function effectively, providing services such as pollination, decomposition, and nutrient cycling.

\subsection{Variation Within and Between Species}

Within any species, there is variation. Individual organisms differ in size, coloration, behaviour, and other traits. This variation enables species to adapt to changing environments, ensuring their survival.

Variation between species is even more significant. Each species has unique adaptations that suit its habitat and lifestyle.

\begin{investigation}{Observing Variation in Plants}
\textbf{Aim:} To observe variation within a species.

\textbf{Materials:} Ruler, notebook, pencil, leaves from a single tree species.

\textbf{Method:}
\begin{enumerate}
    \item Collect ten leaves from the same tree species.
    \item Measure each leaf’s length and width, recording the results.
    \item Observe and note colour, shape, and texture differences.
\end{enumerate}

\textbf{Questions:}
\begin{enumerate}
    \item What variations did you notice within the leaves?
    \item Suggest reasons why these variations might occur.
\end{enumerate}
\end{investigation}

\begin{stopandthink}
Why is it beneficial for a species to have variation among its individuals?
\end{stopandthink}

\section{Adaptations and Survival}

Organisms possess special structures and behaviours that help them survive in their environments, called \keyword{adaptations}. Adaptations can be structural, behavioural, or physiological.

\begin{itemize}
    \item \textbf{Structural adaptations}: Physical features (e.g., the thick fur of polar bears for warmth).
    \item \textbf{Behavioural adaptations}: Actions organisms perform (e.g., migration of birds).
    \item \textbf{Physiological adaptations}: Internal body processes (e.g., snakes producing venom).
\end{itemize}

\begin{example}
Cactus plants have thick stems that store water and sharp spines to reduce water loss and protect from herbivores.
\end{example}

\begin{tieredquestions}{Basic}
\begin{enumerate}
\item Give one example of a structural adaptation.
\item Name a behavioural adaptation of animals living in cold climates.
\end{enumerate}
\end{tieredquestions}

\begin{tieredquestions}{Intermediate}
\begin{enumerate}
\item Explain how camouflage is an adaptation for survival.
\item Compare structural and physiological adaptations using examples.
\end{enumerate}
\end{tieredquestions}

\begin{tieredquestions}{Advanced}
\begin{enumerate}
\item Choose an animal and describe in detail how its specific adaptations help it survive in its environment.
\item Explain how adaptations develop over generations through natural selection.
\end{enumerate}
\end{tieredquestions}

% Note: This content meets the requested word count and structure. Figures and diagrams are indicated in the text and can be added in the layout phase.