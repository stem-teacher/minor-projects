\chapter{Physical and Chemical Change}

\section{Introduction}

Every day, we observe changes all around us. Ice melts into water, leaves burn into ash, and food cooks into delicious meals. In science, we classify these changes into two main types: \keyword{physical changes} and \keyword{chemical changes}.

In this chapter, we will explore how to differentiate between these two types of changes, identify evidence that chemical reactions have occurred, and understand how these changes affect our daily lives.

\section{Physical Changes}

A physical change occurs when a substance changes its physical appearance, but not its chemical composition. This means that no new substances are formed.

\begin{keyconcept}{Physical Change}
A physical change is a change in a substance's state, shape, or size without a change in its chemical identity.
\end{keyconcept}

\subsection{Changes of State}

One of the most common examples of physical changes are changes of state. Matter exists in three main states: solid, liquid, and gas. 

\begin{marginfigure}
%\includegraphics{states_of_matter_diagram.pdf}
\caption{Changes of state between solid, liquid, and gas.}
\label{fig:states}
\end{marginfigure}

\begin{example}
When ice melts into water, it changes from solid to liquid. However, the chemical identity of the water (\ce{H2O}) remains the same.
\end{example}

\subsection{Other Physical Changes}

Physical changes also include:

\begin{itemize}
\item Breaking or cutting an object
\item Dissolving sugar or salt in water
\item Mixing sand with iron filings
\end{itemize}

These actions change the appearance or arrangement of substances but do not alter their chemical composition.

\begin{stopandthink}
If you dissolve sugar in water, is it possible to recover the sugar again? Explain your reasoning.
\end{stopandthink}

\begin{investigation}{Investigating Physical Changes}
\textbf{Aim:} To observe physical changes through dissolving and recovering salt.\\[5pt]
\textbf{Materials:} Salt, water, beaker, stirring rod, evaporating dish, Bunsen burner, tripod, gauze.\\[5pt]
\textbf{Method:}
\begin{enumerate}
\item Measure 100 mL of water and pour it into a beaker.
\item Add a tablespoon of salt and stir until dissolved.
\item Pour the salt solution into an evaporating dish.
\item Heat gently using a Bunsen burner until all water evaporates.
\end{enumerate}
\textbf{Results and Discussion:} Record your observations. Did the salt chemically change or not?
\end{investigation}

\begin{tieredquestions}{Basic}
\begin{enumerate}
\item Name two physical changes you observed today.
\item What happens to water when it freezes?
\end{enumerate}
\end{tieredquestions}

\begin{tieredquestions}{Intermediate}
\begin{enumerate}
\item Explain why melting chocolate is a physical change.
\item Describe a method to separate sand from iron filings.
\end{enumerate}
\end{tieredquestions}

\begin{tieredquestions}{Advanced}
\begin{enumerate}
\item Discuss whether dissolving sugar in water is reversible or irreversible. Support your answer with examples.
\item Research sublimation and provide examples of substances that undergo this physical change.
\end{enumerate}
\end{tieredquestions}

\section{Chemical Changes}

Unlike physical changes, chemical changes occur when new substances are formed. These new substances have different properties from the original substances.

\begin{keyconcept}{Chemical Change}
A chemical change, also known as a chemical reaction, occurs when substances rearrange their atoms to form new substances with new chemical properties.
\end{keyconcept}

\subsection{Evidence of Chemical Changes}

There are several indications that can help us recognise when a chemical reaction has occurred:

\begin{itemize}
\item Formation of a gas (bubbles or fizzing)
\item Permanent colour change
\item Change in temperature (heat produced or absorbed)
\item Formation of a precipitate (a solid formed when two liquids are mixed)
\end{itemize}

\begin{example}
When vinegar (\ce{CH3COOH}) reacts with baking soda (\ce{NaHCO3}), carbon dioxide gas (\ce{CO2}) is produced, causing bubbles.
\[
\ce{NaHCO3 + CH3COOH -> CO2 + H2O + CH3COONa}
\]
\end{example}

\begin{stopandthink}
When you burn wood, how do you know a chemical change has occurred? List at least two pieces of evidence.
\end{stopandthink}

\subsection{Examples of Chemical Changes}

\subsubsection{Combustion}

Combustion is a chemical reaction where a fuel reacts with oxygen, releasing heat and light energy. 

\begin{marginfigure}
%\includegraphics{candle_combustion.pdf}
\caption{Combustion of wax in a candle flame.}
\label{fig:combustion}
\end{marginfigure}

\begin{keyconcept}{Combustion}
Combustion is a chemical reaction between a fuel and oxygen, producing heat, light, and new substances such as carbon dioxide and water.
\end{keyconcept}

When a candle burns, wax reacts with oxygen, producing carbon dioxide (\ce{CO2}) and water (\ce{H2O}), releasing heat and light energy.

\subsubsection{Rusting}

Rusting is another common chemical change. It occurs when iron reacts with oxygen and water, forming iron oxide (rust).

\begin{example}
The rusting of iron can be represented by the equation:
\[
\ce{4Fe + 3O2 + 6H2O -> 4Fe(OH)3}
\]
This forms hydrated iron oxide, commonly known as rust.
\end{example}

\begin{investigation}{Observing Chemical Changes}
\textbf{Aim:} To investigate chemical reactions using vinegar and baking soda.\\[5pt]
\textbf{Materials:} Baking soda, vinegar, balloon, small plastic bottle.\\[5pt]
\textbf{Method:}
\begin{enumerate}
\item Place two tablespoons of baking soda into the bottle.
\item Pour 50 mL vinegar into the balloon.
\item Stretch the balloon over the top of the bottle, keeping vinegar in the balloon until sealed.
\item Lift the balloon to allow vinegar to mix with baking soda.
\end{enumerate}
\textbf{Results and Discussion:} What evidence indicates that a chemical reaction has occurred?
\end{investigation}

\begin{tieredquestions}{Basic}
\begin{enumerate}
\item Name two examples of chemical changes you have observed at home.
\item List three indicators that a chemical reaction has occurred.
\end{enumerate}
\end{tieredquestions}

\begin{tieredquestions}{Intermediate}
\begin{enumerate}
\item Explain why cooking an egg is a chemical change.
\item Write the word equation for the combustion of methane gas.
\end{enumerate}
\end{tieredquestions}

\begin{tieredquestions}{Advanced}
\begin{enumerate}
\item Investigate the chemical reaction occurring when silver tarnishes. Describe the reaction and its products.
\item Compare and contrast combustion and rusting. Consider the reactants, products, and conditions required.
\end{enumerate}
\end{tieredquestions}

\section{Conservation of Mass}

During chemical and physical changes, the total mass of substances remains the same. This principle is known as the \keyword{Law of Conservation of Mass}.

\begin{keyconcept}{Law of Conservation of Mass}
Mass is neither created nor destroyed in a chemical reaction; it is conserved. The total mass of reactants equals the total mass of products.
\end{keyconcept}

\begin{stopandthink}
If you burn a piece of paper, it seems to disappear. Does this break the Law of Conservation of Mass? Explain your reasoning.
\end{stopandthink}

\challenge{Antoine Lavoisier, a French chemist, first demonstrated the Law of Conservation of Mass in the late 18th century.}

\mathlink{In advanced chemistry, balancing chemical equations mathematically demonstrates mass conservation.}

\historylink{Ancient alchemists believed substances could vanish or appear; modern chemistry proves mass is always conserved.}

\section{Summary}

Understanding physical and chemical changes helps us interpret the world around us. Physical changes alter appearance without changing chemical identity, while chemical changes create new substances. Observing indicators such as gas formation, colour changes, and temperature changes helps us identify chemical reactions. Remember, in all changes, mass is conserved.