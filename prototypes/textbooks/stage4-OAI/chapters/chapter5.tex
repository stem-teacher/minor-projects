\chapter{Forces and Motion}

Everything around us—from cars accelerating down the motorway to leaves gently falling to the ground—is influenced by forces. Understanding forces and motion helps us explain and predict how objects behave in everyday life. In this chapter, we will explore different types of forces, how they interact, and the effects they have on the movement of objects.

\section{What are Forces?}

A \keyword{force} is a push, pull, twist or squeeze that can cause an object to move, stop, change shape or change direction. Forces can act between objects that are in direct contact, or even at a distance without touching at all.

\begin{keyconcept}{Types of Forces}
Forces are broadly classified into two types:
\begin{itemize}
    \item \keyword{Contact forces} – act when objects physically touch, such as friction and tension.
    \item \keyword{Non-contact forces} – act from a distance, such as gravity, magnetism and electrostatic forces.
\end{itemize}
\end{keyconcept}

\subsection{Contact Forces}

Contact forces occur when objects are physically touching each other. Common examples include friction, air resistance, tension, and normal forces.

\marginnote[1cm]{\textbf{Friction:} A force resisting motion between two surfaces.}

\paragraph{Friction} is a force we experience every day. It acts opposite to the direction of motion and occurs when two surfaces rub against each other. Friction helps us to walk without slipping, hold objects firmly, and allows cars to brake safely.

\begin{investigation}{Measuring Frictional Force}
\textbf{Aim:} To measure the frictional force between different surfaces.

\textbf{Materials:}
\begin{itemize}
    \item Spring balance
    \item Wooden block
    \item Various surfaces (sandpaper, carpet, smooth wood)
\end{itemize}

\textbf{Method:}
\begin{enumerate}
    \item Attach the spring balance to the wooden block.
    \item Pull the block slowly across each surface, recording the force required.
\end{enumerate}

\textbf{Results:} Record your results in a table and compare the frictional force across different surfaces.

\textbf{Discussion:} Which surface had the highest frictional force? Why do you think this is the case?
\end{investigation}

\marginnote{\challenge{Did you know friction even occurs in space? Astronauts use friction to grip tools during spacewalks.}}

\paragraph{Air resistance} is another type of friction, occurring when objects move through air. You can feel air resistance when cycling fast or running into the wind.

\subsection{Non-contact Forces}

Non-contact forces act at a distance, without any direct physical contact between objects. Gravity, magnetism, and electrostatic forces are examples of non-contact forces.

\paragraph{Gravity} is an important non-contact force that pulls objects towards each other. The Earth's gravity keeps us on the ground and causes objects to fall when dropped.

\marginnote{\historylink{Sir Isaac Newton (1642–1727) was the first scientist to explain gravity mathematically.}}

\paragraph{Magnetism} involves attraction and repulsion between magnetic materials. Magnets have north and south poles, and opposite poles attract while similar poles repel.

\begin{stopandthink}
What would happen to objects on Earth if gravity suddenly disappeared?
\end{stopandthink}

\begin{tieredquestions}{Basic}
\begin{enumerate}
    \item Define force in your own words.
    \item Name two examples each of contact and non-contact forces.
    \item Describe one everyday situation where friction is useful.
\end{enumerate}
\end{tieredquestions}

\begin{tieredquestions}{Intermediate}
\begin{enumerate}
    \item Explain how friction can be both helpful and unhelpful, giving examples.
    \item Identify two factors that affect the amount of friction between surfaces.
\end{enumerate}
\end{tieredquestions}

\begin{tieredquestions}{Advanced}
\begin{enumerate}
    \item Research and explain how engineers reduce friction in high-speed vehicles.
    \item Describe how gravitational force changes as you move away from Earth.
\end{enumerate}
\end{tieredquestions}

\section{Balanced and Unbalanced Forces}

Forces can act simultaneously on an object. The way these forces interact determines whether the object moves, remains stationary, or changes its motion.

\begin{keyconcept}{Balanced and Unbalanced Forces}
\begin{itemize}
    \item \keyword{Balanced forces}: Forces of equal size acting in opposite directions, resulting in no change in motion.
    \item \keyword{Unbalanced forces}: Forces that are unequal, causing changes in an object's speed or direction.
\end{itemize}
\end{keyconcept}

\subsection{Balanced Forces}

When forces are balanced, there is no net force acting on an object. This means the object will either remain stationary or continue to move at a constant speed in a straight line.

\begin{example}
Consider a book resting on a table. Gravity pulls it downwards, but the table pushes upward with an equal force. These forces balance each other out, so the book remains stationary.
\end{example}

\begin{stopandthink}
What other examples of balanced forces can you identify around your classroom?
\end{stopandthink}

\subsection{Unbalanced Forces and Motion}

When forces acting on an object are unbalanced, the object will change its motion. It may start moving, stop moving, speed up, slow down, or change direction.

\begin{example}
Imagine pushing a shopping trolley. Initially stationary, the trolley begins to move forward because the pushing force exceeds frictional forces.
\end{example}

\begin{investigation}{Observing Unbalanced Forces}
\textbf{Aim:} To observe how unbalanced forces cause a change in motion.

\textbf{Materials:}
\begin{itemize}
    \item Toy cars
    \item Inclined ramp
    \item Stopwatch
\end{itemize}

\textbf{Method:}
\begin{enumerate}
    \item Place a toy car at the top of an inclined ramp and release it.
    \item Use a stopwatch to measure how long it takes to reach the bottom.
    \item Repeat for different inclines and record your observations.
\end{enumerate}

\textbf{Discussion:} How does changing the incline angle affect the car's motion?
\end{investigation}

\section{Newton’s First Law of Motion}

In the 17th century, Sir Isaac Newton described three important laws that explain motion. In this chapter, we focus on his first law of motion.

\begin{keyconcept}{Newton’s First Law}
Newton’s First Law states: \textit{An object will remain at rest, or continue to move at a constant velocity, unless acted upon by an unbalanced force.}

This law is also known as the law of inertia.
\end{keyconcept}

\marginnote{\keyword{Inertia} is the tendency of an object to resist changes in its motion.}

\begin{example}
Consider passengers in a car. If the car suddenly brakes, passengers continue moving forward due to inertia, unless stopped by seatbelts.
\end{example}

\begin{stopandthink}
Why is it important to wear seatbelts in cars, considering Newton’s First Law?
\end{stopandthink}

\section{Forces in Everyday Life}

Forces are constantly at play in our daily lives—whether in sport, transport, or even at home.

\subsection{Forces in Sport}

Sports involve many examples of balanced and unbalanced forces. When kicking a football, the unbalanced force from your foot causes it to accelerate forward. Air resistance and gravity eventually slow and bring it back down.

\subsection{Forces in Transport}

Cars, bicycles, buses, and planes all rely on forces to accelerate, slow down, and change direction. Friction between tyres and the road provides control, while engines generate unbalanced forces to move vehicles forward.

\subsection{Falling Objects}

Gravity pulls objects towards Earth. Objects falling through the air experience air resistance, slowing their fall. Without air resistance, all objects would fall at the same rate.

\begin{investigation}{Comparing Falling Objects}
\textbf{Aim:} To investigate how objects fall.

\textbf{Materials:}
\begin{itemize}
    \item Different objects (paper, ball, feather, coin)
\end{itemize}

\textbf{Method:}
\begin{enumerate}
    \item Drop objects from the same height and observe their fall.
\end{enumerate}

\textbf{Discussion:} Why do some objects fall faster than others? Consider air resistance and gravity.
\end{investigation}

\begin{tieredquestions}{Intermediate}
\begin{enumerate}
    \item Explain how Newton’s First Law relates to cycling or skateboarding.
    \item What would happen if there were no frictional forces acting on moving vehicles?
\end{enumerate}
\end{tieredquestions}

\begin{tieredquestions}{Advanced}
\begin{enumerate}
    \item Research and explain the concept of terminal velocity in skydiving.
    \item Describe how forces are balanced and unbalanced during a rocket launch.
\end{enumerate}
\end{tieredquestions}