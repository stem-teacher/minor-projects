\chapter{Cells and Body Systems}

\section{Introduction}

You are made up of trillions of tiny structures called \keyword{cells}. Cells are the basic building blocks of all living organisms—from the simplest bacteria to complex animals and plants. Although cells are microscopic, their organisation and interactions form complex systems that keep organisms alive, healthy, and capable of reproducing.

In this chapter, we will explore cell theory, examine the structure of cells in plants and animals, and investigate how cells combine to form tissues, organs, and body systems. We will also look closely at several key human body systems, including digestive, circulatory, and reproductive systems, to understand how their structures relate to their vital functions.

\section{Cell Theory}

\historylink{The first observation of cells was made by Robert Hooke in 1665 when looking at cork under a simple microscope.}

Cells were first discovered using microscopes, leading to the development of the \keyword{cell theory}, a fundamental concept in biology. Cell theory has three main principles:

\begin{keyconcept}{Principles of Cell Theory}
\begin{enumerate}
    \item All living organisms are composed of one or more cells.
    \item The cell is the basic unit of structure and function in organisms.
    \item All cells come from pre-existing cells.
\end{enumerate}
\end{keyconcept}

\begin{stopandthink}
Why do you think we say cells are the basic units of life? What characteristics make something alive?
\end{stopandthink}

\section{Investigating Cells}

Scientists use microscopes to study cells in detail. Microscopes allow us to see structures too small for the naked eye.

\begin{investigation}{Observing Cells Under a Microscope}
\textbf{Aim:} To observe plant and animal cells under a microscope and identify their structures.

\textbf{Materials:}
\begin{itemize}
\item Light microscope
\item Microscope slides and coverslips
\item Onion skin (plant cells)
\item Cheek cells (animal cells)
\item Iodine solution (for staining onion cells)
\item Methylene blue solution (for staining cheek cells)
\end{itemize}

\textbf{Method:}
\begin{enumerate}
\item Prepare slides of onion cells and cheek cells, staining them gently.
\item Observe each slide under low, medium, and high power magnification.
\item Draw labelled diagrams of each cell type, noting similarities and differences.
\end{enumerate}

\textbf{Results and Discussion:}
Compare your observations and explain how plant and animal cells differ in structure.
\end{investigation}

\section{Structure of Cells}

Cells contain specialised structures called \keyword{organelles}, each with a specific function. Although cells vary greatly, most share common organelles.

\subsection{Animal Cells}

Animal cells contain several organelles, including:

\begin{itemize}
    \item \keyword{Nucleus}: controls cell activities, contains genetic material (DNA).
    \item \keyword{Cytoplasm}: jelly-like substance where chemical reactions occur.
    \item \keyword{Cell membrane}: controls what enters and leaves the cell.
    \item \keyword{Mitochondria}: produce energy by respiration.
\end{itemize}

In animal cells, the shape is usually irregular, allowing flexible movement.

\subsection{Plant Cells}

Plant cells share many of the same organelles as animal cells but have key differences:

\begin{itemize}
    \item \keyword{Cell wall}: rigid outer layer providing structure and support.
    \item \keyword{Chloroplasts}: contain chlorophyll, performing photosynthesis.
    \item \keyword{Large central vacuole}: stores water and nutrients, maintaining turgidity.
\end{itemize}

\begin{stopandthink}
Why do plant cells need chloroplasts, but animal cells do not?
\end{stopandthink}

\begin{tieredquestions}{Basic}
\begin{enumerate}
    \item List two organelles found in both plant and animal cells.
    \item What is the main function of the cell membrane?
\end{enumerate}
\end{tieredquestions}

\begin{tieredquestions}{Intermediate}
\begin{enumerate}
    \item Describe the importance of mitochondria in cells.
    \item Explain why plant cells have a rigid structure compared to animal cells.
\end{enumerate}
\end{tieredquestions}

\begin{tieredquestions}{Advanced}
\begin{enumerate}
    \item Predict what might happen to a plant cell if its central vacuole lost all its water. Explain your reasoning.
    \item How do chloroplasts and mitochondria work together to support plant cell function?
\end{enumerate}
\end{tieredquestions}

\section{Cells to Body Systems}

Multicellular organisms have groups of cells organised into tissues, organs, and systems. 

\begin{keyconcept}{Levels of Organisation}
Cells $\rightarrow$ Tissues $\rightarrow$ Organs $\rightarrow$ Organ Systems $\rightarrow$ Organisms
\end{keyconcept}

\subsection{Tissues and Organs}

A \keyword{tissue} is a group of similar cells working together to perform a specific function. For example, muscle tissue contracts to allow movement, and nerve tissue transmits signals throughout the body.

An \keyword{organ} comprises different tissues working together to perform a particular function. The heart, lungs, and stomach are examples of organs.

\begin{example}
The stomach is an organ that consists of muscle tissue to churn food, epithelial tissue to line its surface, and glandular tissue to produce digestive juices.
\end{example}

\subsection{Body Systems}

Organs work together in \keyword{organ systems} to perform vital functions. We will explore three crucial body systems: digestive, circulatory, and reproductive.

\section{The Digestive System}

The digestive system breaks food down into nutrients that cells can absorb and use for energy, growth, and repair.

\subsection{Structure and Function}

Key digestive organs include:

\begin{itemize}
    \item \textbf{Mouth}: Mechanical and chemical digestion begins here.
    \item \textbf{Stomach}: Food is mixed with enzymes and acids.
    \item \textbf{Small intestine}: Nutrients are absorbed into the bloodstream.
    \item \textbf{Large intestine}: Water is absorbed, forming faeces.
\end{itemize}

\challenge{Did you know your small intestine is approximately 7 metres long? Its length increases the surface area available for nutrient absorption.}

\section{The Circulatory System}

The circulatory system transports nutrients, oxygen, and waste products around the body.

\subsection{The Heart and Blood Vessels}

The heart pumps blood through blood vessels:

\begin{itemize}
    \item \keyword{Arteries} carry oxygen-rich blood away from the heart.
    \item \keyword{Veins} return oxygen-poor blood to the heart.
    \item \keyword{Capillaries} are tiny vessels where exchange of substances occurs.
\end{itemize}

\mathlink{Heart rate can be measured by counting beats per minute (bpm). Calculate your heart rate at rest and after exercise.}

\section{The Reproductive System}

The reproductive system allows organisms to produce offspring, ensuring the survival of their species.

\subsection{Male and Female Systems}

The male reproductive system produces sperm cells and delivers them. The female reproductive system produces egg cells and supports pregnancy.

\begin{stopandthink}
Why is reproduction considered essential for the survival of a species, even though an individual organism can survive without reproducing?
\end{stopandthink}

\section{Coordination of Body Systems}

Body systems must work together to maintain life—a process called \keyword{homeostasis}. For example, during exercise, the circulatory system increases blood flow, providing more oxygen and nutrients to muscles, while the respiratory system increases breathing rate to supply oxygen.

\begin{tieredquestions}{Intermediate}
Describe how the digestive and circulatory systems interact after you eat a meal.
\end{tieredquestions}

\begin{tieredquestions}{Advanced}
Describe the interaction of multiple body systems during vigorous exercise, including at least three different systems.
\end{tieredquestions}

\section{Chapter Summary}

In this chapter, we have explored the fundamental units of life—cells—and how their organisation into tissues, organs, and organ systems supports complex life. Each body system has specialised structures that enable specific functions to maintain life and reproduction. Understanding the organisation and function of cells and body systems helps explain how living things grow, survive, and reproduce, highlighting the intricate coordination essential for life.