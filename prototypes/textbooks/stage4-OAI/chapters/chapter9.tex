\chapter{Earth's Resources and Geological Change}

Our planet Earth provides us with everything we need to survive—from fresh water to fertile soil, minerals and energy sources. Understanding Earth's resources and how geological processes shape and change our planet is essential for making informed decisions about our environment and future.

In this chapter, we explore the types of Earth's resources, how geological processes create and transform these resources, and how humans utilise and impact them. We will also investigate the dynamic geological changes that continuously reshape the Earth's surface, from slow processes like erosion to sudden events such as volcanic eruptions and earthquakes.

\section{Earth's Natural Resources}

Resources are materials from the Earth that we use to support life and meet our needs. Earth's resources can be categorised into renewable, non-renewable, and sustainable resources.

\subsection{Renewable and Non-Renewable Resources}

\keyword{Renewable resources} are resources that naturally replenish within a human lifetime, such as water, wind, sunlight, and timber. In contrast, \keyword{non-renewable resources} are resources that exist in limited quantities and cannot be replenished within human timescales. Examples include fossil fuels (coal, oil, natural gas) and minerals (gold, copper, iron).

\begin{marginfigure}
% Figure placeholder: Renewable vs Non-Renewable resources comparison diagram.
\caption{Comparison of renewable and non-renewable resources.}
\end{marginfigure}

\begin{keyconcept}{Sustainable Resources}
A resource is considered sustainable if it is used at a rate that allows it to replenish and remain available for future generations. Sustainable practices aim to balance human needs with environmental preservation.
\end{keyconcept}

\begin{stopandthink}
List three renewable and three non-renewable resources you have used today. How could you reduce your consumption of non-renewable resources?
\end{stopandthink}

\subsection{Water as a Vital Resource}

Water is a renewable resource essential to life. Although approximately 70\% of Earth's surface is covered in water, only a small fraction (around 2.5\%) is freshwater. Most freshwater is locked away in glaciers and ice caps, leaving a limited amount available for human use.

\begin{marginfigure}
% Figure placeholder: Diagram of Earth's water distribution.
\caption{Distribution of water on Earth.}
\end{marginfigure}

\historylink{Throughout history, civilisations have flourished around freshwater sources such as rivers and lakes, emphasising water's critical role in human development.}

\begin{investigation}{Water Usage at Home}
For one week, track the amount of water you use daily for activities such as showering, drinking, cooking and cleaning. Present your findings in a table and create a graph to show your water usage patterns. Identify areas where you could reduce water consumption.
\end{investigation}

\subsection{Minerals and Rocks}

Minerals and rocks are valuable Earth resources. Minerals are naturally occurring, inorganic substances with a defined chemical composition and structure. Rocks are aggregates of one or more minerals.

\begin{marginfigure}
% Figure placeholder: Common minerals and their uses.
\caption{Examples of common minerals and their everyday uses.}
\end{marginfigure}

\begin{example}
Quartz is a mineral used in making glass, watches, and electronics due to its hardness and transparency. Iron ore, a mineral-rich rock, is extracted to produce iron and steel.
\end{example}

\begin{stopandthink}
Examine the items around you. List three objects and identify the minerals or rocks used to make them.
\end{stopandthink}

\begin{tieredquestions}{Basic}
\begin{enumerate}
\item Define renewable and non-renewable resources.
\item Name two renewable and two non-renewable resources.
\end{enumerate}
\end{tieredquestions}

\begin{tieredquestions}{Intermediate}
\begin{enumerate}
\item Explain why freshwater is considered a renewable yet limited resource.
\item Describe two ways humans can reduce their consumption of non-renewable resources.
\end{enumerate}
\end{tieredquestions}

\begin{tieredquestions}{Advanced}
\begin{enumerate}
\item Evaluate the challenges and benefits of relying heavily on renewable resources.
\item Propose a sustainable practice that your school could adopt to conserve natural resources.
\end{enumerate}
\end{tieredquestions}

\section{Geological Processes and Earth's Resources}

Earth's surface is continually changing due to geological processes including weathering, erosion, sedimentation, volcanic activity, and tectonic movements. These processes shape landscapes and influence the availability and distribution of Earth's resources.

\subsection{Weathering and Erosion}

\keyword{Weathering} is the breakdown of rocks into smaller particles by physical, chemical, or biological processes. \keyword{Erosion} is the movement of these weathered materials by wind, water, ice, or gravity.

\begin{keyconcept}{Sedimentation and Formation of Sedimentary Rocks}
Sedimentation occurs when particles transported by erosion settle in layers, often in water bodies. Over time, these layers compact and cement together, forming sedimentary rocks. Coal, limestone, and sandstone are examples of sedimentary rocks created through this process.
\end{keyconcept}

\begin{investigation}{Observing Weathering and Erosion}
Place several sugar cubes in two separate containers. Shake one container gently and the other vigorously for one minute. Observe and record the differences. Relate your observations to natural erosion processes.
\end{investigation}

\subsection{Volcanic Activity and Igneous Rocks}

Volcanic activity occurs when magma (molten rock) rises from beneath Earth's surface, erupting as lava. When lava cools and solidifies, it forms \keyword{igneous rocks} such as basalt and granite.

\historylink{Indigenous Australians have long understood volcanic landscapes, using basalt and obsidian tools created from volcanic rocks for thousands of years.}

\begin{marginfigure}
% Figure placeholder: Diagram of volcanic eruption and rock formation.
\caption{Formation of igneous rocks through volcanic activity.}
\end{marginfigure}

\begin{stopandthink}
What types of resources might communities living near volcanic regions benefit from?
\end{stopandthink}

\subsection{Metamorphic Rocks and the Rock Cycle}

\keyword{Metamorphic rocks} form when existing rocks (igneous or sedimentary) are transformed by heat, pressure, or chemically active fluids. Marble and slate are examples of metamorphic rocks.

\begin{keyconcept}{The Rock Cycle}
The rock cycle describes how rocks change from one type to another over geological time. Through processes like melting, cooling, weathering, compaction, and metamorphism, rocks continually recycle, creating Earth's diverse geological materials.
\end{keyconcept}

\mathlink{Understanding the rock cycle involves recognising repeated cycles and patterns—key mathematical concepts that help scientists predict geological changes.}

\begin{investigation}{Rock Cycle Simulation}
Using chocolate shavings (sediments), apply pressure to form a solid piece (sedimentary rock). Then gently heat and cool your solid chocolate (igneous rock formation). Finally, apply pressure and gentle heat again (metamorphic rock formation). Document each step with observations and diagrams.
\end{investigation}

\begin{tieredquestions}{Basic}
\begin{enumerate}
\item Define weathering, erosion, and sedimentation.
\item Name the three main rock types.
\end{enumerate}
\end{tieredquestions}

\begin{tieredquestions}{Intermediate}
\begin{enumerate}
\item Describe how sedimentary rocks form.
\item Explain the difference between weathering and erosion.
\end{enumerate}
\end{tieredquestions}

\begin{tieredquestions}{Advanced}
\begin{enumerate}
\item Illustrate and explain the rock cycle, giving examples of each rock type.
\item Evaluate how human activities can accelerate erosion and suggest methods to reduce this impact.
\end{enumerate}
\end{tieredquestions}

\section{Human Impact and Resource Management}

Human activities significantly impact Earth's resources and geological processes. Responsible resource management is crucial for sustainability.

\begin{keyconcept}{Sustainable Resource Management}
Sustainable management involves using resources responsibly to meet current needs without compromising the ability of future generations to meet theirs.
\end{keyconcept}

\begin{stopandthink}
Think about local resources in your community. Suggest one way your community could improve sustainability.
\end{stopandthink}

\begin{investigation}{Resource Management Debate}
Organise a class debate on the statement: "Economic development should always prioritise environmental sustainability." Research and prepare arguments for and against, and present your debate in class.
\end{investigation}

By understanding Earth's resources and geological processes, we can make informed decisions about resource management, ensuring a sustainable future for all Earth's inhabitants.