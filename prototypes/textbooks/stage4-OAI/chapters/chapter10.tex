\chapter{Earth in Space}

In this chapter, you will explore the fascinating interactions between the Earth, the Sun, and the Moon. You will learn how models of our solar system have evolved over time, how scientific understanding develops in response to new evidence, and how everyday phenomena such as day and night, seasons, and lunar phases occur. You will also investigate eclipses and learn to critically evaluate scientific models.

\section{Our Place in the Solar System}

Humans have always been fascinated by the night sky. Ancient cultures developed their own explanations for what they saw, creating myths and stories around the stars and planets. Today, we understand our position in a vast solar system, one that continues to be explored and understood through science.

\begin{keyconcept}{Key Ideas of this Chapter}
By the end of this chapter, you will be able to:
\begin{itemize}
    \item Describe the structure of our solar system and Earth's position within it.
    \item Explain observable phenomena including day and night, seasons, lunar phases, and eclipses.
    \item Outline how scientific models of our solar system have changed over time.
\end{itemize}
\end{keyconcept}

\subsection{The Solar System: An Overview}

Our solar system includes the Sun, eight planets, dwarf planets, moons, asteroids, and comets. The Sun, a massive star, contains about 99.8\% of the solar system's total mass. It provides the gravitational pull that keeps planets orbiting around it.

Earth is the third planet from the Sun, located in a region called the \keyword{habitable zone}—the area around a star where conditions are just right to allow liquid water to exist.

\marginnote{\textbf{Habitable zone:} The orbital region around a star where conditions allow liquid water and potentially life.}

\begin{stopandthink}
Why do you think liquid water is crucial in defining a planet as potentially habitable?
\end{stopandthink}

\section{Historical Models: Geocentric to Heliocentric}

Early scientists sought explanations for the movements of celestial objects. Their theories about the solar system changed dramatically as new evidence emerged, showing clearly the dynamic nature of scientific models.

\subsection{The Geocentric Model}

In ancient Greece, philosophers such as Aristotle and Ptolemy proposed the \keyword{geocentric model}, placing the Earth at the centre of the universe. They imagined planets, stars, the Sun, and Moon revolving around the Earth in perfect circular paths.

\historylink{Ptolemy's geocentric model dominated European astronomy for over 1400 years, influencing both science and culture significantly.}

\subsection{The Heliocentric Revolution}

In the 16th century, Nicolaus Copernicus proposed a radically different idea—the \keyword{heliocentric model}, where the Sun, not Earth, was at the centre of the solar system. Galileo Galilei later provided critical evidence supporting the heliocentric view through telescopic observations.

\historylink{Galileo's observations of Jupiter's moons and Venus's phases strongly supported the heliocentric theory, challenging prevailing beliefs.}

\begin{investigation}{Modelling Our Solar System}

Work in small groups. Using simple materials (such as balls, string, and torches), construct both geocentric and heliocentric models. Discuss the following:

\begin{itemize}
    \item How well does each model explain observed phenomena, such as planet movement and phases of the Moon?
    \item Why do you think the heliocentric model eventually replaced the geocentric model?
\end{itemize}
\end{investigation}

\section{Day and Night}

Earth spins on its own axis, an imaginary line running from the North Pole to the South Pole. This rotation produces the cycle of day and night, with one full rotation taking approximately 24 hours.

\subsection{Explaining Day and Night}

When one half of Earth faces the Sun, it experiences daylight, while the other half facing away experiences night. This rotation explains why the Sun appears to rise in the east and set in the west each day.

\begin{marginfigure}
\includegraphics[width=\linewidth]{placeholder-day-night-diagram}
\caption{Earth's rotation causes day and night.}
\end{marginfigure}

\begin{stopandthink}
If Earth rotated twice as fast, how long would one day-night cycle last? How would this affect life on Earth?
\end{stopandthink}

\section{Seasons on Earth}

Earth’s orbit around the Sun, combined with its tilted axis, gives rise to our seasons.

\subsection{Axis Tilt and Seasons}

Earth is tilted at an angle of approximately 23.5°. As Earth orbits the Sun, this tilt causes different hemispheres to receive varying amounts of solar energy throughout the year. When the southern hemisphere tilts toward the Sun, it experiences summer, while the northern hemisphere experiences winter, and vice versa.

\begin{example}
During December, the southern hemisphere experiences summer because it is tilted towards the Sun, receiving more direct sunlight. At this same time, the northern hemisphere experiences winter, receiving less direct sunlight.
\end{example}

\begin{investigation}{Investigating the Seasons}

Use a globe, lamp, and thermometer to model sunlight falling on Earth at different angles. Measure temperature changes at various angles to see how sunlight intensity affects temperature.

\begin{itemize}
    \item What angle produced the highest temperature? Why?
    \item How does this model explain seasonal temperature changes?
\end{itemize}
\end{investigation}

\section{The Moon and Its Phases}

The Moon orbits Earth approximately every 29.5 days, creating a regular cycle of phases.

\subsection{Understanding Moon Phases}

Moon phases occur due to the changing positions of the Earth, Moon, and Sun. As the Moon orbits Earth, we observe different amounts of the Moon’s illuminated half, creating phases such as new moon, crescent, quarter, gibbous, and full moon.

\begin{marginfigure}
\includegraphics[width=\linewidth]{placeholder-lunar-phases}
\caption{The phases of the Moon as viewed from Earth.}
\end{marginfigure}

\begin{stopandthink}
Why do we always see the same side of the Moon from Earth?
\end{stopandthink}

\section{Eclipses}

Occasionally, the Earth, Moon, and Sun align, causing eclipses.

\subsection{Solar and Lunar Eclipses}

A \keyword{solar eclipse} happens when the Moon passes directly between Earth and the Sun, casting a shadow on Earth. A \keyword{lunar eclipse} occurs when the Earth passes directly between the Sun and Moon, casting Earth's shadow onto the Moon.

\marginnote{\textbf{Solar eclipse:} The Moon blocks sunlight from reaching Earth.}

\marginnote{\textbf{Lunar eclipse:} Earth blocks sunlight from reaching the Moon.}

\begin{investigation}{Modelling Eclipses}

Using a torch (the Sun), a tennis ball (the Moon), and a globe (Earth), simulate lunar and solar eclipses. 

Discuss:
\begin{itemize}
    \item How does alignment affect the occurrence of eclipses?
    \item Why don't eclipses happen every month?
\end{itemize}
\end{investigation}

\section{Review and Extend}

\begin{tieredquestions}{Basic}
\begin{enumerate}
    \item Define heliocentric and geocentric models.
    \item What causes day and night?
    \item Name and describe two moon phases.
\end{enumerate}
\end{tieredquestions}

\begin{tieredquestions}{Intermediate}
\begin{enumerate}
    \item Explain why seasons occur.
    \item Why was Galileo’s evidence important for accepting the heliocentric model?
    \item Describe differences between solar and lunar eclipses.
\end{enumerate}
\end{tieredquestions}

\begin{tieredquestions}{Advanced}
\begin{enumerate}
    \item Analyse the impact shifting from a geocentric to heliocentric model had on scientific thinking.
    \item Predict how Earth's climate might change if its axial tilt increased significantly.
    \item Explain why eclipses do not occur every lunar cycle, including a diagram to illustrate your explanation.
\end{enumerate}
\end{tieredquestions}

\challenge{Investigate the future of space exploration—what role might humans play in colonising other planets? Discuss potential challenges and benefits.}

\mathlink{Calculate the length of a day on different planets given their rotation periods.}

By understanding Earth's place in space, you gain insight into the dynamic nature of scientific knowledge—a process of continuous discovery and refinement based on evidence.