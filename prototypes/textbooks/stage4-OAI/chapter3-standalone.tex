% Chapter 3 Standalone Document

\documentclass{article}

% Essential packages
\usepackage[utf8]{inputenc}
\usepackage[T1]{fontenc}
\usepackage{graphicx}
\usepackage{amsmath,amssymb}
\usepackage{xcolor}
\usepackage{tcolorbox}
\usepackage{enumitem}

% Custom colors
\definecolor{primary}{RGB}{0, 73, 144} % Deep blue

\title{Chapter 3: Mixtures and Separation Techniques}
\author{Emergent Mind Press}
\date{\today}

\begin{document}

\maketitle

\section{Introduction to Mixtures}

Mixtures are a fundamental concept in chemistry. A mixture consists of two or more substances that are physically combined but not chemically bonded.

\begin{tcolorbox}[colback=primary!5,colframe=primary,title=\textbf{Key Properties of Mixtures}]
Unlike compounds, mixtures:
\begin{itemize}
    \item Can be separated by physical means
    \item Retain the properties of their components
    \item Can have variable composition
\end{itemize}
\end{tcolorbox}

\section{Types of Mixtures}

Mixtures can be classified into two main categories:

\subsection{Homogeneous Mixtures}

Homogeneous mixtures have a uniform composition throughout. The components are evenly distributed and not distinguishable by eye.

\subsection{Heterogeneous Mixtures}

Heterogeneous mixtures do not have a uniform composition. The components are unevenly distributed and can be visibly distinguished.

\section{Separation Techniques}

Because the components in mixtures retain their properties, mixtures can be separated physically. Some common separation techniques include:

\subsection{Filtration}
Used to separate an insoluble solid from a liquid.

\subsection{Evaporation}
Used to separate a dissolved solid from a solution.

\subsection{Distillation}
Used to separate liquids with different boiling points.

\subsection{Chromatography}
Used to separate components based on different solubilities.

\section{Conclusion}

Understanding mixtures and how to separate them is essential for many scientific and industrial applications.

\end{document}