\chapter{Earth's Resources and Geological Change}

\section{Introduction: Our Dynamic Earth}

\marginnote{Geology is the scientific study of Earth's composition, history, and processes.}
Earth is an active, ever-changing planet. Its surface and interior undergo constant transformations due to geological processes driven by energy from within and from the Sun. In this chapter, we explore Earth's natural resources, their formation, use, and conservation, and investigate how geological phenomena shape our planet over time.

\begin{keyconcept}{What are Earth's Resources?}
Earth's resources are materials or substances occurring naturally and providing benefits for humans. They include minerals, water, fossil fuels, soil, and biological resources like plants and animals.
\end{keyconcept}

Understanding the nature, use, and management of these resources is vital for sustainable living. Additionally, geological changes, such as volcanic eruptions, earthquakes, and erosion, continuously shape Earth's surface and affect human activity.

\section{Earth's Natural Resources}

\subsection{Renewable and Non-Renewable Resources}

Earth's resources are classified into two main categories: renewable and non-renewable.

\marginnote{\keyword{Renewable resources} can be replenished naturally within a human lifespan.}
\textbf{Renewable resources} include sunlight, wind, water, and biomass. These resources regenerate naturally and sustainably if managed properly.

\marginnote{\keyword{Non-renewable resources} exist in finite amounts and renew very slowly, if at all, within human timescales.}
\textbf{Non-renewable resources} consist of fossil fuels (coal, oil, natural gas), minerals, and metals. Once extracted and used, they cannot be quickly replaced.

\begin{stopandthink}
Can you list three renewable and three non-renewable resources found near your local area? How might their availability affect your community?
\end{stopandthink}

\subsection{Formation and Extraction of Fossil Fuels}

Fossil fuels are essential but limited non-renewable resources. They originated from decomposed plants and animals buried millions of years ago under layers of sediment. High pressure and temperature over geological time transformed these remains into coal, oil, and natural gas.

\marginnote{\historylink{The Industrial Revolution in the 18th century relied heavily on coal, significantly shaping modern society.}}
Coal forms predominantly from ancient forests buried in swampy regions. Oil and natural gas derive from marine organisms that settled on ocean floors.

\begin{example}
Petroleum reserves commonly occur in porous rocks, capped by impermeable layers that trap oil and natural gas. Geologists use seismic surveys to locate these reservoirs.
\end{example}

\begin{investigation}{Modelling Oil Reservoirs}
\textbf{Aim:} To simulate how oil becomes trapped within geological formations.

\textbf{Materials:} Clear container, sand, gravel, clay, vegetable oil, water.

\textbf{Procedure:}
\begin{enumerate}
    \item Layer sand and gravel in the container, representing permeable rock layers.
    \item Place a clay layer on top to mimic an impermeable cap.
    \item Slowly pour vegetable oil into one side, observing how it moves and accumulates.
    \item Add water to simulate groundwater.
\end{enumerate}

\textbf{Observations and Questions:}
\begin{itemize}
    \item Where did the oil accumulate?
    \item What role did the clay layer perform in trapping the oil?
\end{itemize}
\end{investigation}

\subsection{Minerals and Ores}

Minerals are naturally occurring solids with defined chemical compositions and crystal structures. Ores are minerals containing valuable substances that can be economically extracted.

\marginnote{\keyword{Ore} is a mineral deposit containing economically valuable metals or minerals.}

\begin{example}
Australia's iron ore deposits, predominantly hematite (\ce{Fe2O3}), significantly contribute to global steel production.
\end{example}

\begin{stopandthink}
Why is it important for geologists to identify ores accurately before mining begins? Consider economic and environmental reasons.
\end{stopandthink}

\subsection{Water as an Essential Resource}

Freshwater availability is critical for life. Earth is rich in water, but only about 3\% is freshwater, primarily stored in glaciers, rivers, lakes, and aquifers.

\marginnote{\keyword{Aquifer} is a layer of permeable rock or sediment holding groundwater.}

Groundwater extraction must be managed sustainably to avoid depletion and contamination.

\begin{investigation}{Groundwater Filtration Experiment}
\textbf{Aim:} Demonstrate how groundwater is naturally filtered through soil layers.

\textbf{Materials:} Plastic bottle, gravel, sand, soil, dirty water, filter paper.

\textbf{Procedure:}
\begin{enumerate}
    \item Cut bottle in half; invert the top to act as a funnel.
    \item Layer gravel, sand, and soil in the funnel.
    \item Pour dirty water slowly through the layers, collecting filtered water below.
\end{enumerate}

\textbf{Questions:}
\begin{itemize}
    \item Describe differences in water before and after filtration.
    \item How does this experiment relate to natural groundwater purification?
\end{itemize}
\end{investigation}

\begin{tieredquestions}{Basic}
\begin{enumerate}
    \item Define renewable and non-renewable resources.
    \item List three minerals and their common uses.
\end{enumerate}
\end{tieredquestions}

\begin{tieredquestions}{Intermediate}
\begin{enumerate}
    \item Explain how fossil fuels are formed.
    \item Discuss two environmental impacts of extracting mineral resources.
\end{enumerate}
\end{tieredquestions}

\begin{tieredquestions}{Advanced}
\begin{enumerate}
    \item Propose a sustainable plan for managing local freshwater resources.
    \item Evaluate the economic and environmental trade-offs in mining iron ore.
\end{enumerate}
\end{tieredquestions}

\section{Geological Processes and Earth's Changing Surface}

Earth's surface continuously changes through geological processes including weathering, erosion, tectonic activity, and volcanic eruptions.

\subsection{Weathering and Erosion}

Weathering breaks down rocks through physical, chemical, and biological processes. Erosion transports these fragments, reshaping landscapes.

\marginnote{\keyword{Weathering} is the breakdown of rocks at Earth's surface; \keyword{erosion} moves these particles elsewhere.}

\begin{example}
The Twelve Apostles in Victoria were formed by erosion of coastal limestone cliffs.
\end{example}

\begin{stopandthink}
Describe a landscape in your area shaped by weathering or erosion. How might this area change in the future?
\end{stopandthink}

\subsection{Plate Tectonics: Earth's Moving Crust}

Earth's lithosphere is divided into tectonic plates moving slowly due to mantle convection currents. Plate boundaries are sites of earthquakes, volcanoes, and mountain-building.

\marginnote{\keyword{Plate tectonics} explains large-scale movements of Earth's crustal plates driven by mantle convection.}

\begin{keyconcept}{Types of Plate Boundaries}
\begin{itemize}
    \item \textbf{Divergent}: Plates move apart, creating new crust (e.g., mid-ocean ridges).
    \item \textbf{Convergent}: Plates collide, forming mountains or subduction zones.
    \item \textbf{Transform}: Plates slide horizontally, causing earthquakes (e.g., San Andreas Fault).
\end{itemize}
\end{keyconcept}

\begin{investigation}{Modelling Plate Boundaries}
\textbf{Aim:} Simulate interactions between tectonic plates.

\textbf{Materials:} Foam sheets, scissors, water tray.

\textbf{Procedure:}
\begin{enumerate}
    \item Cut foam sheets into plate shapes; float them on water.
    \item Simulate divergent, convergent, and transform boundaries by moving foam pieces accordingly.
\end{enumerate}

\textbf{Questions:}
\begin{itemize}
    \item What geological features form at each boundary type?
\end{itemize}
\end{investigation}

\subsection{Volcanoes and Earthquakes}

Volcanoes and earthquakes occur primarily at plate boundaries. Magma rises through cracks, erupting onto the surface as lava, forming volcanic landscapes. Sudden plate movements release energy causing earthquakes.

\begin{challenge}{Research the Ring of Fire and identify three significant volcanic or seismic events that have occurred there.}
\end{challenge}

\begin{tieredquestions}{Basic}
\begin{enumerate}
    \item Define weathering and erosion.
    \item What is plate tectonics?
\end{enumerate}
\end{tieredquestions}

\begin{tieredquestions}{Intermediate}
\begin{enumerate}
    \item Explain how earthquakes occur at transform boundaries.
    \item Describe two ways volcanic eruptions impact ecosystems.
\end{enumerate}
\end{tieredquestions}

\begin{tieredquestions}{Advanced}
\begin{enumerate}
    \item Assess the risks and benefits of living near a tectonic plate boundary.
    \item Explain how geological evidence supports the theory of plate tectonics.
\end{enumerate}
\end{tieredquestions}

\section{Summary}

This chapter explored Earth's resources, their sustainable management, and geological processes shaping our dynamic planet. Understanding these processes helps us responsibly manage resources and prepare for geological hazards.