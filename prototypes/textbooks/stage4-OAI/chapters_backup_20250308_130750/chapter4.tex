\chapter{Physical and Chemical Change}

Every day, we witness changes around us—from ice melting in our drinks to the rust forming on a bicycle left out in the rain. Some changes are temporary or reversible, while others permanently transform substances into something completely new. Scientists classify these transformations into two main categories: \keyword{physical changes} and \keyword{chemical changes}. Understanding the difference between these two types of changes helps us appreciate the complexities of matter and the nature of the world around us.

\section{Physical Changes}

A \keyword{physical change} occurs when a substance alters its form or appearance without changing its chemical composition. In other words, the molecules that make up the substance remain the same before and after the change.

\begin{marginfigure}
\centering
% Placeholder for figure: Ice melting
\includegraphics{ice-melting-placeholder}
\caption{Ice melting demonstrates a physical change.}
\label{fig:icemelting}
\end{marginfigure}

Common examples of physical changes include changes in state (such as melting, freezing, boiling, and condensing), breaking something into smaller pieces, and dissolving sugar or salt in water.

\begin{keyconcept}{Identifying Physical Changes}
Physical changes:
\begin{itemize}
    \item Are reversible (usually);
    \item Do not form new substances;
    \item Often involve changes in state, shape, or size.
\end{itemize}
\end{keyconcept}

\subsection{Changes of State}

When a substance changes from one state of matter to another—solid, liquid, or gas—it is undergoing a change of state. These transitions occur when substances gain or lose energy (usually as heat), causing particles to move closer together or farther apart.

\begin{marginfigure}
\centering
% Placeholder for figure: Particle arrangement in solids, liquids, gases
\includegraphics{particle-arrangement-placeholder}
\caption{Particle arrangements differ in solids, liquids, and gases.}
\label{fig:particles}
\end{marginfigure}

For example, water freezing into ice or evaporating into steam are both physical changes. The water molecules (\ce{H2O}) are the same in each state, but their energy and arrangement differ.

\begin{stopandthink}
If you melt an ice cube and then freeze it again, is the water chemically different at any point? Explain your reasoning.
\end{stopandthink}

\subsection{Dissolving as a Physical Change}

When you dissolve salt in water, the salt crystal breaks into smaller particles, distributing evenly throughout the water. However, the chemical structure of the salt (\ce{NaCl}) remains unchanged.

\begin{example}
When sugar dissolves in tea, it seems to disappear completely. However, tasting the tea reveals that the sugar molecules are still present, evenly dispersed throughout the liquid.
\end{example}

\historylink{Ancient Greek philosophers believed all matter was made of four elements—earth, air, fire, and water. It wasn't until modern chemistry developed that scientists understood dissolving as a physical rather than chemical change.}

\subsection{Reversibility of Physical Changes}

Most physical changes, such as freezing and melting, are reversible. This means the substance can return to its original form.

\begin{investigation}{Reversible or Irreversible?}
\textbf{Materials:} Ice cubes, salt, sugar, water, beakers, heat source.

\textbf{Procedure:}
\begin{enumerate}
    \item Melt ice cubes by placing them in a warm area. Observe and record changes.
    \item Dissolve sugar and salt separately in water. Heat gently to evaporate the water.
    \item Examine residues left behind. 
\end{enumerate}

\textbf{Discussion:}
Which changes were reversible? Which were irreversible? What does this tell you about physical changes?
\end{investigation}

\begin{tieredquestions}{Basic}
\begin{enumerate}
    \item Define physical change using your own words.
    \item List three examples of physical changes you observe daily.
\end{enumerate}
\end{tieredquestions}

\begin{tieredquestions}{Intermediate}
\begin{enumerate}
    \item Describe what happens to particles during melting and freezing.
    \item Explain why dissolving sugar in water is a physical change.
\end{enumerate}
\end{tieredquestions}

\begin{tieredquestions}{Advanced}
\begin{enumerate}
    \item Predict whether dissolving baking soda (\ce{NaHCO3}) in water is a physical or chemical change. Design an experiment to test your prediction.
\end{enumerate}
\end{tieredquestions}

\section{Chemical Changes}

A \keyword{chemical change} (or chemical reaction) happens when substances combine or break apart to form new substances with different chemical properties. Unlike physical changes, chemical changes usually cannot be reversed by simple physical means.

\begin{keyconcept}{Identifying Chemical Changes}
Chemical changes:
\begin{itemize}
    \item Produce new substances;
    \item Usually irreversible by physical methods;
    \item Often involve observable evidence such as color change, gas production, temperature changes, and formation of precipitates.
\end{itemize}
\end{keyconcept}

\subsection{Evidence of Chemical Changes}

Scientists recognize chemical reactions through several observable clues:

\begin{description}
    \item[Color Change:] A new substance with a distinct color forms.
    \item[Gas Production:] Bubbles or gases are produced.
    \item[Temperature Change:] Heat may be released or absorbed.
    \item[Formation of Precipitate:] A solid substance forms from solutions.
\end{description}

\begin{example}
When iron rusts, it reacts with oxygen and moisture in the air, forming iron oxide (\ce{Fe2O3}). This rust is a new compound with different properties compared to pure iron.
\end{example}

\subsection{Common Chemical Reactions}

Some common chemical reactions include:

\begin{itemize}
    \item \textbf{Combustion:} Burning fuels like wood or petrol produces heat, gases (\ce{CO2}, \ce{H2O}), and sometimes visible flames.
    \item \textbf{Rusting:} Iron reacts slowly with oxygen and moisture to produce rust (\ce{Fe2O3}).
    \item \textbf{Cooking:} Heat changes the chemical structure of food, altering taste, texture, and appearance.
\end{itemize}

\begin{stopandthink}
Is burning paper a physical or chemical change? How can you prove your reasoning scientifically?
\end{stopandthink}

\historylink{Antoine Lavoisier (1743–1794), the "Father of Modern Chemistry," first clearly distinguished between physical and chemical changes through his pioneering experiments in combustion.}

\subsection{Conservation of Mass}

In all chemical changes, the total mass of reactants equals the total mass of products. This principle is called the \keyword{Law of Conservation of Mass}. Atoms are rearranged into new substances, but no atoms are created or destroyed.

\mathlink{In chemical equations, coefficients ensure the number of atoms on each side is equal, reflecting conservation of mass quantitatively.}

\begin{investigation}{Demonstrating Conservation of Mass}
\textbf{Materials:} Vinegar (\ce{CH3COOH}), baking soda (\ce{NaHCO3}), balloon, flask, balance.

\textbf{Procedure:}
\begin{enumerate}
    \item Measure mass of flask, vinegar, baking soda, and balloon separately.
    \item Combine baking soda and vinegar, quickly sealing balloon to flask mouth.
    \item After reaction, measure mass again.
\end{enumerate}

\textbf{Discussion:} 
Did the mass before and after the reaction remain constant? Explain your observations.
\end{investigation}

\begin{tieredquestions}{Basic}
\begin{enumerate}
    \item Define chemical change clearly.
    \item Name two everyday examples of chemical changes.
\end{enumerate}
\end{tieredquestions}

\begin{tieredquestions}{Intermediate}
\begin{enumerate}
    \item List four observable signs of chemical reactions.
    \item Explain why cooking an egg is a chemical change.
\end{enumerate}
\end{tieredquestions}

\begin{tieredquestions}{Advanced}
\begin{enumerate}
    \item Research and describe a chemical reaction involved in digestion.
    \item Design a simple experiment to demonstrate the chemical reaction you chose.
\end{enumerate}
\end{tieredquestions}

\section{Comparing Physical and Chemical Changes}

Understanding both physical and chemical changes allows scientists and engineers to manipulate materials effectively in industry, medicine, cooking, and environmental management.

\begin{stopandthink}
Compare melting chocolate with baking a chocolate cake. Describe clearly which is a physical change and which is a chemical change, justifying your answers scientifically.
\end{stopandthink}

Understanding these differences gives us powerful ways to predict, control, and innovate in science and technology.