\chapter{Earth in Space}

Earth is a small, rocky planet orbiting a medium-sized star—the Sun—on the edge of a galaxy containing billions of stars. From ancient astronomers gazing at the skies, seeking patterns in the movements of planets, to modern spacecraft venturing into the cosmos, humans have always wondered about our planet's place in the universe. In this chapter, you will explore the solar system and how the Earth's movements explain phenomena such as seasons, lunar phases, and eclipses. You will also investigate how scientific models have advanced from ancient Earth-centred (geocentric) perspectives to today's Sun-centred (heliocentric) model, illustrating the dynamic and evolving nature of scientific understanding.

\section{Our Solar System}

The solar system comprises the Sun, eight planets, their moons, dwarf planets, asteroids, and comets. At its centre is the Sun—a star composed primarily of hydrogen and helium, undergoing nuclear fusion to produce light and heat.

\marginnote{\textbf{Solar System:} The Sun and all celestial bodies orbiting it due to gravity.}

\begin{keyconcept}{Planets and Their Orbits}
All planets in our solar system orbit the Sun in elliptical (oval-shaped) paths. Inner planets (Mercury, Venus, Earth, Mars) are rocky and dense, while outer planets (Jupiter, Saturn, Uranus, Neptune) are gas giants composed mainly of hydrogen and helium.
\end{keyconcept}

\historylink{Historically, Pluto was considered the ninth planet until it was reclassified as a dwarf planet by the International Astronomical Union in 2006.}

\begin{stopandthink}
Why do you think scientists decided Pluto was no longer a planet? What criteria might be used to classify celestial bodies?
\end{stopandthink}

\section{Earth's Movements and Phenomena}

Earth's motions include rotation (spinning on its axis) and revolution (orbiting the Sun). These movements are responsible for day and night, seasons, and climatic variations.

\subsection{Rotation and Day-Night Cycle}

Earth rotates around its imaginary axis once every 24 hours, creating the daily cycle of daylight and darkness. At any given moment, half of Earth's surface faces the Sun, experiencing daylight, while the other half faces away, experiencing night.

\marginnote{\textbf{Rotation:} Spinning of Earth on its axis, completing one turn every 24 hours.}

\begin{investigation}{Modelling Day and Night}
\begin{enumerate}
    \item Obtain a globe and a bright lamp representing the Sun.
    \item Place the globe about one metre from the lamp. Darken the room slightly.
    \item Slowly rotate the globe anticlockwise (as viewed from above the North Pole). Observe how different parts experience day and night.
\end{enumerate}
\textbf{Analysis:} How does the model help you understand the day-night cycle? What limitations might this model have?
\end{investigation}

\subsection{Revolution and Seasons}

Earth revolves around the Sun once every 365.25 days. Earth's axis is tilted about \(23.5^\circ\) relative to its orbital plane. This axial tilt causes seasonal changes as different hemispheres receive varying amounts of sunlight throughout the year.

\marginnote{\textbf{Revolution:} Earth's orbit around the Sun, taking one year to complete.}

\begin{example}
In December, the southern hemisphere tilts toward the Sun, experiencing summer. At the same time, the northern hemisphere tilts away, experiencing winter.
\end{example}

\begin{stopandthink}
Why do you think the areas near the equator experience less seasonal variation than those closer to the poles?
\end{stopandthink}

\section{The Moon and Lunar Phases}

The Moon is Earth's natural satellite, orbiting approximately once every 27.3 days. Lunar phases—such as full moon, new moon, and crescent—result from how sunlight illuminates the Moon as viewed from Earth.

\marginnote{\textbf{Satellite:} A celestial body orbiting another body; the Moon is Earth's natural satellite.}

\begin{keyconcept}{Phases of the Moon}
The lunar phases are caused by the changing relative positions of the Sun, Earth, and Moon. The Moon reflects sunlight, so the visible shape of the Moon changes throughout its orbit.
\end{keyconcept}

\begin{investigation}{Observing Lunar Phases}
Over one month, record the shape of the Moon each night. Sketch or photograph the Moon at the same time each evening.
\begin{enumerate}
    \item Organise your observations into a table or graph.
    \item Explain the changes you observed using a model or diagram.
\end{enumerate}
\textbf{Analysis:} Can you predict when the next full moon will occur? How accurate is your prediction?
\end{investigation}

\section{Eclipses}

An eclipse occurs when one celestial body moves into the shadow of another. There are two main types: solar eclipses and lunar eclipses.

\subsection{Solar Eclipse}

A solar eclipse occurs when the Moon passes directly between Earth and the Sun, casting a shadow on Earth's surface. Observers in the shadow's path experience the Sun being temporarily obscured.

\marginnote{\textbf{Solar Eclipse:} Occurs when the Moon blocks sunlight from reaching Earth.}

\subsection{Lunar Eclipse}

A lunar eclipse happens when Earth is positioned directly between the Sun and the Moon, casting Earth's shadow onto the Moon. Lunar eclipses can appear reddish due to sunlight passing through Earth's atmosphere.

\marginnote{\textbf{Lunar Eclipse:} Occurs when Earth's shadow covers the Moon.}

\begin{investigation}{Creating an Eclipse}
\begin{enumerate}
    \item Using a torch (Sun), ball (Moon), and larger globe (Earth), recreate a solar and lunar eclipse.
    \item Draw diagrams demonstrating the positions required for each eclipse.
\end{enumerate}
\textbf{Analysis:} How do your models explain why eclipses are relatively rare events?
\end{investigation}

\section{Modelling the Solar System: Historical Perspectives}

Ancient civilisations observed celestial objects closely, believing Earth was the stationary centre of the universe (geocentric model). Later, scientific evidence led to the heliocentric model, placing the Sun at the centre, proposed by Nicolaus Copernicus and refined by Johannes Kepler and Galileo Galilei.

\begin{keyconcept}{Geocentric vs. Heliocentric}
\begin{itemize}
    \item \textbf{Geocentric Model:} Earth-centred; planets, Sun, and stars orbit Earth.
    \item \textbf{Heliocentric Model:} Sun-centred; planets, including Earth, orbit the Sun.
\end{itemize}
\end{keyconcept}

\historylink{In 1609, Galileo observed moons orbiting Jupiter, challenging geocentric thinking by proving not all objects revolve around Earth.}

\begin{stopandthink}
Why might it have been difficult historically for people to accept the heliocentric model?
\end{stopandthink}

\section{Dynamic Nature of Scientific Models}

Scientific models evolve as new evidence emerges. The transition from geocentric to heliocentric models exemplifies how scientific theories adapt and improve.

\begin{example}
In the early 20th century, astronomers believed our galaxy was the entire universe. Observations by Edwin Hubble proved other galaxies exist, vastly expanding our cosmic perspective.
\end{example}

\challenge{Investigate more about how Edwin Hubble's discoveries changed astronomy at \textit{https://www.nasa.gov/hubble}.}

\begin{tieredquestions}{Basic}
\begin{enumerate}
    \item What causes Earth's seasons?
    \item Define rotation and revolution.
    \item Name the eight planets in our solar system.
\end{enumerate}
\end{tieredquestions}

\begin{tieredquestions}{Intermediate}
\begin{enumerate}
    \item Explain why eclipses do not occur every month.
    \item Describe how a lunar eclipse differs visually from a solar eclipse.
\end{enumerate}
\end{tieredquestions}

\begin{tieredquestions}{Advanced}
\begin{enumerate}
    \item How did Galileo's observations challenge the geocentric model?
    \item Predict what might happen if Earth's axial tilt changed significantly (for example, to \(45^\circ\)).
\end{enumerate}
\end{tieredquestions}

\section{Further Exploration and Research}
Extend your understanding through advanced reading:
\begin{itemize}
    \item "Cosmos" by Carl Sagan
    \item NASA's educational website: \textit{https://spaceplace.nasa.gov/}
\end{itemize}

By investigating Earth's place in space, you've explored important scientific phenomena and seen how scientific understanding evolves when new evidence is uncovered.