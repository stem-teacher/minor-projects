\chapter{Forces and Motion}

Have you ever wondered why a soccer ball slows down after rolling on the grass, or why astronauts float effortlessly in space? What causes your bicycle to start moving when you pedal harder and to stop when you squeeze the brakes? All these questions come down to one key idea: \keyword{forces}. Forces shape how things move, change direction, speed up, slow down, or stay still.

In this chapter, we will explore the nature of forces, how they interact, and how these interactions determine the motion of objects. Through real-life examples and hands-on investigations, you will discover how forces govern everyday actions, from sports and transport to falling objects and beyond.

\section{Understanding Forces}

A force is simply a push or a pull. Forces can make things move, slow things down, speed them up, stop them, or change their shape. When you kick a ball, throw a paper aeroplane, or stretch an elastic band, you're applying forces.

\marginnote{\textbf{Force:} A push or pull that can change an object's motion or shape.}

\begin{keyconcept}{Types of Forces}
Forces are classified into two main categories:
\begin{itemize}
    \item \keyword{Contact forces} involve direct physical interaction, such as friction, air resistance, and tension.
    \item \keyword{Non-contact forces} act at a distance without direct contact, such as gravity, magnetism, and electrostatic forces.
\end{itemize}
\end{keyconcept}

\subsection{Contact Forces}

Contact forces require two objects to be physically touching. Examples include frictional forces, tension in ropes, and buoyancy in water.

\marginnote{\historylink{The concept of friction was studied extensively by Leonardo da Vinci and later by Guillaume Amontons in the late 17th century.}}

\begin{example}
When you ride your bicycle, friction between the bike tires and the ground helps you move forward. Without friction, you would not be able to pedal effectively, and the wheels would just spin without moving forward.
\end{example}

\subsection{Non-Contact Forces}

These forces can act across empty space without physical contact. Gravity, magnetism, and electrostatic forces are examples of non-contact forces.

\marginnote{\textbf{Gravity:} A non-contact force that pulls objects toward one another. On Earth, gravity pulls objects toward the center of the planet.}

\begin{example}
When you drop an apple, gravity pulls it downward toward Earth. Even though Earth and the apple aren't in direct contact initially, gravity still acts on the apple.
\end{example}

\begin{stopandthink}
Classify each of the following forces as either contact or non-contact:
\begin{enumerate}
    \item A magnet attracting a paperclip.
    \item A footballer kicking a ball.
    \item An apple falling from a tree.
    \item Pushing a door open.
\end{enumerate}
Explain your reasoning.
\end{stopandthink}

\section{Effects of Forces}

How do forces affect objects around us? Forces can lead to:

\begin{itemize}
    \item Changing the object's speed (accelerating or decelerating).
    \item Changing the object's direction.
    \item Changing the object's shape.
\end{itemize}

\subsection{Balanced and Unbalanced Forces}

Forces rarely act alone. Usually, many forces act together on an object simultaneously. When multiple forces act on an object, the overall effect depends on whether these forces are balanced or unbalanced.

\begin{keyconcept}{Balanced Forces}
Forces are \keyword{balanced} when they are equal in size but opposite in direction. Balanced forces result in no change in motion.
\end{keyconcept}

\begin{keyconcept}{Unbalanced Forces}
Forces are \keyword{unbalanced} when the net force acting on an object is not zero. Unbalanced forces cause changes in motion—objects can speed up, slow down, or change direction.
\end{keyconcept}

\begin{example}
Imagine two teams playing tug-of-war. If both teams pull with equal strength, they create balanced forces, and the rope does not move. But if one team pulls harder, the forces are unbalanced, and the rope moves toward the stronger team.
\end{example}

\begin{investigation}{Exploring Balanced and Unbalanced Forces}
Materials: Two spring scales, a toy car, and a smooth surface.

\textbf{Procedure:}
\begin{enumerate}
    \item Attach a spring scale to each side of the toy car.
    \item Pull gently on both scales with equal force. Observe the car's motion.
    \item Now pull harder on one side. Observe again.
\end{enumerate}

\textbf{Questions:}
\begin{itemize}
    \item When were the forces balanced? How could you tell?
    \item Describe what happened when the forces became unbalanced.
\end{itemize}
\end{investigation}

\section{Newton's First Law of Motion}

Sir Isaac Newton, an English scientist from the 17th century, formulated three fundamental laws describing the motion of objects. Newton's First Law of Motion describes the relationship between forces and motion clearly and simply.

\marginnote{\historylink{Isaac Newton formulated his laws of motion in 1687 in the groundbreaking book \textit{Principia Mathematica}.}}

\begin{keyconcept}{Newton's First Law (Qualitative)}
An object at rest remains at rest, and an object in motion continues moving at a constant speed in a straight line unless acted upon by an unbalanced force.
\end{keyconcept}

This idea is also known as the law of \keyword{inertia}.

\marginnote{\textbf{Inertia:} The tendency of an object to resist changes in its state of motion.}

\begin{stopandthink}
Use Newton's First Law to explain why passengers in a car lean forward when the car suddenly brakes.
\end{stopandthink}

\begin{investigation}{Observing Inertia}
Materials: A coin, a smooth card, and a glass.

\textbf{Procedure:}
\begin{enumerate}
    \item Place the card over the top of the glass.
    \item Place the coin in the center of the card.
    \item Quickly flick the card horizontally out from under the coin.
\end{enumerate}

\textbf{Questions:}
\begin{itemize}
    \item What happened to the coin when the card was flicked away? Why?
    \item How does this demonstration illustrate inertia?
\end{itemize}
\end{investigation}

\section{Forces in Everyday Life}

Forces operate constantly around us in sports, transportation, and natural phenomena. Let's explore how some specific forces like gravity, friction, and magnetism shape our daily experiences.

\subsection{Gravity and Falling Objects}

Gravity is the force that pulls objects toward Earth. It causes objects to accelerate downward when falling freely.

\begin{example}
When a skydiver jumps from a plane, gravity pulls them downward, accelerating their fall. As their speed increases, air resistance (a frictional force) also increases, eventually balancing gravity and causing the skydiver to reach a constant speed, known as terminal velocity.
\end{example}

\subsection{Friction: Friend or Foe?}

Friction is a force that opposes motion between two surfaces in contact. Friction can be helpful, such as providing traction for bicycles and cars, or problematic, like wearing down machinery parts.

\begin{investigation}{Measuring Frictional Force}
Materials: Spring scale, wooden block, different surfaces (smooth table, sandpaper, carpet).

\textbf{Procedure:}
\begin{enumerate}
    \item Attach the spring scale to the wooden block.
    \item Pull the block across each surface, noting the force required to move it.
\end{enumerate}

\textbf{Questions:}
\begin{itemize}
    \item Which surface had the greatest friction? Which had the least?
    \item Why do you think friction varied between surfaces?
\end{itemize}
\end{investigation}

\section{Summary of Key Ideas}

Forces shape how objects move and interact. Understanding balanced and unbalanced forces, as well as Newton's First Law, helps explain the world around us.

\begin{tieredquestions}{Basic}
\begin{enumerate}
    \item Define force.
    \item Give three examples of contact forces.
\end{enumerate}
\end{tieredquestions}

\begin{tieredquestions}{Intermediate}
\begin{enumerate}
    \item Explain the difference between balanced and unbalanced forces.
    \item Describe how gravity affects falling objects.
\end{enumerate}
\end{tieredquestions}

\begin{tieredquestions}{Advanced}
\begin{enumerate}
    \item Explain inertia and provide two everyday examples.
    \item Discuss how friction can be both beneficial and harmful in daily life.
\end{enumerate}
\end{tieredquestions}