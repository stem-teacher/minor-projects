\chapter{Diversity of Life (Classification and Survival)}

\section{Introduction: The Richness of Life}

Life on Earth is astonishingly diverse, with millions of known species and potentially millions more yet undiscovered. Scientists estimate that around 8.7 million species inhabit our planet, each uniquely adapted to survive and thrive in its environment. From microscopic bacteria to massive whales, the variety of structures, forms and behaviors is breathtaking. This chapter explores how scientists classify living things, the importance of biodiversity, and how organisms' structures and functions are adapted for survival.

\marginnote{\textbf{Biodiversity:} The variety of life found on Earth, including the different species, genetic variations, and ecosystems.}

\begin{stopandthink}
What do you think is the importance of classifying living organisms? How might classification help scientists study life on Earth?
\end{stopandthink}

\section{Characteristics of Living Things}

To study and classify life effectively, scientists first identify common characteristics that distinguish living organisms from non-living objects.

\begin{keyconcept}{What Makes Something Alive?}
All living organisms share the following fundamental characteristics:
\begin{itemize}
    \item Movement
    \item Respiration
    \item Sensitivity (response to stimuli)
    \item Growth
    \item Reproduction
    \item Excretion
    \item Nutrition
\end{itemize}
These can easily be remembered using the acronym \keyword{MRS GREN}.
\end{keyconcept}

\begin{investigation}{Observing Life: Is it Alive?}
\begin{enumerate}
    \item Collect samples such as leaves, rocks, insects, water droplets, and bread mold.
    \item Observe and record whether each sample demonstrates the characteristics of life.
    \item Discuss your observations and create a table to classify the samples as living, once-living, or non-living.
\end{enumerate}
\end{investigation}

\begin{stopandthink}
If a robot moves and responds to its environment, is it considered alive? Justify your answer using MRS GREN.
\end{stopandthink}

\begin{tieredquestions}{Basic}
\begin{enumerate}
    \item List three characteristics of living organisms.
    \item Using MRS GREN, explain why a stone is not a living organism.
\end{enumerate}
\end{tieredquestions}

\begin{tieredquestions}{Intermediate}
\begin{enumerate}
    \item Imagine you discovered an unknown object. Describe how you could test whether it is alive.
    \item Explain why reproduction is essential for the survival of a species.
\end{enumerate}
\end{tieredquestions}

\begin{tieredquestions}{Advanced}
\begin{enumerate}
    \item Viruses can reproduce within host cells but lack many other characteristics of life. Debate whether viruses should be classified as living organisms.
\end{enumerate}
\end{tieredquestions}

\section{Classifying Living Things}

Scientists classify organisms into groups based on shared characteristics. This process, known as \keyword{classification}, helps scientists understand relationships, evolutionary history, and ecological roles.

\subsection{Historical Context of Classification}

\historylink{Carl Linnaeus, an 18th-century Swedish botanist, developed the binomial nomenclature system still used today.}

Early classification systems were simple, but as scientists learned more about life, the need for more complex systems arose. Carl Linnaeus developed a universally recognized naming system called \keyword{binomial nomenclature}, giving each species a two-part scientific name.

\begin{example}
Humans' scientific name is \textit{Homo sapiens}. \textit{Homo} is the genus, and \textit{sapiens} is the species.
\end{example}

\begin{stopandthink}
Why is it important for scientists worldwide to use a standardized system for naming organisms?
\end{stopandthink}

\subsection{Levels of Classification}

Modern classification organizes life into hierarchical categories:

\begin{itemize}
    \item Domain
    \item Kingdom
    \item Phylum
    \item Class
    \item Order
    \item Family
    \item Genus
    \item Species
\end{itemize}

\marginnote{\textbf{Mnemonic:} \textit{Dear King Philip Came Over For Good Soup} helps recall the order of classification.}

\begin{keyconcept}{Domains and Kingdoms}
Currently, life is categorized into three domains:
\begin{itemize}
    \item Archaea (ancient bacteria in extreme environments)
    \item Bacteria (true bacteria)
    \item Eukarya (plants, animals, fungi, protists)
\end{itemize}

Within the Eukarya domain, living organisms are further divided into kingdoms:
\begin{itemize}
    \item Animals
    \item Plants
    \item Fungi
    \item Protists
\end{itemize}
\end{keyconcept}

\begin{investigation}{Classification Keys}
\begin{enumerate}
    \item Collect images of various animals and plants.
    \item Create a dichotomous key to classify the organisms based on observable features.
    \item Exchange your key with another group and test its accuracy.
\end{enumerate}
\end{investigation}

\begin{tieredquestions}{Basic}
\begin{enumerate}
    \item Name the three domains of life.
    \item What makes the Eukarya domain different from Archaea and Bacteria?
\end{enumerate}
\end{tieredquestions}

\begin{tieredquestions}{Intermediate}
\begin{enumerate}
    \item Explain the purpose of a dichotomous key in classification.
    \item Choose an organism and list its classification from domain to species.
\end{enumerate}
\end{tieredquestions}

\begin{tieredquestions}{Advanced}
\begin{enumerate}
    \item Research and debate why scientific classifications sometimes change over time.
\end{enumerate}
\end{tieredquestions}

\section{Adaptations: Survival and Structure}

\subsection{Structural Adaptations}

Organisms have specialized body parts, or \keyword{structural adaptations}, that help them survive in their habitats.

\begin{example}
The thick fur of a polar bear provides insulation from freezing temperatures, while the camel’s hump stores fat for energy in harsh desert environments.
\end{example}

\marginnote{\challenge{Investigate the unique adaptations of Australian marsupials. How do their structures enhance survival?}}

\subsection{Functional Adaptations}

Internal structures and functions such as respiration, digestion, and circulation also contribute to survival. These features, called \keyword{functional adaptations}, help organisms efficiently carry out life processes.

\begin{stopandthink}
How do the gills of fish illustrate a functional adaptation for survival underwater?
\end{stopandthink}

\begin{investigation}{Adaptation Analysis}
\begin{enumerate}
    \item Choose two different animals in contrasting environments (e.g., rainforest frog and desert lizard).
    \item Research and list their structural and functional adaptations.
    \item Present your findings visually through a poster or digital presentation.
\end{enumerate}
\end{investigation}

\begin{tieredquestions}{Basic}
\begin{enumerate}
    \item Define structural adaptation and provide two examples.
\end{enumerate}
\end{tieredquestions}

\begin{tieredquestions}{Intermediate}
\begin{enumerate}
    \item Explain how functional adaptations differ from structural adaptations using examples.
\end{enumerate}
\end{tieredquestions}

\begin{tieredquestions}{Advanced}
\begin{enumerate}
    \item Predict how structural and functional adaptations may change if an organism's environment changes dramatically.
\end{enumerate}
\end{tieredquestions}

\section{Biodiversity and Its Importance}

Biodiversity refers to the variety of life forms within an ecosystem. High biodiversity often indicates a healthy and stable ecosystem.

\begin{keyconcept}{Why Biodiversity Matters}
Biodiversity provides:
\begin{itemize}
    \item Stability through varied food webs
    \item Genetic diversity for adaptation
    \item Resources such as food, medicine, and raw materials
\end{itemize}
Protecting biodiversity ensures ecosystems remain resilient.
\end{keyconcept}

\begin{stopandthink}
What might happen if biodiversity decreases dramatically within an ecosystem you know?
\end{stopandthink}

\begin{investigation}{Local Biodiversity Survey}
\begin{enumerate}
    \item Choose a local habitat (park, garden, or schoolyard).
    \item Conduct a biodiversity survey by observing and recording the types and numbers of organisms present.
    \item Suggest ways to improve biodiversity in your chosen area.
\end{enumerate}
\end{investigation}

This comprehensive exploration of life's diversity, classification, adaptations, and biodiversity provides a solid foundation for understanding and protecting life on Earth.