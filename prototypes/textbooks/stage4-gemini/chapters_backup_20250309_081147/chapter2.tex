\chapter{Properties of Matter (Particle Theory)}

\FloatBarrier
\1

Everything around you is made up of \keyword{matter}. Matter is anything that has mass and occupies space. The air you breathe, the desk you sit at, the water you drink—all are forms of matter. But have you ever wondered what matter actually is, and how it behaves? Scientists have asked these same questions for centuries, developing theories and models to explain their observations.

In this chapter, we will explore the particle theory of matter, a powerful scientific model that helps us understand the properties and behaviour of solids, liquids, and gases. We will examine how this theory explains everyday experiences like why solids hold their shape or why gases can fill any container. Additionally, we'll look at how scientific theories evolve over time as new evidence emerges.

\FloatBarrier
\1

\historylink{Ancient Greek philosophers, such as Democritus, proposed matter was made up of tiny, indivisible particles called `atomos'.}Early philosophers and scientists debated the nature of matter. Was matter continuous (meaning it could be divided endlessly), or was it made up of smaller, indivisible particles?

\begin{keyconcept}{Continuous vs. Particle Model}
Historically, two opposing models were proposed:
\begin{itemize}
    \item \textbf{Continuous Model}: Matter can be divided infinitely without reaching a limit.
    \item \textbf{Particle Model}: Matter consists of discrete, indivisible particles.
\end{itemize}
\end{keyconcept}

For many centuries, Aristotle's continuous model dominated, as it seemed intuitive. However, experiments and observations gradually provided evidence supporting the particle model.

\begin{stopandthink}
What everyday evidence might suggest matter is made up of particles rather than being continuous?
\end{stopandthink}

\FloatBarrier
\1

Today, scientists widely accept the particle theory of matter, also known as the kinetic particle theory. This theory helps explain the properties and behaviour of matter clearly and simply.

\begin{keyconcept}{Main Ideas of Particle Theory}
Particle theory states that:
\begin{enumerate}
    \item All matter consists of tiny particles too small to be seen clearly, even with powerful microscopes.
    \item These particles are always in constant motion.
    \item Particles attract each other, with the strength of attraction depending on their distance apart.
    \item Particles move faster and further apart when heated (expansion) and slower and closer together when cooled (contraction).
\end{enumerate}
\end{keyconcept}

\FloatBarrier
\1

Matter commonly exists in three states: solids, liquids, and gases. Each state has distinct properties and particle arrangements.

\subsection{Solids}

A solid has a definite shape and volume. Particles in a solid are tightly packed together and vibrate in fixed positions.

\begin{marginfigure}
% Figure placeholder: Diagram showing particles in solid state tightly packed and orderly.
\caption{Particles in a solid are closely packed and vibrate in place.}
\end{marginfigure}

\textbf{Properties of solids:}
\begin{itemize}
    \item Fixed shape and volume
    \item Incompressible (cannot be easily compressed)
    \item Particles vibrate but do not move freely
\end{itemize}

\begin{stopandthink}
Why can't you easily compress a wooden block, even if you apply considerable force?
\end{stopandthink}

\subsection{Liquids}

Liquids have a definite volume but no fixed shape. They take the shape of their container. Particles in liquids are close together but can move and slide past each other.

\begin{marginfigure}
% Figure placeholder: Liquid particles closely spaced, able to move and slide past each other.
\caption{Particles in a liquid are close but can flow past one another.}
\end{marginfigure}

\textbf{Properties of liquids:}
\begin{itemize}
    \item Fixed volume but shape can change
    \item Difficult to compress
    \item Particles move freely within the liquid, allowing it to flow
\end{itemize}

\subsection{Gases}

Gases have neither a fixed shape nor volume—they expand to fill their container. Particles in gases move rapidly and are far apart.

\begin{marginfigure}
% Figure placeholder: Gas particles widely spaced, moving rapidly in random directions.
\caption{Particles in a gas move rapidly and randomly, filling available space.}
\end{marginfigure}

\textbf{Properties of gases:}
\begin{itemize}
    \item No fixed shape or volume
    \item Easy to compress because particles are far apart
    \item Particles move quickly and randomly
\end{itemize}

\begin{stopandthink}
When you pump up a bicycle tyre, why can you easily compress air but not water?
\end{stopandthink}

\FloatBarrier
\1

When matter is heated, particles gain energy, move faster, and spread apart. This process is called \keyword{expansion}. Cooling matter causes particles to lose energy, slow down, and move closer together, resulting in \keyword{contraction}.

\begin{investigation}{Observing Expansion and Contraction}
\textbf{Materials:} Balloon, freezer, hot water, measuring tape.

\textbf{Procedure:}
\begin{enumerate}
    \item Partially inflate a balloon, measure and record its circumference.
    \item Place balloon in freezer for 15 minutes, then measure circumference again.
    \item Immerse balloon briefly in warm water and measure circumference again.
\end{enumerate}

\textbf{Questions:}
\begin{enumerate}
    \item Did the balloon expand or contract in each situation? Explain why.
    \item How are your observations explained by particle theory?
\end{enumerate}
\end{investigation}

\FloatBarrier
\1

Compression involves reducing the space between particles. Gases are easily compressed because their particles are far apart. Solids and liquids are difficult to compress due to closely packed particles.

\begin{example}
A syringe filled with air can be easily compressed, but if you fill it with water, it is almost impossible to compress. This shows that gases are compressible, while liquids are practically incompressible.
\end{example}

\FloatBarrier
\1

Scientific knowledge changes over time as new evidence emerges. Our current particle theory evolved from earlier models like Aristotle’s continuous matter and Dalton’s atomic theory.

\historylink{John Dalton (1766–1844) reintroduced the atomic theory, suggesting atoms were indivisible and unique for each element.}

\begin{keyconcept}{Scientific Theories Evolve}
Scientific theories change as new evidence emerges from experiments and observations. This process of refining and changing ideas is central to scientific progress.
\end{keyconcept}

\FloatBarrier
\1

\begin{tieredquestions}{Basic}
\begin{enumerate}
    \item List the three states of matter and one key property of each.
    \item Define expansion and contraction using particle theory.
\end{enumerate}
\end{tieredquestions}

\begin{tieredquestions}{Intermediate}
\begin{enumerate}
    \item Explain why gases are more compressible than liquids or solids.
    \item Describe how heating affects the particles in a solid.
\end{enumerate}
\end{tieredquestions}

\begin{tieredquestions}{Advanced}
\begin{enumerate}
    \item Imagine you have a solid metal ball that cannot fit through a metal ring. When heated, the ring expands. Using particle theory, explain if the ball can now pass through the ring.
    \item Research and summarise one historical experiment that provided evidence supporting particle theory.
\end{enumerate}
\end{tieredquestions}

\FloatBarrier
\1

In this chapter, we have explored how particle theory explains the properties and behaviours of matter in different states—solid, liquid, and gas. We have learnt about historical ideas and seen how scientific understanding changes with new evidence. Understanding particle theory helps us explain everyday phenomena and predict how matter will behave in different circumstances.
\FloatBarrier
