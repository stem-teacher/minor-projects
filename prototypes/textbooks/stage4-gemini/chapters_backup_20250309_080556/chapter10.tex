```latex
\chapter{Earth in Space}

\begin{marginnote}
\textbf{Syllabus Outcomes:}
\begin{itemize}
    \item SC4-12ES Explains Earth’s dynamic geological history and the effect of astronomical events on Earth.
\end{itemize}
\textbf{Inquiry Questions:}
\begin{itemize}
    \item How do models help us understand Earth’s place in space?
    \item What are the effects of the relative positions of the Earth, Sun and Moon?
\end{itemize}
\end{marginnote}

\section*{\centering Chapter Overview}

Welcome to the cosmic perspective! In this chapter, we'll embark on an exciting journey to explore our place in the vast universe. We will investigate Earth’s position in space, focusing on our solar system, the fascinating dance of seasons, and the captivating phases of the Moon.

Understanding Earth in Space is not just about memorising facts; it’s about grasping how our planet functions within a larger system.  This knowledge is crucial for understanding many aspects of our lives, from predicting tides to planning satellite launches, and even appreciating the delicate balance of our environment.

This chapter directly addresses the NSW Stage 4 Science syllabus outcome \textbf{SC4-12ES}, where you will learn to explain Earth’s dynamic geological history and the effect of astronomical events on Earth. We’ll delve into how scientists use \keyword{models} to understand complex phenomena like day and night, lunar phases, and eclipses.  You’ll also discover how our understanding of the solar system has evolved over centuries, shifting from Earth-centred (\keyword{geocentric}) to Sun-centred (\keyword{heliocentric}) models. This journey highlights a key aspect of science – that scientific ideas are not fixed but change and improve as we gather more evidence.

So, get ready to explore the cosmos and discover the wonders of Earth in space!

\FloatBarrier
\1

Imagine taking a step back, far away from Earth, to look at our cosmic neighbourhood. We live in the \keyword{Solar System}, a family of celestial objects bound together by gravity, all orbiting a central star – our Sun.

\begin{keyconcept}{What is the Solar System?}
The Solar System is a system consisting of the Sun, planets, moons, asteroids, comets, and other smaller objects, all held together by the Sun’s gravitational pull.
\end{keyconcept}

\begin{marginfigure}
\includegraphics[width=\marginparwidth]{placeholder-solarsystem.pdf}
\caption{\label{fig:solarsystem}A simplified model of our Solar System. \textit{Image to be added.}}
\end{marginfigure}

\subsection{The Sun: Our Star}

At the heart of our Solar System is the \keyword{Sun}, a giant ball of hot gas that provides light and heat to all the planets.  The Sun is a star, just like the countless stars you see at night, but it's much closer to us. In fact, it’s about 150 million kilometres away!  The Sun is mostly made of hydrogen and helium, and it generates energy through nuclear fusion, a process we will learn more about in later chapters.

\begin{stopandthink}
Why is the Sun so important for life on Earth?
\end{stopandthink}

\subsection{Planets: Orbiting Worlds}

Orbiting the Sun are eight major planets.  These planets are vastly different from each other, each with unique characteristics.  They are categorised into two main groups:

\begin{itemize}
    \item \textbf{Inner, Rocky Planets:} These are closer to the Sun and are made of rock and metal. They are:
        \begin{itemize}
            \item \textbf{Mercury:} The smallest planet and closest to the Sun.
            \item \textbf{Venus:}  A scorching hot planet, often called Earth's "sister planet" due to similar size.
            \item \textbf{Earth:} Our home, the only known planet to support life.
            \item \textbf{Mars:} The "Red Planet," known for its rusty colour and potential for past (or even present!) life.
        \end{itemize}
    \item \textbf{Outer, Gas Giant Planets:}  These are further from the Sun and are much larger, composed mainly of gases like hydrogen and helium. They are:
        \begin{itemize}
            \item \textbf{Jupiter:} The largest planet in our Solar System, famous for its Great Red Spot, a giant storm.
            \item \textbf{Saturn:} Known for its spectacular rings made of ice and rock.
            \item \textbf{Uranus:} An ice giant that rotates on its side.
            \item \textbf{Neptune:} The furthest planet from the Sun, another ice giant with strong winds.
        \end{itemize}
\end{itemize}

\begin{marginnote}
\challenge{Mnemonic for Planet Order:}
Try creating your own mnemonic to remember the order of the planets! For example: \textit{My Very Educated Mother Just Served Us Noodles}.
\end{marginnote}

\begin{tieredquestions}{Basic}
\begin{enumerate}
    \item What is at the centre of our Solar System?
    \item Name the four inner, rocky planets.
    \item Name the four outer, gas giant planets.
\end{enumerate}
\end{tieredquestions}

\begin{tieredquestions}{Intermediate}
\begin{enumerate}
    \item Describe two key differences between the inner and outer planets.
    \item Why is Earth considered unique among the planets in our Solar System?
    \item  Imagine you are an astronaut visiting Mars. Describe what you might see.
\end{enumerate}
\end{tieredquestions}

\begin{tieredquestions}{Advanced}
\begin{enumerate}
    \item Research and compare the atmospheric composition of Venus and Mars. What makes Venus so hot and Mars so cold?
    \item  Explain why the gas giants are located further away from the Sun than the rocky planets. (Hint: Think about the early Solar System and the Sun's heat).
    \item  If a new planet was discovered in our Solar System, where would you expect it to be located and what characteristics might it have? Justify your answer.
\end{enumerate}
\end{tieredquestions}


\subsection{Moons, Asteroids, and Comets: The Rest of the Family}

Planets are not the only objects in our Solar System.  \keyword{Moons} are natural satellites that orbit planets. Earth has one large moon, simply called "the Moon," but many other planets have multiple moons – Jupiter and Saturn have dozens!

\keyword{Asteroids} are rocky and metallic objects that mostly orbit in a belt between Mars and Jupiter, called the asteroid belt. They are much smaller than planets.

\keyword{Comets} are icy bodies that release gas and dust as they get closer to the Sun, creating spectacular tails. They often come from the outer reaches of the Solar System.

\begin{investigation}{Modelling the Solar System}
\textbf{Purpose:} To create a scale model of the distances of the planets from the Sun to understand the vastness of the Solar System.

\textbf{Materials:}
\begin{itemize}
    \item A long roll of paper (at least 5 metres)
    \item Markers or pens
    \item Measuring tape or ruler
\end{itemize}

\textbf{Procedure:}
\begin{enumerate}
    \item Let’s imagine the Sun is at the very beginning of your paper (0 cm mark).
    \item We'll use a scale where 1 cm represents approximately 10 million kilometres.
    \item Using the distances in Table 10.1, calculate the scaled distance of each planet from the Sun.
    \item Mark the position of each planet on your paper, starting from the Sun. Label each planet clearly.
\end{enumerate}

\textbf{Table 10.1: Approximate Distances of Planets from the Sun}
\begin{center}
\begin{tabular}{lc}
\textbf{Planet} & \textbf{Distance from Sun (million km)} \\
\hline
Mercury & 58 \\
Venus & 108 \\
Earth & 150 \\
Mars & 228 \\
Jupiter & 778 \\
Saturn & 1430 \\
Uranus & 2870 \\
Neptune & 4500 \\
\end{tabular}
\end{center}

\textbf{Discussion:}
\begin{enumerate}
    \item What do you notice about the distances between the inner planets compared to the outer planets in your model?
    \item Why is it difficult to create a truly scale model of the Solar System that includes both distances and sizes of planets on a single piece of paper? (Hint: Think about the relative sizes of planets and distances between them).
    \item How does this activity help you understand the scale of our Solar System?
\end{enumerate}
\end{investigation}


\FloatBarrier
\1

Humans have been observing the sky for thousands of years.  Early civilisations developed models to explain the movement of celestial objects.  Two major models emerged: the \keyword{geocentric model} and the \keyword{heliocentric model}.

\subsection{The Geocentric Model: Earth at the Centre}

For a long time, people believed in a \textbf{geocentric} model, which places the Earth at the centre of the universe.  In this model, the Sun, Moon, stars, and planets were thought to orbit around the Earth.  This idea seemed to make sense based on daily observations – the Sun appears to rise in the east and set in the west, as if it's moving around us.  Ancient Greek philosophers like Ptolemy developed complex geocentric models to explain the observed movements of planets.

\begin{marginfigure}
\includegraphics[width=\marginparwidth]{placeholder-geocentric.pdf}
\caption{\label{fig:geocentric}A simplified diagram of the geocentric model. \textit{Image to be added.}}
\end{marginfigure}

\begin{historylink}{Ptolemy and the Almagest}
Ptolemy's \textit{Almagest}, written in the 2nd century AD, was a comprehensive astronomical treatise that detailed the geocentric model and remained the standard model for over 1400 years.
\end{historylink}


\subsection{The Heliocentric Model: Sun at the Centre}

Over time, some astronomers noticed inconsistencies and complexities in the geocentric model. In the 16th century, Nicolaus Copernicus, a Polish astronomer, proposed a revolutionary idea: the \textbf{heliocentric} model. This model places the Sun at the centre of the Solar System, with the Earth and other planets orbiting around it.

\begin{marginfigure}
\includegraphics[width=\marginparwidth]{placeholder-heliocentric.pdf}
\caption{\label{fig:heliocentric}A simplified diagram of the heliocentric model. \textit{Image to be added.}}
\end{marginfigure}

Copernicus's heliocentric model provided a simpler and more elegant explanation for the observed motions of planets, including phenomena like retrograde motion (where planets appear to move backwards in the sky).  Initially, the heliocentric model faced resistance as it contradicted long-held beliefs and religious views. However, with further observations and the work of scientists like Galileo Galilei and Johannes Kepler, the heliocentric model gradually gained acceptance.

\begin{historylink}{Galileo and the Telescope}
Galileo Galilei's observations with his telescope, such as the phases of Venus and the moons of Jupiter, provided crucial evidence supporting the heliocentric model and challenging the geocentric view.
\end{historylink}

\begin{keyconcept}{Evolution of Scientific Models}
The shift from the geocentric to the heliocentric model demonstrates how scientific models evolve.  Models are not static; they are constantly refined or even replaced as new evidence emerges and our understanding deepens. This dynamic nature is a hallmark of scientific progress.
\end{keyconcept}

\begin{stopandthink}
Why was the shift from geocentric to heliocentric model considered a scientific revolution?
\end{stopandthink}

\begin{tieredquestions}{Basic}
\begin{enumerate}
    \item What is the main difference between the geocentric and heliocentric models of the Solar System?
    \item Which model places the Earth at the centre?
    \item Who proposed the heliocentric model?
\end{enumerate}
\end{tieredquestions}

\begin{tieredquestions}{Intermediate}
\begin{enumerate}
    \item Explain why the geocentric model was initially accepted for so long.
    \item What evidence supported the heliocentric model over time?
    \item Describe what is meant by "retrograde motion" of planets and how the heliocentric model explains it more simply than the geocentric model.
\end{enumerate}
\end{tieredquestions}

\begin{tieredquestions}{Advanced}
\begin{enumerate}
    \item Research the contributions of Tycho Brahe and Johannes Kepler to the development of the heliocentric model. How did their work build upon and improve Copernicus's initial ideas?
    \item Discuss the social and cultural impact of the shift from a geocentric to a heliocentric worldview. Why was this change so significant beyond just astronomy?
    \item  Consider the statement: "Scientific models are just theories, they are not necessarily true."  Discuss this statement in the context of the geocentric and heliocentric models. How do we decide which model is "better" in science?
\end{enumerate}
\end{tieredquestions}


\FloatBarrier
\1

The Earth is not stationary; it's constantly in motion.  Two main motions of Earth are crucial for our daily and yearly experiences: \keyword{rotation} and \keyword{revolution}.

\subsection{Earth's Rotation: Day and Night}

Earth spins on its axis, an imaginary line passing through the North and South Poles. This spinning motion is called \textbf{rotation}. One complete rotation takes approximately 24 hours, which we experience as a day and night cycle.

As Earth rotates, different parts of the Earth face the Sun. When your location faces the Sun, it's daytime. As Earth continues to rotate, your location turns away from the Sun, and it becomes night-time.

\begin{marginfigure}
\includegraphics[width=\marginparwidth]{placeholder-earthrotation.pdf}
\caption{\label{fig:earthrotation}Diagram illustrating Earth's rotation and day/night cycle. \textit{Image to be added.}}
\end{marginfigure}

\begin{stopandthink}
If Earth rotated in the opposite direction, how would sunrise and sunset times change?
\end{stopandthink}


\subsection{Earth's Revolution: The Year}

While Earth rotates on its axis, it also travels around the Sun in an elliptical path called an \keyword{orbit}. This movement around the Sun is called \textbf{revolution}. One complete revolution takes approximately 365.25 days, which we call a year.  The extra 0.25 days is why we have a leap year every four years, adding an extra day (February 29th) to keep our calendar aligned with Earth's orbit.

\begin{marginfigure}
\includegraphics[width=\marginparwidth]{placeholder-earthrevolution.pdf}
\caption{\label{fig:earthrevolution}Diagram illustrating Earth's revolution around the Sun and its orbit. \textit{Image to be added.}}
\end{marginfigure}


\subsection{Earth's Tilt and Seasons}

Earth's axis is not perfectly vertical; it is tilted at an angle of about 23.5 degrees relative to its orbital plane (the plane of Earth's orbit around the Sun). This tilt is the reason we experience \keyword{seasons}.

As Earth revolves around the Sun, different hemispheres (Northern and Southern) are tilted more directly towards the Sun at different times of the year.

\begin{itemize}
    \item \textbf{Summer in the Hemisphere tilted towards the Sun:} When the Northern Hemisphere is tilted towards the Sun, it receives more direct sunlight for longer periods. This results in warmer temperatures and longer days – summer in the Northern Hemisphere (and winter in the Southern Hemisphere).
    \item \textbf{Winter in the Hemisphere tilted away from the Sun:} When the Northern Hemisphere is tilted away from the Sun, it receives less direct sunlight for shorter periods. This results in colder temperatures and shorter days – winter in the Northern Hemisphere (and summer in the Southern Hemisphere).
    \item \textbf{Spring and Autumn:} In between summer and winter, during spring and autumn (also called fall), neither hemisphere is tilted directly towards or away from the Sun.  Day and night lengths are more equal, and temperatures are moderate.
\end{itemize}

\begin{marginfigure}
\includegraphics[width=\marginparwidth]{placeholder-earthseasons.pdf}
\caption{\label{fig:earthseasons}Diagram illustrating Earth's tilt and the seasons. \textit{Image to be added.}}
\end{marginfigure}

\begin{keyconcept}{Seasons are due to Earth's Tilt}
It's important to understand that seasons are caused by the Earth's tilt, \textbf{not} by changes in Earth's distance from the Sun in its orbit.  Earth's orbit is slightly elliptical, but this has a very minor effect on seasons compared to the tilt of the axis.
\end{keyconcept}

\begin{tieredquestions}{Basic}
\begin{enumerate}
    \item What is Earth's rotation? What does it cause?
    \item What is Earth's revolution? What does it cause?
    \item What causes the seasons on Earth?
\end{enumerate}
\end{tieredquestions}

\begin{tieredquestions}{Intermediate}
\begin{enumerate}
    \item Explain in your own words how Earth's rotation leads to day and night.
    \item Describe how Earth's tilt and revolution around the Sun result in seasons.
    \item If Earth had no tilt, would we still have seasons? Explain your answer.
\end{enumerate}
\end{tieredquestions}

\begin{tieredquestions}{Advanced}
\begin{enumerate}
    \item Research and explain why the Southern Hemisphere experiences seasons more strongly than the Northern Hemisphere (Hint: Consider the distribution of land and water).
    \item Explain how the angle of sunlight affects the amount of energy received per unit area on Earth's surface. How does this relate to temperature differences between summer and winter?
    \item  Imagine a planet similar to Earth but with a much larger axial tilt (e.g., 60 degrees). Describe what the seasons would be like on this planet. Would life as we know it be possible?
\end{enumerate}
\end{tieredquestions}


\FloatBarrier
\1

Our Moon is Earth's only natural satellite, and it's the brightest object in the night sky after the Sun.  It plays a significant role in Earth's environment, influencing tides and providing nighttime light.

\subsection{Lunar Phases: The Changing Face of the Moon}

As the Moon orbits Earth, we see different amounts of its sunlit surface from our perspective. These changing appearances are called \keyword{lunar phases}. The cycle of lunar phases takes approximately 29.5 days, known as a lunar month.

The main lunar phases are:

\begin{itemize}
    \item \textbf{New Moon:} The Moon is between the Earth and the Sun. The sunlit side faces away from Earth, so we cannot see the Moon from Earth.
    \item \textbf{Waxing Crescent:}  A small sliver of the Moon becomes visible, and it grows larger each night. "Waxing" means growing larger.
    \item \textbf{First Quarter:} Half of the Moon is illuminated (right half as seen from the Northern Hemisphere).
    \item \textbf{Waxing Gibbous:} More than half of the Moon is illuminated, and it continues to grow. "Gibbous" means more than half but not full.
    \item \textbf{Full Moon:} The Earth is between the Sun and the Moon. The entire sunlit side of the Moon faces Earth, so we see a fully illuminated Moon.
    \item \textbf{Waning Gibbous:} After the full moon, the illuminated portion starts to decrease. "Waning" means shrinking.
    \item \textbf{Last Quarter (Third Quarter):} Half of the Moon is illuminated again (left half as seen from the Northern Hemisphere).
    \item \textbf{Waning Crescent:} The illuminated sliver continues to shrink until it disappears into the new moon phase again.
\end{itemize}

\begin{marginfigure}
\includegraphics[width=\marginparwidth]{placeholder-lunarphases.pdf}
\caption{\label{fig:lunarphases}Diagram illustrating the lunar phases. \textit{Image to be added.}}
\end{marginfigure}

\begin{example}{Observing Lunar Phases}
Try to observe the Moon every night for a month and record its appearance. You can draw sketches or take photos. See if you can track the cycle of lunar phases and identify each phase as it occurs.
\end{example}

\begin{stopandthink}
Why do we always see the same side of the Moon from Earth? (Hint: Think about the Moon's rotation and revolution periods).
\end{stopandthink}


\subsection{Eclipses: Shadows in Space}

Sometimes, the Earth, Sun, and Moon align in a straight line, casting shadows and causing \keyword{eclipses}. There are two main types of eclipses: solar eclipses and lunar eclipses.

\begin{itemize}
    \item \textbf{Solar Eclipse:} A solar eclipse occurs when the Moon passes between the Sun and Earth, and the Moon blocks the Sun's light from reaching Earth.  This can only happen during a new moon phase.
        \begin{itemize}
            \item \textbf{Total Solar Eclipse:} If the alignment is perfect, the Moon completely blocks the Sun, creating a total solar eclipse.  During totality, the sky becomes dark, and the Sun's faint outer atmosphere, the corona, becomes visible.
            \item \textbf{Partial Solar Eclipse:} If the alignment is not perfect, only part of the Sun is blocked by the Moon, resulting in a partial solar eclipse.
        \end{itemize}
    \item \textbf{Lunar Eclipse:} A lunar eclipse occurs when the Earth passes between the Sun and the Moon, and Earth's shadow falls on the Moon. This can only happen during a full moon phase.
        \begin{itemize}
            \item \textbf{Total Lunar Eclipse:} If the Moon passes entirely into Earth's umbra (the darkest part of the shadow), we have a total lunar eclipse. The Moon may appear reddish during a total lunar eclipse, sometimes called a "blood moon."
            \item \textbf{Partial Lunar Eclipse:} If only part of the Moon passes into Earth's umbra, we have a partial lunar eclipse.
        \end{itemize}
\end{itemize}

\begin{marginfigure}
\includegraphics[width=\marginparwidth]{placeholder-eclipses.pdf}
\caption{\label{fig:eclipses}Diagram illustrating solar and lunar eclipses. \textit{Image to be added.}}
\end{marginfigure}

\begin{keyconcept}{Conditions for Eclipses}
Eclipses do not happen every month because the Moon's orbit is tilted relative to Earth's orbit around the Sun.  Eclipses only occur when the Sun, Earth, and Moon are closely aligned in three dimensions.
\end{keyconcept}

\begin{tieredquestions}{Basic}
\begin{enumerate}
    \item What are lunar phases? What causes them?
    \item Name the main phases of the Moon in order, starting with the new moon.
    \item What is a solar eclipse? What is a lunar eclipse?
\end{enumerate}
\end{tieredquestions}

\begin{tieredquestions}{Intermediate}
\begin{enumerate}
    \item Explain why we see different phases of the Moon throughout the month.
    \item Describe the positions of the Sun, Earth, and Moon during a solar eclipse and a lunar eclipse.
    \item Why don't we have a solar eclipse and a lunar eclipse every month?
\end{enumerate}
\tieredquestions}

\begin{tieredquestions}{Advanced}
\begin{enumerate}
    \item Research and explain why total solar eclipses are relatively rare events at any given location on Earth.
    \item Describe the difference between the umbra and penumbra shadows during eclipses. How do these shadows relate to total and partial eclipses?
    \item  Predict what would happen to lunar phases and eclipses if the Moon's orbital plane was exactly aligned with Earth's orbital plane around the Sun.
\end{enumerate}
\end{tieredquestions}


\FloatBarrier
\1

In this chapter, we explored Earth's place in space, starting with our Solar System and expanding to understand the motions of Earth and the Moon. We learned about:

\begin{itemize}
    \item \textbf{The Solar System:}  Our cosmic neighbourhood, consisting of the Sun, planets, moons, asteroids, and comets.
    \item \textbf{Models of the Solar System:} The evolution from geocentric (Earth-centred) to heliocentric (Sun-centred) models, highlighting the dynamic nature of scientific understanding.
    \item \textbf{Earth's Rotation and Revolution:} How Earth's spin causes day and night, and its orbit around the Sun defines a year.
    \item \textbf{Seasons:}  The effect of Earth's axial tilt on creating seasons, and the distribution of sunlight across the hemispheres.
    \item \textbf{Lunar Phases:} The changing appearances of the Moon as it orbits Earth, and the cycle of lunar phases.
    \item \textbf{Eclipses:} Solar and lunar eclipses as shadow phenomena caused by the alignment of the Sun, Earth, and Moon.
\end{itemize}

Understanding these concepts provides a fundamental framework for appreciating Earth's place in the cosmos and the interconnectedness of celestial events.  Remember that science is a journey of continuous discovery, and our models of the universe are constantly being refined and improved as we learn more. Keep exploring and questioning – the universe is full of wonders waiting to be uncovered!
```
\FloatBarrier
