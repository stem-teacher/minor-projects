```latex
\chapter{Physical and Chemical Change}

\begin{marginfigure}
\includegraphics[width=\linewidth]{placeholder-physical-chemical-change.jpg}
\captionof{figure}{Observing changes around us helps us understand the world. Some changes are easily reversed, while others are not.}
\end{marginfigure}

\section*{Introduction: Change is All Around Us}

Take a moment to look around you. What do you see?  Perhaps you see a glass of water, a piece of paper, or a plant. Now think about how these things can change. Water can freeze into ice or boil into steam. Paper can be torn or burnt. Plants grow and change colour with the seasons.  \keyword{Change} is a fundamental part of our universe.

In science, we are very interested in understanding different types of changes. Some changes are simple and easily reversed, like melting ice. Others are more dramatic and create entirely new substances, like burning wood. This chapter will explore the fascinating world of change, focusing on two main categories: \keyword{physical changes} and \keyword{chemical changes}. We will learn how to tell them apart, what evidence to look for, and how these changes are important in our everyday lives and the world around us.

\begin{marginnote}
\historylink{Early scientists, like alchemists, were fascinated by change and tried to transform substances, often with limited understanding of physical and chemical processes.}
\end{marginnote}

Get ready to investigate, observe, and think critically about the changes that shape our world!

\FloatBarrier
\1

Imagine you have an ice cube. What happens when you leave it out of the freezer? It melts into liquid water. What happens if you then heat that water? It boils and turns into steam. These are examples of \keyword{physical changes}.

\begin{keyconcept}{Physical Change}
A \keyword{physical change} is a change in the form or appearance of a substance, but it \textbf{does not} create any new substances. The chemical composition of the substance remains the same. Physical changes are often easily reversed.
\end{keyconcept}

\subsection{Changes of State}

One of the most common types of physical change is a \keyword{change of state}. Matter can exist in three main states: solid, liquid, and gas.

\begin{itemize}
    \item \textbf{Solid:}  Has a fixed shape and volume. Ice is solid water.
    \item \textbf{Liquid:} Has a fixed volume but takes the shape of its container. Water is liquid water.
    \item \textbf{Gas:} Has no fixed shape or volume and fills its container. Steam is gaseous water (also called water vapour).
\end{itemize}

\begin{marginnote}
\keyword{Melting}, \keyword{freezing}, \keyword{boiling}, \keyword{condensation}, and \keyword{sublimation} are all terms for changes of state.
\end{marginnote}

Changes of state happen when we add or remove energy, usually in the form of heat.

\begin{itemize}
    \item \textbf{Melting:} Solid to liquid (e.g., ice melting to water).
    \item \textbf{Freezing:} Liquid to solid (e.g., water freezing to ice).
    \item \textbf{Boiling/Vaporisation:} Liquid to gas (e.g., water boiling to steam).
    \item \textbf{Condensation:} Gas to liquid (e.g., steam condensing to water on a cold window).
    \item \textbf{Sublimation:} Solid directly to gas (e.g., dry ice turning into carbon dioxide gas). \challenge{Can you think of other examples of sublimation in everyday life or nature?}
\end{itemize}

In all these changes, we are still dealing with water (\ce{H2O}). The water molecules are just arranged differently and have different amounts of energy.  We haven't created anything new.

\begin{figure}
\centering
\includegraphics[width=0.7\textwidth]{placeholder-states-of-matter.jpg}
\captionof{figure}{The three states of matter: solid, liquid, and gas. Changes between these states are physical changes.}
\end{figure}

\subsection{Other Examples of Physical Changes}

Besides changes of state, there are many other physical changes we see every day:

\begin{itemize}
    \item \textbf{Dissolving:} When a substance (\keyword{solute}) mixes evenly into another substance (\keyword{solvent}) to form a \keyword{solution}. For example, sugar dissolving in water. You still have sugar and water, just mixed together. You can often get the sugar back by evaporating the water – another physical change!
    \item \textbf{Mixing:}  Combining different substances together without them chemically reacting. For instance, mixing sand and gravel. You can still see and separate the sand and gravel.
    \item \textbf{Changes in Shape or Size:}  Cutting paper, crushing a can, bending a wire. You are changing the shape or size, but the substance itself remains the same.  The paper is still paper, the can is still aluminium, and the wire is still whatever metal it's made of.
    \item \textbf{Changes in Texture:}  Sanding wood to make it smoother, or crumpling fabric to change its texture. These are physical changes as the substance is still the same, only its surface appearance has changed.
\end{itemize}

\begin{stopandthink}
Think about making a fruit salad. Is this a physical or chemical change? Explain your reasoning.
\end{stopandthink}

\subsection{Reversibility of Physical Changes}

Many physical changes are \keyword{reversible}, meaning you can change the substance back to its original form.

\begin{example}
Melting ice is reversible. You can freeze liquid water back into ice by lowering the temperature. Dissolving sugar in water is also reversible. You can evaporate the water to get the sugar back.
\end{example}

However, some physical changes are not easily reversible in practice, even though they are still physical changes.

\begin{example}
Tearing paper is a physical change. You can't easily put the paper back together perfectly as it was before, but it is still paper, and no new substance has been formed.
\end{example}

\begin{tieredquestions}{Basic}
\begin{enumerate}
    \item Give three examples of physical changes.
    \item Explain in your own words what a physical change is.
    \item Is boiling water a physical change or a chemical change? Explain why.
\end{enumerate}
\end{tieredquestions}

\begin{tieredquestions}{Intermediate}
\begin{enumerate}
    \item Describe the changes of state involved in making a cup of tea, from heating the water to the steam rising from the cup. Identify each change of state.
    \item Explain why dissolving salt in water is considered a physical change, even though the salt seems to disappear.
    \item  Imagine you have a block of wood. List three different physical changes you could make to it.
\end{enumerate}
\end{tieredquestions}

\begin{tieredquestions}{Advanced}
\begin{enumerate}
    \item  Sublimation is a physical change where a solid turns directly into a gas. Research and explain why dry ice (solid carbon dioxide) is a good example of sublimation, and discuss some of its uses that rely on this property.
    \item  Consider the process of distillation, used to purify water or separate mixtures. Explain how distillation relies on physical changes, and identify the specific physical changes involved.
    \item  Are all changes of state reversible in theory? Are they always reversible in practice? Discuss with examples.
\end{enumerate}
\end{tieredquestions}


\FloatBarrier
\1

Now let's think about burning wood. When wood burns, it doesn't just change shape or state. It transforms into something completely different: ash, smoke, and gases. This is an example of a \keyword{chemical change}.

\begin{keyconcept}{Chemical Change}
A \keyword{chemical change}, also known as a \keyword{chemical reaction}, is a process where one or more substances (\keyword{reactants}) are transformed into new substances (\keyword{products}) with different chemical properties. Chemical changes are usually difficult to reverse.
\end{keyconcept}

\subsection{Evidence of Chemical Reactions}

How do we know if a chemical change has happened? There are several common signs or pieces of evidence that indicate a chemical reaction has taken place:

\begin{itemize}
    \item \textbf{Gas Production:} Bubbles forming when you mix substances (and it's not just boiling!). For example, mixing baking soda and vinegar produces carbon dioxide gas.
    \item \textbf{Temperature Change:}  The mixture getting hotter (\keyword{exothermic reaction}) or colder (\keyword{endothermic reaction}).  Burning wood is exothermic, releasing heat and light. Some instant ice packs become cold due to an endothermic reaction.
    \item \textbf{Colour Change:} A noticeable change in colour that isn't just mixing colours. For example, iron rusting changes from silvery grey to reddish-brown.
    \item \textbf{Precipitate Formation:} A solid (\keyword{precipitate}) forming when you mix two solutions.  This often makes the mixture cloudy or milky.
    \item \textbf{Light or Sound Production:}  Light being emitted (like in burning or fireworks) or sound being produced (like in an explosion) can indicate a chemical reaction.
    \item \textbf{New Smell:** A new odour being released that wasn't present before the substances were mixed. For example, the smell of cooking food is often due to chemical reactions.
\end{itemize}

It's important to note that not all of these signs will be present in every chemical reaction, but observing one or more of these can strongly suggest a chemical change has occurred.

\begin{figure}
\centering
\includegraphics[width=0.8\textwidth]{placeholder-chemical-reaction-evidence.jpg}
\captionof{figure}{Evidence of chemical reactions: gas production (bubbles), temperature change, colour change, and precipitate formation.}
\end{figure}

\subsection{Examples of Chemical Changes}

Let's look at some common examples of chemical changes:

\begin{itemize}
    \item \textbf{Combustion (Burning):}  Burning fuels like wood, gas, or petrol. This is a chemical reaction with oxygen that releases energy in the form of heat and light, and produces new substances like carbon dioxide and water.
    \begin{example}
    Burning methane gas (natural gas) in a cooker:
    \ce{CH4 + 2O2 -> CO2 + 2H2O}
    \end{example}
    \item \textbf{Rusting:} Iron reacting with oxygen and water in the air to form iron oxide (rust). Rust is a new substance with different properties to iron.
    \begin{example}
    Rusting of iron:
    \ce{4Fe + 3O2 + 2H2O -> 2Fe2O3.H2O}  (simplified equation for hydrated iron(III) oxide)
    \end{example}
    \item \textbf{Cooking and Baking:}  Heating food often causes chemical reactions that change its taste, texture, and appearance. For example, baking a cake involves reactions between flour, sugar, eggs, and baking powder.
    \item \textbf{Digestion:}  The process of breaking down food in your body involves many chemical reactions, using enzymes to break down large molecules into smaller ones that your body can absorb.
    \item \textbf{Photosynthesis:} Plants use sunlight, water, and carbon dioxide to produce glucose (sugar) and oxygen. This is a vital chemical reaction for life on Earth.
    \begin{example}
    Photosynthesis (simplified):
    \ce{6CO2 + 6H2O + light -> C6H12O6 + 6O2}
    \end{example}
\end{itemize}

\begin{stopandthink}
Think about cooking an egg. What evidence suggests that cooking an egg is a chemical change, not just a physical change?
\end{stopandthink}

\begin{investigation}{Investigating a Chemical Reaction: Baking Soda and Vinegar}
\textbf{Materials:}
\begin{itemize}
    \item Baking soda (sodium bicarbonate)
    \item Vinegar (acetic acid solution)
    \item Small plastic cup or beaker
    \item Spoon
\end{itemize}

\textbf{Procedure:}
\begin{enumerate}
    \item Place a spoonful of baking soda in the plastic cup.
    \item Slowly pour a small amount of vinegar into the cup with the baking soda.
    \item Observe what happens carefully. Note down your observations. Look for evidence of a chemical reaction.
\end{enumerate}

\textbf{Observations and Questions:}
\begin{enumerate}
    \item What did you observe when you mixed baking soda and vinegar?
    \item Was there any gas produced? How do you know?
    \item Did you notice any temperature change? (Carefully touch the bottom of the cup).
    \item Do you think a chemical change occurred? Explain your reasoning based on your observations and the evidence of chemical reactions you have learned about.
    \item What do you think the new substances formed in this reaction might be? \challenge{Research the reaction between baking soda and vinegar to find out the products!}
\end{enumerate}
\end{investigation}


\subsection{New Substances and New Properties}

The key feature of a chemical change is that \textbf{new substances are formed}. These new substances have different \keyword{chemical properties} compared to the original reactants. \keyword{Chemical properties} describe how a substance reacts with other substances.

\begin{example}
Iron is a strong, grey metal that can rust. Rust (iron oxide) is a reddish-brown, crumbly substance that is much weaker than iron.  The chemical properties have changed completely. Iron reacts with oxygen and water, rust does not easily react further in the same way.
\end{example}

Because new substances are formed, chemical changes are usually much harder to reverse than physical changes.  You can't easily unburn wood to get back the original wood.  You can't easily un-rust iron back into pure iron (though it is possible with chemical processes, it's not a simple reversal).

\begin{tieredquestions}{Basic}
\begin{enumerate}
    \item Give three examples of chemical changes.
    \item What is the main difference between a physical change and a chemical change?
    \item List three pieces of evidence that suggest a chemical reaction has taken place.
\end{enumerate}
\end{tieredquestions}

\begin{tieredquestions}{Intermediate}
\begin{enumerate}
    \item Describe the chemical change that happens when a candle burns. What are the reactants and some of the products?
    \item Explain why cooking food is generally considered a chemical change, not just a physical change.
    \item  Think about fireworks. What evidence do fireworks provide that chemical reactions are happening?
\end{enumerate}
\end{tieredquestions}

\begin{tieredquestions}{Advanced}
\begin{enumerate}
    \item  Explain the difference between exothermic and endothermic reactions. Give an example of each type of reaction from everyday life and describe the energy changes involved.
    \item  Rusting is a chemical reaction that is often considered undesirable. Discuss ways to prevent or slow down the rusting of iron or steel, based on your understanding of the rusting process.
    \item  Consider the chemical reaction in a car engine. What are the reactants and products? What evidence is there that a chemical reaction is occurring? How is this reaction useful?
\end{enumerate}
\end{tieredquestions}


\FloatBarrier
\1

Now that we have explored both physical and chemical changes, let's summarise the key differences and practice identifying them.

\begin{keyconcept}{Key Differences: Physical vs. Chemical Changes}
\begin{itemize}
    \item \textbf{New Substances:} Physical changes \textbf{do not} create new substances; chemical changes \textbf{do} create new substances.
    \item \textbf{Chemical Composition:} In physical changes, the chemical composition of the substance remains the \textbf{same}; in chemical changes, the chemical composition \textbf{changes}.
    \item \textbf{Reversibility:} Physical changes are often \textbf{reversible}; chemical changes are usually \textbf{difficult to reverse}.
    \item \textbf{Evidence:} Chemical changes often show evidence like gas production, temperature change, colour change, precipitate formation, etc., which are usually \textbf{not} seen in physical changes (except for temperature changes during changes of state).
\end{itemize}
\end{keyconcept}

Let's test your understanding. For each of the following examples, decide if it is a physical or chemical change and explain your reasoning:

\begin{enumerate}
    \item  Ice cream melting on a hot day.
    \item  A silver spoon tarnishing (turning black over time).
    \item  Cutting your hair.
    \item  Baking a cake.
    \item  Boiling an egg.
    \item  Dissolving sugar in tea.
    \item  Burning a log in a fireplace.
    \item  Sharpening a pencil.
    \item  Mixing oil and vinegar (they don't mix).
    \item  A car rusting.
\end{enumerate}

\begin{stopandthink}
Think about mixing paint colours. Is this a physical or chemical change? What about paint drying on a wall? Are these both the same type of change? Explain your answers.
\end{stopandthink}


\begin{tieredquestions}{Basic}
\begin{enumerate}
    \item  State whether each of the following is a physical or chemical change:
        \begin{itemize}
            \item  Freezing water
            \item  Burning paper
            \item  Chopping wood
            \item  Cooking pasta
            \item  Melting chocolate
        \end{itemize}
    \item  What is the best way to tell if a chemical change has occurred?
    \item  Can a physical change ever be irreversible? Give an example.
\end{enumerate}
\end{tieredquestions}

\begin{tieredquestions}{Intermediate}
\begin{enumerate}
    \item  For each example in the list above (ice cream melting, silver tarnishing, etc.), explain your reasoning for classifying it as a physical or chemical change, using the key differences you have learned.
    \item  Explain why the change of state from liquid water to steam is a physical change, but the reaction of hydrogen and oxygen to form water is a chemical change.
    \item  Design a simple experiment to demonstrate the difference between a physical change (like dissolving) and a chemical change (like baking soda and vinegar reaction).
\end{enumerate}
\end{tieredquestions}

\begin{tieredquestions}{Advanced}
\begin{enumerate}
    \item  Consider the process of making toast. Identify at least two physical changes and two chemical changes that occur when bread is toasted. Explain your reasoning for each.
    \item  Some changes appear to be physical but might have a subtle chemical component. For example, dissolving some substances can cause a slight temperature change.  Is dissolving always purely a physical change? Discuss with examples and consider the concept of solutions and intermolecular forces. \mathlink{This relates to concepts of enthalpy and thermodynamics in more advanced chemistry.}
    \item  "All chemical changes are irreversible, and all physical changes are reversible."  Discuss whether this statement is entirely true or if there are exceptions or nuances. Provide examples to support your argument.
\end{enumerate}
\end{tieredquestions}


\FloatBarrier
\1

We have learned that chemical changes involve \keyword{chemical reactions}, where reactants are transformed into products.  Let's look a little closer at what happens during a chemical reaction.

\subsection{Reactants and Products}

In a chemical reaction, the substances you start with are called \keyword{reactants}. These are the substances that are going to change. The substances that are formed as a result of the reaction are called \keyword{products}.

\begin{example}
In the reaction of burning methane (\ce{CH4}) with oxygen (\ce{O2}), methane and oxygen are the \textbf{reactants}. Carbon dioxide (\ce{CO2}) and water (\ce{H2O}) are the \textbf{products}.
\ce{CH4 + 2O2 -> CO2 + 2H2O}
\end{example}

We can represent chemical reactions using \keyword{word equations} and \keyword{chemical equations}.

\begin{itemize}
    \item \textbf{Word Equation:} Describes the reaction using the names of the reactants and products.
    \begin{example}
    Methane + Oxygen $\rightarrow$ Carbon Dioxide + Water
    \end{example}
    \item \textbf{Chemical Equation:} Uses chemical formulas to represent the reactants and products.
    \begin{example}
    \ce{CH4 + 2O2 -> CO2 + 2H2O}
    \end{example}
\end{itemize}

Chemical equations can be further classified as \keyword{balanced} or \keyword{unbalanced}. A balanced chemical equation shows that the number of atoms of each element is the same on both sides of the equation (reactants and products), reflecting the \keyword{law of conservation of mass}.

\subsection{Conservation of Mass}

One of the most fundamental principles in chemistry is the \keyword{law of conservation of mass}. This law states that in a chemical reaction, matter is neither created nor destroyed.  In simpler terms, the total mass of the reactants is equal to the total mass of the products in a closed system.

\begin{marginnote}
\historylink{Antoine Lavoisier, a French chemist in the 18th century, is often called the 'father of modern chemistry' partly for his work on the conservation of mass. He conducted careful experiments using closed containers to demonstrate this principle.}
\end{marginnote}

This means that during a chemical reaction, atoms are rearranged, but they are not lost or gained. They are simply combined in different ways to form new molecules.

\begin{figure}
\centering
\includegraphics[width=0.6\textwidth]{placeholder-conservation-of-mass.jpg}
\captionof{figure}{The Law of Conservation of Mass: The total mass of reactants equals the total mass of products in a chemical reaction.}
\end{figure}

\begin{stopandthink}
If you burn a log of wood, it seems like the mass disappears because you are left with only a small amount of ash. Does this contradict the law of conservation of mass? Explain what is happening to the mass.
\end{stopandthink}

To properly demonstrate conservation of mass, you need to consider all reactants and products, including any gases that might be produced or consumed in the reaction.  If you perform a reaction in a closed container, you will find that the total mass before and after the reaction remains the same (within the limits of measurement accuracy).

\begin{investigation}{Conservation of Mass: Reaction in a Closed System}
\textbf{Materials:}
\begin{itemize}
    \item Baking soda (sodium bicarbonate)
    \item Vinegar (acetic acid solution)
    \item Small plastic bag (ziplock bag)
    \item Small plastic cup
    \item Weighing scale
\end{itemize}

\textbf{Procedure:}
\begin{enumerate}
    \item Place a spoonful of baking soda in the small plastic cup.
    \item Carefully place the cup inside the ziplock bag.
    \item Pour a small amount of vinegar into the bag, but \textbf{do not} let it mix with the baking soda yet. Seal the ziplock bag tightly, ensuring as little air as possible is trapped inside.
    \item Weigh the sealed bag with all its contents and record the mass.
    \item Now, carefully tip the cup inside the bag so that the vinegar mixes with the baking soda. Observe the reaction.
    \item After the reaction has finished (no more bubbles), weigh the sealed bag again and record the mass.
    \item Compare the mass before and after the reaction.
\end{enumerate}

\textbf{Observations and Questions:}
\begin{enumerate}
    \item What happened when you mixed the baking soda and vinegar inside the closed bag?
    \item Did you observe any change in mass after the reaction?
    \item Does this experiment support the law of conservation of mass? Explain your answer.
    \item Why was it important to do this experiment in a closed bag? What would happen if you did the same experiment in an open cup?
\end{enumerate}
\end{investigation}

\begin{tieredquestions}{Basic}
\begin{enumerate}
    \item  What are reactants and products in a chemical reaction?
    \item  State the law of conservation of mass in your own words.
    \item  Why is it important to use a closed container to accurately demonstrate the conservation of mass in some chemical reactions?
\end{enumerate}
\end{tieredquestions}

\begin{tieredquestions}{Intermediate}
\begin{enumerate}
    \item  Write a word equation and a chemical equation for the rusting of iron (using iron, oxygen, and water as reactants and rust as the product – you can use the simplified formula for rust: iron(III) oxide).
    \item  Explain how the law of conservation of mass applies to the burning of a candle. Where does the mass go that seems to disappear?
    \item  If you react 10 grams of baking soda with vinegar in a closed container, and you find that the mass of the products and unreacted reactants afterwards is also 10 grams, does this prove the law of conservation of mass? Explain your reasoning.
\end{enumerate}
\end{tieredquestions}

\begin{tieredquestions}{Advanced}
\begin{enumerate}
    \item  Explain why balanced chemical equations are essential for demonstrating and applying the law of conservation of mass. What information does a balanced equation provide about the quantities of reactants and products? \mathlink{This is related to stoichiometry in chemistry.}
    \item  In some nuclear reactions, a very small amount of mass can be converted into a large amount of energy (as described by Einstein's equation \ce{E=mc^2}). Does this contradict the law of conservation of mass? Explain the relationship between mass and energy and the concept of conservation of mass-energy. \challenge{This is an extension into nuclear chemistry and physics.}
    \item  Design an experiment to investigate the conservation of mass in a precipitation reaction (where a solid precipitate forms from mixing two solutions). Describe the reactants, products, procedure, and how you would measure mass to demonstrate conservation of mass.
\end{enumerate}
\end{tieredquestions}


\FloatBarrier
\1

Chemical changes are not just something that happens in science labs. They are happening all around us, all the time!  They are essential for life and are involved in many everyday processes.

\subsection{Cooking and Food}

Cooking is full of chemical changes! When you cook food, you are causing chemical reactions that change its flavour, texture, colour, and nutritional value.

\begin{itemize}
    \item \textbf{Browning of meat:}  The Maillard reaction is a complex series of chemical reactions between amino acids and sugars in food, especially when heated. It's responsible for the delicious brown crust on cooked meat and baked goods.
    \item \textbf{Baking a cake:}  Baking powder or baking soda reacts with acids in the batter to produce carbon dioxide gas, which makes the cake rise. The heat also causes proteins in eggs and flour to denature and coagulate, setting the structure of the cake.
    \item \textbf{Ripening of fruit:} As fruit ripens, enzymes cause chemical reactions that break down complex carbohydrates into simpler sugars, making the fruit sweeter and softer. Colour changes also occur due to chemical changes in pigments.
    \item \textbf{Digestion:}  Your body uses chemical reactions to break down food into smaller molecules that can be absorbed and used for energy and building blocks. Enzymes in your saliva, stomach, and intestines catalyse these reactions.
\end{itemize}

\begin{figure}
\centering
\includegraphics[width=0.7\textwidth]{placeholder-cooking-chemical-changes.jpg}
\captionof{figure}{Cooking involves many chemical changes that transform food.}
\end{figure}

\subsection{Burning Fuels for Energy}

Combustion reactions, like burning fuels, are crucial for providing energy for our homes, transport, and industries.

\begin{itemize}
    \item \textbf{Burning wood, coal, and gas:}  These fuels contain carbon and hydrogen, which react with oxygen in the air to produce carbon dioxide, water, and release a lot of energy in the form of heat and light.
    \item \textbf{Petrol and diesel in cars:}  Engines burn petrol or diesel to power vehicles. These fuels are hydrocarbons that undergo combustion reactions to release energy that moves the car.
    \item \textbf{Power plants:} Many power plants burn fossil fuels (coal, oil, gas) to generate electricity. The heat from combustion boils water to produce steam, which turns turbines to generate electricity.
\end{itemize}

\begin{marginnote}
\challenge{Burning fossil fuels releases carbon dioxide, a greenhouse gas, which contributes to climate change.  Scientists and engineers are working on developing cleaner energy sources that rely on different types of chemical reactions or physical processes.}
\end{marginnote}

\subsection{Other Everyday Chemical Changes}

Chemical changes are involved in countless other aspects of our daily lives:

\begin{itemize}
    \item \textbf{Batteries:** Batteries use chemical reactions to produce electricity. Different types of batteries use different chemical reactions.
    \item \textbf{Cleaning products:** Many cleaning products rely on chemical reactions to remove dirt, stains, and germs. Bleach, for example, uses chemical reactions to break down coloured compounds and disinfect surfaces.
    \item \textbf{Photography:** Traditional photography relies on chemical reactions that are sensitive to light to capture images on film.
    \item \textbf{Glow sticks:** Glow sticks contain chemicals that react together when mixed, producing light. This is called chemiluminescence.
    \item \textbf{Fireworks:** Fireworks get their spectacular colours and effects from chemical reactions. Different metal salts are used to create different colours when they are heated and undergo combustion.
\end{itemize}

\begin{stopandthink}
Think about a bicycle rusting in the rain.  What type of change is this? Why is it a problem? Can you think of ways to prevent this chemical change from happening?
\end{stopandthink}


\begin{tieredquestions}{Basic}
\begin{enumerate}
    \item Give three examples of how chemical changes are important in cooking.
    \item Why is burning fuel considered a chemical change? What are the products of burning fuels like wood or gas?
    \item Name one everyday item that relies on chemical changes to function.
\end{enumerate}
\end{tieredquestions}

\begin{tieredquestions}{Intermediate}
\begin{enumerate}
    \item Explain the Maillard reaction and why it is important in cooking. Give examples of foods where the Maillard reaction is desirable.
    \item Discuss the advantages and disadvantages of using combustion reactions to generate energy. Consider both the energy produced and the environmental impact.
    \item  Describe how a battery works in terms of chemical reactions. What are the reactants and products in a typical battery? (You can research a specific type of battery like an alkaline battery).
\end{enumerate}
\end{tieredquestions}

\begin{tieredquestions}{Advanced}
\begin{enumerate}
    \item Research and explain the chemical reactions involved in how a glow stick produces light (chemiluminescence). What are the reactants and products, and what is the source of the light energy?
    \item  Compare and contrast the chemical reactions involved in burning wood and burning hydrogen gas as fuels. What are the products of each reaction? Which fuel is considered "cleaner" in terms of environmental impact and why? \challenge{Consider concepts like carbon footprint and sustainability.}
    \item  Investigate the chemical reactions involved in developing traditional photographic film (black and white photography). Explain the role of light, silver halides, and developing chemicals in creating a photographic image.
\end{enumerate}
\end{tieredquestions}


\section*{Chapter Summary}

In this chapter, we have explored the fascinating world of change, focusing on physical and chemical changes.

\begin{itemize}
    \item \textbf{Physical changes} alter the form or appearance of a substance but do not create new substances. Examples include changes of state (melting, boiling, freezing), dissolving, mixing, and changes in shape or size. Physical changes are often reversible.
    \item \textbf{Chemical changes} (chemical reactions) transform substances into new substances with different properties. Evidence of chemical changes includes gas production, temperature change, colour change, precipitate formation, and new smells. Chemical changes are usually difficult to reverse.
    \item \textbf{Key differences} between physical and chemical changes lie in whether new substances are formed, whether the chemical composition changes, and the reversibility of the change.
    \item \textbf{Chemical reactions} involve reactants transforming into products. They can be represented by word equations and chemical equations.
    \item \textbf{The law of conservation of mass} states that in a chemical reaction, mass is neither created nor destroyed. The total mass of reactants equals the total mass of products in a closed system.
    \item \textbf{Chemical changes are essential} in everyday life, including cooking, burning fuels for energy, batteries, cleaning products, and many other processes that shape our world.
\end{itemize}

Understanding the difference between physical and chemical changes is fundamental to understanding chemistry and the world around us. By observing changes, asking questions, and investigating, we can unlock the secrets of matter and its transformations.

\section*{Review Questions}

\begin{tieredquestions}{Basic}
\begin{enumerate}
    \item  Define physical change and give three examples.
    \item  Define chemical change and give three examples.
    \item  List three pieces of evidence that indicate a chemical reaction may have occurred.
    \item  State the law of conservation of mass.
    \item  Is melting ice a physical or chemical change? Explain.
    \item  Is burning wood a physical or chemical change? Explain.
\end{enumerate}
\end{tieredquestions}

\begin{tieredquestions}{Intermediate}
\begin{enumerate}
    \item  Explain the difference between reactants and products in a chemical reaction.
    \item  Describe an experiment you could do to demonstrate a chemical change. What would you observe?
    \item  For each of the following, classify it as a physical or chemical change and explain your reasoning:
        \begin{itemize}
            \item  Ironing a shirt
            \item  Milk going sour
            \item  Making ice cubes
            \item  Baking bread
            \item  Dissolving sugar in water
        \end{itemize}
    \item  Explain why cooking an egg is considered a chemical change, not just a physical change.
    \item  How does the law of conservation of mass apply to chemical reactions?
\end{enumerate}
\end{tieredquestions}

\begin{tieredquestions}{Advanced}
\begin{enumerate}
    \item  Design an investigation to compare and contrast a physical change (e.g., dissolving salt in water) and a chemical change (e.g., baking soda and vinegar reaction). What measurements would you take, and what would you compare?
    \item  Discuss the role of chemical changes in providing energy for society. Consider both the benefits and drawbacks of using combustion reactions for energy production.
    \item  Explain how the concept of conservation of mass is related to balanced chemical equations. How do balanced equations reflect the rearrangement of atoms during chemical reactions? \mathlink{Consider stoichiometry and mole ratios.}
    \item  Research and describe a real-world application of chemical changes in a technology or industry that interests you (e.g., battery technology, food preservation, pharmaceuticals, materials science). Explain the chemical principles involved. \challenge{Explore the ethical and environmental considerations related to your chosen application.}
    \item  Critically evaluate the statement: "Physical changes are always easily reversible, and chemical changes are always irreversible." Provide examples to support or refute this statement, and discuss the nuances of reversibility in both types of changes.
\end{enumerate}
\end{tieredquestions}
```
\FloatBarrier
