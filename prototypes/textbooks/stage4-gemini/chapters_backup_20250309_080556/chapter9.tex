```latex
\chapter{Earth's Resources and Geological Change}

\begin{marginfigure}
\includegraphics[width=\marginparwidth]{earth_resources_image.jpg}
\caption*{\textit{Image of diverse Earth resources: minerals, forests, water, and fossil fuels.}}
\end{marginfigure}

\FloatBarrier
\1

Welcome to the fascinating study of our planet, Earth!  Everything we use in our daily lives, from the water we drink to the devices we use, comes from the \keyword{Earth's resources}.  These resources are the materials found in nature that humans use to survive and thrive.  But the Earth is not a static, unchanging object. It is a dynamic system constantly undergoing \keyword{geological change}.  These changes, from slow processes like erosion to dramatic events like volcanic eruptions, shape the planet and influence the availability and distribution of its resources.

In this chapter, we will explore the different types of Earth's resources, how they are formed and used, and the powerful geological forces that constantly reshape our world.  Understanding these concepts is crucial for responsible citizenship and for appreciating the delicate balance of our planet.

\begin{keyconcept}{What are Earth's Resources?}
Earth's resources are materials found in nature that are valuable to humans. They can be broadly classified into renewable and non-renewable resources.
\end{keyconcept}

\FloatBarrier
\1

\subsection{Types of Earth's Resources}

Earth provides us with an incredible variety of resources. We can broadly categorise them into two main types: \keyword{renewable resources} and \keyword{non-renewable resources}.

\begin{marginnote}
\textit{Renewable} resources can be replenished over a relatively short time, while \textit{non-renewable} resources are finite and take millions of years to form.
\end{marginnote}

\textbf{Renewable Resources:} These are resources that can be replenished naturally over a relatively short period of time, often within a human lifespan. Examples include:

\begin{itemize}
    \item \textbf{Solar Energy}:  Energy from the sun, which is constantly being replenished.
    \item \textbf{Wind Energy}:  Energy harnessed from the wind, driven by solar energy and Earth's rotation.
    \item \textbf{Water}:  Although finite in total amount, the \keyword{water cycle} continuously recycles water through evaporation, condensation, and precipitation.
    \item \textbf{Forests (Timber and Biomass)}: Trees and other plants can regrow after harvesting, making them renewable if managed sustainably.
    \item \textbf{Geothermal Energy}: Heat from the Earth's interior, constantly generated by radioactive decay.
    \item \textbf{Air}: The atmosphere, constantly replenished by natural processes (though pollution can affect its quality).
\end{itemize}

\textbf{Non-Renewable Resources:} These resources exist in finite quantities and are formed over millions of years through geological processes. Once used, they cannot be replenished within a human timescale. Examples include:

\begin{itemize}
    \item \textbf{Fossil Fuels (Coal, Oil, Natural Gas)}: Formed from the remains of ancient organisms over millions of years.
    \item \textbf{Minerals and Metals}:  Naturally occurring solid substances with a specific chemical composition and crystal structure. Examples include iron ore, copper, gold, and aluminium.
    \item \textbf{Rocks}: Solid aggregates of minerals, such as granite, limestone, and sandstone, used for building materials and other purposes.
\end{itemize}

\begin{stopandthink}
Think about your home. List three renewable resources and three non-renewable resources that were used to build or furnish it.
\end{stopandthink}

\subsection{Mineral Resources: Building Blocks of Technology}

\begin{marginnote}
\historylink{Ancient civilisations} like the Egyptians and Romans were skilled miners, extracting minerals for tools, jewellery, and building materials.
\end{marginnote}

\keyword{Minerals} are naturally occurring, inorganic solids with a definite chemical composition and a crystalline structure. They are the fundamental building blocks of rocks and are essential for many aspects of modern life. From the smartphones in our pockets to the skyscrapers in our cities, minerals are crucial components.

\textbf{Formation of Mineral Deposits:} Most mineral deposits are formed through geological processes that concentrate minerals in specific locations. These processes include:

\begin{itemize}
    \item \textbf{Igneous Processes}: Magma (molten rock beneath the Earth's surface) cools and solidifies, allowing minerals to crystallise and separate.  Dense minerals like chromite and platinum can sink and accumulate at the bottom of magma chambers.
    \item \textbf{Hydrothermal Processes}: Hot, chemically active water solutions circulate through rocks, dissolving and transporting minerals. When these solutions cool or react with surrounding rocks, minerals precipitate out and form veins or disseminated deposits. Many valuable metal ores, such as gold, silver, and copper, are formed this way.
    \item \textbf{Sedimentary Processes}: Weathering and erosion break down rocks, and the resulting sediments are transported and deposited in layers.  Some minerals, like halite (salt) and gypsum, precipitate directly from evaporating seawater.  Placer deposits, where dense minerals like gold and tin are concentrated by flowing water, are also sedimentary in origin.
    \item \textbf{Metamorphic Processes}:  Existing rocks are transformed by heat, pressure, or chemically active fluids deep within the Earth.  Metamorphism can create new minerals or concentrate existing ones. For example, graphite and garnet are often formed during metamorphism.
\end{itemize}

\textbf{Examples of Important Mineral Resources:}

\begin{itemize}
    \item \textbf{Iron Ore}:  Essential for making steel, the backbone of modern construction and manufacturing.  Major iron minerals include haematite (\ce{Fe2O3}) and magnetite (\ce{Fe3O4}).
    \item \textbf{Copper Ore}:  A highly conductive metal used extensively in electrical wiring, plumbing, and electronics.  Chalcopyrite (\ce{CuFeS2}) and malachite (\ce{Cu2CO3(OH)2}) are common copper ores.
    \item \textbf{Aluminium Ore (Bauxite)}:  A lightweight and strong metal used in aircraft, vehicles, and packaging. Bauxite is a mixture of aluminium hydroxides and oxides.
    \item \textbf{Gold}:  A precious metal valued for its beauty, rarity, and resistance to corrosion. Used in jewellery, electronics, and as a store of value. Gold often occurs in hydrothermal veins or placer deposits.
    \item \textbf{Rare Earth Elements}:  A group of 17 metallic elements crucial for modern technologies like smartphones, wind turbines, and electric vehicles.  They are not necessarily "rare" in abundance but are often dispersed and difficult to extract economically.
\end{itemize}

\begin{investigation}{Mineral Identification}
\textbf{Materials:} Mineral samples (e.g., quartz, feldspar, mica, pyrite, halite), hand lens, streak plate, hardness scale (Mohs scale if available), magnet, dilute hydrochloric acid (optional, for calcite testing - teacher supervision required).

\textbf{Procedure:}
\begin{enumerate}
    \item Observe each mineral sample carefully. Note its colour, lustre (how it reflects light), and any visible crystal shapes.
    \item Test the streak of each mineral by rubbing it on the streak plate. Record the colour of the streak.
    \item Test the hardness of each mineral by trying to scratch it with your fingernail (hardness ~2.5), a copper coin (hardness ~3), and a steel nail (hardness ~5.5). If available, use a Mohs hardness scale kit.
    \item Test for magnetism using a magnet.
    \item (Optional, teacher supervision required) Test for reaction with dilute hydrochloric acid. Place a drop of acid on the mineral and observe if it fizzes (indicates the presence of carbonates like calcite).
\end{enumerate}

\textbf{Analysis:}
\begin{enumerate}
    \item Create a table to record your observations for each mineral (colour, lustre, streak, hardness, magnetism, acid reaction).
    \item Use mineral identification guides (or online resources) to try and identify each mineral based on your observations.
    \item Discuss the properties of each mineral and how these properties are related to its uses.
\end{enumerate}
\end{investigation}

\begin{tieredquestions}{Mineral Resources}
\textbf{Basic:}
\begin{enumerate}
    \item What is a mineral?
    \item Name three examples of mineral resources.
    \item Describe one way mineral deposits can form.
\end{enumerate}
\textbf{Intermediate:}
\begin{enumerate}
    \item Explain the difference between igneous, hydrothermal, and sedimentary processes in mineral formation.
    \item Why are rare earth elements considered important even though they are not always "rare"?
    \item Describe the properties of copper that make it useful in electrical wiring.
\end{enumerate}
\textbf{Advanced:}
\begin{enumerate}
    \item Research and explain the environmental impacts of mining for mineral resources.
    \item Discuss the concept of mineral reserves and resources and how they are estimated.
    \item  How does the geological history of a region influence the types of mineral resources that are likely to be found there?
\end{enumerate}
\end{tieredquestions}

\subsection{Fossil Fuels: Energy from the Past}

\begin{marginnote}
\challenge{Peak Oil} is a concept that suggests that global oil production will eventually reach a peak and then decline, raising concerns about future energy supplies.
\end{marginnote}

\keyword{Fossil fuels} – coal, oil (petroleum), and natural gas – are formed from the remains of ancient organisms (plants and animals) that lived millions of years ago.  These organic remains accumulated in sedimentary basins and were buried under layers of sediment. Over time, heat and pressure transformed them into carbon-rich fuels. Fossil fuels are a major source of energy for electricity generation, transportation, and industry worldwide. However, they are non-renewable and their combustion releases greenhouse gases, contributing to climate change.

\textbf{Formation of Fossil Fuels:}

\begin{itemize}
    \item \textbf{Coal}: Formed from plant matter that accumulated in swampy environments.  Over millions of years, peat (partially decayed plant matter) is compressed and heated, gradually transforming into different grades of coal – lignite, bituminous coal, and anthracite – with increasing carbon content and energy density.
    \item \textbf{Oil and Natural Gas}:  Formed from microscopic marine organisms (plankton and algae) that accumulated on the ocean floor.  Buried under sediment, these organic remains are transformed by heat and pressure into liquid oil and gaseous natural gas.  These fluids migrate upwards through porous rocks until they are trapped by impermeable layers, forming oil and gas reservoirs.
\end{itemize}

\textbf{Extraction and Use of Fossil Fuels:}

\begin{itemize}
    \item \textbf{Coal Mining}: Coal is extracted from underground mines or surface mines (open-pit mining).  Mining can have significant environmental impacts, including habitat destruction, water pollution, and land subsidence.
    \item \textbf{Oil and Gas Drilling}:  Oil and natural gas are extracted by drilling wells into underground reservoirs.  Offshore drilling is also common.  Extraction can involve primary recovery (natural pressure), secondary recovery (water or gas injection), and enhanced oil recovery (EOR) techniques to maximise production.
    \item \textbf{Combustion and Energy Production}: Fossil fuels are burned to release energy.  Coal and natural gas are primarily used to generate electricity in power plants.  Oil is refined into petrol, diesel, and other fuels used in vehicles and other engines.  Burning fossil fuels releases carbon dioxide (\ce{CO2}), a major greenhouse gas, as well as other pollutants.
\end{itemize}

\begin{stopandthink}
Consider the energy you used today.  How much of it likely came from fossil fuels?  Think about electricity, transportation, and manufactured goods.
\end{stopandthink}

\begin{investigation}{Comparing Fuels}
\textbf{Materials:} Samples of different fuels (e.g., wood chips, coal, candle wax, vegetable oil – ensure safety and adult supervision for burning),  small metal containers, matches or lighter, water, thermometer, measuring cylinder, safety goggles.

\textbf{Procedure:}
\begin{enumerate}
    \item  Measure a known volume of water (e.g., 100 ml) and pour it into a metal container. Record the initial temperature of the water.
    \item  Carefully burn a small, measured amount of one fuel sample under the container of water, ensuring proper ventilation and safety precautions.  Continue burning until the fuel is consumed or for a set time (e.g., 5 minutes).
    \item  Record the final temperature of the water.
    \item  Repeat steps 1-3 for each fuel sample, using the same amount of water and fuel (as much as practically possible).
\end{enumerate}

\textbf{Analysis:}
\begin{enumerate}
    \item Calculate the temperature change of the water for each fuel. This is a rough measure of the energy released by burning the fuel.
    \item Compare the temperature changes for different fuels. Which fuel produced the greatest temperature change?
    \item Discuss the advantages and disadvantages of each fuel in terms of energy content, renewability, and environmental impact.
\end{enumerate}
\textbf{Safety Note:} Burning fuels involves fire and potential hazards. This investigation should be conducted with adult supervision and appropriate safety precautions, including wearing safety goggles and ensuring good ventilation.
\end{investigation}


\begin{tieredquestions}{Fossil Fuels}
\textbf{Basic:}
\begin{enumerate}
    \item What are fossil fuels?
    \item Name the three main types of fossil fuels.
    \item Explain why fossil fuels are considered non-renewable.
\end{enumerate}
\textbf{Intermediate:}
\begin{enumerate}
    \item Describe the process of coal formation.
    \item How do oil and natural gas reservoirs form?
    \item What are some environmental consequences of burning fossil fuels?
\end{enumerate}
\textbf{Advanced:}
\begin{enumerate}
    \item Research and discuss the concept of carbon capture and storage (CCS) as a way to mitigate the climate impact of fossil fuel use.
    \item Compare and contrast the environmental impacts of coal mining versus oil drilling.
    \item  Discuss the role of fossil fuels in the global energy mix and the challenges of transitioning to renewable energy sources.
\end{enumerate}
\end{tieredquestions}

\subsection{Water Resources: The Elixir of Life}

\begin{marginnote}
\mathlink{Water Density} is crucial for aquatic life.  Water is most dense at 4°C, causing ice to float and insulating water bodies in winter.
\end{marginnote}

\keyword{Water} is arguably the most essential resource for life on Earth.  It is vital for drinking, agriculture, industry, and ecosystems.  While Earth is often called the "blue planet," only a small percentage of its water is freshwater, and even less is readily accessible.

\textbf{The Water Cycle:} Water is a renewable resource because of the \keyword{water cycle}, also known as the hydrologic cycle. This is a continuous process of water movement on, above, and below the surface of the Earth.  Key processes in the water cycle include:

\begin{itemize}
    \item \textbf{Evaporation}:  The process of liquid water changing into water vapour (gas) and rising into the atmosphere, primarily from oceans, lakes, and rivers, driven by solar energy.
    \item \textbf{Transpiration}:  The release of water vapour from plants into the atmosphere.
    \item \textbf{Condensation}:  The process of water vapour changing back into liquid water, forming clouds as it cools in the atmosphere.
    \item \textbf{Precipitation}:  Water falling back to Earth in the form of rain, snow, hail, or sleet.
    \item \textbf{Infiltration}:  The process of water soaking into the ground and becoming groundwater.
    \item \textbf{Runoff}:  Water flowing over the land surface, eventually reaching rivers, lakes, and oceans.
\end{itemize}

\textbf{Types of Water Resources:}

\begin{itemize}
    \item \textbf{Surface Water}:  Water found on the Earth's surface in rivers, lakes, reservoirs, and wetlands.  Surface water is easily accessible but can be vulnerable to pollution and seasonal variations in supply.
    \item \textbf{Groundwater}:  Water stored underground in aquifers, which are permeable rock or sediment layers that can hold and transmit water. Groundwater is a major source of freshwater, often cleaner and more reliable than surface water, but it can be depleted if extraction rates exceed recharge rates.
    \item \textbf{Atmospheric Water}: Water in the atmosphere as vapour, clouds, and precipitation. While not directly usable as a resource in its atmospheric form, it is the source of all freshwater precipitation.
    \item \textbf{Seawater}:  The vast majority of Earth's water is saltwater in oceans and seas.  Desalination (removing salt from seawater) can provide freshwater, but it is energy-intensive and costly.
\end{itemize}

\textbf{Water Scarcity and Management:}  Many regions around the world face \keyword{water scarcity}, meaning they lack sufficient water resources to meet their needs.  Factors contributing to water scarcity include:

\begin{itemize}
    \item \textbf{Climate Change}:  Altering precipitation patterns, increasing droughts, and melting glaciers and snowpack, which are important water sources.
    \item \textbf{Population Growth}: Increasing demand for water for domestic, agricultural, and industrial uses.
    \item \textbf{Pollution}: Contaminating water sources, reducing the availability of clean, usable water.
    \item \textbf{Inefficient Water Use}:  Wasting water in agriculture, industry, and households.
\end{itemize}

Sustainable water management is crucial for ensuring water security for future generations.  This involves:

\begin{itemize}
    \item \textbf{Water Conservation}: Reducing water use through efficient irrigation, water-saving technologies, and responsible household practices.
    \item \textbf{Water Recycling and Reuse}: Treating and reusing wastewater for irrigation or industrial purposes.
    \item \textbf{Protecting Water Sources}: Preventing pollution of surface and groundwater.
    \item \textbf{Efficient Water Distribution Systems}: Reducing leaks and losses in water supply networks.
\end{itemize}


\begin{stopandthink}
Think about how you use water each day.  Identify at least three ways you could reduce your water consumption.
\end{stopandthink}

\begin{investigation}{Water Filtration}
\textbf{Materials:}  Dirty water sample (e.g., muddy water, water with leaves and debris),  empty plastic bottle (cut in half),  cloth or coffee filter,  sand,  gravel,  activated charcoal (optional),  beaker or jar to collect filtered water.

\textbf{Procedure:}
\begin{enumerate}
    \item  Layer the filtration materials in the top half of the plastic bottle (inverted funnel shape), starting with cloth/filter at the bottom, then sand, then gravel, and optionally a layer of activated charcoal on top.
    \item  Slowly pour the dirty water sample into the top of the filter, allowing it to pass through the layers.
    \item  Collect the filtered water in the beaker or jar.
    \item  Observe the appearance of the filtered water compared to the original dirty water.
\end{enumerate}

\textbf{Analysis:}
\begin{enumerate}
    \item  Describe the changes in the water's appearance after filtration.  Is it clearer?  Are there fewer visible particles?
    \item  Explain how each layer of the filter (cloth, sand, gravel, charcoal) contributes to the filtration process.
    \item  Discuss whether this filtration method makes the water safe to drink.  What further steps might be needed to purify water for drinking?
\end{enumerate}
\textbf{Safety Note:}  The filtered water from this activity is not necessarily safe to drink.  This experiment demonstrates basic filtration principles, but proper water purification for drinking requires more advanced methods to remove bacteria, viruses, and dissolved contaminants.
\end{investigation}


\begin{tieredquestions}{Water Resources}
\textbf{Basic:}
\begin{enumerate}
    \item What is the water cycle?
    \item Name three types of water resources.
    \item Why is water considered a renewable resource?
\end{enumerate}
\textbf{Intermediate:}
\begin{enumerate}
    \item Describe the processes of evaporation, condensation, and precipitation in the water cycle.
    \item Explain the difference between surface water and groundwater.
    \item What are some factors that contribute to water scarcity?
\end{enumerate}
\textbf{Advanced:}
\begin{enumerate}
    \item Research and discuss the challenges of managing water resources in a specific region facing water scarcity (e.g., the Murray-Darling Basin in Australia).
    \item  Explain the concept of a water footprint and how it can be used to assess water use.
    \item  Discuss the potential of desalination as a solution to water scarcity and its environmental and economic implications.
\end{enumerate}
\end{tieredquestions}


\FloatBarrier
\1

\subsection{The Earth's Dynamic Crust: Plate Tectonics}

\begin{marginnote}
\historylink{Alfred Wegener} proposed the theory of continental drift in the early 20th century, but it was initially met with scepticism until evidence for plate tectonics emerged.
\end{marginnote}

The Earth's outer layer, the \keyword{crust}, is not a single solid shell but is broken into several large and small pieces called \keyword{tectonic plates}. These plates are constantly moving, albeit very slowly (centimetres per year), on the semi-molten layer beneath called the \keyword{mantle}. This movement and interaction of plates is the driving force behind \keyword{plate tectonics}, the theory that explains many of Earth's major geological features and events.

\textbf{Plate Boundaries and Interactions:}  Most geological activity, such as earthquakes, volcanoes, and mountain building, occurs at plate boundaries where plates interact. There are three main types of plate boundaries:

\begin{itemize}
    \item \textbf{Convergent Boundaries (Colliding Plates)}: Plates move towards each other.
        \begin{itemize}
            \item \textit{Oceanic-Continental Convergence}: Denser oceanic plate subducts (sinks) beneath the less dense continental plate. This creates ocean trenches, volcanic arcs on the continental side (e.g., Andes Mountains), and earthquakes.
            \item \textit{Oceanic-Oceanic Convergence}: One oceanic plate subducts beneath another. This forms ocean trenches and volcanic island arcs (e.g., Japan, Philippines).
            \item \textit{Continental-Continental Convergence}:  When two continental plates collide, neither subducts easily due to their similar density.  Instead, they crumple and fold, creating massive mountain ranges (e.g., Himalayas).
        \end{itemize}
    \item \textbf{Divergent Boundaries (Spreading Plates)}: Plates move apart from each other.  Magma from the mantle rises to fill the gap, creating new crust.  This occurs at mid-ocean ridges (e.g., Mid-Atlantic Ridge), where seafloor spreading takes place, and continental rift valleys (e.g., East African Rift Valley). Divergent boundaries are associated with volcanism and earthquakes.
    \item \textbf{Transform Boundaries (Sliding Plates)}: Plates slide past each other horizontally.  Crust is neither created nor destroyed. Transform boundaries are characterised by frequent earthquakes, often shallow and powerful (e.g., San Andreas Fault in California).
\end{itemize}

\textbf{Evidence for Plate Tectonics:} The theory of plate tectonics is supported by a wealth of evidence, including:

\begin{itemize}
    \item \textbf{Fit of the Continents}: The shapes of continents, particularly South America and Africa, suggest they were once joined together like puzzle pieces.
    \item \textbf{Fossil Evidence}:  Similar fossils of plants and animals are found on continents now separated by vast oceans, indicating they were once connected.
    \item \textbf{Geological Evidence}:  Matching rock formations and mountain ranges are found on different continents, suggesting they were once part of the same geological structures.
    \item \textbf{Paleomagnetism}:  Rocks preserve a record of Earth's magnetic field at the time they formed.  Studies of paleomagnetism show that the continents have moved over time and that new crust is being created at mid-ocean ridges.
    \item \textbf{Seafloor Spreading}:  The age of the seafloor rocks increases with distance from mid-ocean ridges, indicating that new crust is being generated at the ridges and spreading outwards.
    \item \textbf{Earthquake and Volcano Distribution}:  Earthquakes and volcanoes are concentrated along plate boundaries, confirming that these are zones of intense geological activity.
\end{itemize}

\begin{stopandthink}
Look at a world map showing tectonic plates.  Identify the type of plate boundary that exists near your location (if applicable).  What geological features or events might be associated with this boundary?
\end{stopandthink}

\begin{investigation}{Modelling Plate Boundaries}
\textbf{Materials:}  Playdough or modelling clay (different colours),  knife or string,  large sheet of paper or tray.

\textbf{Procedure:}
\begin{enumerate}
    \item \textbf{Convergent Boundary (Subduction):}  Use two colours of playdough to represent oceanic and continental plates.  Flatten each colour into a rectangular shape.  Place the "oceanic plate" (denser colour) next to the "continental plate". Gently push the oceanic plate under the continental plate to model subduction. Observe what happens to the plates and the surface.
    \item \textbf{Convergent Boundary (Collision):} Use two colours of playdough to represent two continental plates. Flatten each colour into a rectangular shape.  Push the two plates towards each other. Observe what happens when they collide.
    \item \textbf{Divergent Boundary:}  Use one colour of playdough. Flatten it into a rectangular shape.  Cut it in half with a knife or string.  Slowly pull the two halves apart.  Observe what happens in the gap. You can add a different colour of playdough to represent magma rising.
    \item \textbf{Transform Boundary:} Use two colours of playdough. Flatten each colour into a rectangular shape. Place them side by side.  Slide one plate horizontally past the other.  Observe what happens at the boundary.
\end{enumerate}

\textbf{Analysis:}
\begin{enumerate}
    \item  Describe what you observed at each type of plate boundary in your model.
    \item  How does your model represent the geological features and events associated with each type of plate boundary (e.g., volcanoes, mountains, earthquakes)?
    \item  What are the limitations of using playdough to model plate tectonics?
\end{enumerate}
\end{investigation}


\begin{tieredquestions}{Plate Tectonics}
\textbf{Basic:}
\begin{enumerate}
    \item What is plate tectonics?
    \item Name the three main types of plate boundaries.
    \item What geological features are associated with convergent boundaries?
\end{enumerate}
\textbf{Intermediate:}
\begin{enumerate}
    \item Explain the process of subduction at convergent boundaries.
    \item How are mid-ocean ridges formed at divergent boundaries?
    \item Describe the movement of plates at transform boundaries and the type of geological activity associated with them.
\end{enumerate}
\textbf{Advanced:}
\begin{enumerate}
    \item Discuss the evidence that supports the theory of plate tectonics.
    \item  Explain the driving forces behind plate movement (e.g., convection currents in the mantle, ridge push, slab pull).
    \item  How does plate tectonics influence the distribution of Earth's resources, such as mineral deposits and fossil fuels?
\end{enumerate}
\end{tieredquestions}

\subsection{Weathering and Erosion: Sculpting the Landscape}

\begin{marginnote}
\challenge{Differential Weathering} occurs when different rock types weather at different rates, leading to the formation of unique landforms.
\end{marginnote}

\keyword{Weathering} and \keyword{erosion} are continuous geological processes that break down rocks and transport rock fragments, shaping the Earth's surface over time. Weathering breaks down rocks in place, while erosion involves the removal and transport of weathered material.

\textbf{Weathering: Breaking Down Rocks in Place:}

\begin{itemize}
    \item \textbf{Mechanical Weathering (Physical Weathering)}:  The physical disintegration of rocks into smaller pieces without changing their chemical composition. Processes include:
        \begin{itemize}
            \item \textit{Frost Wedging}: Water seeps into cracks in rocks, freezes, and expands, widening the cracks and eventually breaking the rock apart.
            \item \textit{Thermal Expansion and Contraction}:  Repeated heating and cooling of rocks cause them to expand and contract, leading to stress and cracking, especially in deserts with large temperature variations.
            \item \textit{Abrasion}:  Rocks are worn down by friction and impact as they collide with each other, often caused by wind, water, or ice carrying rock particles.
            \item \textit{Biological Activity}:  Plant roots growing into cracks can widen them, and burrowing animals can break down rocks.
        \end{itemize}
    \item \textbf{Chemical Weathering}:  The chemical alteration and decomposition of rocks, changing their mineral composition. Processes include:
        \begin{itemize}
            \item \textit{Solution (Dissolution)}:  Some minerals, like halite (salt) and calcite (in limestone), dissolve in water, especially acidic water.
            \item \textit{Hydrolysis}:  Minerals react with water, causing them to break down and form new minerals. Feldspar weathering to clay minerals is a common example.
            \item \textit{Oxidation}:  Minerals react with oxygen, often resulting in rust (iron oxide) formation.
            \item \textit{Carbonation}:  Carbon dioxide in the atmosphere dissolves in rainwater to form carbonic acid, which can dissolve carbonate rocks like limestone, creating caves and karst landscapes.
        \end{itemize}
\end{itemize}

\textbf{Erosion: Transporting Weathered Material:}

\begin{itemize}
    \item \textbf{Water Erosion}:  The most significant agent of erosion globally.  Running water in rivers and streams carries away weathered material.  Rain splash erosion also occurs when raindrops directly impact soil and rock surfaces.
    \item \textbf{Wind Erosion}:  Effective in dry and sparsely vegetated regions. Wind can pick up and transport fine particles like sand and dust over long distances.
    \item \textbf{Ice Erosion (Glacial Erosion)}:  Glaciers are powerful agents of erosion.  As they move, they grind and scrape the underlying rock, carving out valleys and transporting vast amounts of sediment.
    \item \textbf{Gravity Erosion (Mass Wasting)}:  Downslope movement of rock and soil due to gravity.  Includes landslides, rockfalls, and soil creep.
\end{itemize}

\textbf{Factors Influencing Weathering and Erosion:}

\begin{itemize}
    \item \textbf{Climate}: Temperature and precipitation are major factors.  Warm and humid climates promote chemical weathering, while cold climates favour frost wedging.  Arid climates often experience wind erosion.
    \item \textbf{Rock Type}:  Different rock types have varying resistance to weathering.  For example, granite is more resistant to weathering than limestone.
    \item \textbf{Topography (Slope)}:  Steeper slopes are more prone to erosion due to gravity and faster runoff.
    \item \textbf{Vegetation}: Plant cover protects soil from erosion and reduces runoff.  Deforestation increases erosion rates.
    \item \textbf{Time}:  Weathering and erosion are slow processes that operate over long geological timescales.
\end{itemize}

\begin{stopandthink}
Observe the buildings and landscapes around you.  Identify examples of weathering and erosion that you can see.  What evidence suggests these processes are occurring?
\end{stopandthink}

\begin{investigation}{Investigating Erosion}
\textbf{Materials:}  Two trays or shallow containers, soil, grass seeds (or small plants), watering can, ruler or measuring tape.

\textbf{Procedure:}
\begin{enumerate}
    \item  Fill both trays with soil to the same depth.
    \item  In one tray, plant grass seeds or small plants to create vegetation cover. Leave the other tray bare soil.
    \item  Water both trays equally and regularly, allowing the grass to grow in the vegetated tray.
    \item  Once the grass has grown (or after a set period, e.g., one week), tilt both trays at a slight angle.
    \item  Gently pour a measured amount of water (e.g., 500 ml) from the watering can onto the top end of each tray, simulating rainfall.
    \item  Observe and compare the amount of soil eroded from each tray.  Measure the depth of soil removed or estimate the amount of sediment washed out.
\end{enumerate}

\textbf{Analysis:}
\begin{enumerate}
    \item  Compare the amount of erosion in the vegetated tray versus the bare soil tray.  Which tray experienced more erosion?
    \item  Explain how vegetation cover reduces soil erosion.
    \item  Discuss the importance of vegetation in preventing soil erosion in natural landscapes and agricultural settings.
\end{enumerate}
\end{investigation}


\begin{tieredquestions}{Weathering and Erosion}
\textbf{Basic:}
\begin{enumerate}
    \item What is weathering?
    \item What is erosion?
    \item Name two types of mechanical weathering.
\end{enumerate}
\textbf{Intermediate:}
\begin{enumerate}
    \item Explain the difference between mechanical and chemical weathering.
    \item Describe three agents of erosion.
    \item How does climate influence weathering and erosion rates?
\end{enumerate}
\textbf{Advanced:}
\begin{enumerate}
    \item Discuss the role of weathering and erosion in the formation of soils.
    \item  Explain how human activities can accelerate erosion rates.
    \item  Research and discuss the impact of weathering and erosion on infrastructure and buildings.
\end{enumerate}
\end{tieredquestions}

\subsection{The Rock Cycle: A Continuous Transformation}

\begin{marginnote}
\historylink{James Hutton}, considered the "father of geology," developed the concept of the rock cycle and uniformitarianism – the idea that geological processes operating today are the same as those that operated in the past.
\end{marginnote}

The \keyword{rock cycle} is a fundamental concept in geology that describes the continuous processes of transformation between the three main types of rocks: \keyword{igneous rocks}, \keyword{sedimentary rocks}, and \keyword{metamorphic rocks}.  These rock types are not static; they are constantly being changed from one type to another through various geological processes.

\textbf{The Three Main Rock Types and their Formation:}

\begin{itemize}
    \item \textbf{Igneous Rocks}: Formed from the cooling and solidification of magma or lava.
        \begin{itemize}
            \item \textit{Intrusive (Plutonic) Igneous Rocks}: Formed from magma that cools slowly beneath the Earth's surface. They have large crystals (coarse-grained texture), e.g., granite, gabbro.
            \item \textit{Extrusive (Volcanic) Igneous Rocks}: Formed from lava that cools quickly on the Earth's surface. They have small crystals or no crystals (fine-grained or glassy texture), e.g., basalt, obsidian.
        \end{itemize}
    \item \textbf{Sedimentary Rocks}: Formed from the accumulation and cementation of sediments (fragments of rocks, minerals, and organic matter).
        \begin{itemize}
            \item \textit{Clastic Sedimentary Rocks}: Formed from fragments of other rocks cemented together, e.g., sandstone, shale, conglomerate.
            \item \textit{Chemical Sedimentary Rocks}: Formed from minerals that precipitate out of solution (e.g., seawater or lake water), e.g., limestone (biochemical and chemical), halite (evaporite).
            \item \textit{Organic Sedimentary Rocks}: Formed from the accumulation and compression of organic matter (plant or animal remains), e.g., coal, some types of limestone.
        \end{itemize}
    \item \textbf{Metamorphic Rocks}: Formed from existing igneous, sedimentary, or other metamorphic rocks that are changed by heat, pressure, or chemically active fluids. Metamorphism occurs deep within the Earth.
        \begin{itemize}
            \item \textit{Foliated Metamorphic Rocks}:  Minerals are aligned in parallel layers or bands due to directed pressure, e.g., slate (from shale), schist (from shale or igneous rocks), gneiss (from granite or sedimentary rocks).
            \item \textit{Non-foliated Metamorphic Rocks}:  Minerals are not aligned, often formed by heat or uniform pressure, e.g., marble (from limestone), quartzite (from sandstone).
        \end{itemize}
\end{itemize}

\textbf{Processes Driving the Rock Cycle:}

\begin{itemize}
    \item \textbf{Melting}: Rocks melt to form magma or lava.
    \item \textbf{Cooling and Solidification (Crystallisation)}: Magma or lava cools and solidifies to form igneous rocks.
    \item \textbf{Weathering and Erosion}: Igneous, sedimentary, and metamorphic rocks are broken down into sediments.
    \item \textbf{Transportation and Deposition}: Sediments are transported by water, wind, or ice and deposited in layers.
    \item \textbf{Compaction and Cementation (Lithification)}: Sediments are compacted and cemented together to form sedimentary rocks.
    \item \textbf{Metamorphism}: Existing rocks are transformed by heat, pressure, or chemically active fluids to form metamorphic rocks.
\    item \textbf{Uplift and Exposure}:  Tectonic forces can uplift rocks, bringing them to the surface where they are exposed to weathering and erosion, continuing the cycle.
\end{itemize}

\textbf{The Rock Cycle and Earth's Resources:}  The rock cycle is directly linked to the formation and distribution of many Earth's resources.

\begin{itemize}
    \item \textbf{Mineral Deposits}:  Many mineral deposits are associated with igneous and metamorphic processes (e.g., hydrothermal veins, magmatic segregation). Sedimentary processes also concentrate certain minerals (e.g., placer deposits, evaporites).
    \item \textbf{Fossil Fuels}:  Formed from organic matter within sedimentary rocks (coal, oil shale).
    \item \textbf{Building Materials}:  Various rocks are used as building materials (e.g., granite, limestone, sandstone, marble, slate).
    \item \textbf{Soil Formation}:  Weathering of rocks is the primary source of mineral components of soil.
\end{itemize}

\begin{stopandthink}
Think about a common rock like granite or sandstone.  Trace its journey through the rock cycle.  Where did it come from?  What could it become in the future?
\end{stopandthink}

\begin{investigation}{Rock Cycle in Action (Demonstration)}
\textbf{Materials:}  Modelling clay (different colours to represent rock types),  heat source (hot plate or hairdryer – adult supervision required),  ice cubes,  sand,  gravel,  water,  container for water.

\textbf{Procedure (Demonstration by Teacher):}
\begin{enumerate}
    \item \textbf{Igneous Rock Formation:}  Melt a small amount of one colour of clay using a heat source (hot plate or hairdryer – adult supervision).  Allow it to cool and solidify to represent igneous rock formation from magma/lava.
    \item \textbf{Sedimentary Rock Formation:}  Take small pieces of different coloured clays (representing rock fragments). Mix them with sand and gravel. Add a little water to represent cementation and press them together to form a "sedimentary rock."
    \item \textbf{Metamorphic Rock Formation:}  Take a piece of “sedimentary rock” or “igneous rock” made from clay. Apply pressure and heat (using your hands or carefully placing it near a warm surface – adult supervision) to simulate metamorphism. Observe how the clay changes shape and texture.
    \item \textbf{Weathering and Erosion:}  Place a “rock” made of clay in a container. Pour water over it and use an ice cube to rub against it, simulating weathering and erosion. Observe the breakdown of the clay.
\end{enumerate}

\textbf{Analysis (Discussion):}
\begin{enumerate}
    \item  Describe how each step of the demonstration represents a process in the rock cycle.
    \item  How did the clay "rocks" change during each stage of the demonstration?
    \item  Discuss how the rock cycle is a continuous process of transformation and how it links the different rock types.
\end{enumerate}
\textbf{Safety Note:}  Use heat sources with caution and adult supervision. Avoid direct contact with hot surfaces.
\end{investigation}


\begin{tieredquestions}{The Rock Cycle}
\textbf{Basic:}
\begin{enumerate}
    \item What is the rock cycle?
    \item Name the three main types of rocks.
    \
\FloatBarrier
