```latex
\chapter{Forces and Motion}

\begin{marginfigure}
\includegraphics[width=0.9\linewidth]{placeholder_force_image.jpg}
\caption*{\textit{Think about all the forces acting around you, even when you are still.}}
\end{marginfigure}

\FloatBarrier
\1

Have you ever wondered what makes things move? Or what stops them from moving?  From kicking a football to a spaceship blasting off, \keyword{forces} are at play.  Forces are fundamental to how our world works.  Imagine trying to play basketball without being able to push the ball, or trying to ride a bike if you couldn't push on the pedals!  Forces are everywhere, constantly influencing our lives and the world around us.

\begin{keyconcept}{What is a Force?}
A \keyword{force} is a push or a pull that can cause an object to start moving, stop moving, change direction, or change shape. Forces are measured in \keyword{Newtons} (N).
\marginnote{\textit{Definition of Force}}
\end{keyconcept}

Forces are vector quantities, meaning they have both magnitude (strength) and direction. We often represent forces using arrows, where the length of the arrow indicates the magnitude of the force and the direction of the arrow shows the direction of the force.

\begin{stopandthink}
Think about opening a door.  Are you applying a push or a pull force? What direction is the force in?
\end{stopandthink}

Forces can be broadly categorised into two main types: \keyword{contact forces} and \keyword{non-contact forces}.

\subsection{Contact Forces}

\begin{marginnote}
\challenge{Can you think of situations where contact forces are helpful? And when they might be a hindrance?}
\end{marginnote}
\keyword{Contact forces} are forces that act between objects when they are touching.  There is direct physical contact between the objects involved.  Let's explore some common types of contact forces:

\begin{itemize}
    \item \textbf{Friction:}  \keyword{Friction} is a force that opposes motion when two surfaces rub against each other.  It acts in the opposite direction to the motion or attempted motion.  For example, when you push a book across a table, friction between the book and the table surface resists the movement.  Friction is what allows us to walk without slipping and what slows down a bicycle when you stop pedalling.

    \item \textbf{Tension:} \keyword{Tension} is the force transmitted through a string, rope, cable, or wire when it is pulled tight by forces acting from opposite ends.  Imagine pulling on a rope in a tug-of-war – the tension force is felt throughout the rope.

    \item \textbf{Normal Force:}  The \keyword{normal force} is a support force exerted upon an object that is in contact with another stable object.  It acts perpendicular to the surface of contact.  If you place a book on a table, the table exerts an upward normal force on the book, preventing it from falling through. The normal force is often equal and opposite to the force pressing the object against the surface (like gravity in this example), but it's fundamentally a reaction to contact.

    \item \textbf{Applied Force:}  An \keyword{applied force} is simply a force that is applied to an object by a person or another object.  This could be you pushing a trolley, a motor pulling a lift upwards, or the wind pushing on a sail.

    \item \textbf{Air Resistance (Drag):}  \keyword{Air resistance}, also known as drag, is a type of friction that opposes the motion of objects moving through the air.  The faster an object moves through the air, the greater the air resistance.  Air resistance is why parachutes slow down a falling skydiver and why cars are designed to be streamlined to reduce drag and improve fuel efficiency.  Air resistance is a specific type of fluid friction.

    \item \textbf{Buoyancy:} \keyword{Buoyancy} is an upward force exerted by a fluid (liquid or gas) that opposes the weight of an immersed object.  This is why objects float or seem lighter in water. A boat floats because the buoyant force of the water is equal to the weight of the boat.
\end{itemize}

\begin{tieredquestions}{Basic}
\begin{enumerate}
    \item What is a force?
    \item Give two examples of contact forces.
\end{enumerate}
\end{tieredquestions}

\begin{tieredquestions}{Intermediate}
\begin{enumerate}
    \item Explain the difference between friction and tension.
    \item Describe a situation where multiple contact forces are acting on an object simultaneously.
\end{enumerate}
\end{tieredquestions}

\begin{tieredquestions}{Advanced}
\begin{enumerate}
    \item  Imagine you are pushing a heavy box across a rough floor. Identify all the contact forces acting on the box and describe the direction of each force.
    \item How does air resistance affect the motion of a car?  Explain how car designers try to minimise air resistance.
\end{enumerate}
\end{tieredquestions}


\subsection{Non-Contact Forces}

\begin{marginnote}
\historylink{The concept of ‘action at a distance’, as seen in non-contact forces, was a puzzle for early scientists.  Isaac Newton himself was uneasy with the idea of gravity acting without physical contact.}
\end{marginnote}
\keyword{Non-contact forces}, also known as field forces, are forces that act between objects even when they are not touching.  These forces act over a distance through fields.  Think of it like invisible hands pushing or pulling without physically being there.  Let's look at some key non-contact forces:

\begin{itemize}
    \item \textbf{Gravity:} \keyword{Gravity} is the force of attraction between any two objects with mass.  The more massive the objects, the stronger the gravitational force.  The Earth exerts a gravitational force on everything near it, pulling objects towards its centre.  This is what we experience as weight.  The Sun's gravity keeps the planets in orbit, and the Moon's gravity causes tides in our oceans. Gravity is a universal force, acting between all objects in the universe.

    \item \textbf{Magnetism:} \keyword{Magnetism} is a force associated with moving electric charges.  Magnets have north and south poles, and like poles repel each other, while opposite poles attract.  Magnetic forces can act through space and can attract or repel certain materials like iron, nickel, and cobalt.  Magnets are used in many technologies, from fridge magnets to electric motors and generators.

    \item \textbf{Electrostatic Force:}  \keyword{Electrostatic force} is the force between electrically charged objects.  Like charges repel each other, and opposite charges attract.  This force is much stronger than gravity at the atomic level and is responsible for holding atoms and molecules together.  You experience electrostatic forces when you rub a balloon on your hair and it sticks to a wall, or when you feel a static shock in dry weather.
\end{itemize}

\begin{stopandthink}
Can you think of everyday examples where you see non-contact forces in action?
\end{stopandthink}

\begin{tieredquestions}{Basic}
\begin{enumerate}
    \item What is a non-contact force?
    \item Name two types of non-contact forces.
\end{enumerate}
\end{tieredquestions}

\begin{tieredquestions}{Intermediate}
\begin{enumerate}
    \item Explain how gravity acts as a non-contact force.
    \item How is magnetism different from gravity?
\end{enumerate}
\end{tieredquestions}

\begin{tieredquestions}{Advanced}
\begin{enumerate}
    \item  Describe the similarities and differences between gravitational force and electrostatic force.
    \item  Imagine you have two magnets. Describe how the magnetic force between them changes as you move them closer together and further apart.
\end{enumerate}
\end{tieredquestions}


\FloatBarrier
\1

Let's delve deeper into some of the most common and important types of forces we encounter in our daily lives and in science.

\subsection{Gravity: The Force that Pulls Us Down}

\begin{marginnote}
\mathlink{The strength of gravity depends on mass and distance.  The formula for gravitational force is  $F = G \frac{m_1 m_2}{r^2}$, where $G$ is the gravitational constant, $m_1$ and $m_2$ are the masses of the two objects, and $r$ is the distance between their centres.}
\end{marginnote}
As we touched upon earlier, \keyword{gravity} is a universal force of attraction between objects with mass.  Every object that has mass exerts a gravitational pull on every other object with mass.  You exert a gravitational force on your textbook, and your textbook exerts a gravitational force on you!  However, for everyday objects, these forces are incredibly weak and unnoticeable unless one of the objects is very massive, like the Earth.

The Earth's gravity is what keeps us grounded, makes objects fall to the ground when we drop them, and keeps the Moon in orbit around the Earth.  The strength of the gravitational force depends on two main factors:

\begin{itemize}
    \item \textbf{Mass:} The more massive an object, the stronger its gravitational pull.  A planet like Jupiter, being much more massive than Earth, has a much stronger gravitational field.
    \item \textbf{Distance:}  The greater the distance between two objects, the weaker the gravitational force between them.  This is why the gravitational pull of the Earth is much weaker on the Moon than it is on the surface of the Earth.
\end{itemize}

\subsubsection{Weight vs. Mass}

It's important to distinguish between \keyword{weight} and \keyword{mass}.  While often used interchangeably in everyday language, they are distinct scientific concepts.

\begin{itemize}
    \item \textbf{Mass} is a measure of the amount of matter in an object.  It is a fundamental property of an object and remains constant regardless of location. Mass is measured in kilograms (kg).
    \item \textbf{Weight} is the force of gravity acting on an object's mass.  It is a force, measured in Newtons (N), and it depends on the gravitational field strength at a particular location.
\end{itemize}

Your mass stays the same whether you are on Earth, on the Moon, or in space. However, your weight will change depending on the gravitational field strength.  On the Moon, where gravity is about 1/6th of Earth's gravity, you would weigh only about 1/6th of your weight on Earth, even though your mass remains the same.

\begin{example}
Imagine a person with a mass of 60 kg. On Earth, the acceleration due to gravity is approximately 9.8 m/s\textsuperscript{2}.  Therefore, their weight on Earth is approximately:

Weight = mass $\times$ acceleration due to gravity = 60 kg $\times$ 9.8 m/s\textsuperscript{2} = 588 N.

On the Moon, the acceleration due to gravity is approximately 1.6 m/s\textsuperscript{2}.  Their weight on the Moon would be:

Weight = mass $\times$ acceleration due to gravity = 60 kg $\times$ 1.6 m/s\textsuperscript{2} = 96 N.

Notice how the weight is significantly less on the Moon, but the mass remains 60 kg in both locations.
\end{example}

\begin{stopandthink}
If you were to travel to a planet with twice the gravity of Earth, how would your weight and mass change compared to being on Earth?
\end{stopandthink}

\begin{tieredquestions}{Basic}
\begin{enumerate}
    \item What is gravity?
    \item What is the difference between mass and weight?
\end{enumerate}
\end{tieredquestions}

\begin{tieredquestions}{Intermediate}
\begin{enumerate}
    \item Explain how mass and distance affect the force of gravity between two objects.
    \item If an object weighs 100 N on Earth, approximately how much would it weigh on the Moon? Explain your reasoning.
\end{enumerate}
\end{tieredquestions}

\begin{tieredquestions}{Advanced}
\begin{enumerate}
    \item  Explain why astronauts in the International Space Station appear weightless, even though gravity is still acting on them.
    \item  Research and explain how the concept of gravity has changed over time, from Newton's law of universal gravitation to Einstein's theory of general relativity.
\end{enumerate}
\end{tieredquestions}


\subsection{Friction: The Force that Resists Motion}

\begin{marginnote}
\challenge{Explore different types of friction, such as static friction, kinetic friction, and rolling friction. How do they differ?}
\end{marginnote}
\keyword{Friction} is a contact force that opposes motion between surfaces in contact. It's a ubiquitous force, present in almost all everyday interactions involving movement.  Friction arises because surfaces are not perfectly smooth at a microscopic level.  Even surfaces that appear smooth to the naked eye have microscopic bumps and irregularities.  When two surfaces try to slide past each other, these irregularities interlock and resist the motion.

\subsubsection{Types of Friction}

We can broadly classify friction into a few types:

\begin{itemize}
    \item \textbf{Static Friction:} \keyword{Static friction} is the force that opposes the start of motion. It prevents an object from moving when a force is applied to it.  Imagine trying to push a heavy box that is initially at rest.  You need to apply a certain amount of force to overcome static friction before the box starts to move.  Static friction can vary in magnitude up to a maximum value, which depends on the surfaces in contact and the normal force pressing them together.

    \item \textbf{Kinetic Friction (Sliding Friction):} \keyword{Kinetic friction}, also called sliding friction, is the force that opposes the motion of an object that is already sliding over a surface.  Once the box from the previous example starts moving, kinetic friction acts to slow it down.  Kinetic friction is generally less than the maximum static friction for the same surfaces.

    \item \textbf{Rolling Friction:} \keyword{Rolling friction} is the force that opposes the motion of a rolling object over a surface.  It's generally much less than sliding friction.  This is why it's much easier to roll a heavy object (e.g., using wheels) than to slide it.  Rolling friction occurs due to deformation of both the rolling object and the surface it's rolling on.

    \item \textbf{Fluid Friction:} \keyword{Fluid friction} is the force that opposes the motion of an object through a fluid (a liquid or a gas).  Air resistance and water resistance are examples of fluid friction. The magnitude of fluid friction depends on factors like the speed of the object, the viscosity of the fluid, and the shape of the object.
\end{itemize}

\subsubsection{Factors Affecting Friction}

The magnitude of friction depends on several factors:

\begin{itemize}
    \item \textbf{Nature of Surfaces:}  Rougher surfaces produce more friction than smoother surfaces.  For example, there's more friction between sandpaper and wood than between glass and ice.
    \item \textbf{Normal Force:} The greater the normal force pressing the surfaces together, the greater the friction.  If you push down harder on a book while trying to slide it across a table, it becomes harder to move because the normal force and hence the friction force increases.
\end{itemize}

Friction is independent of the area of contact (within reasonable limits and for solid friction).  For example, the friction between a brick and a table is roughly the same whether you place the brick on its widest face or its narrowest face, as long as the normal force remains the same.

\begin{investigation}{Measuring Frictional Force}
\textbf{Aim:} To investigate the frictional force between different surfaces.

\textbf{Materials:}
\begin{itemize}
    \item Wooden block
    \item Different surfaces (e.g., sandpaper, smooth wood, cloth)
    \item Spring balance or force meter
    \item String
\end{itemize}

\textbf{Procedure:}
\begin{enumerate}
    \item Attach the string to the wooden block.
    \item Place the wooden block on one of the surfaces (e.g., smooth wood).
    \item Attach the other end of the string to the spring balance.
    \item Gently pull the spring balance horizontally, gradually increasing the force until the wooden block just starts to move.  Record the reading on the spring balance just as the block starts to move. This measures the maximum static friction.
    \item Continue pulling the spring balance at a constant speed to keep the block moving steadily across the surface.  Record the reading on the spring balance while the block is moving at a constant speed. This measures the kinetic friction.
    \item Repeat steps 2-5 for different surfaces (sandpaper, cloth, etc.).
\end{enumerate}

\textbf{Observations and Results:}
Record your observations in a table, noting the type of surface and the measured static and kinetic friction forces.

\textbf{Analysis and Conclusion:}
Compare the frictional forces for different surfaces.  Which surface produced the most friction? Which produced the least?  What can you conclude about the relationship between the type of surface and the frictional force?  Is kinetic friction generally less than static friction, as expected?

\end{investigation}

\begin{stopandthink}
Think about situations where friction is helpful and situations where friction is a hindrance.  How do we increase or decrease friction in different situations?
\end{stopandthink}

\begin{tieredquestions}{Basic}
\begin{enumerate}
    \item What is friction?
    \item Give two examples where friction is helpful and two examples where it is a hindrance.
\end{enumerate}
\end{tieredquestions}

\begin{tieredquestions}{Intermediate}
\begin{enumerate}
    \item Explain the difference between static friction and kinetic friction.
    \item Describe two factors that affect the magnitude of frictional force.
\end{enumerate}
\end{tieredquestions}

\begin{tieredquestions}{Advanced}
\begin{enumerate}
    \item  Explain why rolling friction is generally much less than sliding friction.
    \item  Discuss how engineers design machines and vehicles to minimise or maximise friction depending on the application. Give specific examples.
\end{enumerate}
\end{tieredquestions}


\subsection{Magnetism: Forces from Magnets}

\begin{marginnote}
\historylink{The ancient Greeks were aware of lodestones, natural magnets that could attract iron.  The study of magnetism has been crucial for developing technologies like compasses, electric motors, and data storage.}
\end{marginnote}
\keyword{Magnetism} is a non-contact force associated with magnets and moving electric charges.  Magnets exert magnetic forces on each other and on certain materials, like iron, nickel, and cobalt.

\subsubsection{Magnetic Poles and Fields}

Every magnet has two poles: a \keyword{north pole} and a \keyword{south pole}.  These poles are regions where the magnetic force is strongest.  Like poles (north-north or south-south) \textbf{repel} each other, while opposite poles (north-south) \textbf{attract} each other.  This is a fundamental rule of magnetism, often summarised as "like poles repel, unlike poles attract."

The region around a magnet where its magnetic force can be detected is called a \keyword{magnetic field}.  We can visualise magnetic fields using magnetic field lines.  These lines show the direction of the magnetic force that a north magnetic pole would experience if placed in the field.  Magnetic field lines emerge from the north pole of a magnet and enter the south pole, forming closed loops.  The closer the field lines are together, the stronger the magnetic field is in that region.

\begin{figure}
\centering
\includegraphics[width=0.6\linewidth]{placeholder_magnetic_field.jpg}
\caption*{Magnetic field lines around a bar magnet.  Notice how the lines are concentrated at the poles.}
\end{figure}

\subsubsection{Types of Magnets}

Magnets can be classified into different types:

\begin{itemize}
    \item \textbf{Permanent Magnets:}  \keyword{Permanent magnets} are materials that retain their magnetic properties for a long time.  They are made of ferromagnetic materials that have been magnetised.  Examples include bar magnets, horseshoe magnets, and fridge magnets.

    \item \textbf{Temporary Magnets:} \keyword{Temporary magnets} are materials that become magnetised when they are placed in a strong magnetic field, but they lose their magnetism when the field is removed.  Soft iron is a good example of a temporary magnet.

    \item \textbf{Electromagnets:}  \keyword{Electromagnets} are magnets created by passing an electric current through a coil of wire (solenoid).  The magnetic field of an electromagnet can be turned on and off by switching the electric current on and off.  The strength of an electromagnet can be increased by increasing the current, increasing the number of turns in the coil, or by inserting a ferromagnetic core (like iron) inside the coil.  Electromagnets are used in many applications, such as electric motors, generators, and magnetic levitation trains (maglev trains).
\end{itemize}

\begin{stopandthink}
How are magnets used in everyday devices around your home or school?
\end{stopandthink}

\begin{tieredquestions}{Basic}
\begin{enumerate}
    \item What is magnetism?
    \item What are the two poles of a magnet called?
\end{enumerate}
\end{tieredquestions}

\begin{tieredquestions}{Intermediate}
\begin{enumerate}
    \item Explain the rule of attraction and repulsion between magnetic poles.
    \item Describe the difference between permanent magnets and temporary magnets.
\end{enumerate}
\end{tieredquestions}

\begin{tieredquestions}{Advanced}
\begin{enumerate}
    \item  Explain how electromagnets work and describe two applications of electromagnets.
    \item  Research and explain the relationship between electricity and magnetism. How are they interconnected?
\end{enumerate}
\end{tieredquestions}


\FloatBarrier
\1

\begin{marginnote}
\challenge{Think about a car moving at a constant speed on a straight road. Are the forces acting on it balanced or unbalanced? What about a car speeding up?}
\end{marginnote}
Forces don't always result in motion. Sometimes, multiple forces act on an object simultaneously, and their combined effect determines whether the object's motion changes or stays the same.  We can classify forces acting on an object as either \keyword{balanced forces} or \keyword{unbalanced forces}.

\subsection{Balanced Forces}

\begin{keyconcept}{Balanced Forces}
\keyword{Balanced forces} are forces that are equal in magnitude and opposite in direction. When balanced forces act on an object, they cancel each other out, and there is no net force acting on the object.
\marginnote{\textit{Definition of Balanced Forces}}
\end{keyconcept}

When forces are balanced, the object's state of motion remains unchanged.  This means:

\begin{itemize}
    \item If the object is at rest, it will remain at rest.
    \item If the object is moving at a constant velocity (constant speed in a straight line), it will continue to move at that constant velocity.
\end{itemize}

Imagine a book resting on a table.  The force of gravity is pulling the book downwards, but the table exerts an equal and opposite normal force upwards.  These two forces are balanced, resulting in no net force on the book.  Therefore, the book remains at rest.

Another example is a hot air balloon floating at a constant height. The upward buoyant force is balanced by the downward force of gravity (weight of the balloon and its contents).  Because the forces are balanced, the balloon remains at a constant altitude (assuming no wind or other external forces).

\begin{example}
Consider a tug-of-war game where two teams are pulling on a rope with equal force in opposite directions. If the forces are perfectly balanced, the rope will not move.  The net force on the rope is zero.
\end{example}

\begin{stopandthink}
Think of other examples where balanced forces are acting on an object. What is the state of motion of the object in each case?
\end{stopandthink}

\begin{tieredquestions}{Basic}
\begin{enumerate}
    \item What are balanced forces?
    \item What happens to an object when balanced forces act on it?
\end{enumerate}
\end{tieredquestions}

\begin{tieredquestions}{Intermediate}
\begin{enumerate}
    \item Explain why a book resting on a table is an example of balanced forces.
    \item Describe a scenario where an object is moving at a constant velocity and the forces acting on it are balanced.
\end{enumerate}
\end{tieredquestions}

\begin{tieredquestions}{Advanced}
\begin{enumerate}
    \item  Can balanced forces cause an object to deform or change shape? Explain your answer.
    \item  In a tug-of-war, if one team starts to slowly move the rope in their direction, what does this tell you about the forces acting on the rope?
\end{enumerate}
\end{tieredquestions}


\subsection{Unbalanced Forces}

\begin{keyconcept}{Unbalanced Forces}
\keyword{Unbalanced forces} are forces that are not equal and opposite. When unbalanced forces act on an object, there is a net force acting on the object.
\marginnote{\textit{Definition of Unbalanced Forces}}
\end{keyconcept}

When unbalanced forces act on an object, the object's state of motion changes.  This means the object will:

\begin{itemize}
    \item Start moving if it was initially at rest.
    \item Speed up (accelerate) if it was already moving.
    \item Slow down (decelerate) if it was already moving.
    \item Change direction if it was already moving.
\end{itemize}

The \keyword{net force} is the overall force acting on an object when all individual forces are combined.  If the net force is zero, the forces are balanced. If the net force is not zero, the forces are unbalanced, and the object will accelerate in the direction of the net force.

Imagine pushing a stationary box across a floor.  Initially, static friction balances your pushing force.  But if you push harder and overcome static friction, your pushing force becomes greater than the frictional force.  Now, there is a net force in the direction of your push.  This unbalanced force causes the box to accelerate – it starts moving and speeds up.

Consider a car speeding up.  The engine provides a forward force (thrust) that is greater than the opposing forces of friction and air resistance.  This results in a net forward force, causing the car to accelerate.

\begin{example}
Imagine kicking a football that is initially at rest.  Your foot applies an unbalanced force to the ball.  This unbalanced force causes the ball to accelerate from rest and move forward.
\end{example}

\begin{stopandthink}
Think of examples where unbalanced forces cause changes in motion. What changes in motion do you observe in each case?
\end{stopandthink}

\begin{tieredquestions}{Basic}
\begin{enumerate}
    \item What are unbalanced forces?
    \item What happens to an object when unbalanced forces act on it?
\end{enumerate}
\end{tieredquestions}

\begin{tieredquestions}{Intermediate}
\begin{enumerate}
    \item Explain how unbalanced forces cause acceleration.
    \item Describe a scenario where unbalanced forces cause an object to change direction.
\end{enumerate}
\end{tieredquestions}

\begin{tieredquestions}{Advanced}
\begin{enumerate}
    \item  Explain the concept of net force and how it determines whether forces are balanced or unbalanced.
    \item  Imagine a skydiver falling from a plane.  Initially, they accelerate downwards.  Explain how air resistance eventually leads to balanced forces and constant velocity (terminal velocity).
\end{enumerate}
\end{tieredquestions}


\subsection{Newton's First Law of Motion: Inertia}

\begin{marginnote}
\historylink{Newton's First Law is also known as the Law of Inertia.  Galileo Galilei's work on inertia paved the way for Newton's formulation of this fundamental law of motion.}
\end{marginnote}
The concept of balanced and unbalanced forces leads us directly to one of the most fundamental principles in physics: \keyword{Newton's First Law of Motion}, also known as the \keyword{Law of Inertia}.  This law describes what happens to an object's motion when there is no net force acting on it (i.e., when forces are balanced).

\begin{keyconcept}{Newton's First Law of Motion (Law of Inertia)}
An object at rest stays at rest and an object in motion stays in motion with the same speed and in the same direction unless acted upon by an unbalanced force.
\marginnote{\textit{Newton's First Law}}
\end{keyconcept}

In simpler terms, Newton's First Law states that objects tend to resist changes in their state of motion. This resistance to change in motion is called \keyword{inertia}.

\begin{itemize}
    \item \textbf{Inertia of Rest:} An object at rest tends to stay at rest.  You need an unbalanced force to start it moving.  For example, a football will remain stationary on the pitch until a player kicks it (applies an unbalanced force).

    \item \textbf{Inertia of Motion:} An object in motion tends to stay in motion with the same velocity.  It will keep moving at the same speed and in the same direction unless an unbalanced force acts to change its motion.  For example, a hockey puck sliding on ice will continue to slide in a straight line at a constant speed for a long time because friction (the unbalanced force slowing it down) is very small.
\end{itemize}

\begin{investigation}{Observing Inertia}
\textbf{Aim:} To observe inertia of rest and inertia of motion.

\textbf{Materials:}
\begin{itemize}
    \item A glass or beaker
    \item A stiff card (e.g., playing card or index card)
    \item A coin
    \item A toy car or ball
    \item A smooth, flat surface (e.g., table)
\end{itemize}

\textbf{Procedure (Part 1: Inertia of Rest):}
\begin{enumerate}
    \item Place the card on top of the glass, centred over the opening.
    \item Place the coin on top of the card, directly above the centre of the glass opening.
    \item With a quick flick of your finger, sharply strike the card horizontally from the side, trying to remove the card quickly without disturbing the coin too much.
    \item Observe what happens to the coin.
\end{enumerate}

\textbf{Procedure (Part 2: Inertia of Motion):}
\begin{enumerate}
    \item Place the toy car or ball on a smooth, flat surface.
    \item Give the car or ball a gentle push to set it in motion.
    \item Observe the motion of the car or ball. What happens to its speed and direction over time?
    \item Now, try stopping the moving car or ball with your hand. What do you have to do to stop it?
\end{enumerate}

\textbf{Observations and Results:}
Record your observations for both parts of the investigation. What happened to the coin when the card was flicked? What happened to the motion of the toy car or ball?

\textbf{Analysis and Conclusion:}
Explain your observations in terms of inertia.  How does the coin demonstrate inertia of rest? How does the motion of the toy car or ball demonstrate inertia of motion?  What unbalanced forces are involved in changing the motion in each part of the experiment?

\end{investigation}

\begin{example}
Have you ever experienced inertia when riding in a car?  When the car suddenly brakes, your body tends to continue moving forward due to inertia of motion.  This is why we wear seatbelts – to provide an unbalanced force to stop our forward motion and prevent injury.  Similarly, when a car suddenly accelerates from rest, your body tends to stay at rest, and you feel pushed back against the seat due to inertia of rest.
\end{example}

\begin{stopandthink}
Explain how inertia is related to Newton's First Law of Motion.
\end{stopandthink}

\begin{tieredquestions}{Basic}
\begin{enumerate}
    \item State Newton's First Law of Motion.
    \item What is inertia?
\end{enumerate}
\end{tieredquestions}

\begin{tieredquestions}{Intermediate}
\begin{enumerate}
    \item Explain how inertia of rest and inertia of motion are demonstrated in everyday life. Give examples.
    \item How does Newton's First Law relate to balanced and unbalanced forces?
\end{enumerate}
\end{tieredquestions}

\begin{tieredquestions}{Advanced}
\begin{enumerate}
    \item  Discuss why it is more accurate to say that inertia is the tendency of an object to resist changes in its \textit{velocity} rather than just its speed.
    \item  Imagine you are in space, far away from any planets or stars.  If you push a spacecraft, according to Newton's First Law, what would happen to its motion? Explain your reasoning.
\end{enumerate}
\end{tieredquestions}


\FloatBarrier
\1

Forces are not just abstract concepts in textbooks; they are constantly at work all around us, shaping our daily experiences. Let's explore how different types of forces operate in some real-world contexts.

\subsection{Forces in Sports}

Sports provide fantastic examples of forces in action.  Consider these examples:

\begin{itemize}
    \item \textbf{Football (Soccer):} When a footballer kicks a ball, they apply an \textbf{applied force} to it, causing it to accelerate and fly through the air. \textbf{Gravity} pulls the ball downwards, causing it to follow a curved path.  \textbf{Air resistance} opposes the motion of the ball, slowing it down. \textbf{Friction} between the player's boot and the ball helps to transfer force effectively. When the ball lands on the ground, \textbf{friction} between the ball and the ground surface slows it down and eventually brings it to rest.

    \item \textbf{Basketball:}  When a basketball player dribbles the ball, they are applying a downward \textbf{applied force}. The floor exerts an upward \textbf{normal force} on the ball, causing it to bounce back up.  \textbf{Gravity} pulls the ball downwards throughout its flight. \textbf{Air resistance} is also present, though usually less significant than in football due to the lower speeds involved. When shooting a hoop, the player carefully controls the \textbf{applied force} and direction to overcome gravity and air resistance to get the ball into the basket.

    \item \textbf{Cycling:} To start cycling, you apply a force to the pedals, which, through the bicycle mechanism, exerts a force on the wheels.  \textbf{Friction} between the tyres and the road provides the necessary grip to move forward.  \textbf{Air resistance} and \textbf{rolling friction} oppose the motion of the bicycle, slowing it down.  To maintain speed, the cyclist must continuously apply force to overcome these opposing forces. When braking, the brakes apply a frictional force to the wheels, slowing the bicycle down by increasing \textbf{friction}.

    \item \textbf{Swimming:}  Swimmers propel themselves through water by applying forces with their arms and legs. They push water backwards, and according to Newton's Third Law (which we'll explore later), the water pushes them forwards.  \textbf{Water resistance} (fluid friction) is a significant force opposing the swimmer's motion.  Streamlined body position and swimming techniques are designed to minimise water resistance and increase efficiency.  \textbf{Buoyancy} also plays a role, helping swimmers to float and reduce the effort needed to stay at the surface.
\end{itemize}

\begin{stopandthink}
Choose another sport and describe the different types of forces involved in it.
\end{stopandthink}

\begin{tieredquestions}{Basic}
\begin{enumerate}
    \item Give an example of a force used in football.
    \item What force slows down a cyclist?
\end{enumerate}
\end{tieredquestions}

\begin{tieredquestions}{Intermediate}
\begin{enumerate}
    \item Explain how gravity and air resistance affect the motion of a basketball.
    \item Describe the role of friction in cycling. Is it always helpful or sometimes a hindrance?
\end{enumerate}
\end{tieredquestions}

\begin{tieredquestions}{Advanced}
\begin{enumerate}
    \item  Analyse the forces involved in a high jump. Consider forces acting during the run-up, take-off, flight, and landing.
    \item  Discuss how athletes use their understanding of forces and motion to improve their performance in different sports. Give specific examples.
\end{enumerate}
\end{tieredquestions}


\subsection{Forces in Transport}

Transport systems rely heavily on the manipulation and control of forces.

\begin{itemize}
    \item \textbf{Cars:} Cars use the \textbf{friction} between their tyres and the road to accelerate and brake. The engine provides a forward \textbf{applied force} (thrust) to overcome \textbf{air resistance} and \textbf{rolling friction}.  Brakes apply \textbf{friction} to the wheels to slow down or stop the car. \textbf{Gravity} acts downwards on the car, and the road provides an upward \textbf{normal force}.  Car designs are often streamlined to reduce \textbf{air resistance} and improve fuel efficiency.

    \item \textbf{Aeroplanes:} Aeroplanes use powerful engines to generate \textbf{thrust}, a forward force that overcomes \textbf{air resistance} (drag).  The wings are shaped to create \textbf{lift}, an upward force due to the movement of air over and under the wings. \textbf{Gravity} (weight) pulls the aeroplane downwards.  For level flight at a constant speed, thrust must balance drag, and lift must balance weight. To take off, thrust and lift must be greater than drag and weight respectively.  To land, these forces are adjusted to reduce speed and altitude.

    \item \textbf{Boats and Ships:} Boats and ships are propelled through water by engines and propellers (or sails in sailing boats), generating a forward \textbf{thrust}.  \textbf{Water resistance} (drag) opposes the motion through the water.  \textbf{Buoyancy} provides an upward force that keeps the boat afloat, balancing the \textbf{gravity} (weight) of the boat.  The shape of the hull is designed to minimise water resistance and improve efficiency.
\end{itemize}

\begin{stopandthink}
How do different modes of transport use forces to move and control their motion?
\end{stopandthink}

\begin{tieredquestions}{Basic}
\begin{enumerate}
    \item Name a force that helps a car move forward.
    \item What force keeps a boat afloat?
\end{enumerate}
\end{tieredquestions}

\begin{tieredquestions}{Intermediate}
\begin{enumerate}
    \item Explain how lift is generated by aeroplane wings.
    
\FloatBarrier
