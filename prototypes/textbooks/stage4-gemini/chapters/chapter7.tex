```latex
\chapter{Diversity of Life (Classification and Survival)}

\FloatBarrier
% Removed undefined command

Imagine walking through a rainforest. Towering trees reach for the sky, their leaves a vibrant green. Brightly coloured birds call to each other from the branches, insects buzz around your head, and beneath your feet, the soil is alive with unseen organisms. Now picture the vast, silent depths of the ocean, where strange and wonderful creatures drift in the darkness, or the harsh, frozen landscapes of the Arctic, where life clings on in the face of extreme cold.

Our planet is home to an astonishing variety of living things – from the tiniest bacteria, invisible to the naked eye, to the giant blue whale, the largest animal ever to have lived. This incredible \keyword{diversity of life} is what makes Earth so unique and fascinating.  But with so many different types of organisms, how can we possibly make sense of it all? How do we study and understand the relationships between them?

This chapter will introduce you to the science of \keyword{classification}, the system we use to organise and understand the vast array of life on Earth. We will explore the characteristics that define living things, learn about the different levels of classification, and discover how understanding the internal structures of organisms helps us to see how they survive in their environments.  Get ready to delve into the amazing world of biology and discover the incredible diversity of life!

\begin{marginfigure}
\includegraphics[width=\linewidth]{placeholder-rainforest.jpg}
\captionof{margin}{A rainforest scene showcasing the diversity of plant and animal life. \textit{[Figure: Rainforest scene - to be added]}}
\end{marginfigure}

\begin{marginnote}
\textbf{Biodiversity Hotspots:} Some regions on Earth, like rainforests and coral reefs, are particularly rich in biodiversity and are known as biodiversity hotspots. They are crucial for conservation efforts.
\end{marginnote}

\begin{stopandthink}
Why do you think it is important to study the diversity of life on Earth?  Think about practical reasons and reasons related to our understanding of the world.
\end{stopandthink}


\FloatBarrier
% Removed undefined command

Before we can classify living things, we need to understand what makes something ‘alive’ in the first place.  This might seem like a simple question, but it's surprisingly complex!  Think about a car – it can move, use energy (petrol), and even respond to its environment (with sensors).  But we wouldn’t consider a car to be alive.  So, what is the difference?

Scientists have identified several key characteristics that are shared by all living organisms. These characteristics, often summarised by the acronym MRS GREN (or MRS NERG), help us to distinguish between living and non-living things. Let's explore each of these in detail:

\subsection{Characteristics of Living Things (MRS GREN/MRS NERG)}

\begin{enumerate}
    \item \textbf{Movement (M):} All living things exhibit movement, although this can take many forms. Animals move their whole bodies or parts of them. Plants might move their leaves to face the sun, or their roots to find water. Even bacteria, tiny single-celled organisms, can move using whip-like tails called flagella.
    \begin{marginnote}
        \keyword{Movement} is not always obvious. Think about a tree – it doesn't walk around, but it moves its leaves and grows taller over time.
    \end{marginnote}
    \item \textbf{Respiration (R):} Living organisms need energy to carry out their life processes. \keyword{Respiration} is the process of releasing energy from food.  Most living things, including animals and plants, use oxygen in respiration (aerobic respiration). Some organisms, like certain bacteria and yeast, can respire without oxygen (anaerobic respiration).
    \begin{marginnote}
        \textbf{Respiration vs. Breathing:} Breathing is just one part of respiration – it's how we get oxygen into our bodies. Respiration is the chemical process of releasing energy inside cells.
    \end{marginnote}
    \item \textbf{Sensitivity (S):} Living things can detect and respond to changes in their environment. This is called \keyword{sensitivity} or irritability.  Animals have senses like sight, hearing, and touch to help them respond to stimuli. Plants can sense light, gravity, and chemicals in the soil. Even single-celled organisms can sense and move towards food or away from danger.
    \begin{marginnote}
        \challenge{Think about different ways plants show sensitivity. How does a Venus flytrap respond to stimuli?}
    \end{marginnote}
    \item \textbf{Growth (G):} Living organisms \keyword{grow} in size and complexity.  A baby grows into an adult, a seedling grows into a tree. Growth involves increasing the number of cells or the size of cells, or both. Non-living things might get bigger (like a crystal growing), but this is not considered true growth in the biological sense.
    \begin{marginnote}
        \textbf{Development:} Growth is often linked to development, which includes changes in form and function over an organism's lifetime.
    \end{marginnote}
    \item \textbf{Reproduction (R):} Living things are able to produce offspring – to \keyword{reproduce}. This ensures the continuation of their species. Reproduction can be sexual, involving two parents, or asexual, involving only one parent.
    \begin{marginnote}
        \textbf{Species:} A species is a group of organisms that can reproduce with each other and produce fertile offspring.
    \end{marginnote}
    \item \textbf{Excretion (E):}  As living things carry out their life processes, they produce waste products. \keyword{Excretion} is the process of removing these waste products from the body. Animals excrete urine and carbon dioxide; plants excrete oxygen and other waste products.
    \begin{marginnote}
        \textbf{Metabolism:} All the chemical reactions that happen inside a living organism are collectively called metabolism. Excretion is a necessary part of metabolism.
    \end{marginnote}
    \item \textbf{Nutrition (N):} Living organisms need \keyword{nutrition} – they need to take in and use nutrients to provide energy and materials for growth and repair.  Plants make their own food through photosynthesis. Animals obtain nutrients by eating other organisms.
    \begin{marginnote}
        \textbf{Producers and Consumers:} Plants are producers because they produce their own food. Animals are consumers because they consume other organisms for food.
    \end{marginnote}
\end{enumerate}

\begin{keyconcept}{Characteristics of Life}
Living things share several key characteristics: Movement, Respiration, Sensitivity, Growth, Reproduction, Excretion, and Nutrition (MRS GREN/MRS NERG).  These characteristics help us to distinguish living organisms from non-living things.
\end{keyconcept}

\begin{stopandthink}
Think about a virus. Viruses can reproduce and evolve, but they are not made of cells and cannot carry out all life processes on their own.  Are viruses considered living or non-living? This is a complex question that scientists still debate!
\end{stopandthink}

\begin{tieredquestions}{Section 2: What Does it Mean to be Alive?}
\begin{enumerate}
    \item \textbf{Basic:} List three characteristics of living things.
    \item \textbf{Intermediate:} Explain why movement is considered a characteristic of living things, even though plants don't move around like animals.
    \item \textbf{Advanced:}  A crystal can grow larger over time.  Explain why growth in a crystal is not considered the same as growth in a living organism in terms of MRS GREN.
\end{enumerate}
\end{tieredquestions}


\FloatBarrier
% Removed undefined command

Imagine a library with thousands upon thousands of books, but no system for organising them.  It would be almost impossible to find the book you were looking for!  Similarly, with millions of different species on Earth, studying biology would be incredibly difficult without a system to organise and classify them.

\keyword{Classification} is the process of grouping living things based on their shared characteristics. It's like creating categories and subcategories to make sense of a large and complex collection.  Just as we classify books by genre, author, and subject, biologists classify living organisms into different groups based on their similarities and differences.

\subsection{Benefits of Classification}

Classification is essential for many reasons:

\begin{itemize}
    \item \textbf{Organisation and Understanding:} Classification helps us to organise and understand the vast diversity of life. It allows us to see patterns and relationships between different organisms.
    \item \textbf{Communication:}  A universal classification system allows scientists all over the world to communicate effectively about organisms.  Using scientific names avoids confusion caused by common names which can vary from place to place. For example, the bird we call a ‘robin’ in Britain is different from the ‘robin’ in North America.
    \item \textbf{Identification:} Classification keys and systems help us to identify unknown organisms. By observing its characteristics and using a key, we can determine which group an organism belongs to and potentially identify its species.
    \item \textbf{Studying Evolution:} Classification reflects evolutionary relationships. Organisms in the same group are likely to share a common ancestor. By studying classification, we can learn about the history of life on Earth and how different groups of organisms have evolved over time.
    \item \textbf{Conservation:} Understanding classification is crucial for conservation efforts. By identifying species and understanding their relationships, we can better protect endangered species and manage ecosystems.
\end{itemize}

\begin{marginnote}
\textbf{Taxonomy:} The science of classification is called taxonomy. Scientists who specialize in taxonomy are called taxonomists.
\end{marginnote}

\begin{marginnote}
\historylink{Carolus Linnaeus (1707-1778):}  Carl Linnaeus, a Swedish botanist, is considered the father of modern taxonomy. He developed the binomial nomenclature system that we still use today.
\end{marginnote}

\begin{stopandthink}
Think about other areas where classification is used in everyday life.  How do we classify items in a supermarket? How are files organised on a computer? What are the benefits of these classifications?
\end{stopandthink}

\subsection{Levels of Classification: A Hierarchical System}

The classification system used by biologists is hierarchical, meaning it has levels within levels, like a set of nested boxes. The most widely used system is based on the work of Carl Linnaeus and has eight major levels, known as \keyword{taxonomic ranks}, from the broadest to the most specific:

\begin{enumerate}
    \item \textbf{Domain:} The broadest level, domains group organisms based on fundamental differences in their cell structure. There are three domains: Bacteria, Archaea, and Eukarya.
    \item \textbf{Kingdom:} Within each domain, organisms are further grouped into kingdoms.  Traditionally, there were five kingdoms, but now six or even more kingdoms are sometimes recognised, especially within the domain Eukarya.  Common kingdoms include Animalia (animals), Plantae (plants), Fungi, Protista, and Bacteria (sometimes split into Eubacteria and Archaebacteria, now domains Bacteria and Archaea).
    \item \textbf{Phylum:} Kingdoms are divided into phyla (singular: phylum).  Phyla group organisms with a similar body plan. For example, the phylum Chordata includes all animals with a backbone.
    \item \textbf{Class:} Phyla are divided into classes. For example, within the phylum Chordata, the class Mammalia includes all mammals.
    \item \textbf{Order:} Classes are divided into orders.  For example, within the class Mammalia, the order Primates includes monkeys, apes, and humans.
    \item \textbf{Family:} Orders are divided into families. For example, within the order Primates, the family Hominidae includes humans and their extinct close relatives.
    \item \textbf{Genus:} Families are divided into genera (singular: genus). A genus includes a group of very closely related species. For example, the genus \textit{Homo} includes modern humans (\textit{Homo sapiens}) and extinct human species.
    \item \textbf{Species:} The most specific level of classification. A \keyword{species} is a group of organisms that can interbreed and produce fertile offspring.  \textit{Homo sapiens} is the species name for modern humans.
\end{enumerate}

\begin{figure}
\centering
\includegraphics[width=0.7\textwidth]{placeholder-classification-hierarchy.jpg}
\caption{The hierarchical levels of classification, from Domain to Species. As you move down the hierarchy, groups become more specific and organisms within each group share more characteristics. \textit{[Figure: Diagram of classification hierarchy - to be added]}}
\end{figure}

\begin{marginnote}
\textbf{Mnemonic for remembering ranks:}  A common mnemonic to remember the order of taxonomic ranks is: \textbf{D}ear \textbf{K}ing \textbf{P}hilip \textbf{C}ame \textbf{O}ver \textbf{F}or \textbf{G}ood \textbf{S}oup. (Domain, Kingdom, Phylum, Class, Order, Family, Genus, Species).
\end{marginnote}


\subsection{Binomial Nomenclature: Giving Organisms a Scientific Name}

To avoid confusion caused by common names, scientists use a system called \keyword{binomial nomenclature} (meaning "two-name naming system") to give each species a unique scientific name. This system was also developed by Linnaeus.

Each scientific name consists of two parts:

\begin{enumerate}
    \item \textbf{Genus name:} The first part is the genus name, which is always capitalised.
    \item \textbf{Species name:} The second part is the species name, which is always written in lowercase.
\end{enumerate}

Both parts of the scientific name are always written in \textit{italics} or underlined when handwritten.  For example, the scientific name for humans is \textit{Homo sapiens}. \textit{Homo} is the genus, and \textit{sapiens} is the species.

Using binomial nomenclature ensures that every species has a unique and universally recognised name, regardless of language or location.

\begin{example}
Let's classify a domestic dog (\textit{Canis lupus familiaris}):

\begin{itemize}
    \item \textbf{Domain:} Eukarya (cells with a nucleus)
    \item \textbf{Kingdom:} Animalia (multicellular, heterotrophic, mobile)
    \item \textbf{Phylum:} Chordata (has a backbone)
    \item \textbf{Class:} Mammalia (warm-blooded, has fur/hair, produces milk)
    \item \textbf{Order:} Carnivora (meat-eating mammals)
    \item \textbf{Family:} Canidae (dogs, wolves, foxes)
    \item \textbf{Genus:} \textit{Canis} (wolves, dogs, jackals)
    \item \textbf{Species:} \textit{lupus familiaris} (domesticated subspecies of the wolf, \textit{Canis lupus})
\end{itemize}
\end{example}

\begin{stopandthink}
Why is it useful to use scientific names instead of just common names for organisms? Think about the advantages for scientists and for communication around the world.
\end{stopandthink}


\begin{tieredquestions}{Section 3: Why Classify?}
\begin{enumerate}
    \item \textbf{Basic:} What is classification?  Give one reason why classification is important in biology.
    \item \textbf{Intermediate:} Explain the difference between genus and species in binomial nomenclature.  Give an example.
    \item \textbf{Advanced:}  Why is the classification system considered hierarchical? Explain how this hierarchy helps us to understand the relationships between different organisms.
\end{enumerate}
\end{tieredquestions}


\FloatBarrier
% Removed undefined command

At the very top of the classification hierarchy are the \keyword{domains}.  The domain system is a relatively recent development in classification, reflecting our growing understanding of the fundamental differences between living things at the cellular and molecular level.  There are three domains, representing the broadest categories of life: Bacteria, Archaea, and Eukarya.

\subsection{Domain Bacteria}

The domain \keyword{Bacteria} (formerly known as Eubacteria) includes a vast and diverse group of single-celled organisms called bacteria.  Bacteria are \keyword{prokaryotic}, meaning their cells lack a nucleus and other membrane-bound organelles.  They are found everywhere on Earth – in soil, water, air, and even inside and on other organisms.

\begin{itemize}
    \item \textbf{Cell Type:} Prokaryotic (no nucleus)
    \item \textbf{Cell Wall:} Usually present, made of peptidoglycan
    \item \textbf{Number of Cells:} Unicellular (single-celled)
    \item \textbf{Nutrition:}  Diverse – some are autotrophic (making their own food through photosynthesis or chemosynthesis), others are heterotrophic (obtaining food from other organisms).
    \item \textbf{Examples:} \textit{Escherichia coli} (\textit{E. coli}), \textit{Streptococcus}, \textit{Bacillus subtilis}.
\end{itemize}

Bacteria play crucial roles in ecosystems. Some are decomposers, breaking down dead organic matter. Others are involved in nutrient cycling, such as nitrogen fixation.  Some bacteria are also important in industry and medicine, while others can cause diseases.

\begin{marginnote}
\textbf{Prokaryotic vs. Eukaryotic:}  The main difference between prokaryotic and eukaryotic cells is the presence of a nucleus. Eukaryotic cells have a nucleus, which contains the cell's DNA, while prokaryotic cells do not.
\end{marginnote}

\begin{figure}
\centering
\includegraphics[width=0.4\textwidth]{placeholder-bacteria.jpg}
\caption{Bacteria come in various shapes, such as rods, spheres, and spirals. \textit{[Figure: Microscopic image of bacteria - to be added]}}
\end{figure}


\subsection{Domain Archaea}

The domain \keyword{Archaea} (formerly known as Archaebacteria) is another group of prokaryotic organisms.  Initially, archaea were thought to be a type of bacteria, but scientists discovered that they are fundamentally different from bacteria in their genetic makeup and biochemistry.  Archaea often live in extreme environments, such as hot springs, salt lakes, and deep-sea vents. They are sometimes called \keyword{extremophiles}.

\begin{itemize}
    \item \textbf{Cell Type:} Prokaryotic (no nucleus)
    \item \textbf{Cell Wall:} Present, but not made of peptidoglycan (composition varies)
    \item \textbf{Number of Cells:} Unicellular (single-celled)
    \item \textbf{Nutrition:} Diverse – some are autotrophic (chemosynthesis), others are heterotrophic.
    \item \textbf{Examples:} Methanogens (produce methane), halophiles (live in salty environments), thermophiles (live in hot environments).
\end{itemize}

Archaea are also important in nutrient cycling and some have unique metabolic pathways.  They are increasingly recognised for their ecological importance and potential biotechnological applications.

\begin{marginnote}
\textbf{Extremophiles:} Archaea are well-adapted to survive in extreme conditions that would be lethal to most other organisms.
\end{marginnote}

\begin{figure}
\centering
\includegraphics[width=0.4\textwidth]{placeholder-archaea.jpg}
\caption{Archaea often thrive in extreme environments, like these colourful hot springs. \textit{[Figure: Image of hot springs or other archaea habitat - to be added]}}
\end{figure}


\subsection{Domain Eukarya}

The domain \keyword{Eukarya} includes all eukaryotic organisms – organisms whose cells have a nucleus and other membrane-bound organelles.  This domain is incredibly diverse and includes all plants, animals, fungi, and protists.

\begin{itemize}
    \item \textbf{Cell Type:} Eukaryotic (has a nucleus and organelles)
    \item \textbf{Cell Wall:} Present in plants (cellulose) and fungi (chitin), absent in animals and some protists.
    \item \textbf{Number of Cells:} Unicellular or multicellular
    \item \textbf{Nutrition:} Diverse – autotrophic (plants and some protists), heterotrophic (animals, fungi, and some protists).
    \item \textbf{Examples:} Animals (humans, insects, fish), plants (trees, flowers, grasses), fungi (mushrooms, yeasts, moulds), protists (amoeba, paramecium, algae).
\end{itemize}

The domain Eukarya is further divided into kingdoms, which we will explore in the next section.  Eukaryotic cells are more complex than prokaryotic cells, allowing for greater complexity and diversity in eukaryotic organisms.

\begin{marginnote}
\textbf{Organelles:} Organelles are membrane-bound structures within eukaryotic cells that perform specific functions, such as the nucleus (contains DNA), mitochondria (energy production), and chloroplasts (photosynthesis in plants).
\end{marginnote}

\begin{figure}
\centering
\includegraphics[width=0.4\textwidth]{placeholder-eukaryotic-cell.jpg}
\caption{A diagram of a eukaryotic cell, showing the nucleus and other organelles. \textit{[Figure: Diagram of a eukaryotic cell - to be added]}}
\end{figure}


\begin{stopandthink}
Compare and contrast the three domains of life: Bacteria, Archaea, and Eukarya. What are the key differences and similarities between them?
\end{stopandthink}


\begin{tieredquestions}{Section 4: The Domains of Life}
\begin{enumerate}
    \item \textbf{Basic:} Name the three domains of life.
    \item \textbf{Intermediate:} What is the main difference between prokaryotic and eukaryotic cells? Which domains are prokaryotic and which are eukaryotic?
    \item \textbf{Advanced:}  Why were Archaea initially classified as bacteria (Archaebacteria)? What discoveries led scientists to recognise them as a separate domain?
\end{enumerate}
\end{tieredquestions}


\FloatBarrier
% Removed undefined command

Within the domain Eukarya, there are several kingdoms.  While the exact number and definition of kingdoms can vary slightly depending on the classification system used, we will focus on the traditionally recognised kingdoms within Eukarya: Protista, Fungi, Plantae, and Animalia.  (Note: Sometimes Protista is further divided, and the kingdom level classification within Eukarya is still an area of ongoing research and refinement).

\subsection{Kingdom Protista}

The \keyword{Kingdom Protista} is often described as a ‘mismatched’ kingdom because it includes all eukaryotic organisms that are not plants, animals, or fungi. Protists are very diverse and can be unicellular or multicellular, autotrophic or heterotrophic, and have a wide range of lifestyles.  Because of their diversity, some scientists are now classifying many protist groups into their own kingdoms.

\begin{itemize}
    \item \textbf{Cell Type:} Eukaryotic
    \item \textbf{Cell Wall:} May or may not be present (composition varies if present)
    \item \textbf{Number of Cells:} Mostly unicellular, some multicellular (simple organisation)
    \item \textbf{Nutrition:}  Diverse – autotrophic (algae), heterotrophic (amoeba, paramecium), or both (mixotrophic).
    \item \textbf{Movement:}  Some have cilia, flagella, or pseudopods for movement; others are non-motile.
    \item \textbf{Examples:} Amoeba, paramecium, euglena, algae (seaweed, kelp), slime moulds.
\end{itemize}

Protists are ecologically important, especially \keyword{algae}, which are major producers in aquatic ecosystems, carrying out photosynthesis and forming the base of many food webs.  Other protists are important decomposers or play roles in nutrient cycling. Some protists can also cause diseases, such as malaria (caused by the protist \textit{Plasmodium}).

\begin{marginnote}
\textbf{Algae vs. Plants:} Algae are plant-like protists that carry out photosynthesis.  They lack the complex structures of true plants, such as roots, stems, and leaves.
\end{marginnote}

\begin{figure}
\centering
\includegraphics[width=0.4\textwidth]{placeholder-protists.jpg}
\caption{Protists are incredibly diverse, ranging from microscopic amoebas to large seaweeds. \textit{[Figure: Collage of various protists - amoeba, paramecium, algae, etc. - to be added]}}
\end{figure}


\subsection{Kingdom Fungi}

The \keyword{Kingdom Fungi} includes organisms like mushrooms, moulds, yeasts, and mildew. Fungi are eukaryotic, mostly multicellular (yeasts are unicellular), and \keyword{heterotrophic}.  They obtain nutrients by absorbing organic matter from their surroundings.  They play a vital role as decomposers in ecosystems, breaking down dead plants and animals and recycling nutrients.

\begin{itemize}
    \item \textbf{Cell Type:} Eukaryotic
    \item \textbf{Cell Wall:} Present, made of chitin
    \item \textbf{Number of Cells:} Mostly multicellular (except yeasts, which are unicellular)
    \item \textbf{Nutrition:} Heterotrophic (absorptive nutrition)
    \item \textbf{Structure:} Made of hyphae (thread-like filaments) forming a mycelium.
    \item \textbf{Examples:} Mushrooms, toadstools, yeasts, moulds (e.g., \textit{Penicillium}), rusts, smuts.
\end{itemize}

Fungi are essential decomposers in ecosystems. Some fungi are also important in food production (e.g., yeast in baking and brewing, mushrooms as food) and medicine (e.g., penicillin antibiotic from \textit{Penicillium} mould).  However, some fungi can also cause diseases in plants and animals.

\begin{marginnote}
\textbf{Chitin:} Chitin is a strong, flexible material that is also found in the exoskeletons of insects and crustaceans.
\end{marginnote}

\begin{figure}
\centering
\includegraphics[width=0.4\textwidth]{placeholder-fungi.jpg}
\caption{Fungi come in many forms, from mushrooms to moulds and yeasts. \textit{[Figure: Collage of various fungi - mushrooms, mould, yeast cells - to be added]}}
\end{figure}


\subsection{Kingdom Plantae}

The \keyword{Kingdom Plantae} includes all plants – from mosses and ferns to trees and flowering plants. Plants are eukaryotic, multicellular, and \keyword{autotrophic}. They make their own food through \keyword{photosynthesis}, using sunlight, water, and carbon dioxide. Plants are the primary producers in most terrestrial ecosystems, forming the base of food webs and providing oxygen for other organisms.

\begin{itemize}
    \item \textbf{Cell Type:} Eukaryotic
    \item \textbf{Cell Wall:} Present, made of cellulose
    \item \textbf{Number of Cells:} Multicellular
    \item \textbf{Nutrition:} Autotrophic (photosynthesis)
    \item \textbf{Structure:}  Have roots, stems, leaves, and specialised tissues for transport (xylem and phloem).
    \item \textbf{Examples:} Mosses, ferns, conifers (pine trees), flowering plants (roses, grasses, oaks).
\end{itemize}

Plants are essential for life on Earth. They produce oxygen, provide food and shelter for animals, and play a crucial role in regulating climate and maintaining soil health.  The diversity of plants is vast, adapted to a wide range of environments.

\begin{marginnote}
\textbf{Photosynthesis:} The process by which plants use sunlight, water, and carbon dioxide to produce glucose (sugar) and oxygen. The chemical equation for photosynthesis is:
\ce{6CO2 + 6H2O -> C6H12O6 + 6O2}
\mathlink{Link to chemical equations and balancing.}
\end{marginnote}

\begin{figure}
\centering
\includegraphics[width=0.4\textwidth]{placeholder-plants.jpg}
\caption{The plant kingdom includes a huge variety of forms, from small mosses to giant trees. \textit{[Figure: Collage of various plants - moss, fern, tree, flower - to be added]}}
\end{figure}


\subsection{Kingdom Animalia}

The \keyword{Kingdom Animalia} includes all animals – from sponges and insects to fish, birds, and mammals (including humans). Animals are eukaryotic, multicellular, and \keyword{heterotrophic}. They obtain nutrients by consuming other organisms. Animals are typically mobile and have specialised sensory organs and nervous systems, allowing them to respond to their environment and interact with it in complex ways.

\begin{itemize}
    \item \textbf{Cell Type:} Eukaryotic
    \item \textbf{Cell Wall:} Absent
    \item \textbf{Number of Cells:} Multicellular
    \item \textbf{Nutrition:} Heterotrophic (ingestive nutrition)
    \item \textbf{Movement:} Generally motile at some stage in their life cycle.
    \item \textbf{Examples:} Sponges, jellyfish, worms, insects, snails, starfish, fish, amphibians, reptiles, birds, mammals.
\end{itemize}

Animals are incredibly diverse and occupy a wide range of habitats. They play diverse roles in ecosystems, including predation, herbivory, decomposition, and pollination.  Understanding animal diversity is crucial for conservation and for understanding the complex interactions within ecosystems.

\begin{marginnote}
\textbf{Vertebrates and Invertebrates:} Animals are often divided into vertebrates (animals with a backbone) and invertebrates (animals without a backbone). Vertebrates belong to the phylum Chordata.
\end{marginnote}

\begin{figure}
\centering
\includegraphics[width=0.4\textwidth]{placeholder-animals.jpg}
\caption{The animal kingdom is incredibly diverse, with a huge range of body forms and lifestyles. \textit{[Figure: Collage of various animals - insect, fish, bird, mammal - to be added]}}
\end{figure}


\begin{stopandthink}
Choose one kingdom from Eukarya (Protista, Fungi, Plantae, or Animalia).  Describe its key characteristics and give a few examples of organisms within that kingdom.
\end{stopandthink}


\begin{tieredquestions}{Section 5: The Kingdoms of Eukarya}
\begin{enumerate}
    \item \textbf{Basic:} Name the four main kingdoms within the domain Eukarya.
    \item \textbf{Intermediate:} Compare and contrast the kingdoms Plantae and Fungi in terms of cell type, cell wall, and nutrition.
    \item \textbf{Advanced:}  Why is the Kingdom Protista sometimes considered a 'mismatched' kingdom? What does this tell us about the challenges of classifying life and the ongoing nature of scientific discovery?
\end{enumerate}
\end{tieredquestions}


\FloatBarrier
% Removed undefined command

The internal structures of an organism are intricately linked to its functions and its ability to survive in its environment.  Understanding the relationship between \keyword{structure and function} is fundamental to biology.  Let's explore some examples of how internal structures enable survival in different organisms.

\subsection{Plant Structures and Functions}

Plants have evolved specialised structures to carry out essential life functions:

\begin{itemize}
    \item \textbf{Roots:} Anchor the plant in the ground and absorb water and mineral nutrients from the soil. Root hairs increase the surface area for absorption.
    \begin{marginnote}
        \textbf{Adaptations:} Root systems can be adapted to different environments. For example, desert plants often have very deep roots to reach underground water sources.
    \end{marginnote}
    \item \textbf{Stem:} Provides support for the plant, holds leaves up to sunlight, and transports water and nutrients between roots and leaves through vascular tissues (xylem and phloem).
    \begin{marginnote}
        \textbf{Xylem and Phloem:} Xylem vessels transport water and minerals upwards from the roots. Phloem vessels transport sugars (produced in photosynthesis) from the leaves to other parts of the plant.
    \end{marginnote}
    \item \textbf{Leaves:} The main site of photosynthesis in most plants. Leaves are broad and flat to maximise surface area for capturing sunlight. They contain chloroplasts, organelles where photosynthesis occurs. Stomata (tiny pores) on the leaf surface allow for gas exchange (carbon dioxide in, oxygen out).
    \begin{marginnote}
        \textbf{Stomata:} Stomata can open and close to regulate gas exchange and water loss (transpiration).
    \end{marginnote}
    \item \textbf{Flowers, Fruits, and Seeds:} Structures involved in reproduction in flowering plants. Flowers attract pollinators, fruits protect seeds and aid in seed dispersal, and seeds contain the embryo and food reserves for the new plant.
    \begin{marginnote}
        \textbf{Pollination:} The transfer of pollen from the stamen to the stigma of a flower, often aided by wind, insects, or birds.
    \end{marginnote}
\end{itemize}

These structures work together to enable plants to survive, grow, and reproduce in their specific environments.  For example, plants in dry environments may have adaptations like thick waxy cuticles on their leaves to reduce water loss, or succulent stems to store water.

\begin{figure}
\centering
\includegraphics[width=0.6\textwidth]{placeholder-plant-structure.jpg}
\caption{Diagram showing the main structures of a flowering plant: roots, stem, leaves, flower, fruit, and seed. \textit{[Figure: Labelled diagram of plant structure - to be added]}}
\end{figure}


\subsection{Animal Structures and Functions}

Animals also have diverse internal structures that are adapted for their lifestyles and environments:

\begin{itemize}
    \item \textbf{Digestive System:} Breaks down food into smaller molecules that can be absorbed into the bloodstream and used for energy and building materials.  Different animals have different digestive systems adapted to their diets (e.g., herbivores have longer digestive tracts than carnivores to digest plant matter).
    \begin{marginnote}
        \textbf{Enzymes:} Digestive enzymes are biological catalysts that speed up the breakdown of food molecules.
    \end{marginnote}
    \item \textbf{Respiratory System:}  Responsible for gas exchange – taking in oxygen and releasing carbon dioxide.  Animals have various respiratory structures, such as lungs (in mammals, birds, reptiles), gills (in fish), or tracheal systems (in insects).
    \begin{marginnote}
        \textbf{Diffusion:} Gas exchange occurs through diffusion – the movement of molecules from an area of high concentration to an area of low concentration.
    \end{marginnote}
    \item \textbf{Circulatory System:} Transports oxygen, nutrients, hormones, and waste products around the body.  Animals have either open or closed circulatory systems.  Vertebrates have closed circulatory systems with a heart that pumps blood through vessels.
    \begin{marginnote}
        \textbf{Open vs. Closed Circulatory Systems:} In open circulatory systems, blood is not always contained within vessels. In closed circulatory systems, blood is always contained within vessels.
    \end{marginnote}
    \item \textbf{Excretory System:} Removes metabolic waste products from the body.  Animals have various excretory organs, such as kidneys (in vertebrates), Malpighian tubules (in insects), or nephridia (in worms).
    \begin{marginnote}
        \textbf{Kidneys:} Kidneys filter waste products from the blood and produce urine.
    \end{marginnote}
    \item \textbf{Nervous System:}  Detects and responds to stimuli from the environment.  Animals have nervous systems ranging from simple nerve nets (in jellyfish) to complex brains and spinal cords (in vertebrates).
    \begin{marginnote}
        \textbf{Neurons:} Nerve cells (neurons) transmit electrical signals throughout the nervous system.
    \end{marginnote}
\end{itemize}

These systems are interconnected and work together to maintain homeostasis (a stable internal environment) and enable animals to survive and thrive in their environments.  For example, animals living in cold environments may have adaptations like thick fur or blubber for insulation, or efficient circulatory systems to conserve heat.

\begin{figure}
\centering
\includegraphics[width=0.6\textwidth]{placeholder-animal-systems.jpg}
\caption{Diagram showing the major organ systems in a mammal: digestive, respiratory, circulatory, excretory, and nervous systems. \textit{[Figure: Diagram of animal organ systems - to be added]}}
\end{figure}


\begin{stopandthink}
Choose an animal or plant that lives in an extreme environment (e.g., desert, Arctic, deep sea). Describe some of its internal or external adaptations that help it survive in that environment.
\end{stopandthink}


\begin{investigation}{Investigating Plant Adaptations}
\textbf{Title:}  Leaf Adaptations to Different Environments

\textbf{Aim:} To observe and compare leaf structures from plants adapted to different environments and relate these structures to their functions.

\textbf{Materials:}
\begin{itemize}
    \item Leaves from different types of plants (e.g., from a garden, park, or botanical garden). Try to include leaves from plants adapted to different conditions, such as:
    \begin{itemize}
        \item A plant from a sunny, dry location (e.g., succulent, cactus leaf if safe to handle, or a drought-tolerant garden plant leaf).
        \item A plant from a shady, moist location (e.g., fern frond, broad leaf from a shade-loving plant).
        \item A typical broadleaf plant from a garden or park.
    \end{itemize}
    \item Hand lens or microscope (optional, for closer observation).
    \item Ruler or measuring tape.
    \item Notebook and pencil for recording observations.
\end{itemize}

\textbf{Procedure:}
\begin{enumerate}
    \item Collect leaves from different plants. Ensure you have permission to collect leaves and do not damage the plants.
    \item Observe each leaf carefully using your eyes and a hand lens or microscope if available.
    \item Record your observations in your notebook, noting the following for each leaf:
    \begin{itemize}
        \item Leaf shape and size.
        \item Leaf surface (smooth, hairy, waxy, etc.).
        \item Leaf thickness.
        \item Presence of any other noticeable features (e.g., thorns, spines, rolled edges).
    \end{itemize}
    \item Try to identify the type of environment each plant is likely adapted to (sunny/dry, shady/moist, etc.).
    \item Compare the leaf features you observed with the likely environment.
    \item Discuss how the leaf structures might be adaptations for survival in those environments. For example, how might a waxy leaf surface help a plant in a dry environment? How might broad leaves be advantageous in a shady environment?
\end{enumerate}

\textbf{Analysis and Conclusion:}
\begin{enumerate}
    \item What similarities and differences did you observe in the leaf structures?
    \item How do the leaf structures you observed relate to the functions of leaves (photosynthesis, water conservation, etc.)?
    \item How do the leaf structures appear to be adapted to the different environments?
    \item What conclusions can you draw about the relationship between structure and function in plant leaves and their survival in different environments?
\end{enumerate}
\end{investigation}


\begin{tieredquestions}{Section 6: Structure and Function}
\begin{enumerate}
    \item \textbf{Basic:} Name two internal structures found in plants and describe their functions.
    \item \textbf{Intermediate:} Explain how the structure of a leaf is adapted for photosynthesis.
    \item \textbf{Advanced:}  Compare and contrast the respiratory systems of a fish and a mammal. How are their structures adapted to their different environments (aquatic vs. terrestrial)?
\end{enumerate}
\end{tieredquestions}


\FloatBarrier
% Removed undefined command

\keyword{Biodiversity} is the variety of life on Earth at all levels, from genes to ecosystems, and can encompass the evolutionary, ecological, and cultural processes that sustain life. It is not just about the number of species, but also the variation within species and the variety of ecosystems.  Biodiversity is essential for a healthy planet and provides us with many benefits.

\subsection{Importance of Biodiversity}

Biodiversity is important for many reasons:

\begin{itemize}
    \item \textbf{Ecosystem Stability and Function:} Diverse ecosystems are more stable and resilient to changes and disturbances. Different species play different roles in ecosystems (e.g., producers, consumers, decomposers), and a greater variety of species ensures that these roles are filled even if some species are lost.
    \item \textbf{Ecosystem Services:} Biodiversity provides us with essential ecosystem services, such as:
    \begin{itemize}
        \item \textbf{Clean air and water:} Plants and microorganisms help to purify air and water.
        \item \textbf{Pollination:} Insects, birds, and other animals pollinate crops and wild plants.
        \item \textbf{Climate regulation:} Forests and oceans absorb carbon dioxide, helping to regulate the climate.
        \item \textbf{Soil fertility:} Soil organisms maintain soil fertility, essential for agriculture.
        \item \textbf{Food and resources:} We obtain
\FloatBarrier
