```latex
\chapter{Introduction to Scientific Inquiry}

\epigraph{Science is not only a disciple of reason but, also, one of romance and passion.}{Stephen Hawking}

\section{What is Scientific Inquiry?}

Have you ever looked up at the night sky and wondered about the stars? Or maybe you've noticed how plants grow towards sunlight? These are the kinds of questions that spark \keyword{scientific inquiry} – the process of asking questions about the world around us and finding answers through careful investigation.

Science is more than just a collection of facts; it's a way of thinking and exploring. It's about being curious, asking questions, and then systematically trying to find the answers. This chapter will introduce you to the exciting world of scientific inquiry and equip you with the basic skills you need to start your own scientific adventures.

\marginnote{\textit{Definition:} \keyword{Scientific inquiry} is the process of investigating questions and ideas in a systematic way to understand the natural world.}

\begin{keyconcept}{The Nature of Science}
Science is a journey of discovery, driven by curiosity and a desire to understand how the world works. It relies on evidence and logical reasoning to build our knowledge. Scientific ideas are always open to revision as we learn more.
\end{keyconcept}

Science is all around us, from the technology we use every day to understanding our own bodies.  It helps us to solve problems, develop new technologies, and make informed decisions about our world. Whether you dream of becoming a scientist, engineer, doctor, or simply want to be a knowledgeable citizen, understanding scientific inquiry is essential.

\historylink{The word "science" comes from the Latin word "scientia," meaning "knowledge."  For centuries, people have been using observation and reasoning to understand the world, from ancient Greek philosophers to early astronomers.}

\section{Laboratory Safety: Your First Priority}

Before we dive into experiments and investigations, there’s something extremely important we need to discuss: \keyword{laboratory safety}.  A science laboratory is a fantastic place to learn and explore, but it can also be a place where accidents can happen if we are not careful.  Thinking about safety isn’t just about following rules; it's about being responsible and ensuring everyone can learn in a safe environment.

\marginnote{\textit{Safety First:} In any scientific activity, your safety and the safety of others is the top priority.}

\subsection{Why is Lab Safety Important?}

Imagine a kitchen without any rules – people running around with knives, hot pans left unattended, and no one washing their hands! It would be chaotic and dangerous, right?  A science lab can be similar if we don't have safety rules.  We use equipment, chemicals, and sometimes even heat, which all require careful handling.

\begin{itemize}
    \item \textbf{Preventing Accidents:}  Safety rules are designed to minimise the risk of accidents like cuts, burns, chemical spills, and fires.
    \item \textbf{Protecting Yourself and Others:}  Following safety procedures protects you, your classmates, and your teacher from harm.
    \item \textbf{Ensuring Successful Experiments:}  A safe lab is a focused lab. By following safety rules, we can concentrate on our experiments and get accurate results.
    \item \textbf{Respecting the Lab Environment:}  Treating the lab and its equipment with respect ensures it remains a safe and useful learning space for everyone.
\end{itemize}

\subsection{Essential Laboratory Safety Rules}

Let’s go through some essential safety rules that you will need to follow in any science laboratory. These might seem like a lot to remember at first, but they will quickly become second nature as you practice them.

\subsubsection{Personal Protective Equipment (PPE)}

Just like athletes wear protective gear in sports, scientists wear \keyword{Personal Protective Equipment} (PPE) in the lab to shield themselves from potential hazards.

\begin{itemize}
    \item \textbf{Safety Goggles:}  Your eyes are extremely vulnerable. \keyword{Safety goggles} must be worn at all times during experiments, especially when working with chemicals, heat, or glassware. They protect your eyes from splashes, fumes, and flying objects.
    \marginnote{\textit{Eye Protection:} Always wear safety goggles whenever you are in the lab and during experiments.}
    \item \textbf{Lab Coats or Aprons:}  \keyword{Lab coats} or \keyword{aprons} protect your skin and clothing from chemical spills and splashes. They should be buttoned up and worn throughout the experiment.
    \marginnote{\textit{Skin and Clothing Protection:} Lab coats or aprons act as a barrier to protect your skin and clothes.}
    \item \textbf{Closed-toe Shoes:}  Protect your feet from spills, broken glass, and falling objects. Sandals, flip-flops, or open-toe shoes are not allowed in the lab. Sturdy, closed-toe shoes are essential.
    \marginnote{\textit{Foot Protection:} Closed-toe shoes are vital for preventing foot injuries.}
    \item \textbf{Gloves (when needed):}  Sometimes, you will need to wear gloves to protect your hands from chemicals or biological materials. Your teacher will tell you when gloves are required and what type to use.
    \marginnote{\textit{Hand Protection:} Gloves provide a barrier against chemicals and biological materials. Use the correct type of glove as instructed.}
    \item \textbf{Long Hair Tied Back:}  Long hair can be a fire hazard, get in the way of experiments, or dip into chemicals. Tie long hair back securely whenever you are in the lab.
    \marginnote{\textit{Hair Safety:} Tie back long hair to prevent accidents and contamination.}
\end{itemize}

\subsubsection{Handling Chemicals Safely}

Chemicals are the building blocks of matter, and we often use them in experiments. However, some chemicals can be harmful if not handled correctly.

\begin{itemize}
    \item \textbf{Read Labels Carefully:}  Always read the label on a chemical container before using it. Understand the warnings, instructions for use, and any safety precautions.
    \marginnote{\textit{Chemical Labels:} Labels provide crucial information about the chemical and its hazards.}
    \item \textbf{Never Taste or Smell Chemicals Directly:}  Unless specifically instructed by your teacher (which is very rare!), never taste or smell chemicals directly.  To smell a chemical safely, gently waft the fumes towards your nose with your hand. This is called \textit{wafting}.
    \marginnote{\textit{Smelling Chemicals Safely:} Waft fumes towards your nose instead of directly inhaling them.}
    \item \textbf{Dispense Chemicals Carefully:}  Pour chemicals slowly and carefully to avoid spills. Use appropriate equipment like funnels and droppers.
    \marginnote{\textit{Careful Dispensing:} Avoid spills by pouring slowly and using appropriate tools.}
    \item \textbf{Diluting Acids:} If you need to dilute an acid, \textbf{always add acid to water, never water to acid}. This is because adding water to concentrated acid can cause a dangerous exothermic reaction, which can splash acid out of the container. Remember the saying: "Do as you oughta, add acid to water!"
    \marginnote{\textit{Diluting Acids:} Always add acid to water to prevent dangerous reactions and splashing.}
    \begin{example}
        \textbf{Diluting Sulfuric Acid (\ce{H2SO4})}

        To dilute concentrated sulfuric acid, you would carefully and slowly pour the acid into a larger volume of water, stirring gently.  The water absorbs the heat released during the dilution, making it safer.  Adding water to concentrated sulfuric acid can cause it to boil and splash violently due to the heat generated.
    \end{example}
    \item \textbf{Proper Disposal:}  Dispose of chemicals as instructed by your teacher. Do not pour chemicals down the sink unless you are told it is safe to do so.  Some chemicals require special disposal methods.
    \marginnote{\textit{Chemical Disposal:} Follow your teacher's instructions for proper chemical waste disposal.}
\end{itemize}

\subsubsection{Using Equipment Safely}

Laboratory equipment, from glassware to Bunsen burners, helps us conduct experiments. But each piece of equipment needs to be handled with care.

\begin{itemize}
    \item \textbf{Glassware:}  Glassware is fragile and can break easily. Handle it carefully. Report any broken or chipped glassware to your teacher immediately. Never use damaged glassware. Be cautious with hot glassware as it can look the same as cold glassware.
    \marginnote{\textit{Glassware Handling:} Be careful with glassware, and report any damage to your teacher.}
    \item \textbf{Bunsen Burners (if used):}  If your lab uses Bunsen burners, be extremely careful with open flames. Tie back long hair, keep flammable materials away, and never leave a lit burner unattended. Learn how to light and adjust the burner correctly.
    \marginnote{\textit{Bunsen Burner Safety:} Be cautious with open flames. Know how to use a Bunsen burner safely and never leave it unattended.}
    \item \textbf{Electrical Equipment:}  Be careful around electrical equipment. Make sure your hands are dry before using electrical devices. Check for frayed cords or damaged equipment and report any issues to your teacher.  Do not overload electrical outlets.
    \marginnote{\textit{Electrical Safety:} Keep electrical equipment away from water and report any damage.}
    \item \textbf{Heating Substances:} When heating substances, use appropriate equipment like beakers, test tubes, and heating plates. Never heat closed containers as pressure can build up and cause explosions. Use tongs or heat-resistant gloves to handle hot objects.
    \marginnote{\textit{Heating Safety:} Use appropriate equipment for heating and never heat closed containers.}
\end{itemize}

\subsubsection{General Lab Conduct}

Safety is also about how we behave in the lab. A responsible and focused approach prevents many accidents.

\begin{itemize}
    \item \textbf{No Running or Horseplay:}  The lab is not a playground. Running, playing, or engaging in horseplay is strictly prohibited as it can lead to accidents.
    \marginnote{\textit{Lab Behaviour:} Maintain a calm and focused attitude in the lab. No running or horseplay.}
    \item \textbf{Know the Location of Safety Equipment:}  Familiarise yourself with the location of safety equipment in the lab, such as fire extinguishers, fire blankets, eyewash stations, and first aid kits. Know how to use them in case of an emergency.
    \marginnote{\textit{Safety Equipment Locations:} Know where safety equipment is located and how to use it.}
    \item \textbf{Report Accidents Immediately:}  If any accident, spill, or injury occurs, no matter how minor it seems, report it to your teacher immediately.  Early action can prevent problems from becoming worse.
    \marginnote{\textit{Accident Reporting:} Report all accidents and spills to your teacher immediately.}
    \item \textbf{Clean Up Spills Immediately:}  If you spill something, clean it up immediately according to your teacher's instructions. Spills can be slippery and dangerous.
    \marginnote{\textit{Spill Clean-up:} Clean up spills promptly and according to instructions.}
    \item \textbf{Wash Your Hands:** Wash your hands thoroughly with soap and water after every experiment and before leaving the lab. This removes any chemicals or substances you might have come into contact with.
    \marginnote{\textit{Hand Washing:} Always wash your hands after experiments and before leaving the lab.}
    \item \textbf{No Food or Drink in the Lab:** Eating or drinking in the lab is strictly forbidden. Chemicals can contaminate food and drinks, and labware should never be used for food or drinks.
    \marginnote{\textit{No Food or Drink:} Keep food and drinks out of the lab to prevent contamination and accidents.}
    \item \textbf{Follow Instructions:** Always listen carefully to your teacher's instructions and follow them precisely. If you are unsure about anything, ask your teacher for clarification before proceeding.
    \marginnote{\textit{Follow Instructions:} Pay attention to and follow your teacher's instructions at all times.}
\end{itemize}

\begin{stopandthink}
Imagine you accidentally spill a chemical on your hand. What are the first three things you should do?
\end{stopandthink}

\begin{investigation}{Safety Scenarios}
\textbf{Materials:}  Worksheet with safety scenarios (provided by teacher), pen.

\textbf{Procedure:}
\begin{enumerate}
    \item Read each safety scenario on the worksheet carefully.
    \item For each scenario, identify the potential hazards and the safety rules that are being violated (if any).
    \item Describe the correct and safe way to handle each situation.
    \item Discuss your answers with a partner and then with the class.
\end{enumerate}

\textbf{Example Scenario:}  "Sarah is heating a test tube of liquid over a Bunsen burner. She is wearing safety goggles, but her long hair is not tied back, and she is looking directly into the test tube."

\textbf{Analysis:} Identify the hazards and correct actions for each scenario. Discuss why these safety measures are important.
\end{investigation}

\begin{tieredquestions}{Laboratory Safety}
\begin{enumerate}
    \item \textbf{Basic:} List three items of Personal Protective Equipment (PPE) and explain what each protects.
    \item \textbf{Intermediate:} Explain why it is important to add acid to water when diluting acids, rather than water to acid.
    \item \textbf{Advanced:}  Design a short safety checklist that students should go through before starting any experiment in the lab.  Explain why each item on your checklist is important for safety.
\end{enumerate}
\end{tieredquestions}

\section{The Scientific Method: A Systematic Approach}

Scientists don't just guess at answers; they use a systematic approach to investigate questions and find evidence-based explanations. This approach is often called the \keyword{scientific method}. It’s a bit like a recipe for discovery, helping us to explore the world in a structured and reliable way.  While scientists don't always follow these steps in a rigid order, they provide a useful framework for scientific inquiry.

\marginnote{\textit{Systematic Inquiry:} The scientific method is a structured way to investigate questions and find answers.}

\begin{keyconcept}{The Scientific Method}
The scientific method is a process of investigation that typically involves observation, questioning, hypothesis formation, experimentation, data analysis, and conclusion. It is a cyclical process, where conclusions often lead to new questions and further investigations.
\end{keyconcept}

\subsection{Steps of the Scientific Method}

Let's break down the common steps of the scientific method. Remember, this is a guideline, and the process can be flexible and iterative.

\subsubsection{1. Observation and Questioning: Sparking Curiosity}

Scientific inquiry often begins with \keyword{observation}.  This means carefully noticing things around you.  Observations can be made with your senses – sight, hearing, touch, smell, and taste (though taste is rarely used in lab settings and never without explicit instruction!).  Observations can lead to questions about why things are the way they are.

\marginnote{\textit{Observation:}  Carefully noticing things around you using your senses.}
\marginnote{\textit{Questioning:}  Asking "why" or "how" about your observations.}

\begin{example}
\textbf{Observation:} You notice that some plants in your garden grow taller in shady areas than in sunny areas.

\textbf{Question:}  Why do some plants grow taller in the shade than in the sun?
\end{example}

Good scientific questions are usually focused, testable, and lead to further investigation.

\subsubsection{2. Hypothesis Formation: Making an Educated Guess}

Once you have a question, the next step is to form a \keyword{hypothesis}. A hypothesis is a testable explanation or a proposed answer to your question.  It’s like making an educated guess based on your prior knowledge and observations. A good hypothesis is specific and can be tested through experimentation.

\marginnote{\textit{Hypothesis:} A testable explanation or proposed answer to a scientific question.}

A hypothesis is often written as an "if...then..." statement.

\begin{example}
\textbf{Question:} Why do some plants grow taller in the shade than in the sun?

\textbf{Hypothesis:} \textbf{If} plants are grown in shade, \textbf{then} they will grow taller than plants grown in sunlight because shade reduces water loss, allowing for more stem elongation.

\challenge{Consider alternative hypotheses for the plant growth observation. Could it be related to light intensity, temperature, or nutrient availability in different locations?}

\keyword{Variables} are factors that can change or be changed in an experiment. In a hypothesis, we often identify:
\begin{itemize}
    \item \textbf{Independent Variable:} The variable that you change or manipulate in an experiment. It is the factor you are testing.  (In the plant example, it’s the amount of sunlight: shade vs. sun).
    \item \textbf{Dependent Variable:} The variable that you measure or observe. It is the factor that might change in response to the independent variable. (In the plant example, it’s the height of the plants).
    \item \textbf{Controlled Variables (Constants):}  Variables that you keep the same throughout the experiment to ensure that only the independent variable is affecting the dependent variable. (For the plant example, controlled variables could be the type of plant, amount of water, type of soil, temperature, etc.).
\end{itemize}
\marginnote{\textit{Variables:} Factors that can change or be changed in an experiment.}
\marginnote{\textit{Independent Variable:} The variable you change.}
\marginnote{\textit{Dependent Variable:} The variable you measure.}
\marginnote{\textit{Controlled Variables:} Variables kept constant to ensure a fair test.}

\subsubsection{3. Planning and Conducting Investigations: Designing a Fair Test}

To test your hypothesis, you need to design and conduct an \keyword{investigation} or experiment. A well-designed experiment is a \keyword{fair test}, meaning that only one variable (the independent variable) is changed at a time, while all other variables (controlled variables) are kept constant.  This ensures that any changes you observe in the dependent variable are actually due to the independent variable.

\marginnote{\textit{Fair Test:} An experiment where only one variable is changed at a time.}

\textbf{Planning an Investigation involves:}

\begin{itemize}
    \item \textbf{Identifying Variables:} Clearly define your independent, dependent, and controlled variables.
    \item \textbf{Materials and Equipment:** List all the materials and equipment you will need.
    \item \textbf{Procedure:** Write a detailed, step-by-step procedure that clearly outlines how you will conduct your experiment.  Your procedure should be clear enough for someone else to repeat your experiment exactly.
    \item \textbf{Control Group (Optional but often important):}  A \keyword{control group} is a standard of comparison. It is a group in your experiment that does not receive the independent variable. This allows you to see if the independent variable has an effect compared to a normal or baseline condition. In the plant example, plants grown in normal sunlight could be the control group, compared to the experimental group grown in shade.
\end{itemize}
\marginnote{\textit{Control Group:} A group in an experiment that does not receive the independent variable, used for comparison.}

\textbf{Conducting the Investigation involves:}

\begin{itemize}
    \item \textbf{Following the Procedure:** Carefully follow the procedure you have planned.
    \item \textbf{Making Observations:**  Make careful and detailed observations throughout your experiment. Record both qualitative and quantitative data.
    \item \textbf{Collecting Data:**  Collect data systematically. This might involve taking measurements, counting, or noting descriptions.  Organise your data in tables or charts as you collect it.
\end{itemize}

\subsubsection{4. Collecting and Analysing Data: Making Sense of Your Observations}

\keyword{Data} is the information you collect during your investigation. It can be \keyword{qualitative} (descriptions, observations using your senses) or \keyword{quantitative} (numerical measurements).

\marginnote{\textit{Data:} Information collected during an investigation.}
\marginnote{\textit{Qualitative Data:} Descriptive data, observations.}
\marginnote{\textit{Quantitative Data:} Numerical data, measurements.}

\textbf{Collecting Data:}

\begin{itemize}
    \item \textbf{Tables:**  Use tables to organise numerical data in rows and columns. Tables should have clear headings and units.
    \item \textbf{Measurements:**  Use appropriate measuring tools and record measurements accurately with correct units (e.g., centimetres, grams, seconds).
    \item \textbf{Observations:**  Write down detailed descriptions of what you observe during the experiment.
\end{itemize}

\textbf{Analysing Data:}

\begin{itemize}
    \item \textbf{Graphs:**  Graphs are visual representations of data that can help you see patterns and relationships. Common types of graphs include bar graphs, line graphs, and pie charts. Choose the type of graph that best suits your data.
    \marginnote{\textit{Graphs:} Visual representations of data to show patterns.}
    \item \textbf{Calculations (if applicable):}  Sometimes you need to perform calculations on your data, such as averages, percentages, or ratios, to help you analyse it. \mathlink{Mathematical skills are often essential in science!}
    \item \textbf{Identifying Trends and Patterns:**  Look for trends, patterns, or relationships in your data. Does your data support or contradict your hypothesis?
\end{itemize}

\subsubsection{5. Drawing Conclusions and Evaluating: What Does it All Mean?}

\keyword{Conclusions} are statements based on your data and analysis that summarise what you have found out from your investigation.  You need to determine whether your data supports or refutes your hypothesis.

\marginnote{\textit{Conclusions:} Statements based on data summarising findings.}

\textbf{Drawing Conclusions involves:}

\begin{itemize}
    \item \textbf{Relating to Hypothesis:**  State whether your data supports or refutes your hypothesis. Explain why based on your analysed data.
    \item \textbf{Summarising Findings:**  Summarise the key findings of your investigation. What did you learn?
    \item \textbf{Explaining Results:**  Try to explain your results based on scientific knowledge.
\end{itemize}

\textbf{Evaluating the Investigation involves considering:}

\begin{itemize}
    \item \textbf{Validity:** \keyword{Validity} refers to whether your experiment actually tested what you intended to test. Was your experimental design appropriate for your question and hypothesis? Were controlled variables effectively controlled?
    \marginnote{\textit{Validity:}  Whether the experiment tested what it was supposed to test.}
    \item \textbf{Reliability:** \keyword{Reliability} refers to the consistency of your results. If you repeated the experiment, would you get similar results?  Repeating experiments and getting consistent results increases reliability.
    \marginnote{\textit{Reliability:}  Consistency of results if the experiment is repeated.}
    \item \textbf{Limitations and Errors:**  Identify any limitations or potential sources of error in your experiment. Were there any factors that you couldn't control? Could there be errors in your measurements or procedure?  Understanding limitations helps improve future investigations.
\end{itemize}

\subsubsection{6. Communicating Results: Sharing Your Discoveries}

Science is a collaborative process.  Communicating your findings is a crucial part of the scientific method.  Scientists share their work through scientific reports, presentations, and publications.

\marginnote{\textit{Communication:} Sharing scientific findings with others.}

\textbf{Communicating Results involves:}

\begin{itemize}
    \item \textbf{Scientific Reports:**  Writing a clear and structured report that includes your question, hypothesis, procedure, data, analysis, conclusions, and evaluation.
    \item \textbf{Presentations:**  Presenting your findings to others, either orally or using posters. This allows you to share your work and get feedback.
    \item \textbf{Discussions:**  Discussing your findings with classmates, teachers, or other scientists.  This can lead to new ideas and further investigations.
\end{itemize}

The scientific method is not always a linear process. Sometimes, you might need to go back and revise your hypothesis, redesign your experiment, or collect more data based on your initial results.  It's an iterative process of learning and refining our understanding.

\begin{stopandthink}
Think about a simple experiment you have done before. Can you identify the steps of the scientific method in that experiment?
\end{stopandthink}

\begin{investigation}{Designing a Plant Growth Experiment}
\textbf{Question:}  Does the amount of water affect the growth of bean plants?

\textbf{Task:}  Work in small groups to design an experiment to investigate this question using the scientific method.

\textbf{Procedure:}
\begin{enumerate}
    \item \textbf{Formulate a Hypothesis:** Write an "if...then..." hypothesis about how the amount of water might affect bean plant growth.
    \item \textbf{Identify Variables:**  Identify the independent variable, dependent variable, and at least three controlled variables for your experiment.
    \item \textbf{Plan the Procedure:**  Outline the steps you would take to conduct your experiment.  Consider:
        \begin{itemize}
            \item What materials will you need (bean seeds, pots, soil, watering cans, measuring tools, etc.)?
            \item How many plants will you use in each group?
            \item How will you vary the amount of water? (e.g., different volumes of water given daily).
            \item How often will you water the plants?
            \item How long will you run the experiment for?
            \item How will you measure plant growth? (e.g., height, number of leaves).
            \item Will you have a control group? If so, what will it be?
        \end{itemize}
    \item \textbf{Data Collection Plan:**  How will you record your data? Design a table to collect your measurements.
    \item \textbf{Safety Considerations:**  What safety precautions do you need to consider for this experiment?
    \item \textbf{Present Your Plan:**  Present your experimental plan to the class. Be prepared to explain your hypothesis, variables, procedure, and data collection methods. Discuss and refine your plans based on feedback. (You may actually conduct this experiment later if time permits).
\end{enumerate}

\textbf{Analysis:}  Discuss the strengths and weaknesses of different experimental designs presented by the groups. What are some potential challenges in conducting this experiment?
\end{investigation}

\begin{tieredquestions}{The Scientific Method}
\begin{enumerate}
    \item \textbf{Basic:} List the six main steps of the scientific method in order.
    \item \textbf{Intermediate:} Explain the difference between a hypothesis and a conclusion. Give an example of each related to an experiment about testing different types of fertiliser on plant growth.
    \item \textbf{Advanced:}  Critically evaluate a given experimental design for testing the effectiveness of a new cleaning product. Identify potential flaws in the design related to validity and reliability, and suggest improvements.
\end{enumerate}
\end{tieredquestions}

\section{Working Scientifically Skills: Putting it into Practice}

\keyword{Working Scientifically} is not just about knowing scientific facts; it’s about developing a set of skills that scientists use to investigate the world. These skills are essential for conducting scientific inquiry effectively. In Stage 4 Science, you will start developing foundational Working Scientifically skills that will build upon throughout your science education.

\marginnote{\textit{Working Scientifically:} The skills and processes scientists use to investigate and understand the world.}

\begin{keyconcept}{Working Scientifically Skills}
Working Scientifically skills are practical abilities and thinking processes used in scientific inquiry. They include questioning, predicting, planning and conducting investigations, processing and analysing data, and evaluating and communicating.
\end{keyconcept}

For Stage 4, we will focus on some key introductory skills:

\subsection{Questioning and Predicting}

This skill involves asking relevant scientific questions and making predictions based on existing knowledge.

\subsubsection{Formulating Testable Questions}

Good scientific questions are \textbf{testable}. This means that they can be investigated through experiments or observations. They are also usually focused and specific.

\begin{itemize}
    \item \textbf{Start with Observations:** Questions often arise from observations about the world around you.
    \item \textbf{Focus on "How," "What," or "Why":}  Scientific questions often start with these words to explore relationships and explanations.
    \item \textbf{Consider Variables:** Think about the variables that might be involved in your question.
    \item \textbf{Can it be Tested?**  The most important aspect – can you design an experiment or investigation to answer your question? If not, it might not be a testable scientific question (yet).
\end{itemize}

\begin{example}
\textbf{Not Testable Question:** Is blue a prettier colour than green? (This is subjective and based on opinion, not scientific evidence).

\textbf{Testable Question:**  Do people react faster to the colour blue than the colour green? (This can be tested by measuring reaction times).
\end{example}

\subsubsection{Making Predictions}

A \keyword{prediction} is a statement about what you think will happen in an investigation, based on your prior knowledge or understanding.  It’s closely related to a hypothesis, but often more general. Predictions should be logical and based on some reasoning.

\marginnote{\textit{Prediction:} A statement about what you expect to happen in an investigation, based on prior knowledge.}

\begin{itemize}
    \item \textbf{Based on Prior Knowledge:** Use what you already know about the topic to make an educated guess.
    \item \textbf{Logical Reasoning:** Explain your prediction. Why do you think this will happen?
    \item \textbf{Link to the Question:** Your prediction should directly relate to the question you are investigating.
\end{itemize}

\begin{example}
\textbf{Question:} Does salt affect the boiling point of water?

\textbf{Prediction:}  I predict that adding salt to water will increase its boiling point because salt is a solute, and solutes generally raise the boiling point of solvents.
\end{example}

\begin{stopandthink}
Think of a simple everyday observation. Formulate a testable scientific question based on that observation, and then make a prediction about the answer to your question.
\end{stopandthink}

\subsection{Planning Investigations}

This skill involves designing a logical and safe procedure to test your hypothesis or answer your question.

\subsubsection{Identifying and Controlling Variables}

As we discussed earlier, understanding variables is crucial for planning fair tests. You need to be able to:

\begin{itemize}
    \item \textbf{Identify the Independent Variable:** What will you change?
    \item \textbf{Identify the Dependent Variable:** What will you measure or observe?
    \item \textbf{Identify Controlled Variables:** What factors need to be kept constant to ensure a fair test? List as many as possible that are relevant to your investigation.
\end{itemize}

\subsubsection{Selecting and Using Equipment}

Choosing the right equipment is important for accurate and safe investigations. You need to be able to:

\begin{itemize}
    \item \textbf{Select Appropriate Equipment:** Choose tools that are suitable for measuring the variables in your experiment (e.g., rulers, thermometers, beakers, timers).
    \item \textbf{Use Equipment Correctly and Safely:** Know how to use each piece of equipment properly and safely. Ask your teacher for guidance if you are unsure.
    \item \textbf{Consider Accuracy and Precision:** Think about the level of accuracy and precision you need for your measurements. Choose equipment that provides the required level of detail.
\end{itemize}

\subsubsection{Developing Procedures}

A well-written procedure is like a recipe for your experiment. It should be clear, detailed, and step-by-step, so anyone can follow it and repeat your investigation.

\begin{itemize}
    \item \textbf{Step-by-Step Instructions:** Write down each step of your experiment in a numbered list.
    \item \textbf{Be Specific:** Include details about quantities, measurements, timings, and any specific actions you need to take.
    \item \textbf{Logical Order:** Arrange the steps in a logical sequence.
    \item \textbf{Safety Instructions:** Include any necessary safety precautions within your procedure.
\end{itemize}

\subsection{Conducting Investigations}

This skill involves carrying out your planned procedure, making observations, and collecting data systematically and safely.

\subsubsection{Following Procedures}

Carefully follow the procedure you have planned. Deviations from the procedure can affect your results and make your experiment unfair.

\subsubsection{Making Observations and Collecting Data}

\begin{itemize}
    \item \textbf{Make Detailed Observations:** Observe carefully and record all relevant observations, both qualitative and quantitative.
    \item \textbf{Collect Data Systematically:** Collect data in an organised way, using tables, charts, or notebooks. Record data as you go, don't rely on memory.
    \item \textbf{Use Units:** Always include units when recording quantitative data (e.g., cm, °C, s).
    \item \textbf{Repeat Measurements (if appropriate):}  Repeating measurements and taking averages can improve the reliability of your data.
\end{itemize}

\begin{investigation}{Investigating Paper Airplane Flight}
\textbf{Question:} Does the design of a paper airplane affect the distance it flies?

\textbf{Task:}  In groups, investigate how different paper airplane designs affect flight distance, focusing on Working Scientifically skills.

\textbf{Procedure:}
\begin{enumerate}
    \item \textbf{Question and Predict:** As a group, choose 2-3 different paper airplane designs (e.g., dart, glider, bulldog).  Predict which design you think will fly the furthest and explain your reasoning.
    \item \textbf{Plan the Investigation:}
        \begin{itemize}
            \item \textbf{Variables:** Identify the independent variable (paper airplane design), dependent variable (flight distance), and controlled variables (e.g., launching force, starting point, paper type, wind conditions).
            \item \textbf{Materials:** Gather materials (paper, measuring tape, markers).
            \item \textbf{Procedure:** Write a detailed procedure for making each airplane design, launching them consistently, and measuring the flight distance.  Decide how many times you will test each design (repetitions).
            \item \textbf{Safety:** Consider safety – ensure a clear launch area.
        \end{itemize}
    \item \textbf{Conduct the Investigation:}
        \begin{itemize}
            \item Build each paper airplane design carefully.
            \item Follow your procedure to launch each airplane multiple times, trying to launch with consistent force each time.
            \item Measure and record the flight distance for each flight in a table.
        \end{itemize}
    \item \textbf{Analyse and Conclude (in the next chapter):} You will analyse the data and draw conclusions about which airplane design flies furthest in the next chapter.
\end{enumerate}

\textbf{Analysis:}  Discuss as a class: What were the challenges in planning and conducting this investigation? How did you try to control variables? What could be done to improve the investigation?
\end{investigation}

\begin{tieredquestions}{Working Scientifically Skills}
\begin{enumerate}
    \item \textbf{Basic:}  Define the term "controlled variable" and explain why it is important in a fair test.
    \item \textbf{Intermediate:}  Describe the steps involved in planning a scientific investigation. Explain why a detailed procedure is necessary.
    \item \textbf{Advanced:}  Imagine you are designing an experiment to test how different colours of light affect the rate of photosynthesis in pondweed. Identify the independent, dependent, and at least five controlled variables. Outline a procedure for this investigation, including equipment and safety considerations.
\end{enumerate}
\end{tieredquestions}

\section{Chapter Summary}

In this chapter, you have been introduced to the exciting world of scientific inquiry! You have learned that science is a process of asking questions and finding answers through systematic investigation. We covered the crucial importance of laboratory safety, emphasizing the rules and procedures to ensure a safe learning environment. You were introduced to the scientific method, a structured approach to scientific inquiry, involving observation, questioning, hypothesis formation, experimentation, data analysis, and conclusion.  Finally, we began exploring introductory Working Scientifically skills, focusing on questioning, predicting, and planning investigations.

These foundational concepts and skills will be essential as you continue your science journey. In the chapters to come, you will build upon this knowledge and further develop your scientific inquiry skills to explore even more fascinating scientific phenomena!
```