\chapter{Introduction to Scientific Inquiry}

Science plays a crucial role in helping us understand the world around us. From exploring distant galaxies to investigating microscopic organisms, scientists use systematic inquiry to uncover new knowledge and solve problems. In this chapter, you will learn the fundamental skills of scientific inquiry, including laboratory safety, the scientific method, and essential skills for planning and conducting investigations. These skills form the foundation for all future scientific explorations you will undertake.

\FloatBarrier
% Removed undefined command

Science activities often involve using specialised equipment and substances that can pose risks if handled incorrectly. Understanding laboratory safety ensures that we can explore science safely and confidently.

\subsection{General Laboratory Rules}

Before conducting any investigation, make sure you follow these basic safety rules:

\begin{itemize}
    \item Listen to and follow the teacher's instructions carefully.
    \item Always wear appropriate protective equipment, such as safety goggles, gloves, and lab coats.
    \item Never eat, drink, or taste anything in the laboratory.
    \item Tie back long hair and secure loose clothing.
    \item Inform your teacher immediately if accidents or spills occur.
    \item Clean your workspace thoroughly after completing experiments.
\end{itemize}

\begin{stopandthink}
Why do you think it is important not to eat or drink in a science laboratory, even if you are not directly handling chemicals?
\end{stopandthink}

\marginnote{\historylink{\textbf{Historical Note:} Early scientists often conducted experiments without adequate safety measures, leading to injuries and illnesses. Modern safety rules evolved from these early mistakes.}}

\subsection{Using Chemicals Safely}

Chemicals are common in science laboratories. Safe handling of chemicals requires:

\begin{itemize}
    \item Always reading labels carefully and following instructions accurately.
    \item Never directly smelling chemical substances. Instead, gently waft the odour towards your nose if required.
    \item Never returning unused chemicals to their original containers.
    \item Disposing chemicals according to your teacher's instructions.
\end{itemize}

\begin{keyconcept}{Safety Data Sheets (SDS)}
Laboratories use Safety Data Sheets (SDS) to provide detailed information about chemicals, including hazards, handling, storage, and emergency procedures. Always consult the SDS if unsure about chemical properties.
\end{keyconcept}

\subsection{Safety Equipment}

Familiarise yourself with common laboratory safety equipment, including:

\begin{itemize}
    \item \textbf{Safety goggles:} Protect eyes from splashes.
    \item \textbf{Lab coats and aprons:} Protect clothing and skin.
    \item \textbf{Gloves:} Protect hands from hazardous materials.
    \item \textbf{Fire extinguisher and fire blanket:} Used to extinguish fires.
    \item \textbf{Emergency shower and eye-wash station:} Used to rinse chemicals off the body.
\end{itemize}

\begin{tieredquestions}{Basic}
\begin{enumerate}
    \item List three general laboratory safety rules.
    \item Name two pieces of laboratory safety equipment and their purposes.
\end{enumerate}
\end{tieredquestions}

\begin{tieredquestions}{Intermediate}
\begin{enumerate}
    \item Explain why chemicals should never be returned to their original containers after use.
    \item What should you do if you spill a chemical during an experiment?
\end{enumerate}
\end{tieredquestions}

\begin{tieredquestions}{Advanced}
\begin{enumerate}
    \item Describe how laboratory safety has evolved over time, providing examples.
    \item Explain the importance of Safety Data Sheets (SDS) when working with chemicals.
\end{enumerate}
\end{tieredquestions}

\FloatBarrier
% Removed undefined command

Scientific inquiry involves a systematic approach known as the \keyword{scientific method}. This method helps scientists investigate questions, solve problems, and communicate their findings clearly.

\subsection{Steps of the Scientific Method}

The scientific method typically includes the following steps:

\begin{enumerate}
    \item \textbf{Observation:} Making careful observations to identify a question or problem.
    \item \textbf{Questioning:} Formulating clear and concise scientific questions.
    \item \textbf{Hypothesis:} Predicting the answer or explanation to your question.
    \item \textbf{Experimentation:} Designing and performing controlled experiments to test your hypothesis.
    \item \textbf{Data Collection:} Gathering and recording observations and measurements.
    \item \textbf{Analysis:} Interpreting results to determine if they support your hypothesis.
    \item \textbf{Conclusion:} Summarising your findings and communicating your results.
\end{enumerate}

\marginnote{\historylink{\textbf{Historical Context:} The scientific method was formalised by scientists such as Galileo Galilei and Francis Bacon, who emphasised experimentation and evidence-based inquiry.}}

\subsection{Formulating Hypotheses}

A \keyword{hypothesis} is a clear, testable prediction about the outcome of an investigation. It typically follows an "if–then" structure:

\begin{example}
\textbf{Scientific Question:} Does fertiliser help plants grow faster?

\textbf{Hypothesis:} If plants are grown with fertiliser, then they will grow faster than plants grown without fertiliser.
\end{example}

\begin{stopandthink}
Can you create a hypothesis for the question "Does sunlight affect the growth of mould on bread?"
\end{stopandthink}

\FloatBarrier
% Removed undefined command

Scientists design investigations carefully to ensure that the results are reliable and accurate.

\subsection{Variables in Experiments}

Experiments involve different types of variables:

\begin{itemize}
    \item \keyword{Independent variable:} The variable you change intentionally.
    \item \keyword{Dependent variable:} The variable you measure or observe. It changes in response to the independent variable.
    \item \keyword{Controlled variables:} Variables that remain constant to ensure a fair test.
\end{itemize}

\begin{example}
In an experiment testing fertiliser effects on plant growth:

\begin{itemize}
    \item Independent variable: Amount of fertiliser used.
    \item Dependent variable: Plant height.
    \item Controlled variables: Amount of water, sunlight, type of plant.
\end{itemize}
\end{example}

\begin{stopandthink}
Identify the independent, dependent, and controlled variables for an experiment investigating how temperature affects the rate of ice melting.
\end{stopandthink}

\begin{investigation}{Design Your Own Experiment}
Design a simple experiment to test the question "Which type of paper towel absorbs the most water?" Identify your hypothesis, independent variable, dependent variable, and controlled variables. Conduct the experiment and record your data clearly.
\end{investigation}

\FloatBarrier
% Removed undefined command

Accurate observations and measurements are essential in scientific investigations.

\subsection{Types of Observations}

Scientists make two types of observations:

\begin{itemize}
    \item \keyword{Qualitative observations:} Descriptive observations, such as colour, texture, or smell.
    \item \keyword{Quantitative observations:} Numerical measurements, such as mass, length, or temperature.
\end{itemize}

\subsection{Recording Data}

Data should be organised clearly, often in tables, graphs, or charts, to help identify patterns and relationships.

\begin{keyconcept}{Tables and Graphs}
Tables organise data clearly into rows and columns. Graphs visually represent data, making it easier to understand results and trends.
\end{keyconcept}

\begin{investigation}{Making Accurate Observations}
Observe and measure different objects in your classroom using qualitative and quantitative methods. Record your observations in a clear table.
\end{investigation}

\begin{tieredquestions}{Basic}
\begin{enumerate}
    \item Define the terms independent variable and dependent variable.
    \item What is the difference between qualitative and quantitative observations?
\end{enumerate}
\end{tieredquestions}

\begin{tieredquestions}{Intermediate}
\begin{enumerate}
    \item Design a simple experiment to test how the size of a parachute affects its falling speed. List your independent, dependent, and controlled variables.
\end{enumerate}
\end{tieredquestions}

\begin{tieredquestions}{Advanced}
\begin{enumerate}
    \item Explain why it is important to control variables in an experiment, using examples to support your answer.
    \item Discuss the strengths and limitations of qualitative and quantitative data.
\end{enumerate}
\end{tieredquestions}

\marginnote{\challenge{\textbf{Extension:} Explore how mathematical concepts like averages and percentages can help scientists analyse experimental data.}}

Through mastering these essential skills, you are now prepared to undertake scientific investigations confidently and effectively.
\FloatBarrier
