```latex
\chapter{Mixtures and Separation Techniques}

\marginnote{This chapter will introduce you to the world of mixtures and how we can separate them. Get ready to explore!}

\section{What are Mixtures?}

Imagine you are making a delicious fruit salad. You combine different fruits like apples, bananas, and grapes. Each fruit retains its own properties, even when mixed together.  This fruit salad is a perfect example of a \keyword{mixture}.

\begin{keyconcept}{What is a Mixture?}
A \keyword{mixture} is a combination of two or more substances that are physically combined but not chemically joined. Each substance in a mixture keeps its own chemical properties. Mixtures can be separated by physical means.
\end{keyconcept}

In a mixture, the different substances are simply mingled together.  Think about sand and water. You can see both sand and water, and each still behaves as it normally would.  The sand doesn't suddenly dissolve or change into water, and the water doesn't become gritty.

\marginnote{\textit{Physically combined} means the substances are just mixed together, not bonded chemically like in compounds.}
\marginnote{Think of salad dressing – oil and vinegar are mixed but not chemically bonded.}

\begin{stopandthink}
Can you think of three more examples of mixtures you encounter in your daily life?
\end{stopandthink}

\subsection{Pure Substances vs. Mixtures}

To understand mixtures better, it's helpful to compare them to \keyword{pure substances}. A \keyword{pure substance} is made up of only one type of particle.  These particles can be atoms, like in pure gold (\ce{Au}), or molecules, like in pure water (\ce{H2O}).

\begin{keyconcept}{Pure Substances}
A \keyword{pure substance} is a substance that is made up of only one type of particle. Pure substances have fixed compositions and properties. They can be elements or compounds.
\end{keyconcept}

\begin{example}
\textbf{Pure Substances:}
\begin{itemize}
    \item \keyword{Elements}: Gold (\ce{Au}), Oxygen (\ce{O2}), Nitrogen (\ce{N2})
    \item \keyword{Compounds}: Water (\ce{H2O}), Salt (Sodium Chloride, \ce{NaCl}), Sugar (Sucrose, \ce{C12H22O11})
\end{itemize}
\textbf{Mixtures:}
\begin{itemize}
    \item Air (mixture of nitrogen, oxygen, and other gases)
    \item Seawater (mixture of water, salt, and other minerals)
    \item Soil (mixture of minerals, organic matter, air, and water)
\end{itemize}
\end{example}

\marginnote{Elements are the simplest pure substances. Compounds are made of two or more elements chemically combined.}
\marginnote{Distilled water is a purer substance than tap water, which contains dissolved minerals.}

Pure substances have fixed properties, such as melting and boiling points. For example, pure water always boils at 100°C at standard atmospheric pressure. Mixtures, on the other hand, do not have fixed properties. Their properties can vary depending on the amount of each substance present in the mixture.  Think about salt water – the saltier it is, the lower its freezing point and the higher its boiling point compared to pure water.

\begin{stopandthink}
How is a compound different from a mixture? Consider their chemical nature and how they are formed.
\end{stopandthink}

\begin{tieredquestions}{Basic}
\begin{enumerate}
    \item Define the term 'mixture'.
    \item Give two examples of mixtures and two examples of pure substances.
    \item Explain the difference between a pure substance and a mixture in terms of their composition.
\end{enumerate}
\end{tieredquestions}

\begin{tieredquestions}{Intermediate}
\begin{enumerate}
    \item Explain why air is considered a mixture and not a pure substance.
    \item Describe how the properties of a mixture can change compared to the properties of the pure substances it contains. Give an example.
    \item  Is sugar water a pure substance or a mixture? Justify your answer.
\end{enumerate}
\end{tieredquestions}

\begin{tieredquestions}{Advanced}
\begin{enumerate}
    \item  "All matter is either a pure substance or a mixture."  Discuss this statement, explaining any exceptions or edge cases you can think of.
    \item  Imagine you have a sample of something. How could you experimentally determine if it is a pure substance or a mixture?  Think about properties you could measure.
    \item  Research and describe a naturally occurring substance that is very close to being a pure substance but still contains trace impurities.
\end{enumerate}
\end{tieredquestions}


\section{Types of Mixtures}

Mixtures can be broadly classified into two main types based on how well the substances are mixed together: \keyword{homogeneous mixtures} and \keyword{heterogeneous mixtures}.

\subsection{Homogeneous Mixtures (Solutions)}

\marginnote{Homo- means 'same' throughout.}

\begin{keyconcept}{Homogeneous Mixtures}
A \keyword{homogeneous mixture} is a mixture where the components are uniformly distributed throughout the mixture. This means the mixture looks the same throughout and you cannot easily see the different components. Homogeneous mixtures are also called \keyword{solutions}.
\end{keyconcept}

In a homogeneous mixture, the different substances are so evenly mixed that you cannot distinguish them with the naked eye, or even with a simple microscope.  One substance, called the \keyword{solute}, dissolves into another substance, called the \keyword{solvent}. The solute is usually present in a smaller amount, and the solvent in a larger amount.

\begin{example}
\textbf{Examples of Homogeneous Mixtures (Solutions):}
\begin{itemize}
    \item \keyword{Sugar dissolved in water}: Sugar is the solute, and water is the solvent. The sugar particles are evenly distributed throughout the water.
    \item \keyword{Salt water}: Salt (\ce{NaCl}) is the solute, and water is the solvent.
    \item \keyword{Air}: A mixture of gases like nitrogen (\ce{N2}), oxygen (\ce{O2}), carbon dioxide (\ce{CO2}), and others. Gases mix homogeneously.
    \item \keyword{Brass}: A solid solution of copper and zinc.
\end{itemize}
\end{example}

\marginnote{The word 'solution' is often used for liquid homogeneous mixtures but can also apply to gases and solids.}
\marginnote{Brass is an example of a solid solution, also called an alloy.}

When a solid solute dissolves in a liquid solvent, the particles of the solute spread out and fit in between the particles of the solvent. This process is called \keyword{dissolving}.  The ability of a solute to dissolve in a solvent is called \keyword{solubility}.

\begin{stopandthink}
Think about making tea or coffee. Are these homogeneous or heterogeneous mixtures? Explain your reasoning.
\end{stopandthink}

\subsection{Heterogeneous Mixtures}

\marginnote{Hetero- means 'different' throughout.}

\begin{keyconcept}{Heterogeneous Mixtures}
A \keyword{heterogeneous mixture} is a mixture where the components are not uniformly distributed throughout. You can usually see the different components, and the mixture is not the same throughout.
\end{keyconcept}

In a heterogeneous mixture, the different substances are unevenly mixed. You can often see the different parts that make up the mixture.

\begin{example}
\textbf{Examples of Heterogeneous Mixtures:}
\begin{itemize}
    \item \keyword{Sand and water}: You can clearly see the sand and the water as separate components.
    \item \keyword{Oil and water}: Oil and water do not mix and form distinct layers.
    \item \keyword{Salad dressing (like Italian dressing)}: You can see oil, vinegar, herbs, and spices that are not evenly mixed.
    \item \keyword{Concrete}: A mixture of cement, sand, gravel, and water. You can see the different components.
\end{itemize}
\end{example}

\marginnote{Suspensions and colloids are types of heterogeneous mixtures with specific properties.}
\marginnote{Milk is a colloid – it looks homogeneous to the naked eye but is actually heterogeneous under a microscope.}

Some heterogeneous mixtures are suspensions, where solid particles are dispersed in a liquid but will settle out over time (like muddy water). Others are colloids, where the particles are larger than in solutions but do not settle out easily (like milk or fog).

\begin{stopandthink}
Classify each of the following as homogeneous or heterogeneous mixtures and explain your choice:
\begin{itemize}
    \item Orange juice with pulp
    \item Apple juice (clear)
    \item Steel
    \item Pizza
\end{itemize}
\end{stopandthink}

\begin{tieredquestions}{Basic}
\begin{enumerate}
    \item Define the terms 'homogeneous mixture' and 'heterogeneous mixture'.
    \item Give two examples of each type of mixture.
    \item What is another name for a homogeneous mixture?
\end{enumerate}
\end{tieredquestions}

\begin{tieredquestions}{Intermediate}
\begin{enumerate}
    \item Explain the difference between a solution and a suspension.
    \item Describe how you could visually distinguish between a homogeneous mixture and a heterogeneous mixture.
    \item  Think about a glass of iced tea with ice cubes. Is this a homogeneous or heterogeneous mixture? Explain your answer.
\end{enumerate}
\end{tieredquestions}

\begin{tieredquestions}{Advanced}
\begin{enumerate}
    \item  Colloids are sometimes described as being 'in between' homogeneous and heterogeneous mixtures. Explain what this means, considering particle size and properties.
    \item  Design an experiment to determine whether a sample of milk is a homogeneous or heterogeneous mixture without using a microscope. (Hint: Think about the properties of light.)
    \item  Research and describe the Tyndall effect. How does it help distinguish between solutions, colloids, and suspensions?
\end{enumerate}
\end{tieredquestions}


\section{Why Separate Mixtures?}

\marginnote{Separation is essential in chemistry and many industries.}

Often, we need to separate the different components of a mixture.  Why is this important? There are many reasons, and separation techniques are used in everyday life, in science laboratories, and in large industries.

\begin{keyconcept}{Importance of Separation}
Separating mixtures allows us to:
\begin{itemize}
    \item \textbf{Obtain pure substances}:  For scientific research, manufacturing, and everyday use, we often need pure substances.
    \item \textbf{Remove unwanted substances}:  To purify water, clean up pollutants, or remove impurities from materials.
    \item \textbf{Identify components}:  To analyse mixtures and understand their composition.
    \item \textbf{Recycle materials}: To separate valuable materials from waste for reuse.
\end{itemize}
\end{keyconcept}

\subsection{Real-World Applications of Separation}

\begin{itemize}
    \item \textbf{Water Purification}:  Drinking water is purified to remove harmful substances like bacteria, viruses, and dissolved minerals. Separation techniques like filtration and distillation are crucial in water treatment plants.
    \item \textbf{Mining and Resource Extraction}:  Mining involves extracting valuable minerals from ores, which are mixtures of different substances. Separation techniques are used to isolate the desired minerals. For example, gold is often separated from rock and sand using various methods.
    \item \textbf{Food and Beverage Industry}:  Separation techniques are used to produce many food and beverage products. For example, coffee is made by separating soluble compounds from ground coffee beans using hot water. Sugar is extracted and purified from sugar cane or sugar beets.
    \item \textbf{Pharmaceutical Industry}:  In making medicines, it is crucial to separate and purify the active ingredients from other substances. Chromatography is a vital technique in this industry.
    \item \textbf{Environmental Science}:  Separation techniques are used to analyse and clean up environmental pollutants. For example, oil spills can be cleaned up using separation methods.
\end{itemize}

\begin{historylink}{Early Distillation}
Distillation, one of the oldest separation techniques, was used by early alchemists in their attempts to transform substances and create medicines.  Ancient civilisations also used evaporation to obtain salt from seawater.
\end{historylink}

\begin{stopandthink}
Think about a specific industry (e.g., recycling, oil refining, making perfumes). How are separation techniques used in that industry?
\end{stopandthink}

\begin{tieredquestions}{Basic}
\begin{enumerate}
    \item Give three reasons why it is important to separate mixtures.
    \item Name one industry where separation techniques are essential.
    \item What is water purification, and why is it necessary?
\end{enumerate}
\end{tieredquestions}

\begin{tieredquestions}{Intermediate}
\begin{enumerate}
    \item Explain how separation techniques are used in the mining industry.
    \item Describe how the food and beverage industry uses separation techniques. Give a specific example.
    \item  Why is obtaining pure substances important in the pharmaceutical industry?
\end{enumerate}
\end{tieredquestions}

\begin{tieredquestions}{Advanced}
\begin{enumerate}
    \item  Discuss the environmental importance of separation techniques in areas like pollution control and resource recovery (recycling).
    \item  Research and describe a specific real-world example of a complex separation process used in industry.
    \item  Consider the ethical implications of using separation techniques, for example, in resource extraction or waste management.
\end{enumerate}
\end{tieredquestions}


\section{Separation Techniques}

Mixtures can be separated using physical methods that exploit the different physical properties of the substances in the mixture. The choice of separation technique depends on the type of mixture and the properties of its components, such as particle size, solubility, boiling point, and density. Let's explore some common separation techniques.

\marginnote{Remember, we use \textit{physical} methods to separate mixtures because the components are not chemically bonded.}

\subsection{Filtration}

\begin{keyconcept}{Filtration}
\keyword{Filtration} is a technique used to separate insoluble solids from liquids or gases. It works by passing the mixture through a filter medium (like filter paper) that allows the liquid or gas to pass through but traps the solid particles.
\end{keyconcept}

Filtration is effective for separating heterogeneous mixtures where there is a solid that is not dissolved in a liquid. The filter paper has tiny pores (holes) that are large enough for the liquid particles to pass through but too small for the solid particles.

\begin{figure}
\centering
\includegraphics[width=0.5\textwidth]{filtration_setup.pdf}
\caption{Filtration Setup: A mixture of sand and water is poured through filter paper in a funnel. The water passes through as filtrate, and the sand remains on the filter paper as residue.}
\end{figure}

\textbf{How Filtration Works:}
\begin{enumerate}
    \item The mixture is poured into a funnel lined with filter paper.
    \item The liquid passes through the tiny pores in the filter paper due to gravity. This liquid that passes through is called the \keyword{filtrate}.
    \item The solid particles are too large to pass through the pores and are trapped on the filter paper. This solid left behind is called the \keyword{residue}.
\end{enumerate}

\begin{example}
\textbf{Examples of Filtration:}
\begin{itemize}
    \item \keyword{Making coffee}: Coffee grounds are separated from brewed coffee using a filter.
    \item \keyword{Water purification}: Removing sand and other solid impurities from water.
    \item \keyword{Air filtration}: Air filters in cars and homes remove dust and pollen particles from the air.
\end{itemize}
\end{example}

\begin{investigation}{Investigating Filtration}
\textbf{Title:} Separating Sand and Water by Filtration

\textbf{Materials:}
\begin{itemize}
    \item Beaker of sand and water mixture
    \item Filter paper
    \item Funnel
    \item Beaker to collect filtrate
    \item Stirring rod
\end{itemize}

\textbf{Procedure:}
\begin{enumerate}
    \item Fold the filter paper into a cone shape and place it in the funnel.
    \item Place the funnel over the empty beaker.
    \item Gently stir the sand and water mixture.
    \item Slowly pour the mixture into the filter paper in the funnel, ensuring the liquid level stays below the top of the filter paper.
    \item Observe what happens as the mixture passes through the filter paper.
    \item Once all the liquid has passed through, examine the filter paper and the filtrate in the beaker.
\end{enumerate}

\textbf{Observations and Questions:}
\begin{itemize}
    \item What do you observe on the filter paper?
    \item What does the filtrate look like? Is it clear?
    \item What type of mixture is sand and water?
    \item Why is filtration an effective method for separating sand and water?
\end{itemize}

\textbf{Safety:}
\begin{itemize}
    \item Handle glassware carefully to avoid breakage.
    \item Clean up any spills immediately.
\end{itemize}
\end{investigation}

\begin{stopandthink}
Could you use filtration to separate salt from saltwater? Why or why not?
\end{stopandthink}

\begin{tieredquestions}{Basic}
\begin{enumerate}
    \item What is filtration used to separate?
    \item What is the filter medium in filtration, and how does it work?
    \item Give an example of where filtration is used in everyday life.
\end{enumerate}
\end{tieredquestions}

\begin{tieredquestions}{Intermediate}
\begin{enumerate}
    \item Explain why filtration is effective for separating sand and water but not for separating salt and water.
    \item Describe the terms 'filtrate' and 'residue' in the context of filtration.
    \item  Suggest a modification to the filtration setup that might speed up the filtration process.
\end{enumerate}
\end{tieredquestions}

\begin{tieredquestions}{Advanced}
\begin{enumerate}
    \item  Research and describe different types of filter media used in filtration, considering their pore size and applications.
    \item  Explain how vacuum filtration can improve the efficiency of filtration, especially for fine particles.
    \item  Consider a scenario where you need to filter a very large volume of liquid containing solid particles. Describe the industrial filtration process that might be used.
\end{enumerate}
\end{tieredquestions}


\subsection{Evaporation}

\begin{keyconcept}{Evaporation}
\keyword{Evaporation} is a technique used to separate a soluble solid from a liquid solution. It works by heating the solution to evaporate the liquid solvent, leaving the solid solute behind.
\end{keyconcept}

Evaporation relies on the fact that liquids evaporate when heated, turning into a gas (vapour).  The solid solute, which has a much higher boiling point, remains behind as the liquid evaporates.

\begin{figure}
\centering
\includegraphics[width=0.5\textwidth]{evaporation_setup.pdf}
\caption{Evaporation Setup: Saltwater is heated in an evaporating dish. The water evaporates, leaving the salt crystals behind.}
\end{figure}

\textbf{How Evaporation Works:}
\begin{enumerate}
    \item The solution is heated in an open container, such as an evaporating dish.
    \item The liquid solvent absorbs heat energy and evaporates, turning into vapour and escaping into the air.
    \item As the solvent evaporates, the concentration of the solute in the remaining solution increases.
    \item Eventually, all the solvent evaporates, leaving behind the solid solute as a residue in the container.
\end{enumerate}

\begin{example}
\textbf{Examples of Evaporation:}
\begin{itemize}
    \item \keyword{Obtaining salt from seawater}: Seawater is evaporated in salt pans, leaving salt crystals behind.
    \item \keyword{Making sugar from sugar cane juice}: Water is evaporated from sugar cane juice to concentrate the sugar, which then crystallises.
    \item \keyword{Drying clothes}: Water evaporates from wet clothes, leaving them dry.
\end{itemize}
\end{example}

\begin{investigation}{Investigating Evaporation}
\textbf{Title:} Separating Salt from Saltwater by Evaporation

\textbf{Materials:}
\begin{itemize}
    \item Beaker of saltwater solution
    \item Evaporating dish
    \item Bunsen burner or hot plate
    \item Tripod stand and gauze mat
    \item Stirring rod
    \item Heatproof gloves
\end{itemize}

\textbf{Procedure:}
\begin{enumerate}
    \item Pour the saltwater solution into the evaporating dish.
    \item Set up the Bunsen burner or hot plate with the tripod stand and gauze mat.
    \item Carefully place the evaporating dish on the gauze mat.
    \item Gently heat the saltwater solution, stirring occasionally with the stirring rod.
    \item Observe what happens as the solution is heated. Continue heating until all the water has evaporated and only a solid residue remains.
    \item Allow the evaporating dish to cool before handling it. Examine the residue.
\end{enumerate}

\textbf{Observations and Questions:}
\begin{itemize}
    \item What happens to the saltwater solution as it is heated?
    \item What is left behind in the evaporating dish after all the water has evaporated?
    \item What type of mixture is saltwater?
    \item Why is evaporation an effective method for separating salt from saltwater?
    \item What are some limitations of using evaporation to separate mixtures?
\end{itemize}

\textbf{Safety:}
\begin{itemize}
    \item Wear heatproof gloves when handling hot glassware.
    \item Be careful when using a Bunsen burner or hot plate.
    \item Do not look directly into the evaporating dish while heating.
\end{itemize}
\end{investigation}

\begin{stopandthink}
Could you use evaporation to separate sand from saltwater? Why or why not? If not, what technique would you use first?
\end{stopandthink}

\begin{tieredquestions}{Basic}
\begin{enumerate}
    \item What is evaporation used to separate?
    \item What happens to the liquid solvent during evaporation?
    \item Give an example of where evaporation is used in industry.
\end{enumerate}
\end{tieredquestions}

\begin{tieredquestions}{Intermediate}
\begin{enumerate}
    \item Explain why evaporation is effective for separating salt and water but not for separating a mixture of two liquids with different boiling points.
    \item Describe what happens to the concentration of the solute in a solution as evaporation progresses.
    \item  What are some advantages and disadvantages of using evaporation as a separation technique?
\end{enumerate}
\end{tieredquestions}

\begin{tieredquestions}{Advanced}
\begin{enumerate}
    \item  Explain how the rate of evaporation is affected by factors such as temperature, surface area, and humidity.
    \item  Describe the process of crystallisation that often occurs during evaporation when separating a solid from a solution.
    \item  Consider a situation where you want to collect and reuse the evaporated solvent. What modifications would you need to make to the evaporation setup? (Hint: Think about condensation.)
\end{enumerate}
\end{tieredquestions}


\subsection{Distillation}

\begin{keyconcept}{Distillation}
\keyword{Distillation} is a technique used to separate miscible liquids (liquids that mix together) with different boiling points. It involves heating the mixture to boil off the liquid with the lower boiling point, then condensing the vapour back into a liquid and collecting it.
\end{keyconcept}

Distillation is based on the principle that different liquids have different boiling points. When a mixture of liquids is heated, the liquid with the lowest boiling point will vaporise first. The vapour is then cooled and condensed back into a liquid, which is collected separately.

\marginnote{Miscible liquids mix to form a homogeneous solution, like ethanol and water.}
\marginnote{Distillation can also be used to separate a soluble solid from a liquid, but it's more complex than simple evaporation if you want to collect the pure solvent.}

There are different types of distillation, including simple distillation and fractional distillation.

\subsubsection{Simple Distillation}

Simple distillation is used when the boiling points of the liquids are significantly different (usually more than 25°C apart).

\begin{figure}
\centering
\includegraphics[width=0.6\textwidth]{simple_distillation_setup.pdf}
\caption{Simple Distillation Setup: A mixture of liquids is heated in a flask. The vapour of the liquid with the lower boiling point rises, passes through a condenser where it cools and condenses back into a liquid, and is collected in a receiving flask.}
\end{figure}

\textbf{How Simple Distillation Works:}
\begin{enumerate}
    \item The mixture of liquids is placed in a distillation flask.
    \item Heat is applied to the flask. The liquid with the lower boiling point starts to boil and vaporise.
    \item The vapour rises and enters a condenser, which is cooled by cold water circulating around it.
    \item In the condenser, the hot vapour cools down and condenses back into a liquid.
    \item The condensed liquid (called the \keyword{distillate}) is collected in a receiving flask.
    \item The liquid with the higher boiling point remains in the distillation flask.
\end{enumerate}

\begin{example}
\textbf{Examples of Simple Distillation:}
\begin{itemize}
    \item \keyword{Desalination of seawater}: Removing salt from seawater to produce fresh water (though more complex distillation methods are often used).
    \item \keyword{Purifying ethanol from a mixture of ethanol and water}: Ethanol has a lower boiling point than water.
\end{itemize}
\end{example}

\subsubsection{Fractional Distillation}

Fractional distillation is used when the boiling points of the liquids are close together (less than 25°C apart). It uses a fractionating column placed between the distillation flask and the condenser.

\begin{figure}
\centering
\includegraphics[width=0.7\textwidth]{fractional_distillation_setup.pdf}
\caption{Fractional Distillation Setup: Similar to simple distillation but with a fractionating column between the flask and condenser. This column helps to separate liquids with closer boiling points more effectively.}
\end{figure}

\textbf{How Fractional Distillation Works:}
\begin{enumerate}
    \item The mixture of liquids is placed in a distillation flask.
    \item Heat is applied. The mixture starts to boil, and vapours of all liquids present will rise into the fractionating column.
    \item The fractionating column is designed to be cooler at the top and hotter at the bottom.
    \item As the vapours rise through the column, they cool and condense. Liquids with higher boiling points condense lower down in the column and drip back into the flask.
    \item Only the vapour of the liquid with the lowest boiling point reaches the top of the column and enters the condenser.
    \item This vapour is then condensed and collected as the distillate. By carefully controlling the temperature, liquids with progressively higher boiling points can be collected separately.
\end{enumerate}

\begin{example}
\textbf{Examples of Fractional Distillation:}
\begin{itemize}
    \item \keyword{Separating crude oil into different fractions}: Crude oil is a complex mixture of hydrocarbons with different boiling points. Fractional distillation is used in oil refineries to separate it into petrol, diesel, kerosene, and other fractions.
    \item \keyword{Separating ethanol and water mixtures when the concentration of ethanol is high}: For mixtures where the boiling points are closer, fractional distillation provides a better separation than simple distillation.
\end{itemize}
\end{example}

\begin{investigation}{Investigating Simple Distillation}
\textbf{Title:} Separating Water and Coloured Ink by Simple Distillation

\textbf{Materials:}
\begin{itemize}
    \item Mixture of water and coloured ink
    \item Distillation flask
    \item Condenser
    \item Receiving flask
    \item Thermometer
    \item Rubber stoppers and tubing
    \item Bunsen burner or hot plate
    \item Tripod stand and gauze mat
    \item Beakers
    \item Heatproof gloves
\end{itemize}

\textbf{Procedure:}
\begin{enumerate}
    \item Set up the simple distillation apparatus as shown in the diagram (Figure: simple\_distillation\_setup.pdf - *You will need to imagine or sketch this setup*). Ensure all connections are airtight.
    \item Pour the water and ink mixture into the distillation flask, filling it about one-third full. Add a few anti-bumping granules to ensure smooth boiling.
    \item Place the thermometer so that its bulb is level with the side arm of the distillation flask.
    \item Gently heat the distillation flask using a Bunsen burner or hot plate. Observe the temperature reading on the thermometer.
    \item As the water starts to boil, water vapour will rise into the condenser. Cold water should be flowing through the condenser.
    \item Collect the distillate in the receiving flask. Continue distillation until you have collected a reasonable amount of distillate, or the temperature starts to rise significantly.
    \item Observe the distillate and the residue in the distillation flask.
\end{enumerate}

\textbf{Observations and Questions:}
\begin{itemize}
    \item What is the boiling point of water you observed during the experiment?
    \item What does the distillate look like? Is it coloured or colourless?
    \item What is left behind in the distillation flask?
    \item Why is distillation an effective method for separating water and ink (assuming the ink is non-volatile)?
    \item What are some safety precautions you need to take during distillation?
\end{itemize}

\textbf{Safety:}
\begin{itemize}
    \item Handle glassware carefully.
    \item Be careful when using a Bunsen burner or hot plate.
    \item Ensure the distillation apparatus is set up correctly and securely.
    \item Never heat a closed system in distillation.
\end{itemize}
\end{investigation}

\begin{stopandthink}
Why is it important to have cold water flowing through the condenser in distillation? What would happen if the condenser was not cooled?
\end{stopandthink}

\begin{tieredquestions}{Basic}
\begin{enumerate}
    \item What is distillation used to separate?
    \item What physical property is distillation based on?
    \item What is the purpose of the condenser in a distillation setup?
\end{enumerate}
\end{tieredquestions}

\begin{tieredquestions}{Intermediate}
\begin{enumerate}
    \item Explain the difference between simple distillation and fractional distillation. When would you use each technique?
    \item Describe the role of the fractionating column in fractional distillation.
    \item  Why is it important to control the temperature carefully during fractional distillation?
\end{enumerate}
\end{tieredquestions}

\begin{tieredquestions}{Advanced}
\begin{enumerate}
    \item  Explain how fractional distillation is used to separate crude oil into different fractions in oil refineries. Describe the properties of these fractions.
    \item  Research and describe vacuum distillation. Why is it used, and what are its advantages and disadvantages compared to simple distillation?
    \item  Consider a mixture of three miscible liquids with boiling points close to each other. Design a fractional distillation setup and procedure to separate these three liquids as effectively as possible.
\end{enumerate}
\end{tieredquestions}


\subsection{Chromatography}

\begin{keyconcept}{Chromatography}
\keyword{Chromatography} is a technique used to separate components of a mixture based on their different affinities (attraction) to a stationary phase and a mobile phase.  It is particularly useful for separating coloured substances (like dyes in inks or pigments in plants), but it can be used for colourless substances as well.
\end{keyconcept}

Chromatography works because different substances travel at different speeds when carried by a mobile phase through a stationary phase. The \keyword{stationary phase} is a substance that stays fixed in place (e.g., filter paper in paper chromatography). The \keyword{mobile phase} is a substance that moves through the stationary phase, carrying the components of the mixture with it (e.g., a solvent in paper chromatography).

\marginnote{Chroma- comes from the Greek word for colour because early chromatography was used to separate coloured plant pigments.}
\marginnote{There are many types of chromatography, including paper chromatography, thin-layer chromatography (TLC), and column chromatography.}

\subsubsection{Paper Chromatography}

Paper chromatography is a simple and widely used type of chromatography.

\begin{figure}
\centering
\includegraphics[width=0.5\textwidth]{paper_chromatography_setup.pdf}
\caption{Paper Chromatography Setup: A spot of ink is placed on filter paper, and the bottom of the paper is dipped into a solvent. As the solvent moves up the paper, it carries the components of the ink at different rates, separating them into different spots.}
\end{figure}

\textbf{How Paper Chromatography Works:}
\begin{enumerate}
    \item A small spot of the mixture to be separated is placed near the bottom of a piece of special filter paper (chromatography paper). This is called the \keyword{baseline}.
    \item The bottom edge of the paper is dipped into a solvent (the mobile phase), making sure the spot of mixture is above the solvent level.
    \item The solvent slowly moves up the paper by capillary action.
    \item As the solvent moves, it carries the components of the mixture along with it.
    \item Different components of the mixture will travel at different rates depending on:
        \begin{itemize}
            \item Their solubility in the solvent (mobile phase).
            \item Their attraction to the paper (stationary phase).
        \end{itemize}
    \item Components that are more soluble in the solvent and less attracted to the paper will travel further up the paper. Components that are less soluble and more attracted to the paper will travel less distance.
    \item This differential movement causes the components to separate into different spots or bands on the paper. These are called \keyword{spots} or \keyword{bands}.
    \item The paper with the separated components is called a \keyword{chromatogram}.
\end{enumerate}

\begin{example}
\textbf{Examples of Paper Chromatography:}
\begin{itemize}
    \item \keyword{Separating dyes in ink}: Different coloured dyes in a pen ink can be separated to see if the ink is a mixture of dyes.
    \item \keyword{Separating plant pigments}: Chlorophyll and other pigments from plant leaves can be separated.
    \item \keyword{Identifying amino acids}: Paper chromatography can be used to separate and identify amino acids (building blocks of proteins).
\end{itemize}
\end{example}

\begin{investigation}{Investigating Paper Chromatography}
\textbf{Title:} Separating the Dyes in Felt-Tip Pen Ink using Paper Chromatography

\textbf{Materials:}
\begin{itemize}
    \item Chromatography paper or filter paper
    \item Felt-tip pens of different colours (water-based inks work best)
    \item Beaker or tall jar
    \item Ruler
    \item Pencil
    \item Solvent (e.g., water, ethanol, or a mixture)
    \item Paper clips or sticky tape
\end{itemize}

\textbf{Procedure:}
\begin{enumerate}
    \item Cut a strip of chromatography paper or filter paper that is slightly shorter than the height of your beaker or jar.
    \item Use a pencil to draw a baseline about 1-2 cm from the bottom of the paper strip.
    \item Make small spots of different coloured inks along the baseline, using different felt-tip pens. Label each spot with the colour of the pen. Allow the spots to dry.
    \item Pour a small amount of solvent into the beaker or jar (enough to just cover the bottom, about 1 cm deep).
    \item Carefully lower the paper strip into the beaker, ensuring that the baseline with the ink spots is above the solvent level. You can use paper clips or sticky tape to hang the paper from a pencil or glass rod placed across the top of the beaker.
    \item Cover the beaker with a lid or cling film to reduce solvent evaporation.
    \item Observe as the solvent moves up the paper. Allow the solvent to rise until it is about 1-2 cm from the top of the paper.
    \item Remove the paper strip from the beaker and immediately mark the solvent front (the highest point the solvent reached) with a pencil line. Allow the paper to dry.
    \item Observe the chromatogram. Measure the distance travelled by the solvent front and the distance travelled by each dye spot from the baseline.
\end{enumerate}

\textbf{Observations and Questions:}
\begin{itemize}
    \item What happens to the ink spots as the solvent moves up the paper?
    \item How many different dyes are separated from each ink sample? What colours are they?
    \item Did all the ink colours separate into multiple dyes?
    \item Why do different dyes travel different distances up the paper?
    \item What role does the solvent play in paper chromatography?
\end{itemize}

\textbf{Safety:}
\begin{itemize}
    \item Handle solvents with care, especially if using flammable solvents like ethanol. Ensure good ventilation.
    \item Avoid getting ink or solvent on your skin or clothes.
\end{itemize}
\end{investigation}

\begin{stopandthink}
Imagine you are using paper chromatography to separate two dyes. Dye A travels further up the paper than Dye B. What can you conclude about the properties of Dye A compared to Dye B?
\end{stopandthink}

\begin{tieredquestions}{Basic}
\begin{enumerate}
    \item What is chromatography used to separate?
    \item Name the two phases in chromatography and describe their roles.
    \item In paper chromatography, what is the stationary phase and what is the mobile phase?
\end{enumerate}
\end{tieredquestions}

\begin{tieredquestions}{Intermediate}
\begin{enumerate}
    \item Explain why different components of a mixture separate in chromatography.
    \item Describe how paper chromatography can be used to determine if a single coloured ink is actually a mixture of dyes.
    \item  What factors affect the distance a substance travels in paper chromatography?
\end{enumerate}
\end{tieredquestions}

\begin{tieredquestions}{Advanced}
\begin{enumerate