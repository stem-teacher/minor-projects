\chapter{Reactive Chemistry}

Chemistry underpins our daily lives—from the combustion engines driving our transport to the batteries powering our devices. To truly appreciate chemistry, we must understand how substances interact and transform through chemical reactions. \marginnote{Chemical reactions are central to processes as diverse as photosynthesis, cellular respiration, and industrial manufacturing.}

In this chapter, we explore the fascinating world of reactive chemistry, looking closely at types of chemical reactions, how reactions occur, and what influences their speed. We will also delve into practical skills through experimental investigations, enabling you to build a robust understanding of chemical reactivity.

\section{Types of Chemical Reactions}
\FloatBarrier
\FloatBarrier
\FloatBarrier

Chemical reactions can be categorized by their patterns and products. Understanding these types helps chemists predict outcomes, balance equations, and harness reactions for practical purposes. We will now examine combustion, precipitation, acid-base, and redox reactions.

\subsection{Combustion Reactions}
\FloatBarrier
\FloatBarrier
\FloatBarrier

Combustion reactions are rapid reactions involving oxygen, producing energy as heat and light. Commonly associated with burning fuels, combustion reactions are critical for energy generation in engines and industrial processes.
\marginnote{\keyword{Combustion}: A chemical reaction between a substance (fuel) and oxygen, usually releasing heat and light.}

\begin{keyconcept}{General Combustion Reaction}
Combustion typically involves hydrocarbons reacting with oxygen to produce carbon dioxide and water:
\[
\ce{Hydrocarbon + O2 -> CO2 + H2O + energy}
\]
\end{keyconcept}

\begin{example}
Methane (\ce{CH4}), a simple hydrocarbon, combusts according to:
\[
\ce{CH4(g) + 2O2(g) -> CO2(g) + 2H2O(g)}
\]
\end{example}

\begin{stopandthink}
Explain why combustion reactions are considered exothermic.
\end{stopandthink}

\subsection{Precipitation Reactions}
\FloatBarrier
\FloatBarrier
\FloatBarrier

Precipitation reactions occur when two aqueous solutions combine to form an insoluble solid, known as the precipitate. These reactions are crucial for water treatment and qualitative chemical analysis.
\marginnote{\keyword{Precipitate}: Insoluble solid formed from reaction between two aqueous solutions.}

\begin{keyconcept}{Solubility Rules}
Certain ionic compounds are insoluble and form precipitates. For example, all nitrates (\ce{NO3^-}) are soluble, while many carbonates (\ce{CO3^{2-}}) are insoluble, except those of group 1 metals and ammonium.
\end{keyconcept}

\begin{example}
Mixing aqueous silver nitrate with sodium chloride forms white silver chloride precipitate:
\[
\ce{AgNO3(aq) + NaCl(aq) -> AgCl(s) + NaNO3(aq)}
\]
\end{example}

\begin{investigation}{Identifying Unknown Ions}
Design an experiment to identify unknown ionic solutions by testing their reactions and observing precipitate formation. Record observations systematically and interpret results using solubility rules.
\end{investigation}

\begin{stopandthink}
Why is knowledge of solubility rules useful in environmental chemistry?
\end{stopandthink}

\subsection{Acid-Base Reactions}
\FloatBarrier
\FloatBarrier
\FloatBarrier

Acid-base reactions involve proton transfer, resulting in neutralisation. These reactions are fundamental in biological systems, industrial processes, and environmental chemistry.
\marginnote{\keyword{Neutralisation}: Reaction between acid and base producing salt and water.}

\begin{keyconcept}{General Acid-Base Reaction}
Acids donate protons (\ce{H+}), bases accept protons:
\[
\ce{Acid + Base -> Salt + Water}
\]
\end{keyconcept}

\begin{example}
Reaction of hydrochloric acid with sodium hydroxide:
\[
\ce{HCl(aq) + NaOH(aq) -> NaCl(aq) + H2O(l)}
\]
\end{example}

\begin{stopandthink}
Describe the role of acid-base reactions in human digestion.
\end{stopandthink}

\subsection{Redox Reactions}
\FloatBarrier
\FloatBarrier
\FloatBarrier

Redox (reduction-oxidation) reactions involve electron transfer, changing the oxidation states of species involved. Redox reactions are essential in energy production, corrosion, and biological metabolism.
\marginnote{\keyword{Redox Reaction}: Chemical process involving simultaneous oxidation and reduction.}

\begin{keyconcept}{Oxidation and Reduction}
Oxidation is loss of electrons, while reduction is gain. The mnemonic OIL RIG (Oxidation Is Loss, Reduction Is Gain) helps recall this principle.
\end{keyconcept}

\begin{example}
Reaction of zinc metal with copper sulfate solution:
\[
\ce{Zn(s) + CuSO4(aq) -> ZnSO4(aq) + Cu(s)}
\]
Here, zinc is oxidised (\ce{Zn -> Zn^{2+} + 2e^-}) and copper is reduced (\ce{Cu^{2+} + 2e^- -> Cu}).
\end{example}

\begin{stopandthink}
Identify oxidation and reduction in the combustion of methane described earlier.
\end{stopandthink}

\FloatBarrier

\begin{tieredquestions}{Basic}
\begin{enumerate}
    \item Identify the reaction type: 
    \[
    \ce{2H2(g) + O2(g) -> 2H2O(l)}
    \]
    \item Write a balanced equation for neutralisation of sulfuric acid and potassium hydroxide.
\end{enumerate}
\end{tieredquestions}

\begin{tieredquestions}{Intermediate}
\begin{enumerate}
    \item Predict precipitates (if any) formed when solutions of sodium sulfate, barium nitrate, and potassium chloride are mixed.
    \item Explain why combustion reactions are critical in industrial energy production.
\end{enumerate}
\end{tieredquestions}

\begin{tieredquestions}{Advanced}
\begin{enumerate}
    \item Determine oxidation states for all elements in the reaction:
    \[
    \ce{MnO4^- + Fe^{2+} + H+ -> Mn^{2+} + Fe^{3+} + H2O}
    \]
    \item Design a practical investigation to assess environmental impacts of acid-base reactions involving industrial effluents.
\end{enumerate}
\end{tieredquestions}

\section{Reaction Rates and Collision Theory}
\FloatBarrier
\FloatBarrier
\FloatBarrier

Why do some reactions occur instantly, while others seem to take ages? This section explores reaction rates and the factors influencing them, through collision theory.

\subsection{Understanding Reaction Rates}
\FloatBarrier
\FloatBarrier
\FloatBarrier

Reaction rate refers to how quickly reactants are converted into products. Chemists measure reaction rates by changes in concentration of reactants or products per unit time. 
\marginnote{\keyword{Reaction Rate}: Speed at which reactants become products, measured as concentration change per time unit.}

\begin{keyconcept}{Measuring Reaction Rates}
Mathematically, reaction rate can be expressed as:
\[
\text{Rate} = \frac{\Delta [\text{Product}]}{\Delta t} = -\frac{\Delta [\text{Reactant}]}{\Delta t}
\]
\end{keyconcept}

\subsection{Collision Theory Fundamentals}
\FloatBarrier
\FloatBarrier
\FloatBarrier

Collision theory explains rates at molecular level. For a reaction to occur, particles must collide with sufficient energy (activation energy) and correct orientation.
\marginnote{\keyword{Activation Energy}: Minimum energy particles must have to react.}

\begin{stopandthink}
What factors could increase the frequency of successful collisions?
\end{stopandthink}

\begin{investigation}{Factors Affecting Reaction Rate}
Plan experiments to test how temperature, concentration, surface area, and catalysts affect reaction rates. Clearly identify variables and controls, and present data graphically.
\end{investigation}

\challenge{Explore the Maxwell-Boltzmann distribution to understand energy distribution among particles.}

\FloatBarrier

\begin{tieredquestions}{Basic}
\begin{enumerate}
    \item Define activation energy.
    \item List two factors affecting reaction rates.
\end{enumerate}
\end{tieredquestions}

\begin{tieredquestions}{Intermediate}
\begin{enumerate}
    \item Explain, using collision theory, why increasing concentration increases reaction rate.
    \item Why does powdered magnesium react faster than magnesium ribbon?
\end{enumerate}
\end{tieredquestions}

\begin{tieredquestions}{Advanced}
\begin{enumerate}
    \item Using collision theory and activation energy concepts, explain how a catalyst functions.
    \item Research a biochemical reaction in human metabolism and discuss factors influencing its rate.
\end{enumerate}
\end{tieredquestions}

\FloatBarrier

This chapter equips you with foundational knowledge and practical skills in reactive chemistry, preparing you for deeper exploration and application in future chemistry studies.