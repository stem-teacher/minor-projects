\chapter{Properties \& Structure of Matter}

\section{Introduction: Why Matter Matters}
\FloatBarrier
\FloatBarrier
\FloatBarrier

Everything we observe and interact with, from the water we drink to the smartphone in our hands, is composed of matter. Understanding the properties and structure of matter allows scientists to develop new materials, medicines, and technologies essential for modern life. In this chapter, we delve deeply into atomic structure, bonding, periodic trends, and the forces that hold matter together, exploring the historical progression of our knowledge and the sophisticated theories that underpin modern chemistry.

\begin{marginfigure}[0pt][0pt][0pt]
\includegraphics{placeholder_matter_structure}
\caption{The intricate arrangement of atoms determines the properties of all matter around us.}
\label{fig:atomic_structure_intro}
\end{marginfigure}

\section{Atomic Structure}
\FloatBarrier
\FloatBarrier
\FloatBarrier

\subsection{Historical Development of Atomic Theories}
\FloatBarrier
\FloatBarrier
\FloatBarrier

The concept of atoms dates back to ancient Greece with philosophers such as Democritus, who proposed that matter consisted of indivisible particles. However, it was not until the 19th and 20th centuries that empirical evidence emerged, transforming atomic theory into a cornerstone of science.

\historylink{John Dalton (1803) proposed the first scientific atomic theory, stating that elements are composed of indivisible and indestructible atoms.}

\historylink{J.J. Thomson (1897) discovered electrons using cathode-ray tubes, leading to the plum pudding model.}

\historylink{Ernest Rutherford (1911) demonstrated the existence of the atomic nucleus through his gold foil experiment.}

\historylink{Niels Bohr (1913) introduced quantized energy levels for electrons, developing the Bohr model.}

\begin{keyconcept}{The Modern Atomic Model}
Today's atomic model consists of a dense nucleus containing protons and neutrons, surrounded by electrons occupying probabilistic orbitals described by quantum mechanics.
\end{keyconcept}

\begin{stopandthink}
How did each historical atomic model improve upon the previous one? What experiments led to significant changes in the atomic model?
\end{stopandthink}

\subsection{Subatomic Particles}
\FloatBarrier
\FloatBarrier
\FloatBarrier

Atoms consist of three main subatomic particles:

\begin{itemize}
    \item \keyword{Protons}: Positively charged particles located in the nucleus.
    \item \keyword{Neutrons}: Neutral particles in the nucleus, contributing to atomic mass and isotopic variation.
    \item \keyword{Electrons}: Negatively charged particles occupying orbitals around the nucleus.
\end{itemize}

\begin{marginfigure}[0pt][0pt][0pt]
\includegraphics{placeholder_subatomic_particles}
\caption{The arrangement of protons, neutrons, and electrons within an atom.}
\label{fig:subatomic_particles}
\end{marginfigure}

\begin{example}
A neutral carbon atom (\ce{^{12}C}) contains 6 protons, 6 neutrons, and 6 electrons. The atomic number (6) uniquely identifies the element, while the mass number (12) denotes the sum of protons and neutrons.
\end{example}

\begin{tieredquestions}{Basic}
\begin{enumerate}
    \item State the charges and relative masses of protons, neutrons, and electrons.
    \item Determine the number of protons, neutrons, and electrons in a neutral sodium atom (\ce{^{23}Na}).
\end{enumerate}
\end{tieredquestions}

\begin{tieredquestions}{Intermediate}
\begin{enumerate}
    \item Explain isotopes and calculate the average atomic mass of chlorine, given that \ce{^{35}Cl} is 75\% abundant and \ce{^{37}Cl} is 25\% abundant.
\end{enumerate}
\end{tieredquestions}

\begin{tieredquestions}{Advanced}
\begin{enumerate}
    \item Discuss the significance of isotopic analysis in radiometric dating and forensic science.
\end{enumerate}
\end{tieredquestions}

\FloatBarrier

\section{Electron Configuration and Atomic Orbitals}
\FloatBarrier
\FloatBarrier
\FloatBarrier

\subsection{Quantum Numbers and Orbitals}
\FloatBarrier
\FloatBarrier
\FloatBarrier

Electrons occupy specific energy levels and orbitals described by quantum mechanics. Four quantum numbers—principal ($n$), azimuthal ($l$), magnetic ($m_l$), and spin ($m_s$)—fully describe the electron's state within an atom.

\mathlink{Quantum numbers arise from solutions to the Schrödinger equation, reflecting wave-particle duality and electron probabilities.}

\begin{keyconcept}{Electron Configuration}
Electron configuration describes the arrangement of electrons in atomic orbitals following the principles of Aufbau, Pauli exclusion, and Hund's rule.
\end{keyconcept}

\begin{stopandthink}
Why can't two electrons in the same atom have identical quantum numbers? How does this principle affect electron configuration?
\end{stopandthink}

\subsection{Electron Configuration and the Periodic Table}
\FloatBarrier
\FloatBarrier
\FloatBarrier

The periodic table structure directly reflects electron configurations, with the arrangement into blocks ($s$, $p$, $d$, $f$) corresponding to orbital types.

\begin{example}
Write the electron configuration for sulfur (\ce{S}, atomic number 16):

\[
\ce{S}: 1s^2\,2s^2\,2p^6\,3s^2\,3p^4
\]

Sulfur has six valence electrons (in the third energy level), influencing its chemical properties.
\end{example}

\begin{tieredquestions}{Basic}
Write electron configurations for magnesium (\ce{Mg}, Z=12) and chlorine (\ce{Cl}, Z=17).
\end{tieredquestions}

\begin{tieredquestions}{Intermediate}
Explain how electron configurations determine the chemical reactivity and bonding patterns of elements.
\end{tieredquestions}

\begin{tieredquestions}{Advanced}
Using quantum numbers, justify why the 4s orbital fills before the 3d orbital.
\end{tieredquestions}

\FloatBarrier

\section{Periodic Trends}
\FloatBarrier
\FloatBarrier
\FloatBarrier

\subsection{Atomic Radius, Ionisation Energy, and Electronegativity}
\FloatBarrier
\FloatBarrier
\FloatBarrier

Periodic trends describe predictable changes in atomic properties across periods and groups due to electron configuration patterns.

\begin{keyconcept}{Periodic Trends}
\begin{itemize}
    \item \keyword{Atomic radius}: Decreases across periods, increases down groups.
    \item \keyword{Ionisation energy}: Energy required to remove electrons; increases across periods, decreases down groups.
    \item \keyword{Electronegativity}: Atom's ability to attract electrons; increases across periods and decreases down groups.
\end{itemize}
\end{keyconcept}

\begin{investigation}{Trends in Ionisation Energy}
Using tabulated ionisation energies, plot graphs for elements in Period 2 and 3. Analyse patterns and explain anomalies based on electron configuration.
\end{investigation}

\begin{stopandthink}
Why does atomic radius decrease across a period despite the increasing number of electrons?
\end{stopandthink}

\FloatBarrier

\section{Chemical Bonding and Structure}
\FloatBarrier
\FloatBarrier
\FloatBarrier

\subsection{Types of Chemical Bonds}
\FloatBarrier
\FloatBarrier
\FloatBarrier

Atoms bond to achieve stability, forming ionic, covalent, or metallic bonds.

\begin{keyconcept}{Bonding Types}
\begin{itemize}
    \item \keyword{Ionic bonding}: Transfer of electrons between metals and non-metals.
    \item \keyword{Covalent bonding}: Sharing electrons between non-metals.
    \item \keyword{Metallic bonding}: Delocalised electrons in metal lattices.
\end{itemize}
\end{keyconcept}

\subsection{Allotropes and Structural Variations}
\FloatBarrier
\FloatBarrier
\FloatBarrier

Elements can exist in multiple structural forms called \keyword{allotropes}, displaying significantly different properties.

\challenge{Investigate recent research into allotropes of carbon, including graphene and fullerenes, and their potential technological applications.}

\begin{tieredquestions}{Advanced}
Explain the difference in properties between diamond and graphite considering their covalent bonding structures and intermolecular forces.
\end{tieredquestions}

\FloatBarrier

\section{Intermolecular Forces}
\FloatBarrier
\FloatBarrier
\FloatBarrier

Molecules interact via weaker interactions such as dispersion forces, dipole-dipole interactions, and hydrogen bonds, significantly influencing physical properties.

\begin{keyconcept}{Intermolecular Forces}
The strength of intermolecular forces governs boiling points, melting points, solubility, and viscosity.
\end{keyconcept}

\begin{investigation}{Measuring Intermolecular Forces}
Design an experiment to compare boiling points of various compounds. Analyse how intermolecular forces influence your results.
\end{investigation}

\FloatBarrier

\section{Conclusion}
\FloatBarrier
\FloatBarrier
\FloatBarrier

Understanding matter's properties and structure allows chemists to innovate and solve real-world problems. The next chapters build on these foundational concepts, exploring chemical reactions and thermodynamics.

\newpage