\chapter{Equilibrium and Acid Reactions}

\newthought{Chemical reactions} are often depicted as one-way processes, moving from reactants to products. However, in reality, many reactions can proceed in both forward and reverse directions simultaneously, reaching a state of balance known as chemical \keyword{equilibrium}. Equilibrium systems are essential in natural processes, industrial production, and biological functions. Understanding equilibrium and acid-base chemistry equips us with tools to manipulate reactions for practical benefit—ranging from industrial ammonia production to maintaining physiological pH in our bloodstream.

\section{Principles of Chemical Equilibrium}
\FloatBarrier

\subsection{Dynamic Equilibrium}
Consider a reaction where reactants and products coexist in a closed system. Initially, the forward reaction dominates; however, as product concentration increases, the reverse reaction becomes significant. Eventually, the rates of the forward and reverse reactions become equal, establishing \keyword{dynamic equilibrium}.

\begin{keyconcept}{Dynamic Equilibrium Defined}
Dynamic equilibrium occurs when the forward and reverse reactions proceed at equal rates, maintaining constant concentrations of reactants and products over time.
\end{keyconcept}

\begin{marginfigure}[0pt]
\caption{Diagram of a dynamic equilibrium system. Reactant and product molecules continuously interconvert, yet their concentrations remain constant. (Visual to be inserted)}
\end{marginfigure}

\begin{stopandthink}
If equilibrium is dynamic, does this mean the reaction has stopped? Explain your reasoning.
\end{stopandthink}

\subsection{Le Châtelier's Principle}
\historylink{Henri-Louis Le Châtelier (1850–1936), a French chemist, introduced this principle in 1884.} Le Châtelier's principle helps us predict how equilibrium systems respond to disturbances.

\begin{keyconcept}{Le Châtelier's Principle}
If a system at equilibrium is subjected to a change in concentration, pressure, or temperature, the system will shift its equilibrium position to counteract the applied change, partially restoring equilibrium.
\end{keyconcept}

\subsubsection{Concentration Changes}
Adding more reactants pushes equilibrium toward products, while removing products shifts it further toward products. Conversely, adding products or removing reactants shifts equilibrium toward reactants.

\begin{example}
Consider the equilibrium involving nitrogen dioxide and dinitrogen tetroxide:
\[
\ce{2NO2(g) <=> N2O4(g)}
\]
Increasing \ce{NO2} concentration shifts equilibrium to the right, creating more \ce{N2O4}. 
\end{example}

\subsubsection{Pressure Changes}
Pressure changes affect equilibrium involving gases. Increasing pressure favors the side with fewer gas molecules, reducing volume and relieving the increase in pressure.

\subsubsection{Temperature Changes}
Temperature changes affect equilibrium positions depending on reaction enthalpy (\(\Delta H\)):

\begin{itemize}
\item Increasing temperature favors endothermic reactions (\(\Delta H>0\)).
\item Decreasing temperature favors exothermic reactions (\(\Delta H<0\)).
\end{itemize}

\begin{investigation}{Investigating Equilibrium Shifts}
Design a simple experiment using cobalt(II) chloride solutions to observe equilibrium shifts induced by temperature and concentration changes. Record observations, justify equilibrium shifts using Le Châtelier’s principle, and discuss underlying molecular events.
\end{investigation}

\FloatBarrier

\begin{tieredquestions}{Basic}
\begin{enumerate}
\item Define dynamic equilibrium in your own words.
\item Explain how equilibrium responds if the concentration of a reactant is increased.
\end{enumerate}
\end{tieredquestions}

\begin{tieredquestions}{Intermediate}
\begin{enumerate}
\item Given the reaction: \(\ce{N2(g) + 3H2(g) <=> 2NH3(g)}\). Predict the effect of increasing pressure.
\item How would you maximize ammonia yield based on Le Châtelier’s principle?
\end{enumerate}
\end{tieredquestions}

\begin{tieredquestions}{Advanced}
\begin{enumerate}
\item For the equilibrium reaction: \(\ce{H2(g) + I2(g) <=> 2HI(g)}\), explain mathematically how equilibrium concentrations change if additional \ce{H2} is introduced.
\item Discuss limitations of Le Châtelier's principle in industrial contexts.
\end{enumerate}
\end{tieredquestions}

\section{Equilibrium Constants ($K_{eq}$)}
The equilibrium constant quantitatively describes equilibrium positions. For a general reaction:
\[
\ce{aA + bB <=> cC + dD}
\]
the equilibrium constant \(K_{eq}\) is:
\[
K_{eq} = \frac{[\ce{C}]^c [\ce{D}]^d}{[\ce{A}]^a[\ce{B}]^b}
\]

\mathlink{Equilibrium constants are dimensionless. Concentrations are expressed as molarities (mol/L), and partial pressures can be used for gaseous equilibria (\(K_p\)).}

\begin{example}
For the reaction \(\ce{N2(g) + 3H2(g) <=> 2NH3(g)}\), the equilibrium constant expression is:
\[
K_{eq} = \frac{[\ce{NH3}]^2}{[\ce{N2}][\ce{H2}]^3}
\]
\end{example}

\begin{stopandthink}
Why are pure solids and liquids excluded from equilibrium constant expressions?
\end{stopandthink}

\section{Acid-Base Theories}

\subsection{Historical Definitions}

\subsubsection{Arrhenius Theory}
\historylink{Svante Arrhenius (1859–1927) proposed this early definition, earning a Nobel Prize in Chemistry in 1903.} According to Arrhenius:
\begin{itemize}
\item Acids produce \ce{H+} in aqueous solution.
\item Bases produce \ce{OH-} in aqueous solution.
\end{itemize}

\subsubsection{Brønsted-Lowry Theory}
This broader theory defines acids and bases based on proton transfer:
\begin{itemize}
\item Acids donate protons (\ce{H+}).
\item Bases accept protons.
\end{itemize}

\begin{keyconcept}{Conjugate Acid-Base Pairs}
A conjugate acid-base pair differs by one proton (\ce{H+}). For example, \(\ce{NH3}\) and \(\ce{NH4+}\) form a conjugate pair.
\end{keyconcept}

\subsection{pH and pOH}
pH quantitatively measures acidity:
\[
\text{pH} = -\log[\ce{H+}]
\]

Similarly, pOH measures basicity:
\[
\text{pOH} = -\log[\ce{OH-}]
\]

At \(25^\circ C\), we have:
\[
\text{pH} + \text{pOH} = 14
\]

\begin{example}
Calculate the pH of a solution with \(\ce{[H+]} = 1.0 \times 10^{-3}\,\text{mol/L}\).
Solution:
\[
\text{pH} = -\log(1.0 \times 10^{-3}) = 3
\]
\end{example}

\section{Strength vs. Concentration}
Acid/base strength refers to degree of dissociation, whereas concentration is amount per unit volume.

\subsection{Buffer Systems}
Buffers resist pH changes and typically contain weak acids and their conjugate bases.

\begin{keyconcept}{Henderson-Hasselbalch Equation}
For buffer solutions:
\[
\text{pH} = pK_a + \log\frac{[\text{base}]}{[\text{acid}]}
\]
\end{keyconcept}

\begin{investigation}{Buffer Capacity Exploration}
Prepare buffer solutions and measure pH changes upon addition of small amounts of strong acid/base. Analyze results quantitatively using Henderson-Hasselbalch equation.
\end{investigation}

\FloatBarrier

\section*{Chapter Summary}
Chemical equilibrium and acid-base reactions underpin numerous natural and industrial processes. Mastery of equilibrium constants, Le Châtelier's principle, acid-base theories, and buffer systems provides a powerful toolkit for chemical control and analysis.

\challenge{Advanced students may explore equilibrium modeling software or research current advancements in buffer applications in biotechnology or environmental science.}

\FloatBarrier