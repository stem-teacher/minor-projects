\chapter{Organic Chemistry}

Organic chemistry is the study of carbon-containing compounds, which encompass an astonishing variety of substances essential to life and modern industry. From fuels to pharmaceuticals, polymers to biofuels, organic chemistry shapes every aspect of daily life and technological advancement. 

\historylink{The term ``organic chemistry'' originally referred to chemistry of living organisms. In 1828, Friedrich Wöhler synthesized urea (\ce{CH4N2O}), demonstrating that organic compounds could be synthesized from inorganic precursors.}

In this module, we explore the structure, nomenclature, reactions, and applications of hydrocarbons and their derivatives, highlighting their relevance to environmental sustainability, medicine, and advanced materials.

\section{Hydrocarbons and Functional Groups}
\FloatBarrier
\FloatBarrier
\FloatBarrier

Hydrocarbons are compounds made exclusively of carbon and hydrogen atoms. They form the simplest foundation of organic chemistry.

\subsection{Alkanes, Alkenes, and Alkynes}
\FloatBarrier
\FloatBarrier
\FloatBarrier

\keyword{Hydrocarbons} are classified into three major types based on their bonding structure:

\begin{itemize}
    \item \keyword{Alkanes} (\ce{C_nH_{2n+2}}): hydrocarbons containing only single covalent bonds.
    \item \keyword{Alkenes} (\ce{C_nH_{2n}}): contain at least one carbon-carbon double bond (\ce{C=C}).
    \item \keyword{Alkynes} (\ce{C_nH_{2n-2}}): contain at least one carbon-carbon triple bond (\ce{C#C}).
\end{itemize}

\begin{marginfigure}[0pt][0pt][0pt]
    % Figure placeholder: Structural formulae of ethane, ethene, ethyne
    \caption{Structures of ethane (\ce{C2H6}), ethene (\ce{C2H4}), and ethyne (\ce{C2H2}).}
\end{marginfigure}

\begin{keyconcept}{Saturated and Unsaturated Hydrocarbons}
Alkanes are \keyword{saturated}, as they contain only single bonds and cannot accommodate additional atoms without breaking existing bonds. Alkenes and alkynes are \keyword{unsaturated}, containing double or triple bonds that can undergo addition reactions.
\end{keyconcept}

\begin{stopandthink}
Why do you think unsaturated hydrocarbons are typically more reactive than saturated hydrocarbons?
\end{stopandthink}

\subsection{Functional Groups}
\FloatBarrier
\FloatBarrier
\FloatBarrier

\keyword{Functional groups} are specific groups of atoms within molecules responsible for their characteristic chemical reactions. Key functional groups we will study include:

\begin{itemize}
    \item \keyword{Alcohols} (\ce{-OH})
    \item \keyword{Aldehydes} (\ce{-CHO})
    \item \keyword{Ketones} (\ce{-CO-})
    \item \keyword{Carboxylic acids} (\ce{-COOH})
    \item \keyword{Esters} (\ce{-COO-})
\end{itemize}

\begin{marginfigure}[0pt][0pt][0pt]
    % Figure placeholder: Structures of main functional groups
    \caption{General structural formulae of key functional groups.}
\end{marginfigure}

\begin{tieredquestions}{Basic}
\begin{enumerate}
    \item Identify the functional group present in ethanol (\ce{CH3CH2OH}).
    \item What is the difference between an aldehyde and a ketone?
\end{enumerate}
\end{tieredquestions}

\begin{tieredquestions}{Advanced}
\begin{enumerate}
    \item Predict the reactivity order of alkanes, alkenes, and alkynes towards addition reactions and justify your reasoning.
    \item Propose a method to distinguish experimentally between a primary alcohol and a carboxylic acid.
\end{enumerate}
\end{tieredquestions}

\FloatBarrier

\section{IUPAC Nomenclature and Isomerism}
\FloatBarrier
\FloatBarrier
\FloatBarrier

Systematic naming of organic compounds follows the International Union of Pure and Applied Chemistry (IUPAC) rules, ensuring clarity and consistency worldwide.

\subsection{Naming Organic Compounds}
\FloatBarrier
\FloatBarrier
\FloatBarrier

The systematic nomenclature involves identifying the longest carbon chain, numbering substituents and functional groups, and using appropriate suffixes or prefixes.

\begin{example}
Name the following compound: \ce{CH3-CH(CH3)-CH2-CH3}

\textbf{Solution:} The longest chain has four carbon atoms (butane). A methyl substituent is attached to carbon-2. Thus, the compound is named \textbf{2-methylbutane}.
\end{example}

\challenge{Research and compare older nomenclature systems (e.g., common names) with current IUPAC standards.}

\subsection{Structural and Geometric Isomerism}
\FloatBarrier
\FloatBarrier
\FloatBarrier

\keyword{Isomers} have identical molecular formulas but different structural arrangements, resulting in diverse chemical and physical properties.

\begin{keyconcept}{Types of Isomerism}
\begin{itemize}
    \item \textbf{Structural isomers} differ in bonding arrangements.
    \item \textbf{Geometric isomers} (cis-trans) differ in spatial orientation around double bonds.
\end{itemize}
\end{keyconcept}

\begin{stopandthink}
Why can't alkanes exhibit geometric isomerism?
\end{stopandthink}

\begin{investigation}{Modeling Isomers}
Use molecular model kits (or software simulations) to build and compare the structures and properties of structural and geometric isomers of selected hydrocarbons (e.g., \ce{C4H8}).
\end{investigation}

\FloatBarrier

\section{Reaction Pathways}
\FloatBarrier
\FloatBarrier
\FloatBarrier

Organic reactions often involve predictable reaction pathways. Understanding these enables chemists to synthesize complex molecules.

\subsection{Substitution Reactions}
\FloatBarrier
\FloatBarrier
\FloatBarrier

Alkanes undergo \keyword{substitution reactions} when atoms are replaced by other atoms or groups. For example, chlorination of methane:

\[
\ce{CH4 + Cl2 -> CH3Cl + HCl}
\]

\challenge{Explore the mechanism of radical substitution reactions using curly arrow notation.}

\subsection{Addition Reactions}
\FloatBarrier
\FloatBarrier
\FloatBarrier

Alkenes and alkynes readily undergo \keyword{addition reactions}, where double or triple bonds open to incorporate new atoms.

\begin{example}
Ethene reacts with bromine (\ce{Br2}) rapidly, forming 1,2-dibromoethane:

\[
\ce{CH2=CH2 + Br2 -> CH2Br-CH2Br}
\]
\end{example}

\subsection{Condensation Reactions}
\FloatBarrier
\FloatBarrier
\FloatBarrier

In \keyword{condensation reactions}, molecules combine, releasing small molecules like water. Esters form through condensation between alcohols and carboxylic acids (\keyword{esterification}):

\[
\ce{CH3COOH + CH3OH <=> CH3COOCH3 + H2O}
\]

\mathlink{Calculate equilibrium constants (\ce{Kc}) for esterification reactions using equilibrium concentrations.}

\begin{tieredquestions}{Intermediate}
\begin{enumerate}
    \item Draw the reaction pathway for the hydrogenation of propene (\ce{CH3-CH=CH2}) to propane.
    \item Describe the role of catalysts in addition reactions.
\end{enumerate}
\end{tieredquestions}

\FloatBarrier

\section{Polymers, Biofuels, and Current Advances}
\FloatBarrier
\FloatBarrier
\FloatBarrier

Organic chemistry continues to push the boundaries of materials science, energy production, and environmental sustainability.

\subsection{Polymers}
\FloatBarrier
\FloatBarrier
\FloatBarrier

Polymers are large molecules formed by repeated linking of monomers. Addition polymerization produces polymers like polyethylene, while condensation polymerization yields polyesters.

\subsection{Biofuels}
\FloatBarrier
\FloatBarrier
\FloatBarrier

Biofuels derived from organic matter (e.g., ethanol, biodiesel) offer renewable energy alternatives to fossil fuels, reducing greenhouse gas emissions.

\subsection{Current Advances in Organic Synthesis}
\FloatBarrier
\FloatBarrier
\FloatBarrier

Modern techniques such as green chemistry, catalysis, and synthetic biology are revolutionizing organic synthesis, making processes safer, cheaper, and environmentally friendly.

\begin{investigation}{Evaluating Biofuel Efficiency}
Research and compare the energy content, production methods, and environmental impacts of ethanol and biodiesel as biofuels.
\end{investigation}

\FloatBarrier

\section*{Chapter Review}

\begin{tieredquestions}{Basic}
\begin{enumerate}
    \item Define hydrocarbons and give examples.
    \item List five key functional groups.
\end{enumerate}
\end{tieredquestions}

\begin{tieredquestions}{Advanced}
\begin{enumerate}
    \item Design a synthetic route for converting ethene to ethanoic acid.
    \item Discuss recent advances in polymer synthesis and their implications for sustainability.
\end{enumerate}
\end{tieredquestions}