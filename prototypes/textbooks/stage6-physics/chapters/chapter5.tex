\chapter{Advanced Mechanics}

\section{Introduction and Context}
\FloatBarrier
Mechanics forms the backbone of physics, governing everything from the motion of planets to the spin of electrons. In earlier studies, we explored linear motion and basic forces. This chapter delves deeper into rotational and orbital dynamics, offering insights into the universe's motion at a profound level. From the orbiting satellites enabling global communication, to the spinning wheels of a bicycle, advanced mechanics is central to understanding our world.

\begin{marginfigure}[0pt]
\includegraphics[width=\linewidth]{placeholder_satellite_orbit}
\caption{Satellites and their orbits provide essential services such as communication, navigation, and weather forecasting.}
\end{marginfigure}

\section{Uniform Circular Motion and Centripetal Force}
\FloatBarrier

\subsection{Defining Uniform Circular Motion}
\FloatBarrier

\keyword{Uniform circular motion} describes the motion of an object moving in a circle at a constant speed. Although the speed is constant, the velocity is continuously changing due to a constantly altering direction. 

\begin{keyconcept}{Velocity versus Speed}
Remember that velocity is a vector quantity, meaning it has both magnitude and direction. In uniform circular motion, the magnitude of velocity (speed) remains constant, but the direction continually changes, producing acceleration.
\end{keyconcept}

\begin{marginfigure}[0pt]
\includegraphics[width=\linewidth]{placeholder_circular_motion_vector}
\caption{Velocity vectors in circular motion continually change direction, pointing tangentially to the circle.}
\end{marginfigure}

The acceleration experienced in uniform circular motion is called \keyword{centripetal acceleration}, directed towards the centre of the circle, and calculated by:

\[
a_c = \frac{v^2}{r}
\]

where \( v \) is the object's speed and \( r \) is the radius of the circular path.

\begin{stopandthink}
Why is there an acceleration if the speed is constant during uniform circular motion? Explain using vector concepts.
\end{stopandthink}

\subsection{Centripetal Force}
\FloatBarrier

For an object to follow a circular path, a net force must act towards the centre of the circle. This force is known as the \keyword{centripetal force}, calculated by:

\[
F_c = \frac{mv^2}{r}
\]

where \( m \) is the mass of the object.

\begin{example}
A 1200 kg car takes a turn of radius 50 m at a constant speed of 15 m/s. Calculate the centripetal force required.

\textbf{Solution:}
\[
F_c = \frac{mv^2}{r} = \frac{1200 \times 15^2}{50} = 5400\,\text{N}
\]

The force of 5400 N acts towards the centre of the circular path, provided by friction between the tyres and the road.
\end{example}

\begin{investigation}{Determining Centripetal Force Experimentally}
\textbf{Aim:} To explore the relationship between centripetal force, mass, velocity, and radius.

\textbf{Equipment:} Rotational apparatus, masses, stopwatch, meter ruler, spring balance.

\textbf{Procedure:}
\begin{enumerate}
  \item Attach a known mass to the rotational apparatus.
  \item Spin the apparatus at different speeds, measuring the force from the spring balance.
  \item Record the radius of rotation and calculate the theoretical centripetal force.
\end{enumerate}

\textbf{Analysis:} Compare experimental results with theoretical calculations. Discuss discrepancies.
\end{investigation}

\begin{tieredquestions}{Basic}
\begin{enumerate}
  \item Define centripetal force.
  \item List three examples of centripetal forces in everyday life.
  \item Calculate the centripetal acceleration of a ball spinning at 4 m/s in a circle of radius 2 m.
\end{enumerate}
\end{tieredquestions}

\begin{tieredquestions}{Intermediate}
\begin{enumerate}
  \item A 0.5 kg object moves in a circle of radius 0.8 m with a frequency of 2 revolutions per second. Calculate the centripetal force.
  \item Explain the role of friction as a centripetal force when vehicles turn corners.
\end{enumerate}
\end{tieredquestions}

\begin{tieredquestions}{Advanced}
\begin{enumerate}
  \item Derive the equation for centripetal acceleration from first principles, considering vector changes in velocity.
  \item Discuss how centripetal forces influence the design of roller coaster rides.
\end{enumerate}
\end{tieredquestions}

\FloatBarrier

\section{Rotational Dynamics: Torque and Rotational Inertia}
\FloatBarrier

\subsection{Understanding Torque}
\FloatBarrier

\keyword{Torque} (\(\tau\)) is the rotational equivalent of linear force, defined as a measure of a force's ability to rotate an object about an axis. Mathematically:

\[
\tau = rF \sin(\theta)
\]

where \( r \) is the distance from the pivot (lever arm), \( F \) is the applied force, and \( \theta \) is the angle between the force vector and the lever arm.

\begin{marginfigure}[0pt]
\includegraphics[width=\linewidth]{placeholder_torque_diagram}
\caption{Torque depends on the magnitude of force, distance from pivot, and angle of application.}
\end{marginfigure}

\subsection{Rotational Inertia (Moment of Inertia)}
\FloatBarrier

\keyword{Rotational inertia}, also known as moment of inertia (\(I\)), quantifies an object's resistance to changes in rotational motion. Objects with mass distributed further from the axis of rotation have greater rotational inertia.

Qualitatively, rotational inertia depends on mass distribution. Semi-quantitatively, it can be expressed as:

\[
I = \sum m_i r_i^2
\]

\challenge{Advanced students may explore integral calculus approaches for continuous mass distributions.}

\begin{example}
Explain why figure skaters spin faster when they pull their arms inward.

\textbf{Solution:}
As skaters pull their arms closer to their body, the mass distribution moves closer to the rotation axis, reducing rotational inertia (\(I\)). According to angular momentum conservation (\(L = I \omega\)), a decrease in \(I\) results in an increased angular velocity (\(\omega\)), causing the skater to spin faster.
\end{example}

\begin{investigation}{Rotational Inertia of Different Objects}
\textbf{Aim:} To compare rotational inertia of objects with different mass distributions.

\textbf{Procedure:}
Roll various objects (solid sphere, hollow cylinder, solid cylinder) down an inclined plane. Measure and compare their acceleration.

\textbf{Analysis:} Discuss how mass distribution affects rotational inertia and acceleration.
\end{investigation}

\begin{stopandthink}
Which rolls faster down a slope: a solid sphere or a hollow cylinder of identical mass and radius? Why?
\end{stopandthink}

\FloatBarrier

\section{Gravitational Fields and Orbital Motion}
\FloatBarrier

\subsection{Newton's Law of Universal Gravitation}
\FloatBarrier

Isaac Newton discovered that gravitational force between two masses is directly proportional to the product of their masses and inversely proportional to the square of the distance between their centres:

\[
F = G \frac{m_1 m_2}{r^2}
\]

where \(G\) is the gravitational constant.

\historylink{Newton formulated this law in 1687 in his work \textit{Principia Mathematica}.}

\subsection{Orbital Motion: Satellites and Planets}
\FloatBarrier

Objects in orbit experience gravitational force as a centripetal force. For circular orbits:

\[
\frac{GMm}{r^2} = \frac{mv^2}{r}
\]

Simplifying gives orbital velocity:

\[
v = \sqrt{\frac{GM}{r}}
\]

\challenge{Gifted students can explore elliptical orbits and Kepler's laws in greater detail.}

\begin{example}
Calculate the orbital velocity of a satellite orbiting 300 km above Earth's surface. Earth's radius \( R_E = 6.371 \times 10^6\,\text{m} \), mass \(M = 5.972 \times 10^{24}\,\text{kg}\), and \(G = 6.67 \times 10^{-11}\,\text{Nm}^2/\text{kg}^2\).

\textbf{Solution:}
\[
v = \sqrt{\frac{GM}{R_E + h}} \approx 7.7\,\text{km/s}
\]
\end{example}

\FloatBarrier

\section{Chapter Summary and Review}
\FloatBarrier

This chapter explored advanced mechanics concepts—uniform circular motion, rotational dynamics, and gravitational orbital motion—integral to understanding complex systems from everyday machines to celestial motions.

\challenge{For further study, consider exploring gyroscopic motion, rotating reference frames, and advanced orbital mechanics topics.}

\begin{tieredquestions}{Advanced}
\begin{enumerate}
  \item Investigate and discuss the implications of gravitational field variations on satellite orbits.
  \item Explore current research on gravitational wave detection and its impact on astrophysics.
\end{enumerate}
\end{tieredquestions}