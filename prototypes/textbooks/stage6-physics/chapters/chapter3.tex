\chapter{Waves \& Thermodynamics}

Waves and thermodynamics underpin much of our daily experience, from the sound of music and communication technology to energy transfer in heating and cooling systems. Understanding these phenomena provides insight into both the natural world and modern technological advances.

This chapter explores the fundamental properties of waves, introduces sound and electromagnetic waves, and provides an overview of basic thermodynamics concepts such as temperature, heat transfer, and thermal energy. We will examine these topics through clear explanations, practical investigations, and challenging exercises to deepen your understanding and scientific thinking.

\section{The Nature of Waves}
\FloatBarrier

\subsection{What is a Wave?}
\FloatBarrier

Waves are disturbances that transfer energy from one location to another without transporting matter. Waves can propagate through various media or even through empty space.

\begin{marginfigure}[0pt]
% placeholder for wave propagation diagram
\caption{Wave propagation transfers energy without moving matter.}
\end{marginfigure}

\begin{keyconcept}{Wave Basics}
A \keyword{wave} is a disturbance transferring energy through a medium or space, characterized by properties like wavelength, frequency, amplitude, and speed.
\end{keyconcept}

\historylink{The concept of waves has been studied since antiquity, with early Greek philosophers observing water waves to understand natural phenomena.}

\subsection{Properties of Waves}
\FloatBarrier

To describe waves quantitatively, we define several key properties:

\begin{itemize}
\item \keyword{Amplitude} ($A$): Maximum displacement from equilibrium, related to wave energy.
\item \keyword{Wavelength} ($\lambda$): Distance between successive identical points (e.g., two crests).
\item \keyword{Frequency} ($f$): Number of waves passing a point per second, measured in Hertz (Hz).
\item \keyword{Period} ($T$): Time taken for one complete wave cycle, where \( T = \frac{1}{f} \).
\item \keyword{Wave speed} ($v$): Speed at which the wave propagates, given by the wave equation:
\[
v = f\lambda
\]
\mathlink{This relationship is fundamental and will be applied extensively throughout your study of physics.}
\end{itemize}

\begin{stopandthink}
If you double the frequency of a wave, what happens to its wavelength (assuming the speed remains constant)?
\end{stopandthink}

\begin{example}
A wave has a frequency of 60 Hz and a wavelength of 0.5 m. Find the speed of this wave.

\textbf{Solution:}
Using the wave equation:
\[
v = f\lambda = (60\,\text{Hz})(0.5\,\text{m}) = 30\,\text{m/s}
\]
Therefore, the wave speed is \(30\,\text{m/s}\).
\end{example}

\begin{investigation}{Measuring Wave Speed in a Slinky}
\textbf{Aim:} Determine the speed of transverse waves along a slinky.

\textbf{Method:}
\begin{enumerate}
\item Stretch a slinky along a flat surface, measuring the total length.
\item Generate a pulse wave by quickly moving one end of the slinky sideways.
\item Use a stopwatch to measure the time taken for the pulse to travel from one end to the other.
\item Repeat several times for accuracy.
\end{enumerate}

\textbf{Analysis:}
Calculate the average time and use the equation:
\[
v = \frac{\text{distance}}{\text{time}}
\]

\textbf{Discussion:}
Discuss possible sources of error and how wave speed might vary under different conditions.
\end{investigation}

\begin{tieredquestions}{Basic}
\begin{enumerate}
\item Define amplitude, wavelength, and frequency.
\item A wave has frequency 10 Hz and wavelength 2 m. Calculate its wave speed.
\end{enumerate}
\end{tieredquestions}

\begin{tieredquestions}{Intermediate}
\begin{enumerate}
\item Explain how amplitude relates to wave energy.
\item If a wave travels at 300 m/s with a wavelength of 1.5 m, determine its frequency.
\end{enumerate}
\end{tieredquestions}

\begin{tieredquestions}{Advanced}
\begin{enumerate}
\item Derive the relationship between period and frequency mathematically.
\item Discuss how wave properties might change when transitioning from one medium to another.
\end{enumerate}
\end{tieredquestions}

\FloatBarrier

\section{Sound and Electromagnetic Waves}
\FloatBarrier

\subsection{Sound Waves}
\FloatBarrier

Sound waves are mechanical, longitudinal waves produced by vibrating objects and require a medium (solid, liquid, or gas) to travel. The speed of sound varies depending on medium density and temperature.

\begin{keyconcept}{Longitudinal Waves}
In a \keyword{longitudinal wave}, particle oscillations occur parallel to the wave propagation direction. Sound waves are a primary example.
\end{keyconcept}

\historylink{Robert Boyle demonstrated that sound requires a medium for transmission through his bell-in-a-vacuum experiment in 1660.}

\begin{stopandthink}
Why can't astronauts communicate verbally in space without radio equipment?
\end{stopandthink}

\subsection{Electromagnetic Waves}
\FloatBarrier

Electromagnetic waves (EM waves) differ significantly from sound waves. They do not require a medium and travel at the speed of light (\(c \approx 3.0\times10^8\,\text{m/s}\)) in vacuum.

\begin{marginfigure}[0pt]
% placeholder for electromagnetic spectrum diagram
\caption{Electromagnetic spectrum showing wave types from radio waves to gamma rays.}
\end{marginfigure}

\begin{keyconcept}{Electromagnetic Spectrum}
The electromagnetic spectrum encompasses a range of EM waves varying by wavelength and frequency, including radio waves, microwaves, infrared, visible light, ultraviolet, X-rays, and gamma rays.
\end{keyconcept}

\challenge{Explore the use of electromagnetic waves in modern technologies like wireless communication, medical imaging, and astronomy.}

\begin{investigation}{Measuring the Speed of Sound}
\textbf{Aim:} Experimentally determine the speed of sound in air.

\textbf{Method:}
\begin{enumerate}
\item Position two students a known distance apart (e.g., 100 m).
\item One student bangs two wooden blocks together; the second measures time delay between seeing the action and hearing the sound.
\item Repeat and average results.
\end{enumerate}

\textbf{Analysis:}
Calculate sound speed using \(v = d/t\), considering possible reaction time errors.

\textbf{Extension:}
Investigate how temperature affects sound speed.
\end{investigation}

\FloatBarrier

\section{Introduction to Thermodynamics}
\FloatBarrier

Thermodynamics studies energy transformations involving heat and temperature. These concepts underpin processes from weather systems to engines and refrigeration.

\subsection{Temperature vs. Thermal Energy}
\FloatBarrier

Temperature measures the average kinetic energy of particles in a substance. Thermal energy depends on temperature, particle number, and material type.

\begin{keyconcept}{Thermal Energy}
\keyword{Thermal energy} is the internal energy due to particle motion, dependent on temperature, mass, and specific heat capacity.
\end{keyconcept}

\begin{stopandthink}
Which contains more thermal energy: a cup of water at 90°C or a swimming pool at 30°C? Explain your reasoning.
\end{stopandthink}

\subsection{Heat Transfer Mechanisms}
\FloatBarrier

Heat transfer occurs via three mechanisms: conduction, convection, and radiation.

\begin{itemize}
\item \keyword{Conduction}: Heat transfer through direct particle collisions, common in solids.
\item \keyword{Convection}: Heat transfer through fluid motion (liquids or gases), driven by density differences.
\item \keyword{Radiation}: Transfer of heat through electromagnetic waves, requiring no medium.
\end{itemize}

\mathlink{Fourier's Law mathematically describes conductive heat flow, while Stefan-Boltzmann law applies to radiation heat transfer.}

\begin{investigation}{Comparing Heat Transfer Methods}
Design an experiment to compare heat transfer efficiency through conduction, convection, and radiation.
\end{investigation}

\begin{tieredquestions}{Advanced}
\begin{enumerate}
\item Describe mathematically how thermal equilibrium is achieved between two substances at different temperatures.
\item Relate thermodynamics principles to real-world applications such as climate control systems and engines.
\end{enumerate}
\end{tieredquestions}

\FloatBarrier

This chapter has provided foundational understanding of waves and thermodynamics essential for future physics studies and real-world applications.