\chapter{Electromagnetism}

\section{Introduction to Electromagnetism}
\FloatBarrier

Electromagnetism is one of the cornerstones of physics, influencing almost every aspect of our modern technological world. From powering our homes and communication systems to enabling medical imaging techniques such as MRI, electromagnetic phenomena are ubiquitous and indispensable.

\begin{marginfigure}[0pt]
\includegraphics{placeholder_electric_grid.jpg}
\caption{Electrical grids rely heavily on electromagnetism principles.}
\end{marginfigure}

In this chapter, we explore electromagnetic induction, examine the workings of transformers, contrast AC and DC power distribution, and introduce Maxwell's equations qualitatively.

\section{Electromagnetic Induction}
\FloatBarrier

\subsection{Faraday's Law of Induction}
\FloatBarrier

In 1831, Michael Faraday discovered that a changing magnetic field could induce an electric current in a conductor. This phenomenon is called \keyword{electromagnetic induction}.

\historylink{Faraday's meticulous experimentation laid the foundation for electrical power generation.}

\begin{keyconcept}{Faraday's Law}
The induced electromotive force (emf), \(\varepsilon\), in a circuit is directly proportional to the rate of change of magnetic flux, \(\Phi_B\), through the circuit:
\[
\varepsilon = -\frac{d\Phi_B}{dt}
\]
\end{keyconcept}

The negative sign indicates the direction of the induced emf according to Lenz's law.

\begin{stopandthink}
How does increasing the speed of a magnet moving through a coil affect the magnitude of the induced emf?
\end{stopandthink}

\subsection{Lenz's Law and Conservation of Energy}
\FloatBarrier

\begin{keyconcept}{Lenz's Law}
The direction of the induced current is such that it opposes the change in magnetic flux causing it.
\end{keyconcept}

\mathlink{Lenz's law is a direct consequence of the law of conservation of energy.}

\begin{example}
A bar magnet approaches a coil with its north pole first. According to Lenz's law, the coil will induce a current creating a north pole facing the magnet, thereby opposing the magnet's approach.
\end{example}

\begin{investigation}{Observing Electromagnetic Induction}
\begin{enumerate}
    \item Connect a galvanometer to a coil.
    \item Quickly insert and remove a bar magnet into the coil.
    \item Observe and record galvanometer deflection.
    \item Repeat with different speeds and orientations of the magnet.
\end{enumerate}
Explain your observations using Faraday's and Lenz's laws.
\end{investigation}

\begin{tieredquestions}{Basic}
\begin{enumerate}
    \item State Faraday's law of electromagnetic induction in your own words.
    \item What does Lenz's law tell us about the direction of induced currents?
\end{enumerate}
\end{tieredquestions}

\begin{tieredquestions}{Intermediate}
\begin{enumerate}
    \item Explain how the negative sign in Faraday's law relates to Lenz's law.
    \item Describe the energy transformations occurring during electromagnetic induction.
\end{enumerate}
\end{tieredquestions}

\begin{tieredquestions}{Advanced}
\begin{enumerate}
    \item Derive an expression for induced emf in a coil rotating within a uniform magnetic field.
    \item Research and describe an application of electromagnetic induction in modern technology, such as regenerative braking systems.
\end{enumerate}
\end{tieredquestions}

\FloatBarrier

\section{Transformers and Power Distribution}
\FloatBarrier

\subsection{Principles of Transformers}
\FloatBarrier

Transformers are devices that use electromagnetic induction to increase or decrease alternating voltage levels efficiently.

\begin{marginfigure}[0pt]
\includegraphics{placeholder_transformer_diagram.jpg}
\caption{A basic transformer consists of primary and secondary coils wrapped around an iron core.}
\end{marginfigure}

The transformer equation is given by:
\[
\frac{V_p}{V_s} = \frac{N_p}{N_s}
\]

\challenge{Investigate the efficiency of transformers and factors affecting energy loss.}

\begin{keyconcept}{Step-up and Step-down Transformers}
A step-up transformer increases voltage (more secondary than primary winding turns), while a step-down transformer decreases voltage (fewer secondary turns than primary).
\end{keyconcept}

\subsection{Alternating Current (AC) vs. Direct Current (DC)}
\FloatBarrier

Alternating current periodically reverses direction, while direct current flows steadily in one direction.

\historylink{The debate between AC and DC distribution systems—known as the "War of Currents"—involved inventors Thomas Edison (DC) and Nikola Tesla (AC).}

AC allows efficient voltage transformation and transmission over long distances, reducing energy loss.

\begin{stopandthink}
Why is AC rather than DC predominantly used for power distribution in modern electrical grids?
\end{stopandthink}

\begin{investigation}{Building a Simple Transformer}
Construct a transformer using insulated copper wire, an iron core, and an AC power supply.
\begin{enumerate}
    \item Measure primary and secondary voltages.
    \item Change the number of windings and repeat measurements.
    \item Discuss your results in terms of transformer theory.
\end{enumerate}
\end{investigation}

\begin{tieredquestions}{Basic}
\begin{enumerate}
    \item Explain the main purpose of a transformer.
    \item What differentiates AC from DC current?
\end{enumerate}
\end{tieredquestions}

\begin{tieredquestions}{Intermediate}
\begin{enumerate}
    \item Calculate the number of turns required to step down 240\,V to 12\,V, given the primary coil has 600 turns.
    \item Discuss energy losses in transformers and suggest methods to reduce these losses.
\end{enumerate}
\end{tieredquestions}

\begin{tieredquestions}{Advanced}
\begin{enumerate}
    \item Analyze the impact transformers have had on the development of global electric power infrastructure.
    \item Critically evaluate the potential of high-voltage DC (HVDC) transmission as an alternative to AC transmission.
\end{enumerate}
\end{tieredquestions}

\FloatBarrier

\section{Introducing Maxwell's Equations}
\FloatBarrier

Maxwell's equations elegantly summarize electromagnetism, describing the relationships between electric and magnetic fields.

\challenge{Maxwell's equations unified electricity, magnetism, and optics, setting the stage for Einstein's special relativity.}

\subsection{Qualitative Overview of Maxwell's Equations}
\FloatBarrier

\begin{enumerate}
    \item \textbf{Gauss's Law for Electricity}: Electric charges produce electric fields.
    \item \textbf{Gauss's Law for Magnetism}: Magnetic monopoles do not exist; magnetic field lines form closed loops.
    \item \textbf{Faraday's Law of Induction}: A changing magnetic field induces an electric field.
    \item \textbf{Ampère-Maxwell Law}: Electric currents and changing electric fields produce magnetic fields.
\end{enumerate}

\mathlink{These equations are typically expressed mathematically using vector calculus in advanced physics courses.}

\begin{stopandthink}
How did Maxwell's contribution help scientists understand the nature of light?
\end{stopandthink}

\begin{tieredquestions}{Basic}
\begin{enumerate}
    \item List Maxwell's equations and briefly describe what each represents.
\end{enumerate}
\end{tieredquestions}

\begin{tieredquestions}{Intermediate}
\begin{enumerate}
    \item Explain the significance of Maxwell adding the displacement current to Ampère's law.
\end{enumerate}
\end{tieredquestions}

\begin{tieredquestions}{Advanced}
\begin{enumerate}
    \item Research and outline the experimental verification of electromagnetic waves predicted by Maxwell.
    \item Discuss the role Maxwell's equations played in the development of modern communication technologies.
\end{enumerate}
\end{tieredquestions}

\FloatBarrier

\section{Chapter Summary}
\FloatBarrier

Electromagnetism is pivotal in understanding modern technological society. Mastery of electromagnetic induction, transformers, and Maxwell's equations forms an essential foundation for future studies in physics and engineering.

\begin{marginfigure}[0pt]
\includegraphics{placeholder_em_spectrum.jpg}
\caption{Electromagnetism encompasses a broad spectrum of phenomena, including visible light, radio waves, and x-rays.}
\end{marginfigure}

\section{Further Reading and Research}
\FloatBarrier

Expand your understanding by exploring contemporary applications such as wireless power transfer, MRI technology, and renewable energy systems harnessing electromagnetic principles.

\FloatBarrier