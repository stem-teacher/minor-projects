\chapter{Electricity \& Magnetism}

\section{Introduction to Electricity and Magnetism}
\FloatBarrier

Electricity and magnetism are fundamental forces that shape the universe we live in. From powering our homes and gadgets to guiding birds in migration, these phenomena play a crucial role in natural and technological contexts. In this module, we explore how electric charges, electric currents, and magnetic fields interact, laying the groundwork for understanding devices such as electric motors, generators, and modern electronic circuits.

\historylink{Danish physicist Hans Christian Ørsted first observed the interaction between electricity and magnetism in 1820, bridging two previously separate fields.}

\keyword{Electricity} refers to phenomena associated with electric charges, while \keyword{magnetism} is the force exerted by magnets and magnetic fields. We will explore these interconnected concepts through theory, practical examples, and scientific inquiry.

\section{Electric Fields and Forces}
\FloatBarrier

\subsection{Electric Charge and Coulomb's Law}
\FloatBarrier

All matter is composed of atoms, which in turn consist of charged particles: protons (positive charge), electrons (negative charge), and neutrons (neutral). Electric charge is the fundamental property driving electric phenomena.

\begin{keyconcept}{Coulomb's Law}
The electric force ($F$) between two point charges ($q_1$ and $q_2$) separated by a distance ($r$) is given by:
\[
F = k\frac{q_1 q_2}{r^2}
\]
where $k = 8.99 \times 10^9~\text{Nm}^2/\text{C}^2$.
\end{keyconcept}

\challenge{In advanced electromagnetism, Coulomb's Law is generalized using Gauss's Law and Maxwell's equations.}

\begin{stopandthink}
If the distance between two charges doubles, how does the electric force between them change?
\end{stopandthink}

\subsection{Electric Fields}
\FloatBarrier

An \keyword{electric field} describes how an electric charge affects the space around it. The electric field ($E$) at a point is defined as the force per unit positive charge:
\[
E = \frac{F}{q}
\]

\begin{example}
Calculate the strength of the electric field 3.0 m away from a point charge of $+5.0\,\mu\text{C}$.
\end{example}

\challenge{Electric fields can also be visualized using field lines. The number of lines per unit area indicates field strength.}

\subsection{Investigation: Mapping Electric Fields}
\FloatBarrier

\begin{investigation}{Electric Field Patterns}
Use conductive paper, electrodes, and a voltmeter to map electric field patterns around different charged objects. Sketch the patterns observed and compare to theoretical predictions.
\end{investigation}

\begin{tieredquestions}{Basic}
\begin{enumerate}
    \item What is the electric field direction around a positive charge?
    \item Calculate the force between charges of $+3.0\,\mu\text{C}$ and $-4.0\,\mu\text{C}$ separated by 0.05 m.
\end{enumerate}
\end{tieredquestions}

\begin{tieredquestions}{Intermediate}
\begin{enumerate}
    \item Explain how the concept of electric fields simplifies the description of electric forces.
    \item If the electric field at a point is $2000~\text{N/C}$, what force would act on an electron placed at this point?
\end{enumerate}
\end{tieredquestions}

\begin{tieredquestions}{Advanced}
\begin{enumerate}
    \item Derive Coulomb's law from Gauss's law for a single point charge.
    \item Discuss the significance of field lines density and direction.
\end{enumerate}
\end{tieredquestions}

\FloatBarrier

\section{Electric Current, Voltage, and Resistance}
\FloatBarrier

\subsection{Understanding Current and Voltage}
\FloatBarrier

An electric current ($I$) is the rate of flow of electric charge. It is measured in amperes (A), where $1\,\text{A} = 1\,\text{C}/\text{s}$.

\begin{keyconcept}{Voltage}
Voltage ($V$), or potential difference, is the energy transferred per unit charge between two points in a circuit:
\[
V = \frac{W}{q}
\]
where $W$ is energy in joules (J), and $q$ is charge in coulombs (C).
\end{keyconcept}

\subsection{Resistance and Ohm's Law}
\FloatBarrier

Resistance ($R$) measures how strongly a material opposes electric current. It is measured in ohms ($\Omega$).

\begin{keyconcept}{Ohm's Law}
Ohm's law relates voltage ($V$), current ($I$), and resistance ($R$):
\[
V = IR
\]
\end{keyconcept}

\historylink{Georg Simon Ohm (1789-1854) formulated this relationship empirically in 1827.}

\begin{stopandthink}
If voltage across a resistor remains constant and resistance doubles, what happens to the current?
\end{stopandthink}

\subsection{Circuit Analysis}
\FloatBarrier

Electric circuits provide pathways for current to flow. Circuits can be series, parallel, or a combination of both.

\begin{example}
Calculate the total resistance and current in a circuit with three resistors of $2\,\Omega$, $3\,\Omega$, and $6\,\Omega$ connected in series with a $12\,\text{V}$ battery.
\end{example}

\subsection{Investigation: Ohm's Law Experiment}
\FloatBarrier

\begin{investigation}{Verifying Ohm's Law}
Set up a simple circuit with a variable resistor, ammeter, voltmeter, and power supply. Measure current and voltage across the resistor at different resistances. Graph the results and verify Ohm’s law.
\end{investigation}

\begin{tieredquestions}{Basic}
\begin{enumerate}
    \item Define electric current and voltage.
    \item What is the resistance of a lightbulb if a $3\,\text{V}$ battery causes $0.5\,\text{A}$ of current to flow?
\end{enumerate}
\end{tieredquestions}

\begin{tieredquestions}{Intermediate}
\begin{enumerate}
    \item Explain why voltage is also called potential difference.
    \item Calculate the equivalent resistance of two resistors ($4\,\Omega$ and $6\,\Omega$) in parallel.
\end{enumerate}
\end{tieredquestions}

\begin{tieredquestions}{Advanced}
\begin{enumerate}
    \item Derive the equations for equivalent resistance in series and parallel circuits from first principles.
    \item Analyze the factors affecting resistance of conducting wires, including temperature and material.
\end{enumerate}
\end{tieredquestions}

\FloatBarrier

\section{Magnetic Fields and Electromagnetism}
\FloatBarrier

\subsection{Magnetic Fields}
\FloatBarrier

A \keyword{magnetic field} is a region where magnetic forces are felt. Fields originate from magnetic poles and moving electric charges.

\begin{keyconcept}{Magnetic Field Around a Current-Carrying Conductor}
A current-carrying conductor generates a circular magnetic field around itself, described by the right-hand grip rule.
\end{keyconcept}

\subsection{Electromagnetism Fundamentals}
\FloatBarrier

Electromagnetism describes the interaction between electric currents and magnetic fields.

\begin{investigation}{Magnetic Field Around a Wire}
Use a compass and current-carrying wire to explore magnetic field direction and strength around straight conductors and coils.
\end{investigation}

\subsection{DC Electric Motors and Induction}
\FloatBarrier

A \keyword{DC motor} converts electrical energy to mechanical energy by exploiting magnetic forces acting on current-carrying coils.

\begin{keyconcept}{Electromagnetic Induction}
Electromagnetic induction is the generation of voltage (emf) across a conductor moving through a magnetic field:
\[
\varepsilon = B l v \sin\theta
\]
\end{keyconcept}

\challenge{Explore Faraday's Law and Lenz's Law for deeper understanding of induction.}

\begin{stopandthink}
How does changing the speed of the coil's rotation affect the induced voltage?
\end{stopandthink}

\historylink{Michael Faraday discovered electromagnetic induction in 1831, paving the way for electricity generation.}

\FloatBarrier

% The remainder of the chapter continues similarly, including advanced exploration and mathematics, structured inquiry, and differentiated tasks to support and challenge diverse learners.