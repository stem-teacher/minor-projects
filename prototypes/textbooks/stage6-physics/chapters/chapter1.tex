\chapter{Kinematics}
\label{ch:kinematics}

\newthought{Motion is fundamental} to understanding the physical world. From analyzing the trajectory of a spacecraft to understanding the mechanics behind athletes' movements, kinematics plays a central role in our interpretation of reality. In this chapter, we explore the concepts of displacement, velocity, acceleration, graphical representations, and the critical distinction between vectors and scalars.  

\section{Describing Motion}
\FloatBarrier

Motion is everywhere. Cars driving, birds flying, and planets orbiting are all examples of objects in motion. But how exactly do physicists describe and quantify motion?

\subsection{Scalars and Vectors}
\FloatBarrier

Physics uses two distinct types of measurements: \keyword{scalars} and \keyword{vectors}.

\begin{keyconcept}{Scalars and Vectors}
\begin{description}
    \item[Scalars] Quantities that have magnitude only, such as temperature, mass, and speed.
    \item[Vectors] Quantities that have both magnitude and direction, such as velocity, displacement, and acceleration.
\end{description}
\end{keyconcept}

\marginnote[\baselineskip]{\textbf{Scalar examples:} 
\begin{itemize}
    \item Mass: $3\,\mathrm{kg}$
    \item Speed: $50\,\mathrm{km\,h^{-1}}$
\end{itemize}}

\marginnote[\baselineskip]{\textbf{Vector examples:} 
\begin{itemize}
    \item Velocity: $50\,\mathrm{km\,h^{-1}}$ north
    \item Displacement: $10\,\mathrm{m}$ east
\end{itemize}}

\begin{stopandthink}
Classify these quantities as scalar or vector: distance, force, energy, momentum, temperature.
\end{stopandthink}

\subsection{Displacement and Distance}
\FloatBarrier

\keyword{Distance} is the total path length travelled, whereas \keyword{displacement} is the shortest straight-line path from initial to final position, including direction.

\begin{example}
Sophia walks $3\,\mathrm{km}$ east, then $4\,\mathrm{km}$ north. Calculate her displacement and distance travelled.
\end{example}

\begin{solution}
Distance travelled = $3 + 4 = 7\,\mathrm{km}$.

Displacement is a vector quantity:
\[
\text{Displacement magnitude} = \sqrt{3^2 + 4^2} = 5\,\mathrm{km},\quad \text{Direction: Northeast (53.1° from east)}
\]
\end{solution}

\begin{stopandthink}
If you walk around a circular track of circumference $400\,\mathrm{m}$ exactly once, what is your displacement?
\end{stopandthink}

\FloatBarrier

\subsection{Speed and Velocity}
\FloatBarrier

\keyword{Speed} is the rate of change of distance travelled, while \keyword{velocity} is the rate of change of displacement.

\begin{marginfigure}[0pt]
% figure placeholder: speedometer vs. compass+speed illustration
\caption{Speed indicates how fast, while velocity includes direction.}
\end{marginfigure}

\begin{keyconcept}{Average vs Instantaneous Velocity}
\begin{itemize}
    \item \textbf{Average Velocity:} $\vec{v}_{avg} = \frac{\Delta \vec{s}}{\Delta t}$
    \item \textbf{Instantaneous Velocity:} velocity at a specific instant or time interval approaching zero.
\end{itemize}
\end{keyconcept}

\begin{example}
A car travels $150\,\mathrm{km}$ north in $3$ hours, then $50\,\mathrm{km}$ south in $1$ hour. Calculate the average velocity.
\end{example}

\begin{solution}
Net displacement = $150 - 50 = 100\,\mathrm{km}$ north. Total time = $4\,\mathrm{h}$.

Average velocity = $\frac{100\,\mathrm{km}}{4\,\mathrm{h}} = 25\,\mathrm{km\,h^{-1}}$ north.
\end{solution}

\begin{stopandthink}
Can an object have zero velocity but non-zero speed? Explain clearly.
\end{stopandthink}

\FloatBarrier

\subsection{Acceleration}
\FloatBarrier

\keyword{Acceleration} measures the rate at which velocity changes with time.

\begin{keyconcept}{Acceleration}
Defined mathematically as:
\[
\vec{a} = \frac{\Delta \vec{v}}{\Delta t}
\]
Units: $\mathrm{m\,s^{-2}}$
\end{keyconcept}

\challenge{Acceleration can be positive, negative, or directional. Negative acceleration does not necessarily mean slowing down; it depends on the direction chosen as positive.}

\begin{example}
A train increases velocity uniformly from rest to $20\,\mathrm{m\,s^{-1}}$ in $10\,\mathrm{s}$. Find its acceleration.
\end{example}

\begin{solution}
Initial velocity $u = 0$; final velocity $v = 20\,\mathrm{m\,s^{-1}}$; time $t = 10\,\mathrm{s}$.

Acceleration $a = \frac{v - u}{t} = \frac{20 - 0}{10} = 2\,\mathrm{m\,s^{-2}}$.
\end{solution}

\begin{stopandthink}
If velocity is constant, what is the acceleration? Justify mathematically.
\end{stopandthink}

\FloatBarrier

\begin{tieredquestions}{Basic}
\begin{enumerate}
    \item Define scalar and vector quantities. Give two examples each.
    \item A cyclist travels $12\,\mathrm{km}$ east and then $5\,\mathrm{km}$ west. What is the cyclist's displacement?
\end{enumerate}
\end{tieredquestions}

\begin{tieredquestions}{Intermediate}
\begin{enumerate}
    \item A runner completes one lap around a $400\,\mathrm{m}$ track in $50\,\mathrm{s}$. Determine average speed and average velocity.
    \item Differentiate clearly between instantaneous velocity and average velocity.
\end{enumerate}
\end{tieredquestions}

\begin{tieredquestions}{Advanced}
A car accelerates uniformly from rest at $3\,\mathrm{m\,s^{-2}}$ for $4\,\mathrm{s}$, then continues at constant velocity for $10\,\mathrm{s}$, and finally decelerates uniformly to rest in $5\,\mathrm{s}$. Calculate its total displacement.
\end{tieredquestions}

\FloatBarrier

\section{Graphical Analysis of Motion}
\FloatBarrier

Graphs provide an intuitive visual representation of motion. The two primary graphs used in kinematics are displacement-time and velocity-time graphs.

\subsection{Displacement-Time Graphs}
\FloatBarrier

\begin{keyconcept}{Displacement-Time Graphs}
\begin{itemize}
    \item Gradient (slope) represents velocity.
    \item A straight line indicates uniform velocity.
    \item Curvature indicates acceleration (changing velocity).
\end{itemize}
\end{keyconcept}

\begin{investigation}{Plotting Your Motion}
Using a stopwatch and measuring tape, record your displacement every second as you walk at a steady pace. Plot a displacement-time graph. Repeat for a fast walk or run. Compare the slopes and discuss the differences.
\end{investigation}

\challenge{What would a vertical line on a displacement-time graph represent? Is this physically possible?}

\FloatBarrier

\subsection{Velocity-Time Graphs}
\FloatBarrier

\begin{keyconcept}{Velocity-Time Graphs}
\begin{itemize}
    \item Gradient represents acceleration.
    \item Area under the graph represents displacement.
\end{itemize}
\end{keyconcept}

\begin{stopandthink}
Describe the motion represented by a horizontal line below the time axis on a velocity-time graph.
\end{stopandthink}

\begin{investigation}{Measuring Acceleration}
Use a trolley on an inclined track and motion sensor to produce a velocity-time graph. Determine acceleration from the gradient and displacement by calculating the area under the graph.
\end{investigation}

\FloatBarrier

\begin{tieredquestions}{Intermediate}
Given a velocity-time graph:
\begin{enumerate}
    \item Describe how you would find displacement.
    \item Explain how acceleration is determined.
\end{enumerate}
\end{tieredquestions}

\begin{tieredquestions}{Advanced}
Sketch a velocity-time graph for an object thrown vertically upwards and returning to its original position. Label key points clearly, explaining changes in velocity, acceleration, and displacement.
\end{tieredquestions}

\historylink{Galileo Galilei pioneered graphical representations by plotting distance-time relationships in his experiments on acceleration.}

\FloatBarrier

This concludes our exploration of kinematics fundamentals. Mastery in interpreting and analyzing motion is foundational for understanding more complex physical phenomena that we will encounter throughout this course.