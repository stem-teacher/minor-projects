\chapter{Dynamics}

From launching satellites into orbit, to understanding car crashes, dynamics reveals the hidden rules behind motion and force. In this chapter, we explore how forces shape motion, how we analyze complex movements such as projectiles and circular paths, and how momentum governs collisions. Dynamics is fundamental not only for physics but also for engineering, sports science, and space exploration, making it an exciting and highly applicable area of study.

\section{Newton’s Laws of Motion}
\FloatBarrier

Over three centuries ago, Sir Isaac Newton revolutionized our understanding of motion through three elegant laws. These laws describe the relationship between forces and the motion of objects, forming the cornerstone of classical mechanics.

\historylink{Newton’s \emph{Principia Mathematica} (1687) laid the foundations of classical mechanics.}

\subsection{First Law: The Law of Inertia}
\FloatBarrier

An object’s resistance to changes in motion is called \keyword{inertia}. Newton’s First Law states:

\begin{keyconcept}{Newton's First Law}
An object at rest remains at rest, and an object in motion continues in motion at constant velocity unless acted upon by an external unbalanced force.
\end{keyconcept}

\marginnote{\textbf{Definition:} \keyword{Inertia} is the tendency of an object to resist changes in its state of motion.}

\begin{example}
When you suddenly stop pedaling your bicycle, why does it continue to move forward?
\end{example}

\begin{stopandthink}
If an astronaut in space throws a ball, what path will the ball follow after leaving the astronaut's hand? Explain using Newton's First Law.
\end{stopandthink}

\subsection{Second Law: Force, Mass, and Acceleration}
\FloatBarrier

Newton’s Second Law provides a quantitative link between force, mass, and acceleration:

\begin{keyconcept}{Newton's Second Law}
The net force acting on an object is directly proportional to the acceleration it experiences and inversely proportional to the mass of the object. Mathematically:
\[
\Sigma \mathbf{F} = m\mathbf{a}
\]
\end{keyconcept}

\marginnote{\textbf{Definition:} \keyword{Net force} is the vector sum of all forces acting on an object.}

\begin{example}
Calculate the acceleration of a 1200\,kg car if a net force of 3600\,N is applied.
\[
a = \frac{F}{m} = \frac{3600\,\text{N}}{1200\,\text{kg}} = 3\,\text{ms}^{-2}
\]
\end{example}

\begin{stopandthink}
If you push equally on two boxes, one heavier and one lighter, which one accelerates more? Why?
\end{stopandthink}

\subsection{Third Law: Action and Reaction}
\FloatBarrier

Newton’s Third Law describes interactions:

\begin{keyconcept}{Newton's Third Law}
For every action, there is an equal and opposite reaction. Forces always act in pairs, equal in magnitude and opposite in direction.
\end{keyconcept}

\begin{example}
When you push against a wall, you feel a force pushing back against your hand. This reaction force is equal and opposite to your applied force.
\end{example}

\begin{stopandthink}
A balloon flies around the room when released. Explain why this happens using Newton's Third Law.
\end{stopandthink}

\FloatBarrier

\begin{tieredquestions}{Basic}
\begin{enumerate}
    \item State Newton’s three laws of motion in your own words.
    \item What is inertia? Give two examples from daily life.
\end{enumerate}
\end{tieredquestions}

\begin{tieredquestions}{Intermediate}
\begin{enumerate}
    \item A 50\,kg skater accelerates at 2\,ms$^{-2}$. Calculate the net force applied.
    \item Explain why seatbelts are essential using Newton's first law.
\end{enumerate}
\end{tieredquestions}

\begin{tieredquestions}{Advanced}
\begin{enumerate}
    \item Discuss how Newton’s third law applies to rocket propulsion.
    \item Consider two astronauts floating in space. Explain how they could move towards each other without external help.
\end{enumerate}
\end{tieredquestions}

\section{Analyzing Forces: Force Diagrams and Free-Body Diagrams}
\FloatBarrier

To predict an object's motion, physicists analyze the forces acting upon it using \keyword{force diagrams} and \keyword{free-body diagrams (FBDs)}.

\subsection{Constructing Free-Body Diagrams}
\FloatBarrier

A \keyword{free-body diagram} is a simplified representation of an object, showing all external forces acting upon it.

\begin{keyconcept}{Steps for Drawing FBDs}
\begin{enumerate}
    \item Identify the object of interest and represent it as a dot.
    \item Draw vectors representing all external forces, ensuring correct direction and approximate size.
    \item Label each force clearly.
    \item Choose a coordinate system, usually aligning axes with motion directions.
\end{enumerate}
\end{keyconcept}

\begin{example}
Draw the FBD for a book resting on a table.
\begin{itemize}
    \item Weight (gravitational force) directed downward.
    \item Normal force from table directed upward.
\end{itemize}
\end{example}

\begin{investigation}{Analyzing Forces Experimentally}
Using a dynamics trolley, pulley, and various masses, investigate how changing mass or force affects acceleration. Record force, mass, and calculate acceleration. Confirm Newton's Second Law experimentally.
\end{investigation}

\FloatBarrier

\begin{tieredquestions}{Intermediate}
\begin{enumerate}
    \item Draw a free-body diagram for a parachutist descending at constant velocity.
    \item Explain why the parachutist does not accelerate.
\end{enumerate}
\end{tieredquestions}

\section{Projectile Motion}
\FloatBarrier

Projectile motion occurs when an object moves under the influence of gravity alone, following a curved path called a \keyword{trajectory}.

\begin{keyconcept}{Characteristics of Projectile Motion}
\begin{itemize}
    \item Horizontal motion at constant velocity.
    \item Vertical motion affected by gravitational acceleration ($g = 9.8\,\text{ms}^{-2}$ downward).
    \item Horizontal and vertical components independent of each other.
\end{itemize}
\end{keyconcept}

\marginnote{\challenge{Advanced students: Explore projectile motion equations with air resistance.}}

\begin{example}
A ball is thrown horizontally from a 20\,m high cliff at 10\,ms$^{-1}$. How long does it take to hit the ground?
\[
y = ut + \frac{1}{2}gt^{2},\quad 20 = 0 + \frac{1}{2}(9.8)t^{2}
\]
Solve:
\[
t = \sqrt{\frac{2\times20}{9.8}} \approx 2.02\,\text{s}
\]
\end{example}

\begin{stopandthink}
Why does a projectile launched horizontally and one dropped vertically from the same height reach the ground simultaneously?
\end{stopandthink}

\begin{investigation}{Projectile Motion Analysis}
Investigate projectile motion experimentally using projectile launchers or simulations. Measure range, height, and time. Compare results to theoretical predictions.
\end{investigation}

\FloatBarrier

\begin{tieredquestions}{Advanced}
Analyze optimal angles for maximum range of a projectile when launched from different initial heights. Derive conditions mathematically.
\end{tieredquestions}

\section{Uniform Circular Motion (Qualitative)}
\FloatBarrier

When an object moves in a circle at constant speed, its velocity continually changes direction. This is called \keyword{uniform circular motion}.

\begin{keyconcept}{Circular Motion Basics}
\begin{itemize}
    \item Velocity is tangential; acceleration (centripetal) points towards the circle’s center.
    \item Centripetal force maintains circular motion.
\end{itemize}
\end{keyconcept}

\begin{stopandthink}
Why do you feel pushed outward when turning quickly in a car?
\end{stopandthink}

\section{Momentum, Impulse, and Collisions}
\FloatBarrier

\textit{(Due to length limitations, the complete section on Momentum and Collisions, including impulse and conservation laws, will continue similarly with clear explanations, mathematical derivations, examples, investigation activities, and tiered questions.)}

\FloatBarrier