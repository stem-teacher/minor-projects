\chapter{From the Universe to the Atom}

In our quest to understand the universe, physicists delve deeply into the fundamental building blocks of nature, exploring phenomena on scales from the unimaginably large down to the infinitesimally small. This journey from cosmology—the study of the universe as a whole—to particle physics—the exploration of subatomic particles—reveals an extraordinary interconnectedness of natural laws. Understanding these concepts not only satisfies human curiosity but also leads to crucial technological advancements, including nuclear energy and medical imaging technologies.

In this chapter, we will explore three interconnected domains of modern physics:
\begin{enumerate}
    \item The Standard Model of Particle Physics
    \item Nuclear Physics: Stability, Decay, and Applications
    \item Cosmology and the Origin of the Universe
\end{enumerate}

\section{The Standard Model of Particle Physics}
\FloatBarrier

\marginnote{\historylink{The Standard Model was developed throughout the 20th century, influenced by experimental discoveries and theoretical breakthroughs.}}

For centuries, scientists have sought to identify the smallest building blocks of matter. Today, the most successful theory describing these fundamental particles and their interactions is the \keyword{Standard Model of particle physics}.

\subsection{Fundamental Particles}
\FloatBarrier

The Standard Model classifies known particles into two major categories: \keyword{fermions} and \keyword{bosons}. Fermions, which include \keyword{quarks} and \keyword{leptons}, are the building blocks of matter. Bosons mediate forces between particles.

\begin{keyconcept}{Fermions and Bosons}
\begin{itemize}
    \item Fermions have half-integer spin and follow the Pauli exclusion principle. Examples include electrons and quarks.
    \item Bosons carry integer spin and mediate fundamental forces. Examples include photons and gluons.
\end{itemize}
\end{keyconcept}

\subsection{Quarks and Leptons}
\FloatBarrier

Matter is composed of two types of fermions: quarks and leptons.

\subsubsection{Quarks}

Quarks are fundamental particles that combine to form composite particles like protons and neutrons (collectively known as \keyword{hadrons}). There are six types, or \keyword{flavours}, of quarks, grouped into three pairs:

\begin{itemize}
    \item Up (u) and Down (d)
    \item Charm (c) and Strange (s)
    \item Top (t) and Bottom (b)
\end{itemize}

Each quark flavour has an associated fractional electric charge and other quantum numbers.

\marginnote{\challenge{Quarks exhibit a property known as colour charge, leading to interactions described by quantum chromodynamics (QCD).}}

\subsubsection{Leptons}

Leptons include electrons and neutrinos and similarly occur in three generations:

\begin{itemize}
    \item Electron (e) and Electron neutrino ($\nu_e$)
    \item Muon ($\mu$) and Muon neutrino ($\nu_\mu$)
    \item Tau ($\tau$) and Tau neutrino ($\nu_\tau$)
\end{itemize}

Unlike quarks, leptons do not experience the strong nuclear force.

\begin{stopandthink}
Why are neutrinos difficult to detect experimentally, despite being very common in the universe?
\end{stopandthink}

\subsection{Interactions and Force-Carrying Particles}
\FloatBarrier

The Standard Model describes three fundamental forces through specific force-carrying particles known as gauge bosons:

\begin{itemize}
    \item Electromagnetic force: mediated by photons ($\gamma$)
    \item Weak nuclear force: mediated by W and Z bosons
    \item Strong nuclear force: mediated by gluons (g)
\end{itemize}

Gravity remains outside the Standard Model framework and is described by general relativity.

\begin{keyconcept}{Interaction Strength and Range}
\begin{itemize}
    \item Strong force: strongest, short-range (within atomic nuclei)
    \item Electromagnetic force: infinite range, weaker than strong force
    \item Weak force: responsible for radioactive decay, very short-range
\end{itemize}
\end{keyconcept}

\begin{investigation}{Cloud Chamber Particle Tracks}
Construct a simple cloud chamber to visualize tracks from cosmic rays and radioactive sources. Document and classify particle tracks using known properties of alpha, beta, and gamma radiation.
\end{investigation}

\begin{tieredquestions}{Basic}
\begin{enumerate}
    \item List all known fundamental forces and their corresponding gauge bosons.
    \item Define fermions and bosons. Give two examples of each.
\end{enumerate}
\end{tieredquestions}

\begin{tieredquestions}{Intermediate}
\begin{enumerate}
    \item Explain how quarks combine to form protons and neutrons.
    \item Describe the difference between leptons and quarks in terms of their interactions.
\end{enumerate}
\end{tieredquestions}

\begin{tieredquestions}{Advanced}
\begin{enumerate}
    \item Discuss why researchers believe neutrinos might have mass, despite initial assumptions to the contrary.
    \item Examine the limitations of the Standard Model and describe current research efforts aimed at extending it.
\end{enumerate}
\end{tieredquestions}

\FloatBarrier

\section{Nuclear Physics: Stability, Decay, and Applications}
\FloatBarrier

\subsection{Nuclear Stability}
\FloatBarrier

Atomic nuclei consist of protons and neutrons bound by the strong nuclear force. The stability of a nucleus depends on the balance between this attractive force and electrostatic repulsion among protons.

\begin{keyconcept}{Nuclear Stability and Binding Energy}
A nucleus's stability is determined by its \keyword{binding energy}, the energy required to separate the nucleus into individual nucleons (protons and neutrons).
\end{keyconcept}

\subsection{Radioactive Decay}
\FloatBarrier

Unstable nuclei undergo \keyword{radioactive decay} to achieve stability. Common decay modes include:

\begin{itemize}
    \item Alpha decay ($\alpha$): emission of helium nucleus ($^4_2$He)
    \item Beta decay ($\beta$): neutron-to-proton or proton-to-neutron conversion with electron or positron emission
    \item Gamma decay ($\gamma$): emission of high-energy photons
\end{itemize}

\begin{example}
The alpha decay of uranium-238:
\[
^{238}_{92}\text{U}\rightarrow\,^{234}_{90}\text{Th}+\,^{4}_{2}\text{He}
\]
\end{example}

\begin{stopandthink}
What determines whether a nucleus undergoes alpha decay or beta decay?
\end{stopandthink}

\subsection{Applications of Nuclear Physics}
\FloatBarrier

Nuclear physics has critical applications, including:
\begin{itemize}
    \item Nuclear energy generation
    \item Medical imaging and radiotherapy
    \item Radiometric dating techniques
\end{itemize}

\begin{investigation}{Modelling Radioactive Decay}
Using dice or coins, simulate radioactive decay to understand half-life and exponential decay processes. Collect data and graph the results to determine decay constants and half-lives.
\end{investigation}

\FloatBarrier

\section{Cosmology Basics: The Origin and Evolution of Our Universe}
\FloatBarrier

Cosmology studies the universe's origin, structure, and evolution. The predominant theory describing the universe's history is known as the \keyword{Big Bang theory}.

\subsection{The Big Bang Theory}
\FloatBarrier

The Big Bang theory suggests the universe began approximately 13.8 billion years ago as an extremely hot, dense point, subsequently expanding and cooling.

\subsection{Cosmic Microwave Background Radiation}
\FloatBarrier

Discovered in 1965, the \keyword{cosmic microwave background} (CMB) radiation provides compelling evidence for the Big Bang. It represents the thermal radiation leftover from the universe's hot, dense early state.

\begin{stopandthink}
Why does the CMB radiation appear relatively uniform in all directions?
\end{stopandthink}

\subsection{Expansion of the Universe}
\FloatBarrier

Observations show galaxies are moving away from each other. This expansion is described by Hubble's law:

\[
v = H_0 d
\]

where $v$ is the recession velocity of a galaxy, $d$ is distance, and $H_0$ is the Hubble constant.

\marginnote{\challenge{Current research explores the mysterious accelerating expansion attributed to dark energy, a major unsolved problem in cosmology.}}

\FloatBarrier

\section*{Conclusion}

From the particles that constitute matter to the vastness of the cosmos, physics provides a coherent framework for understanding and investigating the universe. Continued inquiry at both extremes—in particle accelerators and astronomical observatories—promises further insight into the fundamental laws governing reality.

As you reflect on these concepts, consider the profound connections between the microcosm and macrocosm, which continue to inspire scientific exploration and discovery.