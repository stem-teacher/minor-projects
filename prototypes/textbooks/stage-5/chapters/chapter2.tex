\chapter{Atoms, Elements and Compounds}

\section*{Chapter Overview}

\begin{quote}
    In this chapter, we'll explore the fundamental building blocks of matter—atoms, elements, and compounds. Building on your Stage 4 understanding, we'll examine atomic structure in greater detail, investigate patterns in the periodic table, and analyze how atoms combine to form compounds. You'll also learn about chemical bonding and how the arrangement of electrons influences the properties of materials.
\end{quote}

\noindent This chapter aligns with the following NSW Syllabus outcomes:
\begin{itemize}
    \item SC5-16CW: Explains how models, theories and laws about matter have been refined as new scientific evidence becomes available
    \item SC5-17CW: Discusses the importance of chemical reactions in the production of a range of substances, and the influence of society on the development of new materials
\end{itemize}

\begin{stopandthink}
\begin{enumerate}
    \item Draw and label a model of an atom. What particles does it contain and where are they located?
    \item How is the periodic table organized? What patterns can you identify?
    \item What's the difference between an element, a compound, and a mixture?
    \item How do atoms form bonds with other atoms? Can you name different types of bonds?
\end{enumerate}
\end{stopandthink}

\section{The Structure of the Atom}

\newthought{Our modern understanding} of the atom has evolved through centuries of scientific inquiry. Let's examine the current model of atomic structure in greater detail.

\subsection{Subatomic Particles Revisited}

Recall that atoms contain three main types of subatomic particles:

\begin{keyconcept}{Subatomic Particles}
\begin{description}
    \item[Protons] Positively charged particles, located in the nucleus, with a relative mass of 1 atomic mass unit (amu)
    \item[Neutrons] Neutral particles (no charge), located in the nucleus, with a relative mass of 1 amu
    \item[Electrons] Negatively charged particles, located in electron orbitals around the nucleus, with a very small mass (1/1836 of a proton)
\end{description}
\end{keyconcept}

These particles have specific properties:

\begin{center}
\begin{tabular}{|l|c|c|c|}
\hline
\textbf{Particle} & \textbf{Relative Charge} & \textbf{Relative Mass} & \textbf{Location} \\
\hline
Proton & +1 & 1 & Nucleus \\
\hline
Neutron & 0 & 1 & Nucleus \\
\hline
Electron & -1 & 1/1836 & Electron orbitals \\
\hline
\end{tabular}
\end{center}

\historylink{The term "electron" was coined by Irish physicist George Johnstone Stoney in 1891, and J.J. Thomson demonstrated their existence in 1897. Protons were discovered by Ernest Rutherford in 1919, and neutrons by James Chadwick in 1932.}

\subsection{The Quantum Mechanical Model}

At Stage 5, we can explore a more sophisticated model of the atom. The quantum mechanical model describes electrons not as particles orbiting the nucleus like planets around the sun, but rather as existing in "probability clouds" or orbitals.

\begin{keyconcept}{Electron Orbitals}
An electron orbital is a region of space around the nucleus where an electron is likely to be found. These orbitals have distinctive three-dimensional shapes and can hold a maximum of two electrons.
\end{keyconcept}

Key principles of the quantum mechanical model include:

\begin{itemize}
    \item Electrons exist in specific energy levels (shells) around the nucleus
    \item Each energy level contains sublevel orbitals (s, p, d, f) with specific shapes
    \item Electrons fill the lowest energy orbitals first (Aufbau principle)
    \item Each orbital can hold a maximum of two electrons with opposite spins (Pauli exclusion principle)
    \item Electrons occupy orbitals of equal energy singly before pairing up (Hund's rule)
\end{itemize}

\mathlink{The energy of an electron in a hydrogen atom can be calculated using the equation: $E_n = -R_H/n^2$, where $R_H$ is the Rydberg constant (2.18 × 10$^{-18}$ J) and $n$ is the principal quantum number (energy level).}

% \begin{marginfigure}
%     \centering
%     \includegraphics[width=\linewidth]{electron_orbitals.png}
%     \caption{The shapes of s and p orbitals. These represent probability distributions, not actual paths of electrons.}
% \end{marginfigure}

\subsection{Electron Configuration}

The arrangement of electrons in an atom's orbitals is called its electron configuration. This is crucial for understanding chemical bonding and the properties of elements.

\begin{itemize}
    \item The first energy level (n=1) has one sublevel (1s) and can hold 2 electrons
    \item The second energy level (n=2) has two sublevels (2s and 2p) and can hold 8 electrons
    \item The third energy level (n=3) has three sublevels (3s, 3p, and 3d) and can hold 18 electrons
\end{itemize}

Electron configurations can be written using notation like 1s$^2$ 2s$^2$ 2p$^6$, where the number and letter indicate the energy level and sublevel, and the superscript shows how many electrons occupy that sublevel.

\begin{example}
The electron configuration of oxygen (atomic number 8) is 1s$^2$ 2s$^2$ 2p$^4$, indicating two electrons in the 1s orbital, two in the 2s orbital, and four in the 2p orbitals.
\end{example}

\challenge{Research and explain the "diagonal rule" (or Madelung rule) for determining the order in which electron orbitals are filled. How does this relate to the arrangement of elements in the periodic table?}

\begin{investigation}{Flame Tests and Electron Transitions}
\textbf{Purpose:} To observe how electron transitions produce characteristic colors when elements are heated.

\textbf{Materials:}
\begin{itemize}
    \item Bunsen burner
    \item Wooden splints soaked in solutions of different metal salts (e.g., sodium chloride, potassium chloride, copper sulfate, strontium chloride, barium chloride)
    \item Safety goggles
    \item Tongs
    \item Cobalt blue glass (if available)
\end{itemize}

\textbf{Procedure:}
\begin{enumerate}
    \item Put on safety goggles.
    \item Light the Bunsen burner and adjust it to produce a blue flame.
    \item Using tongs, hold a splint soaked in a metal salt solution at the edge of the flame.
    \item Observe the color produced in the flame.
    \item Repeat with different metal salt solutions.
    \item If available, view the sodium flame through cobalt blue glass and note any differences.
\end{enumerate}

\textbf{Results:}
Record the color produced by each metal salt solution.

\textbf{Analysis:}
\begin{enumerate}
    \item Why do different elements produce different flame colors?
    \item How does this relate to the electron configuration of atoms?
    \item Why are these colors characteristic for each element?
    \item How do scientists use this phenomenon to identify elements?
    \item Research how these flame tests relate to emission spectra and quantum theory.
\end{enumerate}

\textbf{Extension:} Research how flame tests led to the discovery of new elements, such as cesium (from the Latin caesius, meaning "sky blue") and rubidium (from the Latin rubidus, meaning "deep red").
\end{investigation}

\section{The Periodic Table of Elements}

\newthought{The periodic table} is one of the most powerful tools in chemistry. At Stage 5, we'll explore the underlying patterns and trends in greater detail.

\subsection{Organization of the Periodic Table}

The modern periodic table organizes elements by:

\begin{itemize}
    \item Increasing atomic number (number of protons) from left to right
    \item Similar chemical properties in vertical columns (groups)
    \item Similar electron configurations in horizontal rows (periods)
\end{itemize}

\begin{keyconcept}{Regions of the Periodic Table}
\begin{description}
    \item[Metals] Located on the left and center of the table; good conductors, malleable, usually solid at room temperature
    \item[Non-metals] Located on the right side of the table; poor conductors, brittle as solids, may be solid, liquid, or gas at room temperature
    \item[Metalloids] Located along the "stair step" line separating metals and non-metals; have properties of both metals and non-metals
    \item[Main Group Elements] Groups 1, 2, and 13-18; also called representative elements
    \item[Transition Elements] Groups 3-12; often form colored compounds and have multiple oxidation states
\end{description}
\end{keyconcept}

\historylink{The periodic table was developed by Dmitri Mendeleev in 1869. What made his table revolutionary was that he left gaps for undiscovered elements and accurately predicted their properties. For example, he predicted the properties of "eka-silicon" (now known as germanium), which was discovered in 1886.}

\subsection{Electron Configuration and Periodic Trends}

The organization of the periodic table directly relates to electron configurations:

\begin{itemize}
    \item Group 1 (alkali metals): [noble gas] ns$^1$ (one electron in the outermost s orbital)
    \item Group 2 (alkaline earth metals): [noble gas] ns$^2$ (two electrons in the outermost s orbital)
    \item Group 17 (halogens): [noble gas] ns$^2$np$^5$ (seven electrons in the outermost s and p orbitals)
    \item Group 18 (noble gases): [noble gas] ns$^2$np$^6$ (eight electrons in the outermost s and p orbitals, except helium)
\end{itemize}

This electronic structure explains why elements in the same group have similar chemical properties.

\subsection{Periodic Trends}

The periodic table shows clear trends in physical and chemical properties:

\begin{keyconcept}{Key Periodic Trends}
\begin{description}
    \item[Atomic radius] Generally decreases across a period (left to right) and increases down a group
    \item[Ionization energy] Energy required to remove an electron; increases across a period and decreases down a group
    \item[Electronegativity] Ability to attract electrons in a bond; increases across a period and decreases down a group
    \item[Reactivity] For metals, increases down a group; for non-metals, increases across a period (right to left)
\end{description}
\end{keyconcept}

\mathlink{The effective nuclear charge ($Z_{eff}$) an electron experiences can be approximately calculated using Slater's rule: $Z_{eff} = Z - S$, where $Z$ is the nuclear charge (atomic number) and $S$ is the screening constant. This helps explain why atomic radius decreases across a period.}

% \begin{marginfigure}
%     \centering
%     \includegraphics[width=\linewidth]{periodic_trends.png}
%     \caption{Visualization of periodic trends across the periodic table. Note how properties change systematically.}
% \end{marginfigure}

\begin{investigation}{Patterns in the Periodic Table}
\textbf{Purpose:} To analyze periodic trends and relate them to electron configuration.

\textbf{Materials:}
\begin{itemize}
    \item Periodic table
    \item Data tables of atomic radius, ionization energy, electronegativity, and melting point for elements
    \item Graph paper or graphing software
\end{itemize}

\textbf{Procedure:}
\begin{enumerate}
    \item For the elements lithium (Li) through neon (Ne), create a graph plotting atomic number (x-axis) against:
    \begin{itemize}
        \item Atomic radius
        \item First ionization energy
        \item Electronegativity
        \item Melting point
    \end{itemize}
    
    \item Repeat by graphing these properties for the elements in Group 1 (alkali metals) from lithium (Li) to francium (Fr).
    
    \item For each graph, identify any patterns or trends.
\end{enumerate}

\textbf{Analysis:}
\begin{enumerate}
    \item Describe the trends you observed across Period 2 (Li to Ne) for each property.
    \item Describe the trends you observed down Group 1 for each property.
    \item How does the electron configuration of these elements explain the patterns you observed?
    \item Which elements don't seem to follow the expected trends? Research why these exceptions occur.
    \item How do these trends relate to the reactivity of elements?
\end{enumerate}

\textbf{Extension:} Research and explain how these periodic trends influence the biological roles of elements. Why are certain elements essential for life while others are toxic?
\end{investigation}

\section{Chemical Bonding}

\newthought{Chemical bonding} is the process by which atoms form bonds with other atoms, leading to compounds with distinct properties. The type of bonding depends primarily on electron configurations.

\subsection{Types of Chemical Bonds}

\begin{keyconcept}{Major Types of Chemical Bonds}
\begin{description}
    \item[Ionic bonding] Transfer of electrons between a metal and a non-metal, resulting in oppositely charged ions held together by electrostatic attraction
    \item[Covalent bonding] Sharing of electrons between non-metals, forming molecules
    \item[Metallic bonding] Sharing of electrons in a "sea" of delocalized electrons among metal atoms
\end{description}
\end{keyconcept}

\subsection{Ionic Bonding}

Ionic bonds form between metals (which tend to lose electrons) and non-metals (which tend to gain electrons). The resulting ions have opposite charges and are attracted to each other.

\begin{example}
Sodium chloride (NaCl) forms when sodium atoms lose one electron (forming Na$^+$ ions) and chlorine atoms gain one electron (forming Cl$^-$ ions). The electrostatic attraction between these ions creates a three-dimensional crystal lattice.
\end{example}

Properties of ionic compounds include:
\begin{itemize}
    \item High melting and boiling points
    \item Crystalline structure at room temperature
    \item Conduct electricity when dissolved in water or melted
    \item Often soluble in water but insoluble in non-polar solvents
\end{itemize}

\historylink{The concept of ions was introduced by Michael Faraday in 1834, but it was Svante Arrhenius who proposed in 1884 that salts dissociate into charged particles when dissolved in water, a theory for which he was awarded the Nobel Prize in Chemistry in 1903.}

\subsection{Covalent Bonding}

Covalent bonds form when atoms share electrons to achieve stable electron configurations. These bonds can be classified based on the degree of electron sharing:

\begin{itemize}
    \item \textbf{Non-polar covalent bonds:} Electrons are shared equally between atoms of similar electronegativity
    \item \textbf{Polar covalent bonds:} Electrons are shared unequally between atoms of different electronegativity
\end{itemize}

Lewis structures (electron dot diagrams) are used to represent the sharing of electrons in covalent bonds.

\begin{example}
In a water molecule (H$_2$O), the oxygen atom shares electrons with two hydrogen atoms. Since oxygen is more electronegative, the shared electrons are pulled closer to the oxygen, creating a polar molecule with partial charges ($\delta$+ on hydrogen, $\delta$- on oxygen).
\end{example}

Properties of covalent compounds include:
\begin{itemize}
    \item Generally lower melting and boiling points than ionic compounds
    \item Often exist as gases, liquids, or soft solids at room temperature
    \item Generally poor conductors of electricity
    \item Solubility varies ("like dissolves like" - polar molecules dissolve in polar solvents, non-polar molecules in non-polar solvents)
\end{itemize}

\mathlink{The polarity of a bond can be estimated by the difference in electronegativity ($\Delta$EN) between the bonded atoms. As a general rule: $\Delta$EN < 0.5: non-polar covalent; 0.5 $\leq$ $\Delta$EN $\leq$ 2.0: polar covalent; $\Delta$EN > 2.0: ionic.}

% \begin{marginfigure}
%     \centering
%     \includegraphics[width=\linewidth]{bond_polarity.png}
%     \caption{Spectrum of bond types from non-polar covalent to ionic, based on electronegativity differences.}
% \end{marginfigure}

\subsection{Metallic Bonding}

Metallic bonds form in metals where valence electrons are relatively free to move throughout the structure.

\begin{keyconcept}{Metallic Bonding}
In metallic bonding, the valence electrons from metal atoms form a "sea" of delocalized electrons that move freely throughout the crystal. The positively charged metal ions form a regular lattice structure and are held together by their attraction to this electron sea.
\end{keyconcept}

Properties of metals resulting from this bonding include:
\begin{itemize}
    \item Good conductors of heat and electricity
    \item Malleable and ductile
    \item Lustrous appearance
    \item Generally high melting and boiling points (except for mercury and gallium)
\end{itemize}

\challenge{Research and explain how the concepts of "band theory" expand on the simple "electron sea" model of metallic bonding. How does band theory explain why some materials are conductors, some are insulators, and others are semiconductors?}

\begin{investigation}{Comparing Bond Types}
\textbf{Purpose:} To investigate how different types of chemical bonding affect the properties of substances.

\textbf{Materials:}
\begin{itemize}
    \item Samples of substances with different bond types:
    \begin{itemize}
        \item Ionic: sodium chloride (table salt), calcium chloride
        \item Covalent (molecular): sugar (sucrose), vegetable oil, wax
        \item Covalent (network): graphite, quartz (if available)
        \item Metallic: copper wire, aluminum foil
    \end{itemize}
    \item Conductivity tester
    \item Beakers
    \item Distilled water
    \item Various solvents (water, ethanol, vegetable oil)
    \item Thermometer
    \item Heat source
\end{itemize}

\textbf{Procedure:}
\begin{enumerate}
    \item Examine the physical properties of each substance (appearance, hardness, brittleness).
    
    \item Test the electrical conductivity of each substance in solid form.
    
    \item For soluble substances, create solutions and test their electrical conductivity.
    
    \item Test the solubility of each substance in water, ethanol, and vegetable oil.
    
    \item Determine or research the melting points of the substances.
\end{enumerate}

\textbf{Results:}
Create a data table organizing your observations.

\textbf{Analysis:}
\begin{enumerate}
    \item How do the physical properties of each substance relate to its bond type?
    
    \item Explain why some substances conduct electricity in solid form, some only when dissolved, and others not at all.
    
    \item How does solubility relate to bond type and molecular polarity?
    
    \item Compare the melting points of substances with different bond types. What patterns do you notice?
    
    \item How does the bonding in graphite (a form of carbon) explain why it can be used as a lubricant and in pencil lead?
\end{enumerate}

\textbf{Extension:} Research how semiconductor materials combine properties of conductors and insulators, and how this relates to their bonding and electronic structure.
\end{investigation}

\section{Molecular Structure and Properties}

\newthought{The arrangement of atoms} in a molecule, along with the nature of the bonds between them, determines the properties of substances. In this section, we'll explore how molecular structure influences physical and chemical properties.

\subsection{VSEPR Theory and Molecular Geometry}

The Valence Shell Electron Pair Repulsion (VSEPR) theory helps predict the three-dimensional shape of molecules.

\begin{keyconcept}{VSEPR Theory}
VSEPR theory states that electron pairs around a central atom repel each other and arrange themselves to minimize repulsion, determining the shape of the molecule. Both bonding and non-bonding electron pairs affect molecular geometry.
\end{keyconcept}

Common molecular geometries include:
\begin{itemize}
    \item Linear: two electron groups (e.g., CO$_2$, BeF$_2$)
    \item Trigonal planar: three electron groups (e.g., BF$_3$, CO$_3^{2-}$)
    \item Tetrahedral: four electron groups (e.g., CH$_4$, NH$_4^+$)
    \item Trigonal pyramidal: four electron groups with one lone pair (e.g., NH$_3$)
    \item Bent/angular: four electron groups with two lone pairs (e.g., H$_2$O)
\end{itemize}

\mathlink{The bond angle in a tetrahedral molecule is approximately 109.5°. This can be calculated using the inverse cosine function: $\cos^{-1}(-1/3) = 109.5°$. This is the angle that maximizes the distance between four points on a sphere.}

% \begin{marginfigure}
%     \centering
%     \includegraphics[width=\linewidth]{molecular_geometries.png}
%     \caption{Common molecular geometries predicted by VSEPR theory.}
% \end{marginfigure}

\subsection{Polarity of Molecules}

The polarity of a molecule depends on:
\begin{itemize}
    \item The polarity of individual bonds (electronegativity difference)
    \item The molecular geometry (arrangement of polar bonds)
\end{itemize}

A molecule with polar bonds may still be non-polar overall if the bond dipoles cancel due to symmetrical arrangement.

\begin{example}
Carbon dioxide (CO$_2$) has polar C=O bonds (oxygen is more electronegative than carbon), but because the molecule is linear with bonds at 180° to each other, the dipoles cancel, making CO$_2$ a non-polar molecule.

In contrast, water (H$_2$O) has polar O–H bonds, and its bent structure means the dipoles do not cancel, resulting in a polar molecule.
\end{example}

Molecular polarity affects properties such as:
\begin{itemize}
    \item Solubility ("like dissolves like")
    \item Boiling and melting points
    \item Intermolecular forces
    \item Chemical reactivity
\end{itemize}

\subsection{Intermolecular Forces}

Intermolecular forces are attractions between molecules that influence physical properties like boiling point, melting point, and solubility.

\begin{keyconcept}{Types of Intermolecular Forces}
\begin{description}
    \item[Dispersion forces (London forces)] Weak attractions present in all molecules, caused by temporary dipoles; strength increases with molecular size
    \item[Dipole-dipole forces] Attractions between polar molecules; stronger than dispersion forces
    \item[Hydrogen bonding] Special case of dipole-dipole forces between H atoms bonded to highly electronegative atoms (N, O, F) and lone pairs on other electronegative atoms; particularly strong
    \item[Ion-dipole forces] Attractions between ions and polar molecules (important in solutions of ionic compounds in polar solvents)
\end{description}
\end{keyconcept}

\historylink{Hydrogen bonding was first described by Wendell Latimer and Worth Rodebush in 1920. This concept was crucial for understanding the structure of DNA, as discovered by Watson and Crick in 1953, where hydrogen bonds between complementary base pairs hold the two DNA strands together.}

\begin{investigation}{Investigating Surface Tension}
\textbf{Purpose:} To investigate how intermolecular forces contribute to surface tension in liquids.

\textbf{Materials:}
\begin{itemize}
    \item Different liquids: water, soapy water, ethanol, vegetable oil
    \item Small containers
    \item Droppers
    \item Coins
    \item Paper clips
    \item Small squares of wax paper
    \item Dish detergent
    \item Camera (optional)
\end{itemize}

\textbf{Procedure:}
\begin{enumerate}
    \item Coin Test:
    \begin{itemize}
        \item Place a clean, dry coin on a flat surface.
        \item Using a dropper, carefully add drops of water to the coin, counting how many drops can be added before the water spills.
        \item Repeat with the other liquids.
        \item Observe the shape of the liquid drops on the coin surface.
    \end{itemize}
    
    \item Paper Clip Float:
    \begin{itemize}
        \item Fill containers with each liquid.
        \item Carefully place a paper clip on the surface of each liquid, using a fork to gently lower it.
        \item Observe whether the paper clip floats or sinks.
        \item For liquids where the paper clip floats, add a drop of dish detergent and observe what happens.
    \end{itemize}
    
    \item Wax Paper Drops:
    \begin{itemize}
        \item Place drops of each liquid on wax paper.
        \item Observe the shape of the drops and measure their diameter.
        \item Tilt the wax paper and observe how the drops behave.
    \end{itemize}
\end{enumerate}

\textbf{Analysis:}
\begin{enumerate}
    \item Which liquid demonstrated the highest surface tension? Which had the lowest?
    
    \item Explain how intermolecular forces create surface tension.
    
    \item Why does adding detergent reduce surface tension?
    
    \item How does the shape of drops on wax paper relate to the polarity of the liquid?
    
    \item Research and explain how hydrogen bonding gives water its unusually high surface tension compared to other small molecules.
    
    \item How do the results of your investigation explain natural phenomena like water striders walking on water or the water cycle?
\end{enumerate}

\textbf{Extension:} Research and explain how surfactants work at the molecular level to reduce surface tension, and why this property makes them useful in cleaning products.
\end{investigation}

\section{Naming Compounds and Writing Formulas}

\newthought{Systematic naming} of chemical compounds allows chemists to communicate precisely about substances. At Stage 5, we'll expand on the basic principles you learned earlier.

\subsection{Naming Binary Ionic Compounds}

Binary ionic compounds contain a metal and a non-metal. The naming convention is:
\begin{itemize}
    \item Name the metal (cation) first, using the element name
    \item Name the non-metal (anion) second, changing the ending to -ide
    \item For metals that can form multiple ions (transition metals), use Roman numerals to indicate the charge
\end{itemize}

\begin{example}
\begin{itemize}
    \item NaCl: sodium chloride
    \item CaO: calcium oxide
    \item Fe$_2$O$_3$: iron(III) oxide
    \item CuCl$_2$: copper(II) chloride
\end{itemize}
\end{example}

\subsection{Naming Compounds with Polyatomic Ions}

Many compounds contain polyatomic ions—charged groups of atoms that behave as a unit.

\begin{keyconcept}{Common Polyatomic Ions}
\begin{description}
    \item[Hydroxide] OH$^-$ 
    \item[Ammonium] NH$_4^+$
    \item[Carbonate] CO$_3^{2-}$
    \item[Nitrate] NO$_3^-$
    \item[Sulfate] SO$_4^{2-}$
    \item[Phosphate] PO$_4^{3-}$
\end{description}
\end{keyconcept}

Naming follows similar patterns as binary compounds:
\begin{itemize}
    \item Name the cation first (metal or ammonium)
    \item Name the polyatomic anion second, keeping its name
\end{itemize}

\begin{example}
\begin{itemize}
    \item NaOH: sodium hydroxide
    \item (NH$_4$)$_2$SO$_4$: ammonium sulfate
    \item Ca(NO$_3$)$_2$: calcium nitrate
    \item CuSO$_4$: copper(II) sulfate
\end{itemize}
\end{example}

\subsection{Naming Covalent Compounds}

Binary covalent compounds contain two non-metals. The naming convention is:
\begin{itemize}
    \item Name the first element using its full name
    \item Name the second element, changing the ending to -ide
    \item Use prefixes to indicate the number of atoms (mono-, di-, tri-, tetra-, penta-, hexa-, etc.)
    \item The prefix "mono-" is usually omitted for the first element
\end{itemize}

\begin{example}
\begin{itemize}
    \item CO$_2$: carbon dioxide
    \item N$_2$O$_4$: dinitrogen tetroxide
    \item SF$_6$: sulfur hexafluoride
    \item P$_4$O$_{10}$: tetraphosphorus decaoxide
\end{itemize}
\end{example}

\subsection{Writing Chemical Formulas}

To write a chemical formula from a name:
\begin{itemize}
    \item Identify the elements or ions involved
    \item Determine the charge of each ion
    \item Combine them in the ratio that makes the compound neutral (total positive charge = total negative charge)
\end{itemize}

For compounds with polyatomic ions, use parentheses around the polyatomic ion when more than one is needed.

\begin{example}
To write the formula for aluminum sulfate:
\begin{itemize}
    \item Aluminum ion: Al$^{3+}$ (charge of 3+)
    \item Sulfate ion: SO$_4^{2-}$ (charge of 2-)
    \item To balance charges: 2 Al$^{3+}$ (total 6+) and 3 SO$_4^{2-}$ (total 6-)
    \item Formula: Al$_2$(SO$_4$)$_3$
\end{itemize}
\end{example}

\begin{investigation}{Naming and Formula Writing Practice}
\textbf{Purpose:} To develop proficiency in naming compounds and writing chemical formulas.

\textbf{Materials:}
\begin{itemize}
    \item Chemical formula and name cards
    \item Periodic table
    \item Reference sheet with common polyatomic ions
\end{itemize}

\textbf{Procedure (Part 1 - Matching Game):}
\begin{enumerate}
    \item Create a set of cards with chemical formulas on one set and corresponding names on another.
    \item Shuffle the cards and place them face down in two separate piles.
    \item Take turns drawing one card from each pile.
    \item If the formula and name match, keep the pair. If not, return them face down.
    \item The game continues until all pairs are matched.
\end{enumerate}

\textbf{Procedure (Part 2 - Formula Challenge):}
\begin{enumerate}
    \item For each compound name below, write the correct chemical formula:
    \begin{itemize}
        \item Potassium permanganate
        \item Ammonium phosphate
        \item Magnesium hydroxide
        \item Iron(II) sulfate
        \item Copper(II) nitrate
        \item Carbon tetrachloride
        \item Dinitrogen pentoxide
    \end{itemize}
    
    \item For each chemical formula below, write the correct systematic name:
    \begin{itemize}
        \item Al$_2$O$_3$
        \item Pb(NO$_3$)$_2$
        \item NH$_4$Cl
        \item K$_2$Cr$_2$O$_7$
        \item PCl$_5$
        \item Fe(OH)$_3$
        \item CuSO$_4$·5H$_2$O
    \end{itemize}
\end{enumerate}

\textbf{Procedure (Part 3 - Compound Creation):}
\begin{enumerate}
    \item Using a periodic table and list of polyatomic ions, create five compounds that include:
    \begin{itemize}
        \item A binary ionic compound
        \item A compound containing a transition metal with multiple possible charges
        \item A compound containing a polyatomic ion
        \item A binary covalent compound
        \item A compound containing both a transition metal and a polyatomic ion
    \end{itemize}
    
    \item For each compound, write both the chemical formula and systematic name.
    \item Research common uses for each compound you created.
\end{enumerate}

\textbf{Extension:} Research the historical or common names of some compounds (e.g., baking soda, limestone, laughing gas) and explain how these differ from systematic chemical nomenclature. Why do scientists prefer systematic naming?
\end{investigation}

\section{Chapter Review and Practice}

\newthought{Let's review} the key concepts we've covered in this chapter:

\begin{enumerate}
    \item Atoms are composed of protons, neutrons, and electrons, with electrons arranged in orbitals according to quantum mechanical principles
    \item The periodic table organizes elements based on atomic number and electron configuration, revealing patterns in properties
    \item Chemical bonds (ionic, covalent, and metallic) form when atoms interact to achieve stable electron configurations
    \item Molecular structure influences the physical and chemical properties of substances
    \item Systematic nomenclature provides a consistent way to name compounds and write chemical formulas
\end{enumerate}

\begin{tieredquestions}{Level 1 - Basic Understanding}
\begin{enumerate}
    \item What are the three main subatomic particles, and where are they located in an atom?
    \item Define atomic number and mass number, and explain how they relate to the number of protons, neutrons, and electrons.
    \item Distinguish between metals, non-metals, and metalloids on the periodic table.
    \item Explain the difference between ionic, covalent, and metallic bonding.
    \item Write the chemical formulas for these compounds: sodium oxide, calcium nitrate, carbon dioxide, iron(III) chloride.
\end{enumerate}
\end{tieredquestions}

\begin{tieredquestions}{Level 2 - Application}
\begin{enumerate}
    \item Draw the electron configuration diagram for oxygen (O) and explain how this relates to its position in the periodic table.
    \item Predict the type of bond that would form between each pair of elements: Na and Cl, C and O, Fe and Fe. Explain your reasoning.
    \item Draw Lewis structures for water (H$_2$O), ammonia (NH$_3$), and methane (CH$_4$). Predict the molecular geometry of each based on VSEPR theory.
    \item Explain why water has a higher boiling point than expected for a molecule of its size, referring to intermolecular forces.
    \item For the compound Al$_2$(SO$_4$)$_3$: name it, calculate the total number of atoms per formula unit, and determine the charge of each ion.
\end{enumerate}
\end{tieredquestions}

\begin{tieredquestions}{Level 3 - Extension and Analysis}
\begin{enumerate}
    \item The element technetium (Tc) was the first artificially produced element. Analyze its electron configuration and predict where it would be placed in the periodic table if it hadn't been discovered. What properties would you expect it to have?
    
    \item Compare and contrast the models of the atom proposed by Thomson, Rutherford, Bohr, and the quantum mechanical model. Explain how each model addressed limitations of previous models and how experimental evidence supported each development.
    
    \item Analyze why carbon can form millions of different compounds while some elements form very few. Refer to carbon's electron configuration, bonding capabilities, and ability to form single, double, and triple bonds.
    
    \item Design an experiment to distinguish between samples of ionic, covalent, and metallic substances using only household materials and simple tests.
    
    \item Research and explain the concept of resonance structures in molecules like benzene (C$_6$H$_6$) or the nitrate ion (NO$_3^-$). How does resonance affect the properties of these substances?
\end{enumerate}
\end{tieredquestions}

\section{Glossary of Key Terms}

\begin{description}
    \item[Atomic number] The number of protons in an atom's nucleus, which determines the element's identity.
    \item[Electron configuration] The arrangement of electrons in an atom's orbitals.
    \item[Electronegativity] A measure of an atom's tendency to attract electrons in a chemical bond.
    \item[Ion] An atom or molecule that has gained or lost one or more electrons, giving it a positive or negative charge.
    \item[Ionic bond] A chemical bond formed by the electrostatic attraction between oppositely charged ions.
    \item[Covalent bond] A chemical bond formed when atoms share electron pairs.
    \item[Metallic bond] A chemical bond formed by the attraction between metal ions and delocalized electrons.
    \item[Lewis structure] A diagram showing valence electrons as dots around atomic symbols to represent bonding.
    \item[Molecule] A group of atoms bonded together, representing the smallest fundamental unit of a chemical compound.
    \item[Orbital] A region around an atomic nucleus where an electron is likely to be found.
    \item[Polyatomic ion] A charged group of atoms that behaves as a unit in chemical reactions.
    \item[Valence electrons] Electrons in the outermost energy level of an atom, typically involved in bonding.
    \item[VSEPR theory] A model that predicts the three-dimensional arrangement of atoms in a molecule based on the repulsion between electron pairs.
\end{description}

\section{Beyond the Basics: Exploring Further}

\newthought{Want to learn more?} Here are some suggestions for further exploration:

\begin{itemize}
    \item \textbf{Research Project:} Investigate how atomic theory evolved from ancient Greek philosophy to modern quantum mechanics, identifying key experiments and discoveries.
    
    \item \textbf{Digital Exploration:} Use molecular visualization software (like Jmol, which is free) to explore 3D structures of molecules and how they relate to properties.
    
    \item \textbf{STEM Career Connection:} Interview a materials scientist about how understanding atomic structure and bonding leads to the development of new materials with specific properties.
    
    \item \textbf{Cross-Curricular Link:} Explore how electronic configurations influence the colors of transition metal compounds, connecting chemistry to art and design.
    
    \item \textbf{Critical Thinking:} Research the "island of stability" hypothesis in nuclear physics, which predicts that some superheavy elements might have unusually long half-lives.
\end{itemize}