\chapter{Human Biology and Disease}

\section{Introduction}

Humans are complex organisms made up of intricate organ systems that function collaboratively to sustain life. The biological mechanisms within our bodies are finely regulated and balanced, yet they remain susceptible to various diseases and disorders. Studying human biology and disease not only helps us understand our own bodies better but also empowers us to maintain health, recognise symptoms, and seek appropriate medical intervention.

In this chapter, we will explore the organisation and functions of human body systems, investigate common diseases and disorders, examine how the body defends itself, and describe modern medical advances that help us prevent and treat illness. 

\section{Organisation of the Human Body}

The human body is organised into distinct structural levels, each building upon the previous level to form the complete organism. These levels range from the microscopic to the macroscopic.

\begin{keyconcept}{Levels of Organisation}
The human body is organised into cells, tissues, organs, and organ systems. Each level has a distinct structure and function, contributing to overall health and homeostasis.
\end{keyconcept}

\subsection{Cells: The Basic Units of Life}

Cells are the fundamental structural and functional units of all living organisms. Human cells vary widely in shape and function, from oxygen-carrying red blood cells to electrically excitable nerve cells.

\marginnote{\keyword{Cell}: The smallest structural and functional unit of life.}

\begin{example}
Red blood cells (\keyword{erythrocytes}) are specialised cells that transport oxygen throughout the body. Their biconcave shape increases surface area for efficient gas exchange.
\end{example}

\subsection{Tissues: Groups of Specialised Cells}

Cells of similar structure and function group together to form tissues. Four main tissue types exist within the body:
\begin{itemize}
    \item Epithelial tissue
    \item Connective tissue
    \item Muscle tissue
    \item Nervous tissue
\end{itemize}

\begin{stopandthink}
Why do muscle tissues contain more mitochondria compared to other tissue types?
\end{stopandthink}

\subsection{Organs and Organ Systems}

Organs are structures comprised of two or more tissue types that work together to perform specific functions. Groups of organs form organ systems, which coordinate to carry out complex bodily functions.

\marginnote{\keyword{Organ}: A collection of tissues working together to perform a specific function.}

\begin{example}
The heart is an organ composed predominantly of cardiac muscle tissue, connective tissue, and nerve tissue. It works as part of the circulatory system, pumping blood around the body.
\end{example}

\subsection{Homeostasis: Maintaining Balance}

Homeostasis refers to the body's ability to maintain a stable internal environment despite changes in external conditions. Organ systems interact continuously to regulate temperature, water balance, blood glucose, and other vital parameters.

\marginnote{\historylink{Claude Bernard first described the concept of homeostasis in the 19th century, highlighting that the stability of the internal environment is essential for life.}}

\begin{investigation}{Measuring Heart Rate and Homeostasis}
\begin{enumerate}
    \item Measure your resting heart rate by counting your pulse for one minute.
    \item Exercise moderately (e.g., jogging on the spot) for two minutes.
    \item Immediately measure your heart rate again.
    \item Rest for five minutes and measure your heart rate once more.
    \item Record and compare your results. Discuss how your body maintains homeostasis after exercise.
\end{enumerate}
\end{investigation}

\begin{tieredquestions}{Basic}
\begin{enumerate}
    \item List the four levels of organisation within the human body.
    \item Define homeostasis and give one example.
\end{enumerate}
\end{tieredquestions}

\begin{tieredquestions}{Intermediate}
\begin{enumerate}
    \item Explain how tissues differ from cells.
    \item Describe the role of two specific organ systems in maintaining homeostasis.
\end{enumerate}
\end{tieredquestions}

\begin{tieredquestions}{Advanced}
\begin{enumerate}
    \item Discuss how hormonal and nervous systems interact to maintain body temperature.
    \item Predict the consequences if homeostasis mechanisms fail in the human body.
\end{enumerate}
\end{tieredquestions}

\section{Pathogens and Disease}

Diseases can arise from multiple causes including genetic factors, lifestyle choices, and pathogens. Pathogens are microorganisms that cause infectious disease.

\subsection{Types of Pathogens}

Pathogens include bacteria, viruses, fungi, protozoa, and parasites. Each type has distinct biological characteristics and modes of transmission.

\marginnote{\keyword{Pathogen}: A microorganism capable of causing disease.}

\begin{keyconcept}{Bacteria and Viruses}
Bacteria are single-celled organisms that reproduce rapidly by binary fission. Viruses are smaller non-cellular entities that rely on host cells to replicate.
\end{keyconcept}

\begin{example}
The bacterium \textit{Streptococcus pyogenes} causes strep throat, whereas the influenza virus causes flu infections. Treatment of bacterial infections often involves antibiotics, but antibiotics are ineffective against viruses.
\end{example}

\begin{stopandthink}
Why is it important to correctly identify whether an infection is viral or bacterial before prescribing medication?
\end{stopandthink}

\subsection{Transmission of Disease}

Diseases can spread through several pathways, including direct contact, airborne droplets, contaminated food or water, and vectors like mosquitoes.

\marginnote{\keyword{Vector}: An organism, often an insect, that transmits pathogens from one host to another.}

\begin{investigation}{Simulating Disease Transmission}
\begin{enumerate}
    \item Obtain cups containing clear liquid; one cup secretly contains sodium carbonate solution, others contain water.
    \item Exchange liquid samples with classmates randomly.
    \item After exchanges, add phenolphthalein indicator to each cup.
    \item Observe the colour change indicating infection.
    \item Discuss how quickly and easily a disease can spread.
\end{enumerate}
\end{investigation}

\subsection{Preventing Infectious Disease}

Preventing disease involves hygiene practices, vaccination, and public health measures. Vaccinations stimulate the immune system to protect against specific pathogens.

\marginnote{\historylink{Edward Jenner developed the first successful vaccine against smallpox in 1796, paving the way for modern immunisation practices.}}

\begin{tieredquestions}{Basic}
\begin{enumerate}
    \item List three ways pathogens can spread.
    \item What is a vaccine and how does it protect us from disease?
\end{enumerate}
\end{tieredquestions}

\begin{tieredquestions}{Intermediate}
\begin{enumerate}
    \item Compare the structure of bacteria and viruses.
    \item Explain how good hygiene reduces disease transmission.
\end{enumerate}
\end{tieredquestions}

\begin{tieredquestions}{Advanced}
\begin{enumerate}
    \item Discuss the rise of antibiotic-resistant bacteria and strategies to combat this problem.
    \item Evaluate the importance of global vaccination programmes in disease prevention.
\end{enumerate}
\end{tieredquestions}

\section{The Immune System: Defence Against Disease}

The human immune system protects the body from pathogens through a complex network of cells, tissues, and organs.

\subsection{Innate Immunity}

Innate immunity is the body's immediate, non-specific defence mechanism, including physical barriers like skin and mucous membranes, and cellular responses involving white blood cells.

\begin{keyconcept}{Inflammation}
Inflammation is an innate immune response characterised by redness, swelling, heat, and pain, helping to isolate and destroy pathogens.
\end{keyconcept}

\subsection{Adaptive Immunity}

Adaptive immunity provides specific and long-lasting protection through specialised cells like lymphocytes. These cells recognise specific pathogens and produce antibodies.

\marginnote{\keyword{Antibody}: A protein produced by lymphocytes that binds specifically to foreign antigens.}

\begin{investigation}{Observing Blood Smears}
\begin{enumerate}
    \item Observe prepared microscope slides of human blood.
    \item Identify red blood cells, white blood cells, and platelets.
    \item Sketch and label the cells observed and discuss their roles in immunity.
\end{enumerate}
\end{investigation}

(Continued in next section due to length restrictions...)