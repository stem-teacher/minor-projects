\chapter{Atoms, Elements and Compounds}

\section{Introduction}

Everything around us, from the air we breathe to the materials making up our bodies, is built from tiny particles called \keyword{atoms}. Understanding atoms, elements and compounds is fundamental for exploring chemistry and the world we live in. Throughout this chapter, we will explore what atoms are made of, how they combine to form substances, and how scientists have developed our current understanding of matter.

\section{Atoms: The Building Blocks of Matter}

Atoms are the fundamental particles that make up everything in the universe. They are incredibly small, yet their interactions and combinations determine the properties of all materials.

\subsection{Structure of the Atom}

Atoms consist of three main subatomic particles:
\begin{itemize}
    \item \keyword{Protons} — positively charged particles located in the atom’s nucleus.
    \item \keyword{Neutrons} — particles with no charge, also found in the nucleus.
    \item \keyword{Electrons} — negatively charged particles orbiting the nucleus in electron shells.
\end{itemize}

\marginpar{\historylink{The idea of atoms dates back to Ancient Greece. Philosophers like Democritus first proposed that matter consisted of indivisible units called `atomos'. Modern atomic theory, however, began developing in the 19th century with John Dalton.}}

Atoms have no overall charge because the number of protons (positive) equals the number of electrons (negative). For example, a carbon atom has 6 protons and therefore 6 electrons.

\begin{keyconcept}{Atomic Number and Mass Number}
Every element is defined by its atomic number, which is the number of protons in its nucleus. The mass number is the total number of protons and neutrons in an atom's nucleus.
\[\text{Mass Number} = \text{Protons} + \text{Neutrons}\]
\end{keyconcept}

\begin{example}
Find the number of neutrons in a fluorine atom with an atomic number of 9 and a mass number of 19.

\textbf{Solution:}
\[
\text{Neutrons} = \text{Mass number} - \text{Atomic number} = 19 - 9 = 10
\]
\end{example}

\marginpar{\challenge{Isotopes are atoms of the same element with the same atomic number but different mass numbers due to varying numbers of neutrons.}}

\subsection{Electron Configuration}

Electrons orbit the nucleus in shells or energy levels. Each shell can only hold a certain number of electrons:
\begin{itemize}
    \item First shell: maximum 2 electrons
    \item Second shell: maximum 8 electrons
    \item Third shell: maximum 8 electrons (for the first 20 elements)
\end{itemize}

The arrangement of electrons greatly influences how atoms react chemically.

\begin{stopandthink}
Draw the electron configuration for a chlorine atom. Chlorine has an atomic number of 17. How many electrons are in each shell?
\end{stopandthink}

\section{Elements: Substances Made from One Type of Atom}

An \keyword{element} is a pure substance made of only one type of atom. Each element has unique properties and is represented by a chemical symbol, for example, oxygen (\ce{O}), sodium (\ce{Na}), chlorine (\ce{Cl}), and gold (\ce{Au}).

\marginpar{\historylink{The modern periodic table was first organised by Dmitri Mendeleev in 1869.}}

\subsection{The Periodic Table}

The periodic table arranges elements by their atomic number, electron configuration and chemical properties. Rows are called \keyword{periods}, and columns are called \keyword{groups}.

\begin{keyconcept}{Groups and Periods}
Elements in the same group have similar chemical properties because they have the same number of electrons in their outermost shell. Elements in the same period have the same number of electron shells.
\end{keyconcept}

\begin{example}
Sodium and potassium are both in Group 1. Sodium (\ce{Na}) has the configuration 2,8,1, and potassium (\ce{K}) has configuration 2,8,8,1. Both have one electron in their outer shell, making them chemically similar.
\end{example}

\subsection{Metals, Non-metals, and Metalloids}

Elements are classified into three main groups based on their properties:
\begin{itemize}
    \item \keyword{Metals}: Conduct electricity and heat, malleable, ductile, often shiny.
    \item \keyword{Non-metals}: Poor conductors, brittle, dull appearance.
    \item \keyword{Metalloids}: Have properties between metals and non-metals.
\end{itemize}

\begin{stopandthink}
Identify three metals, three non-metals, and two metalloids on the periodic table.
\end{stopandthink}

\begin{tieredquestions}{Basic}
\begin{enumerate}
    \item State the difference between an atom and an element.
    \item Name the three subatomic particles in an atom.
\end{enumerate}
\end{tieredquestions}

\begin{tieredquestions}{Intermediate}
\begin{enumerate}
    \item Explain why elements in the same group have similar chemical properties.
    \item Draw electron configurations for lithium (\ce{Li}) and sulphur (\ce{S}).
\end{enumerate}
\end{tieredquestions}

\begin{tieredquestions}{Advanced}
\begin{enumerate}
    \item An element has an atomic number of 15. Predict its electron configuration and position on the periodic table.
    \item Compare isotopes of hydrogen (\ce{^1H}, \ce{^2H}, \ce{^3H}) in terms of structure and properties.
\end{enumerate}
\end{tieredquestions}

\section{Compounds: Chemical Combinations of Elements}

Atoms rarely exist independently. Instead, they chemically bond with others to form \keyword{compounds}.

\subsection{Chemical Bonds}

Chemical bonds hold atoms together in compounds. There are two main types:
\begin{itemize}
    \item \keyword{Ionic bonds}: Formed when electrons are transferred from one atom to another, creating ions that attract each other.
    \item \keyword{Covalent bonds}: Formed when atoms share electrons.
\end{itemize}

\begin{example}
Sodium chloride (\ce{NaCl}), common table salt, is formed by ionic bonding. Sodium (\ce{Na}) loses an electron, becoming positively charged (\ce{Na+}), while chlorine (\ce{Cl}) gains an electron, becoming negatively charged (\ce{Cl-}).
\end{example}

\subsection{Chemical Formulas}

Chemical formulas show the type and number of atoms in a compound. For example, water (\ce{H2O}) contains two hydrogen atoms and one oxygen atom.

\begin{stopandthink}
Identify the number of atoms in each element of calcium carbonate (\ce{CaCO3}).
\end{stopandthink}

\begin{investigation}{Observing Ionic Compounds}
\textbf{Aim:} To observe properties of ionic compounds.

\textbf{Materials:} Samples of sodium chloride, copper sulfate, distilled water, test tubes.

\textbf{Method:}
\begin{enumerate}
    \item Observe and record the appearance of each compound.
    \item Place each compound in distilled water and observe if it dissolves.
    \item Test the conductivity of each solution using a conductivity probe.
\end{enumerate}

\textbf{Discussion:}
\begin{itemize}
    \item Which compounds dissolved readily?
    \item Did the dissolved compounds conduct electricity? Explain why.
\end{itemize}
\end{investigation}

\section{Chapter Review}

This chapter has introduced the basic building blocks of matter—atoms, elements, and compounds. You have explored atomic structure, how the periodic table organises elements, and how atoms form chemical bonds to create compounds.

\begin{tieredquestions}{Basic}
\begin{enumerate}
    \item Define an atom, element, and compound.
    \item Provide examples of two ionic and two covalent compounds.
\end{enumerate}
\end{tieredquestions}

\begin{tieredquestions}{Intermediate}
\begin{enumerate}
    \item Explain the difference between ionic and covalent bonding.
    \item Draw electron shell diagrams for sodium chloride (\ce{NaCl}) formation.
\end{enumerate}
\end{tieredquestions}

\begin{tieredquestions}{Advanced}
\begin{enumerate}
    \item Predict the formula of the ionic compound formed between calcium (\ce{Ca}) and chlorine (\ce{Cl}).
    \item Discuss how the periodic table aids in predicting chemical bonding and reactivity.
\end{enumerate}
\end{tieredquestions}

In the next chapter, we will examine chemical reactions and how atoms and compounds rearrange to produce new substances.