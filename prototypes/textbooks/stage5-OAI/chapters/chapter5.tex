\chapter{Genetics and Evolution}

Genetics and evolution are fundamental concepts underpinning much of modern biology. Understanding these topics helps us appreciate the diversity of life on Earth, the connections between different organisms, and how species adapt to their environments over time. In this chapter, we will explore how genetic information is inherited, how genetic variation arises, and how evolution shapes the diversity and adaptations we observe in nature.

\section{Introduction to Genetics}

Genetics is the science of heredity, focusing on how traits are passed from parents to offspring. At its core, genetics investigates the structures, functions, and behaviours of genes, the fundamental units of heredity located within cells.

\marginnote{\keyword{Gene}: A segment of DNA that encodes for a specific protein or function.}

\begin{keyconcept}{What is DNA?}
Deoxyribonucleic acid (\keyword{DNA}) is the chemical that carries genetic information in all living organisms. DNA is organised into structures called chromosomes, which are found within the nucleus of each cell. Human cells, for example, usually contain 46 chromosomes arranged in 23 pairs.
\end{keyconcept}

DNA molecules have a characteristic structure—the double helix—composed of two strands twisted around each other. Each strand consists of repeating units called nucleotides, composed of a sugar molecule (deoxyribose), phosphate groups, and nitrogenous bases. The four bases found in DNA are adenine (A), thymine (T), cytosine (C), and guanine (G). 

\marginnote{\historylink{The structure of DNA was discovered in 1953 by James Watson and Francis Crick, with key contributions from Rosalind Franklin's X-ray diffraction images.}}

\subsection{Genes and Alleles}

Each gene is found at a specific locus (plural loci) on a chromosome. Genes can exist in slightly different forms called \keyword{alleles}. These alleles are responsible for variations in inherited characteristics, such as eye colour or blood type.

\begin{example}
The gene for eye colour has multiple alleles. The allele for brown eyes is dominant, while the allele for blue eyes is recessive. A person with at least one brown-eye allele typically has brown eyes.
\end{example}

\subsection{Dominant and Recessive Traits}

Traits controlled by single genes can show dominant-recessive patterns of inheritance. A \keyword{dominant} allele masks the expression of a \keyword{recessive} allele when both are present. The recessive trait only appears if both alleles inherited are recessive.

\begin{stopandthink}
If a person inherits one dominant allele for brown eyes (B) and one recessive allele for blue eyes (b), what will their eye colour most likely be? Explain your reasoning.
\end{stopandthink}

\begin{tieredquestions}{Basic}
\begin{enumerate}
    \item Define the terms gene, allele, dominant, and recessive.
    \item What is DNA and where is it located within a cell?
\end{enumerate}
\end{tieredquestions}

\begin{tieredquestions}{Intermediate}
\begin{enumerate}
    \item Explain the structure of DNA, including its base-pairing rules.
    \item Describe how dominant and recessive alleles affect inheritance patterns.
\end{enumerate}
\end{tieredquestions}

\begin{tieredquestions}{Advanced}
\begin{enumerate}
    \item A couple, both with brown eyes, have a child with blue eyes. Explain how this could happen using genetic principles.
    \item Discuss the impact of the discovery of DNA’s structure on modern genetics.
\end{enumerate}
\end{tieredquestions}

\section{Inheritance Patterns}

\subsection{Punnett Squares}

A \keyword{Punnett square} is a simple graphical method used to predict possible genotypes and phenotypes of offspring from a genetic cross.

\begin{example}
Consider parents who both carry the allele for cystic fibrosis (a recessive genetic disorder). Representing the normal allele as F (dominant) and the cystic fibrosis allele as f (recessive), we can use a Punnett square to predict the probability of their children inheriting cystic fibrosis.
\end{example}

\mathlink{Punnett squares visually represent probability as fractions or percentages, connecting genetics with mathematical probability.}

\begin{stopandthink}
How many squares in a Punnett square represent possible offspring outcomes when studying one single trait from two parents? Why?
\end{stopandthink}

\subsection{Pedigree Charts}

\keyword{Pedigree charts} illustrate how traits are inherited across generations within families. Symbols represent males, females, and affected individuals, allowing scientists and medical professionals to trace genetic conditions.

\begin{investigation}{Analysing a Family Pedigree}
Obtain or construct a pedigree chart showing inheritance of a trait (such as colour blindness or attached earlobes). Identify dominant and recessive patterns. Write a brief analysis, explaining how the trait is passed down.
\end{investigation}

\section{Genetic Variation and Mutation}

Genetic variation is essential for species' survival as it provides the raw material for evolution. Variation arises primarily through mutations, or random changes in DNA sequences.

\subsection{Types of Mutations}

Mutations can be beneficial, neutral, or harmful. They include:
\begin{itemize}
    \item Substitution (one base replaced by another)
    \item Insertion (an extra base added)
    \item Deletion (a base removed)
\end{itemize}

\begin{keyconcept}{Impact of Mutations}
Beneficial mutations enhance an organism's ability to survive and reproduce, neutral mutations have no noticeable effect, and harmful mutations decrease an organism's fitness.
\end{keyconcept}

\begin{stopandthink}
Why are beneficial mutations important in the process of evolution?
\end{stopandthink}

\section{Evolution and Natural Selection}

Evolution is the process by which species change over generations. Natural selection, a mechanism proposed by Charles Darwin, explains how evolution occurs.

\marginnote{\historylink{Charles Darwin published his groundbreaking book, \textit{On the Origin of Species}, in 1859, introducing the theory of evolution by natural selection.}}

\subsection{Natural Selection}

Natural selection occurs when organisms with advantageous traits survive and reproduce more successfully than those without these traits. Over generations, beneficial adaptations become more common in a population.

\begin{example}
Peppered moth populations in Britain changed colour during the Industrial Revolution. Dark moths became more common because they blended into soot-covered trees, avoiding predators.
\end{example}

\subsection{Evidence for Evolution}

Several types of evidence support the theory of evolution:
\begin{itemize}
    \item Fossil records
    \item Comparative anatomy
    \item Embryology
    \item Molecular biology (DNA comparisons)
\end{itemize}

\begin{investigation}{Simulating Natural Selection}
Use coloured paper or beans to simulate prey animals, and students acting as predators to simulate natural selection. Record and explain changes in prey populations over several "generations".
\end{investigation}

\begin{tieredquestions}{Basic}
\begin{enumerate}
    \item Define mutation and natural selection.
    \item What is an adaptation? Provide an example.
\end{enumerate}
\end{tieredquestions}

\begin{tieredquestions}{Intermediate}
\begin{enumerate}
    \item Explain how mutations contribute to genetic variation.
    \item Describe one example of natural selection observed in nature.
\end{enumerate}
\end{tieredquestions}

\begin{tieredquestions}{Advanced}
\begin{enumerate}
    \item Evaluate the evidence supporting evolution. Which type of evidence do you find most compelling, and why?
    \item Discuss how the understanding of natural selection has impacted current issues such as antibiotic resistance.
\end{enumerate}
\end{tieredquestions}

\section{Applications and Implications of Genetics}

Advances in genetics have significant implications for medicine, agriculture, and society.

\subsection{Genetic Engineering}

\keyword{Genetic engineering} involves directly modifying an organism's DNA, allowing scientists to introduce new traits or remove undesirable ones.

\subsection{Ethical Considerations}

As genetic technology advances, ethical questions arise about its use. Considerations include privacy, genetic discrimination, and the ethics of altering human embryos.

\challenge{Research CRISPR gene-editing technology. How does it work, and what are its potential benefits and risks?}

\begin{investigation}{Debating Genetic Technologies}
Organise a class debate on the ethical implications of genetic engineering, including GM crops, gene therapy, and designer babies. Prepare evidence-based arguments for and against these technologies.
\end{investigation}

\marginnote{\historylink{The Human Genome Project, completed in 2003, mapped the entire human genetic code, revolutionising genetics and medicine.}}

In this chapter, we have explored foundational concepts in genetics and evolution, examining inheritance, genetic variation, natural selection, and societal implications. Understanding these ideas provides a deeper appreciation of biology and the complexity of life on Earth.