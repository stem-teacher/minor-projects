\chapter{Scientific Investigations and Research Skills}

Science is not just a collection of facts, but rather a systematic and logical approach to understanding the natural world. In Stage 5, you are stepping forward as independent investigators, developing your skills to design rigorous experiments, collect precise data, and draw valid conclusions. This chapter revises and builds upon the skills introduced in Stage 4, preparing you for your mandatory Student Research Project (SRP). You will explore advanced experimental design, understand different types of variables, and learn how to ensure the reliability and validity of your investigations. The chapter also stresses the importance of ethical conduct, accuracy, and clear communication through scientific reports.

\section{The Nature of Scientific Investigations}

Scientific investigations involve a structured approach to answering questions about the natural world. Such investigations can take many forms, including controlled experiments, observational studies, and modelling. Regardless of the approach, good scientific practice involves careful planning, systematic observation, accurate data recording, and critical analysis.

\begin{keyconcept}{Types of Scientific Investigations}
Scientific investigations generally fall into three main categories:
\begin{itemize}
    \item \textbf{Controlled experiments} – A test where one factor is deliberately changed while others are kept constant.
    \item \textbf{Observational studies} – Researchers collect data without manipulating variables.
    \item \textbf{Modelling} – Using mathematical and computational techniques to simulate real-world phenomena.
\end{itemize}
\end{keyconcept}

\marginnote[0.5cm]{\keyword{Scientific method}: A systematic approach involving observation, hypothesis formulation, experimentation, and conclusion.}

\begin{example}
Observing bird behaviour at different times of the day without interference is an observational study. Testing the effects of fertiliser concentration on plant growth in a greenhouse is a controlled experiment.
\end{example}

\begin{stopandthink}
Why might a scientist choose to conduct an observational study rather than a controlled experiment?
\end{stopandthink}

\subsection{Historical Context of Scientific Investigations}

Scientific investigation methods have evolved significantly over history. Ancient Greek philosophers relied primarily on observation and logical reasoning, while modern scientists emphasise empirical evidence gathered through rigorous experimentation.

\marginnote{\historylink{The modern scientific method was significantly influenced by Galileo Galilei (1564–1642), who emphasised experimentation and observation over purely theoretical reasoning.}}

\section{Variables in Scientific Investigations}

When you conduct a scientific investigation, understanding and managing variables is crucial. Variables are factors or conditions that can change and therefore potentially affect the outcome of your experiment.

\begin{keyconcept}{Types of Variables}
There are three primary types of variables:
\begin{itemize}
    \item \keyword{Independent variable} – The factor you deliberately change.
    \item \keyword{Dependent variable} – The factor observed and measured for changes.
    \item \keyword{Controlled variables} – Factors kept constant to ensure a fair test.
\end{itemize}
\end{keyconcept}

\begin{example}
If you investigate how the concentration of salt affects plant growth, the independent variable is salt concentration, the dependent variable is plant growth, and controlled variables include plant species, soil type, amount of water, and temperature.
\end{example}

\begin{stopandthink}
Identify the independent, dependent, and controlled variables in an experiment testing the effect of temperature on enzyme activity.
\end{stopandthink}

\section{Reliability and Validity in Investigations}

Two central concepts underlying accurate scientific experiments are reliability and validity. Understanding these terms helps in designing effective investigations and interpreting results accurately.

\begin{keyconcept}{Reliability and Validity}
\begin{itemize}
    \item \keyword{Reliability} refers to consistency and repeatability of results. Reliable results can be reproduced under similar conditions.
    \item \keyword{Validity} refers to whether an experiment accurately measures what it intends to measure.
\end{itemize}
\end{keyconcept}

\begin{stopandthink}
If an experiment's results vary significantly every time it is repeated, what does this indicate about its reliability?
\end{stopandthink}

\begin{investigation}{Assessing Reliability}
Design a simple experiment to measure the reaction time of classmates catching a falling ruler. Conduct multiple trials for each individual and record your data. Discuss how repeating trials affects reliability and how you might improve your experimental design.
\end{investigation}

\section{Formulating a Research Question}

The first step in a scientific investigation is formulating a clear, focused research question. A good research question guides your investigation, providing clarity and direction.

\begin{keyconcept}{Characteristics of Good Research Questions}
A good research question is:
\begin{itemize}
    \item Clear and specific
    \item Testable through experimentation or observation
    \item Relevant and meaningful
    \item Ethical and safe to investigate
\end{itemize}
\end{keyconcept}

\begin{example}
A vague question: "Does exercise affect people?"

A better, focused question: "How does regular aerobic exercise affect resting heart rate in teenagers aged 15–18?"
\end{example}

\begin{stopandthink}
Rewrite the question "Does sleep affect learning?" to make it clearer, more specific, and testable.
\end{stopandthink}

\section{Conducting Background Research}

Before conducting an experiment, scientists gather background information to develop an understanding of the topic. Background research identifies what is already known and helps refine your question and experimental design.

\begin{keyconcept}{Effective Background Research}
When conducting background research:
\begin{itemize}
    \item Use credible sources such as scientific journals, textbooks, and reputable websites.
    \item Take clear notes, summarising key points.
    \item Record references accurately for your report.
\end{itemize}
\end{keyconcept}

\marginnote{\challenge{Advanced students can explore databases like Google Scholar or JSTOR to find peer-reviewed scientific articles.}}

\section{Collecting and Analysing Data}

Data collection must be systematic and accurate. Clear tables, graphs, and calculations help interpret your data effectively.

\begin{keyconcept}{Data Collection and Analysis}
\begin{itemize}
    \item Record data clearly in tables.
    \item Present data visually through graphs.
    \item Analyse data using calculations such as averages and percentages.
\end{itemize}
\end{keyconcept}

\marginnote{\mathlink{Statistical analysis helps assess the significance of your results. Calculating the mean (average) and the range can provide insights into trends and variability.}}

\begin{investigation}{Data Analysis Practice}
Measure the heights of at least ten classmates. Present your data clearly in a table, then create an appropriate graph to visualise the data. Calculate the average height and comment on the range of data you collected.
\end{investigation}

\section{Writing a Scientific Report}

Scientific reports communicate your findings clearly and systematically. Learning to write scientific reports prepares you for your SRP and future scientific studies.

\begin{keyconcept}{Structure of a Scientific Report}
A scientific report typically includes:
\begin{itemize}
    \item Title
    \item Abstract (brief summary)
    \item Introduction (including aim and hypothesis)
    \item Method (clear description of procedure and materials)
    \item Results (tables, graphs, and observations)
    \item Discussion (interpretation of results, reliability, validity, improvements)
    \item Conclusion (briefly restating your findings)
    \item References (sources used)
\end{itemize}
\end{keyconcept}

\begin{stopandthink}
Why is it important for scientists to clearly describe their methods in a report?
\end{stopandthink}

\section{Ethics and Accuracy in Scientific Investigations}

Ethical conduct and accuracy are foundational to scientific integrity. Ethical scientists respect the welfare of living organisms, accurately report their findings, and acknowledge sources.

\begin{keyconcept}{Ethical Conduct in Science}
Scientists must:
\begin{itemize}
    \item Ensure the welfare of animals and humans involved in experiments.
    \item Report results truthfully and completely.
    \item Acknowledge all sources and avoid plagiarism.
\end{itemize}
\end{keyconcept}

\marginnote{\historylink{Ethical guidelines became increasingly important in science following unethical experiments, such as the Tuskegee Syphilis Study (1932–1972), highlighting the necessity of strict ethical standards.}}

\begin{tieredquestions}{Basic}
\item What is an independent variable?
\item Define reliability in scientific experiments.
\item List two controlled variables when investigating plant growth.
\end{tieredquestions}

\begin{tieredquestions}{Intermediate}
\item Explain the difference between reliability and validity.
\item Why is it important to repeat an experiment multiple times?
\item Suggest improvements to increase the validity of an experiment testing memory recall.
\end{tieredquestions}

\begin{tieredquestions}{Advanced}
\item Evaluate why ethical considerations are essential in scientific investigations.
\item Discuss how background research influences experimental design.
\item Design a controlled experiment to test how caffeine affects reaction time.
\end{tieredquestions}