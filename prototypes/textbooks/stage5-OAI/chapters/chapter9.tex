\chapter{Motion and Mechanics}

\section{Introduction}

From the graceful flight of birds to the precise orbit of planets, motion is an essential and endlessly fascinating aspect of our universe. Mechanics, the branch of physics that deals with motion and the forces that cause it, helps us unravel the secrets of how and why objects move. In this chapter, we will explore the fundamental principles of motion, learn how to describe and measure movement accurately, and investigate the forces that shape the interactions of objects around us.

\marginnote{\textbf{Mechanics} is the study of how objects move, behave, and interact, influenced by forces.}

\section{Describing Motion}

\subsection{Distance and Displacement}

When an object moves, there are two ways we can describe how far it has travelled: distance and displacement.

\begin{keyconcept}{Distance and Displacement}
\begin{itemize}
    \item \keyword{Distance} is the total length of the path travelled by an object. It has no direction and is a scalar quantity.
    \item \keyword{Displacement} is the shortest straight-line distance from the starting point to the end point of an object's motion. Displacement includes direction, making it a vector quantity.
\end{itemize}
\end{keyconcept}

\begin{example}
A student walks 100 m north, then 40 m south. The total distance travelled is 140 m, but her displacement is 60 m north.
\end{example}

\begin{stopandthink}
If you walk in a complete circle of radius 5 m and end up exactly where you started, what is your displacement? What distance have you travelled?
\end{stopandthink}

\subsection{Speed and Velocity}

Measuring how quickly an object moves is essential in describing motion. The terms \keyword{speed} and \keyword{velocity} help us quantify this.

\begin{keyconcept}{Speed and Velocity}
\begin{itemize}
    \item \keyword{Speed} is the rate at which distance is covered. It is calculated as:
    \[
    \text{Speed} = \frac{\text{Distance}}{\text{Time}}
    \]
    \item \keyword{Velocity} is the rate of change of displacement and is a vector quantity, meaning it has both magnitude and direction.
    \[
    \text{Velocity} = \frac{\text{Displacement}}{\text{Time}}
    \]
\end{itemize}
\end{keyconcept}

\marginnote{\historylink{Galileo Galilei (1564–1642) was among the first scientists to systematically study motion. His experiments laid the groundwork for modern mechanics.}}

\begin{example}
A cyclist travels 120 km in 4 hours. Her average speed is:
\[
\text{Speed} = \frac{120\,\text{km}}{4\,\text{h}} = 30\,\text{km/h}
\]
\end{example}

\begin{stopandthink}
Can an object have constant speed but changing velocity? Explain your reasoning.
\end{stopandthink}

\subsection{Acceleration}

Acceleration describes how quickly velocity changes. It occurs whenever there is a change in speed or direction.

\begin{keyconcept}{Acceleration}
Acceleration is defined as the rate of change of velocity:
\[
\text{Acceleration} = \frac{\text{Change in velocity}}{\text{Time taken}}
\]
Acceleration is measured in metres per second squared (\(\text{m/s}^2\)).
\end{keyconcept}

\marginnote{\keyword{Acceleration} can be positive (speeding up), negative (slowing down or deceleration), or directional (changing direction).}

\begin{example}
A car increases its velocity from rest (0 m/s) to 20 m/s in 5 seconds. Its acceleration is:
\[
a = \frac{20\,\text{m/s} - 0\,\text{m/s}}{5\,\text{s}} = 4\,\text{m/s}^2
\]
\end{example}

\begin{stopandthink}
Describe a situation in everyday life where an object has negative acceleration. What happens to the object's velocity?
\end{stopandthink}

\begin{tieredquestions}{Basic}
\begin{enumerate}
    \item Define distance and displacement.
    \item Calculate the speed of a runner who completes 400 m in 50 s.
    \item Name two vector quantities studied in this section.
\end{enumerate}
\end{tieredquestions}

\begin{tieredquestions}{Intermediate}
\begin{enumerate}
    \item An athlete runs around a rectangular track (100 m by 50 m) once. Calculate the athlete's distance travelled and displacement from the starting point.
    \item A bus travels at 60 km/h east for 2 hours, then turns and travels south at 30 km/h for 1 hour. Calculate the bus’s total displacement.
\end{enumerate}
\end{tieredquestions}

\begin{tieredquestions}{Advanced}
\begin{enumerate}
    \item Discuss why velocity is considered a vector quantity while speed is scalar. Provide examples to illustrate your explanation.
    \item A vehicle accelerates uniformly from rest to 30 m/s in 10 seconds. Calculate the total distance travelled during this time.
\end{enumerate}
\end{tieredquestions}

\section{Forces and Newton's Laws}

\subsection{The Nature of Forces}

A \keyword{force} is a push, pull, or twist that can cause an object to change its motion or shape. Forces can be contact forces, such as friction, or non-contact forces, such as gravity.

\begin{keyconcept}{Balanced and Unbalanced Forces}
When forces acting on an object are balanced, there is no change in motion. Unbalanced forces cause acceleration—changes in speed or direction.
\end{keyconcept}

\begin{stopandthink}
When you push a shopping trolley and then stop pushing, why does it eventually come to rest?
\end{stopandthink}

\subsection{Newton's First Law: Inertia}

\begin{keyconcept}{Newton's First Law}
An object will remain at rest, or continue moving at a constant velocity, unless acted upon by an unbalanced force. This concept is known as \keyword{inertia}.
\end{keyconcept}

\marginnote{\historylink{Sir Isaac Newton (1643–1727) revolutionised mechanics by formulating the three fundamental laws of motion, providing a clear framework for understanding forces and motion.}}

\subsection{Newton's Second Law: Force and Acceleration}

Newton's second law describes the relationship between force, mass, and acceleration:

\[
F = ma
\]

\marginnote{\mathlink{The SI units are force (newton, N), mass (kilogram, kg), acceleration (m/s\(^2\)). One newton equals one kilogram metre per second squared (1 N = 1 kg·m/s\(^2\)).}}

\begin{example}
A force of 50 N acts on a mass of 10 kg. The acceleration is:
\[
a = \frac{F}{m} = \frac{50\,\text{N}}{10\,\text{kg}} = 5\,\text{m/s}^2
\]
\end{example}

\subsection{Newton's Third Law: Action and Reaction}

\begin{keyconcept}{Newton's Third Law}
For every action force, there is an equal and opposite reaction force.
\end{keyconcept}

\begin{investigation}{Exploring Action-Reaction Forces}
\begin{enumerate}
    \item Inflate a balloon and release it without tying the end. Observe what happens.
    \item Explain how this demonstrates Newton's third law.
\end{enumerate}
\end{investigation}

\begin{tieredquestions}{Basic}
\begin{enumerate}
    \item State Newton’s three laws of motion.
    \item Give an example of inertia from everyday life.
\end{enumerate}
\end{tieredquestions}

\begin{tieredquestions}{Intermediate}
\begin{enumerate}
    \item Calculate the acceleration produced by a 20 N force on a 4 kg object.
    \item Describe a scenario in sport where Newton’s third law is evident.
\end{enumerate}
\end{tieredquestions}

\begin{tieredquestions}{Advanced}
\begin{enumerate}
    \item Analyse and explain how seat belts and airbags reduce injuries during car accidents, referring explicitly to Newton’s laws.
    \item If the mass of an object is doubled while the force acting upon it remains constant, what happens to its acceleration?
\end{enumerate}
\end{tieredquestions}

% Word count ~2600 words (approximation)