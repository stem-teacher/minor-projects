\chapter{Atomic Structure and the Periodic Table}

\section{Introduction}
All matter around us is composed of tiny building blocks we call \keyword{atoms}. Understanding atomic structure and the periodic table helps us make sense of how substances behave, react, and relate to one another. In this chapter, we will explore the fascinating world inside atoms, discover how scientists have built our current understanding, and learn how the periodic table organises elements into patterns and families.

\section{The Structure of the Atom}

Atoms are so small that millions can fit onto the tip of a pin. Despite their tiny size, atoms consist of even smaller particles: protons, neutrons, and electrons. These particles give atoms their properties and determine how they interact.

\subsection{Subatomic Particles}

\begin{keyconcept}{Inside the Atom}
An atom consists of three primary particles:
\begin{itemize}
	\item \keyword{Protons}: Positively charged particles found in the nucleus.
	\item \keyword{Neutrons}: Neutral particles also found in the nucleus.
	\item \keyword{Electrons}: Negatively charged particles orbiting the nucleus.
\end{itemize}
\end{keyconcept}

Protons and neutrons cluster tightly together at the centre of the atom, forming a dense core called the \keyword{nucleus}. Electrons surround the nucleus in regions known as \keyword{electron shells} or energy levels.

\historylink{The electron was first discovered by J.J. Thomson in 1897 through cathode ray experiments. Later, Ernest Rutherford's famous gold foil experiment in 1911 revealed that atoms have a tiny, positively charged nucleus.}

\begin{marginfigure}
\centering
%\includegraphics{atomic_structure_diagram}
\caption{A simplified diagram of atomic structure.}
\end{marginfigure}

\begin{stopandthink}
Why are electrons important for chemical reactions?
\end{stopandthink}

\subsection{Atomic Number and Mass Number}

Each element's identity is determined by the number of protons in its nucleus, known as its \keyword{atomic number}. For example, carbon always has 6 protons, so its atomic number is 6. The number of protons plus neutrons gives an atom's \keyword{mass number}.

\begin{example}
A sodium atom has 11 protons and 12 neutrons. This means sodium has an atomic number of 11 and a mass number of 23.
\end{example}

Atoms of the same element can have different numbers of neutrons. These variations are called \keyword{isotopes}.

\begin{tieredquestions}{Basic}
\begin{enumerate}
	\item Name the three subatomic particles found in atoms.
	\item Define the atomic number.
\end{enumerate}
\end{tieredquestions}

\begin{tieredquestions}{Intermediate}
\begin{enumerate}
	\item An atom has 16 protons and 16 neutrons. Identify its atomic number and mass number. 
	\item Describe the difference between isotopes of the same element.
\end{enumerate}
\end{tieredquestions}

\begin{tieredquestions}{Advanced}
\begin{enumerate}
	\item Explain how Rutherford's gold foil experiment led to the discovery of the atomic nucleus.
	\item If the isotope carbon-14 has 6 protons, how many neutrons does it have? Explain its significance in archaeological dating.
\end{enumerate}
\end{tieredquestions}

\section{Electron Configuration}

Electrons orbit the nucleus in distinct shells, each capable of holding a specific number of electrons. The arrangement of electrons around an atom affects how it interacts chemically.

\subsection{Electron Shells and Valence Electrons}

Electron shells can hold a certain maximum number of electrons:

\begin{itemize}
	\item First shell: 2 electrons
	\item Second shell: 8 electrons
	\item Third shell: 18 electrons (though stable configurations often have 8 electrons)
\end{itemize}

Electrons in the outermost shell are called \keyword{valence electrons}. They determine how atoms bond to form compounds.

\begin{example}
Oxygen has 8 electrons. Its electron configuration is 2,6. Thus, oxygen has 6 valence electrons.
\end{example}

\begin{stopandthink}
How many valence electrons does calcium (\ce{Ca}, atomic number 20) have? Predict its chemical behaviour based on this number.
\end{stopandthink}

\begin{investigation}{Modelling Electron Shells}
Using a set of coloured beads and wire or string, create models of electron shells for elements with atomic numbers from 1 to 20. Investigate patterns in valence electrons and predict chemical properties based on your models.
\end{investigation}

\begin{tieredquestions}{Basic}
\begin{enumerate}
	\item What are valence electrons?
	\item Draw the electron shell configuration for magnesium (\ce{Mg}), atomic number 12.
\end{enumerate}
\end{tieredquestions}

\begin{tieredquestions}{Intermediate}
\begin{enumerate}
	\item How does electron configuration relate to an element's position in the periodic table?
	\item Predict the valence electron pattern for elements in Group 17.
\end{enumerate}
\end{tieredquestions}

\begin{tieredquestions}{Advanced}
\begin{enumerate}
	\item Use electron configurations to explain why helium (\ce{He}) and neon (\ce{Ne}) are chemically inert.
	\item Explain the relationship between valence electrons and chemical reactivity using sodium (\ce{Na}) and chlorine (\ce{Cl}) as examples.
\end{enumerate}
\end{tieredquestions}

\section{Structure and Organisation of the Periodic Table}

The periodic table organises elements based on their atomic number and electron configuration, helping scientists predict chemical behaviour.

\subsection{Groups and Periods}

Vertical columns in the periodic table are called \keyword{groups}, while horizontal rows are called \keyword{periods}. Elements in the same group have similar chemical properties because they have the same number of valence electrons.

\begin{marginfigure}
\centering
%\includegraphics{periodic_table_groups}
\caption{The periodic table organised into groups and periods.}
\end{marginfigure}

\historylink{Dmitri Mendeleev first arranged elements into a periodic table in 1869, predicting undiscovered elements based on the patterns he observed.}

\subsection{Metals, Non-metals, and Metalloids}

Elements are broadly categorised into three groups:

\begin{itemize}
	\item \keyword{Metals}: Typically shiny, conductive, and malleable, e.g., copper (\ce{Cu}), iron (\ce{Fe}).
	\item \keyword{Non-metals}: Generally dull, brittle, and poor conductors, e.g., sulfur (\ce{S}), carbon (\ce{C}).
	\item \keyword{Metalloids}: Exhibit properties intermediate between metals and non-metals, e.g., silicon (\ce{Si}).
\end{itemize}

\begin{stopandthink}
Why do elements in the same group have similar chemical properties?
\end{stopandthink}

\begin{investigation}{Exploring Properties of Elements}
Gather samples of various elements or household materials containing elements. Observe and test their properties such as conductivity, malleability, and appearance. Classify each sample as a metal, non-metal, or metalloid based on your observations.
\end{investigation}

\begin{tieredquestions}{Basic}
\begin{enumerate}
	\item What are groups and periods in the periodic table?
	\item List three properties of metals.
\end{enumerate}
\end{tieredquestions}

\begin{tieredquestions}{Intermediate}
\begin{enumerate}
	\item Identify the group and period of chlorine (\ce{Cl}) and potassium (\ce{K}).
	\item Explain why silicon is considered a metalloid.
\end{enumerate}
\end{tieredquestions}

\begin{tieredquestions}{Advanced}
\begin{enumerate}
	\item Predict properties of an unknown element based on its position in Group 2, Period 4.
	\item Discuss how Mendeleev’s periodic table was significant in the history of science.
\end{enumerate}
\end{tieredquestions}

\section{Chapter Summary}
Atoms form the basis of all matter. Understanding atomic structure helps us explain chemical properties and reactions. The periodic table organises elements systematically, revealing patterns that help scientists predict chemical behaviours and properties.