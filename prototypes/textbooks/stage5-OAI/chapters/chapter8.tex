\chapter{Applied Chemistry and Environmental Chemistry}

\section{Introduction}

Chemistry is more than a study of reactions in laboratories; it is intimately woven into the fabric of our daily lives and the environment around us. Applied chemistry takes the fundamental principles of chemical science and harnesses them to solve real-world problems, from creating sustainable materials to developing life-saving medicines. Environmental chemistry, on the other hand, focuses on the chemical processes occurring in our natural environment and the impact human activity has on these processes. In this chapter, we will explore how chemistry is applied in practical contexts and how chemical knowledge can be used to address pressing environmental challenges.

\section{Everyday Applications of Chemistry}

Applied chemistry surrounds us in everyday life. Every product we use, from toothpaste to mobile devices, has chemistry at its core.

\subsection{Household Chemicals}

Household products such as detergents, cleaning agents, and personal care items rely heavily on chemical reactions and substances.

\begin{keyconcept}{Surfactants}
Surfactants are molecules that reduce the surface tension of water, allowing it to interact more effectively with dirt and grease. They consist of a hydrophilic (water-loving) head and a hydrophobic (water-hating) tail, enabling them to remove dirt and oil from surfaces and fabrics.
\end{keyconcept}

\marginnote{\keyword{Surfactant}: A substance that reduces surface tension, aiding in cleaning processes.}

\begin{example}
Common household detergents contain sodium lauryl sulfate (\ce{CH3(CH2)11OSO3Na}), a surfactant that effectively removes oils and grease from dishes and clothing.
\end{example}

\begin{stopandthink}
Why are surfactants essential in laundry detergents? What would happen if water alone were used to clean oily fabrics?
\end{stopandthink}

\subsection{Polymers and Plastics}

Polymers are large chemical compounds consisting of repeated smaller units called monomers. Plastics, a common type of polymer, are versatile materials used in countless products.

\begin{keyconcept}{Polymerisation}
Polymerisation is the chemical reaction in which monomers bond together, forming long polymer chains. There are two main types: addition polymerisation and condensation polymerisation.
\end{keyconcept}

\marginnote{\keyword{Polymerisation}: Formation of large molecules by linking monomers.}

\begin{example}
Polyethylene, a widely-used plastic, is produced through addition polymerisation of ethylene molecules (\ce{C2H4}):
\[
n\,\ce{C2H4} \rightarrow (\ce{C2H4})_n
\]
\end{example}

\begin{stopandthink}
List four polymer-based products you use daily. Can you identify alternative materials that could replace these polymers?
\end{stopandthink}

\begin{investigation}{Comparing Biodegradability of Plastics}
\textbf{Aim:} Investigate the biodegradability of different types of plastics.\\
\textbf{Materials:} Samples of polyethylene, polyethylene terephthalate (PET), starch-based biodegradable plastic; compost soil; containers.\\
\textbf{Method:}
\begin{enumerate}
\item Place each plastic sample in separate containers filled with compost soil.
\item Maintain moisture and temperature conditions suitable for composting.
\item Observe and record physical changes weekly over two months.
\item Analyse results and discuss implications for plastic waste management.
\end{enumerate}
\end{investigation}

\begin{tieredquestions}{Basic}
\begin{enumerate}
\item Define the term `surfactant' and provide an example.
\item Describe briefly how polymers are formed.
\end{enumerate}
\end{tieredquestions}

\begin{tieredquestions}{Intermediate}
\begin{enumerate}
\item Compare the two types of polymerisation processes.
\item Explain why biodegradable plastics are considered environmentally friendly. Give an example of such a plastic.
\end{enumerate}
\end{tieredquestions}

\begin{tieredquestions}{Advanced}
\begin{enumerate}
\item Evaluate the environmental impact of synthetic polymers compared to natural polymer alternatives.
\item Propose a method for reducing plastic waste in your local community, considering chemical and practical perspectives.
\end{enumerate}
\end{tieredquestions}

\section{Environmental Chemistry}

Environmental chemistry investigates the chemical processes occurring naturally in the environment and those influenced by human activities. It is crucial for understanding and solving environmental challenges.

\subsection{The Atmosphere and Air Pollution}

The Earth's atmosphere consists mainly of nitrogen (\ce{N2}), oxygen (\ce{O2}), argon (\ce{Ar}), and trace gases. Human activities introduce pollutants that affect air quality and health.

\begin{keyconcept}{Air Pollutants}
Major air pollutants include carbon monoxide (\ce{CO}), sulfur dioxide (\ce{SO2}), nitrogen oxides (\ce{NO_x}), particulate matter, and volatile organic compounds (VOCs). These substances result from combustion processes, industrial activities, and vehicle emissions.
\end{keyconcept}

\marginnote{\keyword{Volatile Organic Compounds (VOCs)}: Organic chemicals that easily vaporise, contributing significantly to air pollution.}

\begin{example}
Vehicle exhaust releases nitrogen monoxide (\ce{NO}), which reacts with oxygen to form nitrogen dioxide (\ce{NO2}), a harmful gas causing respiratory issues.
\[
\ce{2NO + O2 -> 2NO2}
\]
\end{example}

\begin{stopandthink}
Identify three human activities that contribute significantly to air pollution. Suggest ways to minimise their impact.
\end{stopandthink}

\begin{investigation}{Detecting Particulate Pollution}
\textbf{Aim:} Investigate the presence of particulate matter in the air around your school.\\
\textbf{Materials:} Petroleum jelly, microscope slides, magnifying glass, markers.\\
\textbf{Method:}
\begin{enumerate}
\item Coat microscope slides thinly with petroleum jelly.
\item Place slides in various school locations for one week.
\item Collect slides, observe under magnification, and count particles.
\item Compare results and discuss sources and health implications of particulate pollution.
\end{enumerate}
\end{investigation}

\subsection{Water Chemistry and Pollution}

Water chemistry studies chemical substances and reactions occurring in aquatic environments. Pollutants such as heavy metals, nitrates, phosphates, and organic compounds disrupt ecosystems and affect water quality.

\begin{keyconcept}{Eutrophication}
Excessive nutrients, particularly nitrates (\ce{NO3^-}) and phosphates (\ce{PO4^{3-}}), lead to eutrophication. This process results in rapid algae growth, depleting oxygen and harming aquatic life.
\end{keyconcept}

\marginnote{\keyword{Eutrophication}: Nutrient enrichment causing excessive algae growth and oxygen depletion in water bodies.}

\begin{example}
Agricultural fertilisers often contain nitrates and phosphates. Run-off into rivers and lakes can trigger eutrophication, harming aquatic ecosystems.
\end{example}

\begin{stopandthink}
Explain how eutrophication affects aquatic life. Suggest agricultural practices to reduce this problem.
\end{stopandthink}

\begin{investigation}{Testing Water Quality}
\textbf{Aim:} Assess water quality in local water bodies by testing for nitrates, phosphates, and pH levels.\\
\textbf{Materials:} Water testing kits, sample bottles, notebook.\\
\textbf{Method:}
\begin{enumerate}
\item Collect water samples from various sources.
\item Follow test kit instructions to measure nitrates, phosphates, and pH.
\item Record and analyse results, comparing them with safe water quality standards.
\item Discuss findings and propose solutions for water quality improvement.
\end{enumerate}
\end{investigation}

\begin{tieredquestions}{Basic}
\begin{enumerate}
\item Name two pollutants commonly found in air.
\item Describe eutrophication in simple terms.
\end{enumerate}
\end{tieredquestions}

\begin{tieredquestions}{Intermediate}
\begin{enumerate}
\item Explain the chemical reaction forming nitrogen dioxide in vehicle emissions.
\item Discuss the environmental impacts of eutrophication.
\end{enumerate}
\end{tieredquestions}

\begin{tieredquestions}{Advanced}
\begin{enumerate}
\item Evaluate various strategies for reducing air pollution in urban areas.
\item Design an experiment to determine the effects of fertiliser run-off on local aquatic ecosystems.
\end{enumerate}
\end{tieredquestions}

\section{Conclusion}

Applied chemistry and environmental chemistry demonstrate the profound influence chemistry has on our lives and the environment. Understanding these concepts empowers us to use chemical knowledge responsibly and sustainably. As future scientists and informed citizens, we must continue exploring chemistry's role in improving our quality of life and preserving Earth's ecosystems for generations to come.