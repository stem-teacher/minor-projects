\documentclass[justified]{tufte-book}
\usepackage{amsmath}
\usepackage{graphicx}
\usepackage[version=4]{mhchem}
\usepackage{xcolor}
\usepackage{hyperref}
\hypersetup{colorlinks=true, linkcolor=blue, urlcolor=blue}

\newcommand{\keyword}[2]{\textbf{#1}\marginnote{\small #2}}

\title{Stage 4 Science -- Forces and Motion}
\author{Your Name \\ Your Institution}
\date{\today}

\begin{document}

\maketitle

% Chapter start
\chapter[Forces and Motion]{Forces and Motion}
\label{ch:forces-motion}

In this chapter, you will learn how different types of forces influence the motion of objects. You'll explore everyday applications of forces and motion, conduct experiments, and practice using evidence to support scientific explanations. This chapter aligns with the Stage 4 Physics outcomes of the NSW Science Syllabus, emphasising real-world examples, inquiry-based learning, and differentiation strategies for gifted and neurodiverse learners.

\section{What is a Force?}

A \keyword{force}{A push or pull upon an object resulting from its interaction with another object.} is any interaction that changes the motion of an object or its state of rest. Forces can be broadly classified as either \textit{contact forces} or \textit{non-contact forces}.

\begin{marginfigure}[-3cm]
  \includegraphics[width=\linewidth]{force_diagram.png}
  \caption{Forces acting on an object.}
  \label{fig:force-diagram}
\end{marginfigure}

\textbf{Contact forces:}
\begin{itemize}
  \item Friction
  \item Air resistance
  \item Pushes and pulls
\end{itemize}

\textbf{Non-contact forces:}
\begin{itemize}
  \item Gravity
  \item Magnetic force
  \item Electrostatic force
\end{itemize}

\section{Balanced and Unbalanced Forces}
Objects remain stationary or keep moving at the same velocity unless acted upon by an unbalanced force. Balanced forces produce no change in motion, while unbalanced forces result in acceleration.

\subsection*{Experiment: Tug-of-War}
\textbf{Purpose:} To clearly understand balanced versus unbalanced forces through hands-on experience.

\textbf{Materials:} Strong rope, marking tape

\textbf{Method:}

\begin{enumerate}
  \item Mark a midpoint clearly on the rope using tape.
  \item Teams of equal strength stand on either side and pull (balanced).
  \item Record movement of midpoint.
  \item Change one team's strength or number of participants (unbalanced). Measure and record the movement again.
\end{enumerate}

\textbf{Conclusion:} Discuss when the forces were balanced versus unbalanced, supporting with your observations.

\section{Newton's First Law: Inertia}
\keyword{Inertia}{The resistance of any object to change its motion or rest state.} Objects tend to keep doing what they're already doing. This is known as Newton's First Law of Motion or the Principle of Inertia.

\subsection*{Activity: Observing Inertia}
\textbf{Purpose:} Connect everyday experience to Newton's First Law.

\textbf{Materials:} Cardboard, coin, glass or cup

\textbf{Method:}

\begin{enumerate}
  \item Place cardboard over the cup and set the coin on it.
  \item Quickly flick/pull the cardboard out.
  \item Observe the movement of the coin.
\end{enumerate}

\textbf{Conclusion:} Explain using Newton's First Law why the coin fell into the cup.

\section{Understanding Friction}
Friction is a contact force opposing movement between surfaces. It occurs because of microscopic irregularities between surfaces in contact.

\subsection*{Experiment: Measuring Friction}
\textbf{Purpose:} Quantitatively investigate friction on different surfaces.

\textbf{Materials:} Blocks, spring scale (Newton-meter), various surfaces (e.g., glass, wood, sandpaper)

\textbf{Method:}

\begin{enumerate}
  \item Attach spring scale to block.
  \item Record the force required to start the block moving on different surfaces.
  \item Compare and analyse your results.
\end{enumerate}

\textbf{Conclusion:} Why does friction vary among different surfaces? Discuss your results.

\section{Gravity and Motion}
Gravity is a universal \keyword{non-contact force}{Forces that can act at a distance, without direct contact between objects.} pulling objects towards each other, noticeable as weight and free-fall motion towards Earth. The weight of an object is the force of gravity acting upon it, measurable using:

\[
\text{Weight (N)} = \text{Mass (kg)} \times \text{Gravity} (9.8~m/s^2)
\]

\begin{marginfigure}[-1cm]
  \includegraphics[width=\linewidth]{gravity_ball_drop.png}
  \caption{Gravity causes the ball to fall straight down.}
  \label{fig:gravity-drop}
\end{marginfigure}

\subsection*{Experiment: Free-Fall Motion}
\textbf{Purpose:} Observe and measure acceleration due to gravity.

\textbf{Materials:} Stopwatch, balls of different weights, measuring tape

\textbf{Method:}

\begin{enumerate}
  \item Drop objects from a fixed height (safe and supervised).
  \item Measure and record the time each takes to hit the ground.
  \item Compare results. Does object's mass affect falling speed significantly?
\end{enumerate}

\textbf{Conclusion:} Discuss why objects fall at similar rates regardless of their mass.

\section{Differentiation and Enrichment}

\begin{itemize}
  \item \textit{Tiered Activities:} Advanced learners may try calculating acceleration numerically from your data in the gravity experiment.

  \item \textit{Choice Activity:} Research and present how friction is managed or utilised in sports equipment (e.g., shoes, skis, Formula 1 cars).

  \item \textit{Extension Challenge:} Can you design an experiment to test the effect of air resistance using paper shapes dropped from heights?
\end{itemize}

\section{Literacy and Numeracy Integration}
\textbf{Scientific Vocabulary:} inertia, acceleration, contact/non-contact

\textbf{Numeracy Skills:} Measurement of forces (Newton), calculating averages (mean), interpreting simple data charts.

\section*{Chapter Review}

\begin{enumerate}
  \item What type of force is friction? Give two everyday examples.
  \item Explain Newton's First Law in your words.
  \item Why do heavier objects not necessarily fall faster?
  \item Describe one real-world situation of balanced and one of unbalanced forces.
  \item (Challenge Question) Using force diagrams, explain what happens when skydivers reach terminal velocity.
\end{enumerate}

\section*{PISA-Style Scenario (Integration)}
A sports company develops two new soccer shoes. Shoe A has better friction, helping players turn quickly but slowing run speed slightly. Shoe B has lower friction, increasing speed but making turns difficult. Based on your knowledge, advise the sports company which shoe is best. Justify your reasoning using scientific concepts learned in this chapter.

\end{document}
