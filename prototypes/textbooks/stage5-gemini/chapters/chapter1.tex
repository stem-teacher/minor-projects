```latex
\chapter{Scientific Investigations and Research Skills}

\marginnote{Welcome to Stage 5 Science!} Welcome to Stage 5 Science! This year, you will be developing your skills as young scientists, learning to investigate the world around you in a more structured and independent way. Get ready to explore, question, and discover!

\section{Introduction: Becoming a Scientific Investigator}

Science is more than just facts and figures; it's a way of thinking, a method of exploring the world and answering questions through careful observation and experimentation.  In Stage 4, you began to learn the basics of \keyword{Working Scientifically}. Now, in Stage 5, we will build upon these foundations, equipping you with more advanced tools and techniques to conduct your own \keyword{scientific investigations}.

Think of scientists as detectives, meticulously piecing together clues to solve mysteries of the universe.  Whether it’s understanding the smallest particles of matter or the vast expanse of space, scientists use a systematic approach to ask questions, gather evidence, and draw conclusions. This chapter will guide you in developing these very skills.

\begin{stopandthink}
Think about a time you acted like a scientist, even without realising it. Perhaps you tried to figure out why a plant wasn't growing, or why a toy wasn't working. What steps did you take?
\end{stopandthink}

This year, a significant part of your science journey will be the \keyword{Student Research Project (SRP)}. This chapter is specifically designed to prepare you for this exciting challenge.  We will cover essential skills like designing experiments, understanding variables, ensuring your results are trustworthy, formulating research questions, and effectively communicating your findings.  By the end of this chapter, you will be well on your way to becoming confident and capable scientific investigators.

\section{Experimental Design: Planning Your Investigation}

Effective scientific investigations begin with careful planning.  A well-designed experiment is crucial for obtaining meaningful and reliable results.  Poorly planned experiments can lead to confusion, wasted time, and conclusions that are not supported by evidence.

\subsection{The Importance of Planning}

Imagine building a house without blueprints. It would be chaotic, inefficient, and likely not very sturdy!  Similarly, in science, a detailed plan – or \keyword{experimental design} – acts as your blueprint for investigation.  It outlines the steps you will take, the materials you will need, and how you will collect and analyse your data. A good experimental design ensures that your investigation is focused, organised, and capable of answering your research question.

\marginnote{Analogy} Just as an architect plans a building before construction, a scientist plans an experiment before starting data collection.

\subsection{Variables in Experiments}

At the heart of experimental design is the concept of \keyword{variables}. Variables are factors that can change or be changed in an experiment. Understanding different types of variables is essential for designing controlled and informative investigations.  There are three main types of variables we focus on in experimental design:

\begin{keyconcept}{Variables in Experiments}
\begin{itemize}
    \item \textbf{Independent Variable}: The variable that you \textit{change} or manipulate in your experiment. It is the factor you are testing.
    \item \textbf{Dependent Variable}: The variable that you \textit{measure} or observe. It is the factor that is expected to change in response to changes in the independent variable.
    \item \textbf{Controlled Variables}:  All other variables that you keep \textit{constant} throughout the experiment to ensure that only the independent variable is affecting the dependent variable.
\end{itemize}
\end{keyconcept}

Let's break down each type with examples:

\subsubsection{Independent Variable}

The \keyword{independent variable} is the 'cause' in your experiment. It's what you deliberately alter to see its effect on something else.  Think of it as the variable you are 'in control' of changing.  You might choose different levels or amounts of the independent variable to test.

\begin{example}
\textbf{Scenario:} A student wants to investigate how different amounts of fertiliser affect the growth of bean plants.

\textbf{Independent Variable:} The \textbf{amount of fertiliser} given to each plant.  The student will likely use different concentrations or volumes of fertiliser for different groups of plants.
\end{example}

\subsubsection{Dependent Variable}

The \keyword{dependent variable} is the 'effect' you are measuring. It's what you observe and record as data in your experiment.  It is expected to change in response to the changes you make to the independent variable.

\begin{example}
\textbf{Continuing the fertiliser example:}

\textbf{Dependent Variable:} The \textbf{growth of the bean plants}. This could be measured in various ways, such as plant height, number of leaves, or biomass (total weight of the plant).  The student will measure plant growth to see if it is affected by the different amounts of fertiliser.
\end{example}

\subsubsection{Controlled Variables}

\keyword{Controlled variables} are crucial for ensuring that your experiment is a fair test.  These are all the other factors that could potentially affect the dependent variable, but you need to keep them constant across all experimental groups.  By controlling these variables, you can be more confident that any changes you observe in the dependent variable are indeed due to the changes in the independent variable, and not something else.

\begin{example}
\textbf{Still with the fertiliser example:}

\textbf{Controlled Variables:}  To ensure a fair test, the student would need to control variables such as:
\begin{itemize}
    \item \textbf{Type of bean plant}:  Using the same variety of bean plant for all groups.
    \item \textbf{Type of soil}:  Using the same type and amount of soil in each pot.
    \item \textbf{Amount of water}:  Watering each plant with the same amount of water at the same frequency.
    \item \textbf{Light exposure}:  Ensuring all plants receive the same amount and intensity of light.
    \item \textbf{Temperature}:  Keeping all plants in the same temperature conditions.
    \item \textbf{Size of pot}: Using pots of the same size.
\end{itemize}
If any of these controlled variables were not kept constant, it would be difficult to know whether changes in plant growth were due to the fertiliser, or due to differences in watering, light, or soil, for instance.
\end{example}

\begin{figure}
\centering
\fbox{\textbf{Figure 1.1:} Diagram illustrating the relationship between independent, dependent, and controlled variables in an experiment. (Figure to be added later depicting arrows showing how the independent variable is manipulated, controlled variables are kept constant, and the dependent variable is measured in response.)}
\caption{Visual representation of variables in an experiment.}
\end{figure}

\begin{stopandthink}
Imagine you are investigating how the type of exercise affects heart rate. Identify the independent, dependent, and at least three controlled variables in this experiment.
\end{stopandthink}

\begin{tieredquestions}{Basic}
\begin{enumerate}
    \item Define the term 'variable' in the context of scientific experiments.
    \item What is the purpose of controlling variables in an experiment?
    \item In an experiment testing the effect of sunlight on plant growth, identify the:
    \begin{enumerate}
        \item Independent variable
        \item Dependent variable
        \item Two controlled variables
    \end{enumerate}
\end{enumerate}
\end{tieredquestions}

\begin{tieredquestions}{Intermediate}
\begin{enumerate}
    \item Explain the difference between an independent and a dependent variable, using your own example.
    \item Why is it important to only change one independent variable at a time in an experiment?
    \item Design an experiment to investigate how the temperature of water affects how quickly sugar dissolves.  Identify the independent, dependent, and at least three controlled variables.
\end{enumerate}
\end{tieredquestions}

\begin{tieredquestions}{Advanced}
\begin{enumerate}
    \item  Critically evaluate the following experimental design: A student wants to test if listening to music helps them concentrate while studying. They study for one hour with music and one hour without music and compare their test scores.  Identify potential flaws in this design and suggest improvements, focusing on variables.
    \item Explain how failing to control variables can impact the validity of experimental results.
    \item  Design an experiment to investigate a factor that affects the rate of a chemical reaction (e.g., reaction between baking soda and vinegar). Clearly identify all variables and outline your experimental procedure.
\end{enumerate}
\end{tieredquestions}

\subsection{Control Groups and Experimental Groups}

To further strengthen experimental design, we often use \keyword{control groups} and \keyword{experimental groups}.

\begin{keyconcept}{Control and Experimental Groups}
\begin{itemize}
    \item \textbf{Control Group}: A group in an experiment that does \textit{not} receive the treatment or change in the independent variable. It serves as a baseline for comparison.
    \item \textbf{Experimental Group(s)}:  The group(s) in an experiment that \textit{do} receive the treatment or change in the independent variable.
\end{itemize}
\end{keyconcept}

The \keyword{control group} acts as a standard against which you can compare the results of your \keyword{experimental group(s)}.  It helps to isolate the effect of the independent variable.

\begin{example}
\textbf{Returning to the fertiliser experiment:}

\begin{itemize}
    \item \textbf{Control Group}: Plants that receive \textbf{no fertiliser} (or a standard, very low amount). This group shows the 'normal' growth of bean plants without added fertiliser.
    \item \textbf{Experimental Groups}: Plants that receive \textbf{different amounts of fertiliser} (e.g., low, medium, high concentrations). These groups show how plant growth is affected by varying levels of fertiliser.
\end{itemize}
By comparing the growth of plants in the experimental groups to the growth in the control group, the student can determine if and how fertiliser affects plant growth. If the experimental groups show significantly different growth compared to the control group, it suggests that the fertiliser is having an effect.

\end{example}

\begin{investigation}{Designing a Simple Experiment: Paper Aeroplanes}
\textbf{Aim:} To investigate how the design of a paper aeroplane affects the distance it flies.

\textbf{Materials:}
\begin{itemize}
    \item A4 paper (same type and weight)
    \item Ruler
    \item Measuring tape
    \item Protractor (optional, for precise folding)
    \item Open space for flying aeroplanes
\end{itemize}

\textbf{Procedure:}
\begin{enumerate}
    \item Research different paper aeroplane designs online or in books. Choose at least three different designs that you think will fly different distances.  Ensure one design is relatively simple and another is more complex.
    \item For each design, create at least three paper aeroplanes, making sure to fold them as consistently as possible.
    \item Choose one design to be your 'control' aeroplane (perhaps the simplest design). The other designs will be your 'experimental' aeroplanes.
    \item Decide on a consistent method for launching the aeroplanes (e.g., same throwing angle, same force).  Practise your throwing technique to make it as consistent as possible.
    \item In a clear and open space, launch each aeroplane design at least three times. Measure and record the distance flown for each flight using the measuring tape.  Record your results in a table.
    \item Calculate the average distance flown for each aeroplane design.
    \item Create a bar graph to represent your results, showing the average distance flown for each design.
    \item Analyse your results.  Which aeroplane design flew the furthest? Was there a significant difference in distance between the designs?
    \item Discuss any limitations of your experiment and suggest improvements for future investigations.  Consider variables that might have been difficult to control.
\end{enumerate}

\textbf{Variables to consider:}
\begin{itemize}
    \item \textbf{Independent Variable:} The \textbf{design of the paper aeroplane}.
    \item \textbf{Dependent Variable:} The \textbf{distance the paper aeroplane flies}.
    \item \textbf{Controlled Variables:} Type of paper, size of paper, folding technique (try to be consistent), launching force, launching angle, environmental conditions (wind).
\end{itemize}

\textbf{Safety:} Ensure you have a clear and safe space to fly paper aeroplanes. Be mindful of others around you.

\end{investigation}

\begin{stopandthink}
Why is it important to have a control group in the paper aeroplane investigation? What would you learn if you only tested different aeroplane designs without a control?
\end{stopandthink}

\begin{tieredquestions}{Basic}
\begin{enumerate}
    \item What is the purpose of a control group in an experiment?
    \item Give an example of a situation where a control group might not be necessary.
    \item In the paper aeroplane investigation, which aeroplane design acted as the control group (as suggested in the procedure)?
\end{enumerate}
\end{tieredquestions}

\begin{tieredquestions}{Intermediate}
\begin{enumerate}
    \item Explain, in your own words, the difference between a control group and an experimental group.
    \item Describe a scenario where having multiple experimental groups would be beneficial in an experiment.
    \item  Design an experiment to test the effectiveness of different brands of washing-up liquid in removing grease.  Include a control group and at least two experimental groups in your design.
\end{enumerate}
\end{tieredquestions}

\begin{tieredquestions}{Advanced}
\begin{enumerate}
    \item  Discuss the ethical considerations of using control groups in medical research, particularly when testing new treatments for serious illnesses.
    \item  Explain how the use of a placebo control group helps to improve the validity of clinical trials for new drugs. \marginnote{\challenge{Placebo Effect} Research the 'placebo effect' and how it can influence experimental results, especially in studies involving human subjects.}
    \item  In the paper aeroplane investigation, suggest additional control measures that could be implemented to improve the reliability of the results.  Consider factors beyond those listed in the procedure.
\end{enumerate}
\end{tieredquestions}

\section{Reliability and Validity: Ensuring Trustworthy Results}

Once you have designed your experiment and collected data, you need to consider how trustworthy your results are.  Two key concepts help us evaluate the quality of scientific data: \keyword{reliability} and \keyword{validity}.

\subsection{What is Reliability?}

\keyword{Reliability} refers to the consistency and repeatability of your measurements and results.  A reliable experiment is one that, if repeated multiple times under the same conditions, would produce similar results.  Think of it as how consistently your experiment 'performs'.

\begin{keyconcept}{Reliability}
\textbf{Reliability} is the extent to which your measurements are consistent and repeatable.  Reliable results are reproducible.
\end{keyconcept}

\marginnote{Analogy} Imagine using a measuring tape to measure the length of a table multiple times. If you consistently get the same measurement (or very similar measurements each time), the measuring tape (and your measurement process) is reliable. If you get wildly different measurements each time, it is unreliable.

Factors that can affect the reliability of your results include:

\begin{itemize}
    \item \textbf{Sample size}:  Larger sample sizes generally lead to more reliable results.  Repeating measurements or testing more subjects reduces the impact of random variations.
    \item \textbf{Repetitions}: Repeating the experiment multiple times and obtaining similar results increases reliability.
    \item \textbf{Precision of instruments}: Using precise measuring instruments (e.g., a digital scale rather than estimations) improves reliability.
    \item \textbf{Control of variables}:  Consistent control of variables across trials contributes to reliability.
    \item \textbf{Clear procedures}:  Following a well-defined and documented procedure ensures that the experiment can be replicated reliably by yourself or others.
\end{itemize}

\begin{figure}
\centering
\fbox{\textbf{Figure 1.2:} Graph showing reliable vs. unreliable data. (Figure to be added later depicting two sets of data points on a graph. One set showing data points clustered closely together around a line of best fit (reliable), and another set showing data points scattered widely (unreliable).)}
\caption{Visual representation of reliable and unreliable data.}
\end{figure}

\subsection{What is Validity?}

\keyword{Validity} refers to whether your experiment is actually measuring what you intend to measure and whether your conclusions are justified and accurate.  A valid experiment answers your research question and provides meaningful insights.  Think of validity as how accurately your experiment 'hits the target' of your investigation.

\begin{keyconcept}{Validity}
\textbf{Validity} is the extent to which your experiment measures what it is supposed to measure and whether your conclusions are accurate and justified. Valid results are meaningful and relevant to the research question.
\end{keyconcept}

\marginnote{Analogy} Imagine you are aiming at a target with arrows.  \textbf{Reliability} is like consistently hitting the same area of the target (whether or not it's the bullseye). \textbf{Validity} is like hitting the bullseye – actually hitting what you are aiming for. You can be reliable without being valid (consistently hitting the wrong area), but you can't be valid without being at least somewhat reliable (you need some consistency to hit the bullseye).

Factors that can affect the validity of your results include:

\begin{itemize}
    \item \textbf{Experimental design}: A poorly designed experiment may not accurately test your hypothesis or answer your research question.
    \item \textbf{Uncontrolled variables}: If important variables are not controlled, they may influence the dependent variable, making it unclear whether the independent variable is actually responsible for the observed changes.
    \item \textbf{Measurement errors}:  Inaccurate or biased measurements can compromise validity.
    \item \textbf{Sampling bias}: If your sample is not representative of the population you are studying, your conclusions may not be valid for the broader population.
    \item \textbf{Confounding variables}: Variables that are not controlled and are related to both the independent and dependent variables can lead to false conclusions about cause and effect.
\end{itemize}

\begin{figure}
\centering
\fbox{\textbf{Figure 1.3:} Graph showing valid vs. invalid data. (Figure to be added later depicting two scenarios: One showing data that aligns with the intended measurement (valid), and another showing data that is measuring something else or is irrelevant to the research question (invalid).)}
\caption{Visual representation of valid and invalid data.}
\end{figure}

\subsection{Reliability vs. Validity: Key Differences}

It is crucial to understand the difference between reliability and validity.  Results can be reliable but not valid, and ideally, we strive for both reliability and validity in scientific investigations.

\begin{itemize}
    \item \textbf{Reliability is about consistency; validity is about accuracy.}
    \item \textbf{Reliability is necessary but not sufficient for validity.} You can have reliable results that are consistently wrong if your experiment is not valid.
    \item \textbf{Validity is more important than reliability.} If your experiment is valid, it means you are measuring what you intend to measure and drawing accurate conclusions, even if there is some variability in your measurements. However, high reliability increases confidence in validity.
\end{itemize}

\begin{stopandthink}
Consider an experiment where a student is using a broken ruler to measure the length of several objects. Would their measurements be reliable? Would they be valid for determining the true length of the objects? Explain your reasoning.
\end{stopandthink}

\begin{tieredquestions}{Basic}
\begin{enumerate}
    \item Define the terms 'reliability' and 'validity' in the context of scientific experiments.
    \item Explain why reliable results are important in science.
    \item Give one example of a factor that can reduce the reliability of an experiment.
\end{enumerate}
\end{tieredquestions}

\begin{tieredquestions}{Intermediate}
\begin{enumerate}
    \item Explain the difference between reliability and validity using an analogy (other than the ones provided in the text).
    \item Can results be reliable but not valid? Explain with an example.
    \item  Describe how increasing the sample size in an experiment can improve reliability.
\end{enumerate}
\end{tieredquestions}

\begin{tieredquestions}{Advanced}
\begin{enumerate}
    \item Critically analyse the following statement: "High reliability always guarantees high validity."  Do you agree or disagree? Justify your answer.
    \item  Design an experiment to investigate the reaction time of students using an online reaction time test. Discuss how you would ensure both reliability and validity in your experiment. Consider potential sources of error and bias.
    \item  Explain the concept of 'internal validity' and 'external validity' in the context of experimental design.  How are these different aspects of validity important in scientific research? \marginnote{\challenge{Types of Validity} Research different types of validity in scientific research, such as internal validity, external validity, construct validity, and content validity. How are they relevant to different types of scientific investigations?}
\end{enumerate}
\end{tieredquestions}

\section{Formulating a Research Question: Guiding Your Research}

The starting point of any scientific investigation, especially your Student Research Project (SRP), is a well-formulated \keyword{research question}. A research question is a clear, focused, and specific question about the natural world that you aim to answer through your investigation. It acts as the compass guiding your entire research journey.

\subsection{What Makes a Good Research Question?}

Not all questions are good research questions. A good research question possesses several key characteristics:

\begin{keyconcept}{Characteristics of a Good Research Question}
A good research question is:
\begin{itemize}
    \item \textbf{Focused}: It is specific and addresses a narrow topic, rather than being too broad.
    \item \textbf{Specific}: It is clear and unambiguous, leaving no room for misinterpretation.
    \item \textbf{Researchable}: It can be investigated through scientific methods, involving data collection and analysis.
    \item \textbf{Relevant}: It is interesting and important, contributing to scientific knowledge or addressing a practical problem.
    \item \textbf{Feasible}: It can be answered within the available time, resources, and ethical constraints.
\end{itemize}
\end{keyconcept}

\begin{example}
\textbf{Example of a weak research question (Too broad):}

\textit{How do humans affect the environment?}

This question is far too broad. 'Humans' and 'environment' are vast topics. It's impossible to answer this question effectively in a single research project.

\textbf{Improved, focused research question:}

\textit{How does plastic pollution affect the growth of seagrass in local coastal areas?}

This question is much more focused and specific. It narrows down the topic to plastic pollution, seagrass, and a local area.  It is also more researchable and feasible.
\end{example}

\begin{figure}
\centering}
\fbox{\textbf{Figure 1.4:} Flowchart showing the steps in formulating a research question. (Figure to be added later depicting a flowchart starting with 'Broad Topic' -> 'Identify Area of Interest' -> 'Brainstorm Questions' -> 'Refine and Focus' -> 'Check Feasibility and Relevance' -> 'Final Research Question').}
\caption{Steps in formulating a research question.}
\end{figure}

\subsection{From Topic to Research Question}

Formulating a good research question is often an iterative process. You might start with a broad topic of interest and then refine it step-by-step into a specific and researchable question. Here's a general process:

\begin{enumerate}
    \item \textbf{Choose a broad topic area}: Start with a general area of science that interests you (e.g., plants, animals, chemistry, physics, space).
    \item \textbf{Narrow down your topic}:  Within your broad topic, identify a more specific area of interest.  For example, if your broad topic is 'plants', you might narrow it down to 'plant growth', 'plant diseases', or 'plant adaptations'.
    \item \textbf{Brainstorm questions}:  Think of questions related to your narrowed topic.  What are you curious about? What problems or phenomena do you want to understand better?
    \item \textbf{Evaluate your questions}:  Review your brainstormed questions and assess them against the criteria for a good research question (focused, specific, researchable, relevant, feasible).
    \item \textbf{Refine and focus your best question}:  Select the question that best meets the criteria.  Refine its wording to make it even clearer and more specific. Ensure it is researchable within your constraints.
\end{enumerate}

\begin{example}
\textbf{Example of refining a research question:}

\textbf{Broad Topic:}  Food and Nutrition

\textbf{Narrowed Topic:}  Effects of Sugar

\textbf{Brainstormed Questions:}
\begin{itemize}
    \item Is sugar bad for you? (Too general)
    \item What are the effects of sugar on health? (Still too broad)
    \item How does sugar affect children? (Better, but still broad)
    \item Does sugar make children hyperactive? (More specific, but potentially based on a misconception)
    \item \textbf{How does the consumption of sugary drinks affect the concentration levels of teenagers during study sessions?} (Focused, specific, researchable, relevant, feasible)
\end{itemize}

\textbf{Final Research Question:} \textit{How does the consumption of sugary drinks compared to water affect the concentration levels of teenagers during a one-hour study session?}
\end{example}

\begin{stopandthink}
Think about a scientific topic that interests you.  It could be anything from space exploration to cooking.  Try to brainstorm a broad research question related to this topic. Then, try to refine it into a more focused and specific research question, considering the criteria for a good research question.
\end{stopandthink}

\begin{tieredquestions}{Basic}
\begin{enumerate}
    \item What is a research question?
    \item List three characteristics of a good research question.
    \item Identify which of the following is a better research question and explain why:
    \begin{enumerate}
        \item \textit{Are plants important?}
        \item \textit{How does the amount of water affect the growth rate of tomato plants?}
    \end{enumerate}
\end{enumerate}
\end{tieredquestions}

\begin{tieredquestions}{Intermediate}
\begin{enumerate}
    \item Explain why a broad research question is less effective than a focused research question for a scientific investigation.
    \item  Take the broad topic of 'pollution' and develop two different research questions that are more focused and specific.
    \item  Evaluate the following research question: \textit{What is the best type of music?}  Explain why this is not a good research question and suggest how it could be improved to be more researchable in a scientific context.
\end{enumerate}
\end{tieredquestions}

\begin{tieredquestions}{Advanced}
\begin{enumerate}
    \item  Discuss the importance of 'relevance' and 'feasibility' when formulating a research question, especially for a student research project.
    \item  Develop three different research questions related to the topic of climate change, each focusing on a different aspect (e.g., impacts on ecosystems, technological solutions, social implications). Ensure your questions are focused, specific, researchable, relevant, and feasible for a Stage 5 research project.
    \item  Consider a research question that might be ethically challenging to investigate directly in humans (e.g., effects of sleep deprivation on cognitive function).  Discuss alternative approaches to investigate this question ethically and suggest a modified, ethically sound research question. \marginnote{\challenge{Ethical Research} Research ethical guidelines for scientific research, particularly when involving human or animal subjects. Understand the principles of informed consent, beneficence, non-maleficence, and justice in research ethics.}
\end{enumerate}

\section{Background Research: Building Your Knowledge Base}

Once you have a well-defined research question, the next crucial step is to conduct \keyword{background research}. This involves gathering information about your topic from reliable sources to understand what is already known, identify gaps in knowledge, and refine your research question further if needed.

\subsection{Why is Background Research Important?}

Background research is essential for several reasons:

\begin{itemize}
    \item \textbf{Understanding existing knowledge}: It helps you learn what scientists already know about your topic. You don't want to 'reinvent the wheel' or investigate something that has already been thoroughly studied and answered.
    \item \textbf{Identifying gaps in knowledge}: Background research can reveal areas where there are still unanswered questions or disagreements in the scientific community. These gaps can become the focus of your own research.
    \item \textbf{Refining your research question}:  As you learn more about your topic, you may realise that your initial research question is too broad, too narrow, or not feasible. Background research can help you refine and focus your question to make it more researchable and meaningful.
    \item \textbf{Developing your methodology}: Reading about how other scientists have investigated similar questions can give you ideas for your own experimental design, data collection methods, and analysis techniques.
    \item \textbf{Providing context for your findings}:  When you write your scientific report, you will need to compare your results to what is already known. Background research provides the necessary context for interpreting and discussing your findings.
\end{itemize}

\subsection{Finding Credible Sources}

Not all sources of information are equally reliable.  In scientific research, it is crucial to use \keyword{credible sources} – sources that are trustworthy, accurate, and based on evidence.  Examples of credible sources include:

\begin{itemize}
    \item \textbf{Scientific journals}: These are publications where scientists publish the results of their research after a rigorous peer-review process (where other experts in the field evaluate the quality and validity of the research).  Examples include journals like \textit{Nature}, \textit{Science}, \textit{The Lancet}, and journals specific to different scientific disciplines.
    \item \textbf{Textbooks}:  Science textbooks, especially those used at university level, are generally reliable sources of established scientific knowledge.
    \item \textbf{Reputable science websites}: Websites of well-known scientific organisations, universities, government agencies (e.g., CSIRO, NASA, National Geographic, BBC Science) often provide accurate and up-to-date science information for the public.
    \item \textbf{Encyclopedias and scientific databases}:  Reputable encyclopedias (like \textit{Encyclopaedia Britannica}) and scientific databases (like \textit{PubMed}, \textit{Google Scholar}) can be good starting points for finding information and research articles.
\end{itemize}

\begin{figure}
\centering
\fbox{\textbf{Figure 1.5:} Example of a credible scientific journal article cover. (Figure to be added later showing a sample cover of a scientific journal, highlighting journal title, volume, issue, and example article title).}
\caption{Example of a scientific journal.}
\end{figure}

\marginnote{\historylink{Peer Review} The peer-review process in scientific journals is a cornerstone of scientific quality control. It has evolved over centuries to ensure the rigour and validity of published research. Investigate the history and importance of peer review in science.}

\textbf{Sources to be cautious with (less credible for scientific research):}

\begin{itemize}
    \item \textbf{General websites (Wikipedia, blogs, forums):} While these can sometimes be helpful for initial exploration, they are often not peer-reviewed and may contain inaccurate or biased information.  Wikipedia can be a starting point, but always verify information from more credible sources.
    \item \textbf{Social media}: Social media platforms are not designed for scientific accuracy and often spread misinformation.
    \item \textbf{Websites with obvious bias or commercial interests}: Be wary of websites that promote specific products, have strong opinions without evidence, or are associated with organisations with a clear agenda.
\end{itemize}

\subsection{Evaluating Source Credibility}

When evaluating a source, consider the following:

\begin{itemize}
    \item \textbf{Author/Source}: Who is the author or organisation? Are they experts in the field? What are their credentials or affiliations?
    \item \textbf{Publisher}: Who published the source? Is it a reputable scientific publisher, university press, or well-known scientific organisation?
    \item \textbf{Date of publication}: Is the information up-to-date? Science is constantly evolving, so recent sources are often preferable, especially in rapidly advancing fields. However, seminal older works can also be important.
    \item \textbf{Purpose}: What is the purpose of the source? Is it to inform, educate, persuade, or sell something? Be aware of potential biases.
    \item \textbf{Evidence and referencing}: Does the source provide evidence to support its claims? Does it cite other credible sources?  Look for references or citations to original research.
    \end{itemize}

\begin{investigation}{Evaluating Online Sources: Climate Change}
\textbf{Aim:} To evaluate the credibility of different online sources of information about climate change.

\textbf{Materials:}
\begin{itemize}
    \item Internet access
    \item Worksheet for recording source evaluation (see example below)
\end{itemize}

\textbf{Procedure:}
\begin{enumerate}
    \item Search online for information about climate change using a search engine (e.g., Google, Bing). Use search terms like "climate change evidence", "effects of climate change", "climate change solutions".
    \item Select at least three different websites that appear in your search results. Try to choose a variety of source types (e.g., a website from a scientific organisation, a news website, a blog, a government agency website).
    \item For each website, evaluate its credibility using the criteria discussed above (author/source, publisher, date, purpose, evidence and referencing).  Use a worksheet like the example below to record your evaluation.
    \item Based on your evaluation, rank the websites from most credible to least credible.
    \item Discuss your findings with classmates or in a group.  Which sources were considered most credible? Why? Which sources were less credible? What were the red flags?
\end{enumerate}

\textbf{Example Worksheet for Source Evaluation:}

| Website URL | Author/Source (Who?) | Publisher (Who?) | Date of Publication | Purpose (Why?) | Evidence & Referencing (How?) | Credibility Rating (1-5, 5=Highest) | Justification for Rating |
|---|---|---|---|---|---|---|---|
| [Website URL 1] |  |  |  |  |  |  |  |
| [Website URL 2] |  |  |  |  |  |  |  |
| [Website URL 3] |  |  |  |  |  |  |  |

\textbf{Discussion Points:}
\begin{itemize}
    \item What are the key indicators of a credible online source for scientific information?
    \item How can you distinguish between reliable and unreliable information online?
    \item Why is it important to use credible sources in scientific research, including your SRP?
\end{itemize}

\end{investigation}

\begin{stopandthink}
