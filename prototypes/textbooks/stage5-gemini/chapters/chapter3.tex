```latex
\chapter{Ecosystems and Environmental Science}

\begin{marginfigure}
\includegraphics[width=\textwidth]{placeholder-ecosystem.jpg}
\caption{A diverse ecosystem: a coral reef teeming with life. \textit{Image to be added later depicting a vibrant coral reef ecosystem.}}
\end{marginfigure}

\section{Introducing Ecosystems: The Web of Life}

Welcome to the fascinating world of \keyword{ecosystems}!  Imagine a bustling city, but instead of buildings and roads, think of forests, oceans, deserts, or even your own garden.  Each of these is an ecosystem, a complex and dynamic community where living organisms interact with each other and their non-living environment.  Understanding ecosystems is crucial because they are the foundation of life on Earth, providing us with everything from the air we breathe to the food we eat.

\begin{marginnote}
\textit{Etymology of Ecosystem:} The term "ecosystem" was coined in 1935 by British ecologist Arthur Roy Clapham, and later popularised by Sir Arthur George Tansley.  It combines "eco" (from ecology) and "system" to highlight the interconnectedness of living and non-living components. \historylink{Arthur Roy Clapham, Arthur George Tansley}
\end{marginnote}

In this chapter, we will explore the key components of ecosystems, how they function, and the crucial role they play in maintaining the health of our planet. We will also delve into \keyword{environmental science}, the study of how humans interact with the environment and the challenges we face in ensuring a sustainable future.  Get ready to become an ecosystem explorer!

\begin{stopandthink}
Think about a local park or green space near you. What living things do you see? What non-living things are present? How do you think they interact with each other?
\end{stopandthink}

\section{What is an Ecosystem?}

At its heart, an \keyword{ecosystem} is a community of living organisms (plants, animals, and microorganisms) in conjunction with the non-living components of their environment (things like air, water, soil, and sunlight), interacting as a system. These interactions are vital for the flow of energy and the cycling of nutrients, which are essential for life.

\subsection{Biotic and Abiotic Factors}

Every ecosystem is made up of two main types of components:

\begin{itemize}
    \item \textbf{Biotic factors}: These are the living parts of an ecosystem. They include:
    \begin{itemize}
        \item \keyword{Producers}: Organisms that make their own food, usually through \keyword{photosynthesis}.  Plants are the primary producers in most terrestrial ecosystems, while algae and phytoplankton are key producers in aquatic ecosystems.
        \item \keyword{Consumers}: Organisms that cannot make their own food and must eat other organisms. Consumers can be further divided into herbivores (plant-eaters), carnivores (meat-eaters), omnivores (eat both plants and animals), and detritivores (eat dead organic matter).
        \item \keyword{Decomposers}: Organisms like bacteria and fungi that break down dead organic matter, returning essential nutrients to the ecosystem.
    \end{itemize}
    \item \textbf{Abiotic factors}: These are the non-living parts of an ecosystem. They include:
    \begin{itemize}
        \item \keyword{Sunlight}: The primary source of energy for most ecosystems.
        \item \keyword{Water}: Essential for all living organisms; its availability varies greatly between ecosystems.
        \item \keyword{Temperature}: Affects the metabolic rates of organisms and the types of species that can survive in an ecosystem.
        \item \keyword{Soil/Substrate}: Provides nutrients and physical support for plants and other organisms. In aquatic ecosystems, this might be the sediment at the bottom of a lake or ocean.
        \item \keyword{Air/Atmosphere}: Provides gases like oxygen and carbon dioxide, crucial for respiration and photosynthesis.
        \item \keyword{Nutrients}: Minerals and chemicals required for growth and survival, such as nitrogen, phosphorus, and potassium.
    \end{itemize}
\end{itemize}

\begin{marginnote}
\textit{Photosynthesis:} The process by which green plants and some other organisms use sunlight to synthesise foods with the help of chlorophyll pigment. \ce{6CO2 + 6H2O -> C6H12O6 + 6O2} \mathlink{Photosynthesis equation}
\end{marginnote}

The interplay between biotic and abiotic factors determines the characteristics and functioning of an ecosystem.  For example, in a desert ecosystem, limited water availability (abiotic factor) shapes the types of plants and animals (biotic factors) that can survive there. Cacti and camels are well-adapted to desert conditions, while lush rainforest plants and fish would not thrive.

\begin{stopandthink}
Consider a pond ecosystem. List three biotic factors and three abiotic factors that are important in this ecosystem. Explain how one of the abiotic factors might influence one of the biotic factors.
\end{stopandthink}

\subsection{Types of Ecosystems}

Ecosystems can be broadly classified into different types based on their dominant environment.  Here are some major categories:

\begin{itemize}
    \item \textbf{Terrestrial Ecosystems}: These are ecosystems found on land. Examples include:
    \begin{itemize}
        \item \keyword{Forests}: Dominated by trees, forests are vital for carbon storage and biodiversity.  They can be further classified into rainforests, deciduous forests, coniferous forests, etc.
        \item \keyword{Grasslands}: Characterised by grasses as the dominant vegetation.  Examples include savannas and prairies.
        \item \keyword{Deserts}:  Dry ecosystems with sparse vegetation, adapted to conserve water.
        \item \keyword{Tundra}: Cold, treeless regions found in high latitudes or altitudes, with permafrost (permanently frozen soil).
    \end{itemize}
    \item \textbf{Aquatic Ecosystems}: These are ecosystems found in water. Examples include:
    \begin{itemize}
        \item \keyword{Freshwater Ecosystems}:  Include lakes, rivers, ponds, streams, and wetlands. These are characterised by low salt concentration.
        \item \keyword{Marine Ecosystems}: Found in oceans and seas, with high salt concentration.  Examples include coral reefs, estuaries, open ocean, and deep-sea vents.
    \end{itemize}
    \item \textbf{Artificial Ecosystems}: These are ecosystems created or significantly modified by humans. Examples include:
    \begin{itemize}
        \item \keyword{Agricultural Ecosystems}: Farmland used for growing crops or raising livestock.
        \item \keyword{Urban Ecosystems}: Cities and towns, which include parks, gardens, and built environments.
        \item \keyword{Aquariums and Terrariums}: Controlled environments created to mimic natural ecosystems, often for study or display.
    \end{itemize}
\end{itemize}

\begin{marginnote}
\challenge{Biome vs. Ecosystem:} While often used interchangeably, a \keyword{biome} is a larger-scale ecosystem classification, defined by climate and dominant vegetation type.  For example, a tropical rainforest is a biome, and within it, there are many individual ecosystems.
\end{marginnote}

Each type of ecosystem has unique characteristics and supports a specific set of organisms adapted to its conditions.  Understanding these different types helps us appreciate the diversity of life on Earth and the specific challenges they face.

\begin{investigation}{Ecosystem Exploration: Your Local Environment}
\textbf{Objective:} To identify and describe the biotic and abiotic components of a local ecosystem.

\textbf{Materials:} Notebook, pen/pencil, camera (optional), field guide for local plants and animals (optional).

\textbf{Procedure:}
\begin{enumerate}
    \item Choose a local ecosystem to study. This could be your garden, a park, a schoolyard, a nearby pond, or even a patch of roadside vegetation.
    \item Observe the ecosystem carefully for at least 30 minutes.
    \item Identify and list as many biotic factors as you can. Try to classify them as producers, consumers, or decomposers if possible.  Note down the types of plants, animals (insects, birds, mammals etc.), and any signs of decomposers (like fungi or decaying leaves).
    \item Identify and list as many abiotic factors as you can. Consider sunlight, temperature, water availability (is it dry or damp?), soil type (sandy, clayey, etc.), and any other relevant non-living components.
    \item Draw a simple diagram of the ecosystem, labelling the biotic and abiotic components you observed.
    \item Write a short paragraph describing the interactions you observed between biotic and abiotic factors. For example, how do plants use sunlight? What are animals eating? How does the soil affect plant growth?
\end{enumerate}

\textbf{Analysis:}
\begin{enumerate}
    \item What were the most common biotic factors you observed in your chosen ecosystem?
    \item Which abiotic factor do you think is most important in shaping this particular ecosystem? Why?
    \item How might human activities impact this local ecosystem?
\end{enumerate}

\textbf{Extension:} Research and identify the type of biome your local ecosystem belongs to.  How does your local ecosystem fit into the larger biome classification?
\end{investigation}

\begin{tieredquestions}{Section 2: What is an Ecosystem?}
\textbf{Basic:}
\begin{enumerate}
    \item Define the term "ecosystem".
    \item List three biotic factors and three abiotic factors in an ecosystem.
    \item Give one example of a terrestrial ecosystem and one example of an aquatic ecosystem.
\end{enumerate}

\textbf{Intermediate:}
\begin{enumerate}
    \item Explain the difference between biotic and abiotic factors and give examples of how they interact in an ecosystem.
    \item Describe the roles of producers, consumers, and decomposers in an ecosystem.
    \item Compare and contrast freshwater and marine ecosystems, highlighting key differences in their abiotic factors.
\end{enumerate}

\textbf{Advanced:}
\begin{enumerate}
    \item  "Ecosystems are dynamic and constantly changing." Discuss this statement, providing examples of both natural and human-induced changes that can occur in ecosystems.
    \item Critically evaluate the concept of "artificial ecosystems." Are they truly ecosystems? What are their limitations and potential benefits?
    \item  Imagine you are designing a terrarium ecosystem. Describe the biotic and abiotic factors you would include and explain how you would ensure its sustainability.
\end{enumerate}
\end{tieredquestions}


\section{Components and Functioning of Ecosystems}

Ecosystems are not just collections of organisms and their environment; they are dynamic systems where energy flows and nutrients cycle. Understanding these processes is key to appreciating how ecosystems function and why they are so important.

\subsection{Energy Flow in Ecosystems}

Energy in most ecosystems originates from the sun.  Producers, like plants, capture this solar energy through \keyword{photosynthesis} and convert it into chemical energy in the form of glucose (sugar). This energy then flows through the ecosystem as organisms consume each other.

\begin{itemize}
    \item \textbf{Food Chains and Food Webs}:  Energy transfer can be represented by \keyword{food chains} and \keyword{food webs}. A food chain is a linear sequence showing how energy and nutrients are transferred from one organism to another. For example:

    \begin{example}
        \textbf{Food Chain Example:} Grass $\rightarrow$ Grasshopper $\rightarrow$ Frog $\rightarrow$ Snake $\rightarrow$ Hawk
    \end{example}

    In this food chain, energy flows from the grass (producer) to the grasshopper (herbivore/primary consumer), then to the frog (carnivore/secondary consumer), and so on.

    \begin{marginnote}
    \textit{Trophic Levels:} Each step in a food chain or food web is called a \keyword{trophic level}. Producers are at the first trophic level, primary consumers at the second, and so on.
    \end{marginnote}

    However, ecosystems are rarely this simple.  Most organisms eat a variety of food sources and are eaten by multiple predators. This interconnected network of food chains is called a \keyword{food web}. Food webs provide a more realistic representation of energy flow in ecosystems.

    \begin{figure}[h]
        \includegraphics[width=0.7\textwidth]{placeholder-foodweb.jpg}
        \caption{A simplified food web in a grassland ecosystem. Arrows indicate the direction of energy flow. \textit{Image to be added later depicting a grassland food web with various plants, herbivores, carnivores, and decomposers.}}
    \end{figure}

    \item \textbf{Ecological Pyramids}:  Energy transfer between trophic levels is not very efficient.  Only about 10% of the energy is transferred from one trophic level to the next. The rest is lost as heat during metabolic processes. This inefficiency limits the length of food chains and the number of trophic levels in most ecosystems.  This energy loss can be visually represented by \keyword{ecological pyramids}.

    \begin{itemize}
        \item \textbf{Pyramid of Energy}: Shows the amount of energy available at each trophic level.  The base of the pyramid (producers) has the most energy, and energy decreases at each successive level.
        \item \textbf{Pyramid of Biomass}: Represents the total mass of living organisms at each trophic level.  Generally, biomass also decreases at higher trophic levels.
        \item \textbf{Pyramid of Numbers}: Shows the number of organisms at each trophic level.  This pyramid can sometimes be inverted (e.g., in a forest ecosystem, there may be fewer trees than insects feeding on them).
    \end{itemize}

    \begin{figure}[h]
        \includegraphics[width=0.6\textwidth]{placeholder-ecologicalpyramids.jpg}
        \caption{Ecological pyramids: Pyramid of energy, biomass, and numbers. Note the decreasing energy and biomass at higher trophic levels. \textit{Image to be added later depicting examples of each type of ecological pyramid.}}
    \end{figure}
\end{itemize}

\begin{stopandthink}
Why is energy transfer between trophic levels only about 10% efficient? What happens to the "lost" energy?
\end{stopandthink}

\subsection{Nutrient Cycles in Ecosystems}

Unlike energy, which flows through an ecosystem and is eventually lost as heat, nutrients are recycled within ecosystems. These \keyword{nutrient cycles} are essential for maintaining life.  Key nutrient cycles include:

\begin{itemize}
    \item \textbf{Water Cycle (Hydrologic Cycle)}:  The continuous movement of water on, above, and below the surface of the Earth.  Processes include evaporation, transpiration (water release from plants), condensation, precipitation, and runoff.

    \begin{figure}[h]
        \includegraphics[width=0.7\textwidth]{placeholder-watercycle.jpg}
        \caption{The Water Cycle.  Note the different processes involved in the movement of water. \textit{Image to be added later depicting the water cycle with labels for evaporation, condensation, precipitation, transpiration, runoff, etc.}}
    \end{figure}

    \item \textbf{Carbon Cycle}:  The movement of carbon through the biosphere, atmosphere, hydrosphere, and geosphere.  Key processes include photosynthesis, respiration, combustion, and decomposition.  Carbon is a fundamental building block of organic molecules.

    \begin{marginnote}
    \challenge{Carbon Sequestration:}  Forests and oceans play a vital role in \keyword{carbon sequestration}, removing carbon dioxide from the atmosphere and storing it.  Deforestation and ocean acidification reduce carbon sequestration capacity.
    \end{marginnote}

    \begin{figure}[h]
        \includegraphics[width=0.7\textwidth]{placeholder-carboncycle.jpg}
        \caption{The Carbon Cycle.  Illustrates the exchange of carbon between the atmosphere, living organisms, and the Earth. \textit{Image to be added later depicting the carbon cycle with labels for photosynthesis, respiration, combustion, decomposition, etc.}}
    \end{figure}

    \item \textbf{Nitrogen Cycle}: The complex series of processes by which nitrogen is converted between different chemical forms. Nitrogen is essential for proteins and nucleic acids. Key processes include nitrogen fixation (conversion of atmospheric nitrogen to ammonia), nitrification (conversion of ammonia to nitrates and nitrites), denitrification (conversion of nitrates back to nitrogen gas), and assimilation (uptake of nitrogen by plants).

    \begin{marginnote}
    \textit{Nitrogen Fixation:}  Most organisms cannot directly use atmospheric nitrogen (\ce{N2}).  Nitrogen fixation is primarily carried out by certain bacteria, converting \ce{N2} into ammonia (\ce{NH3}) that plants can use. \historylink{Nitrogen Fixation - Haber-Bosch process (industrial fixation)}
    \end{marginnote}

    \begin{figure}[h]
        \includegraphics[width=0.7\textwidth]{placeholder-nitrogencycle.jpg}
        \caption{The Nitrogen Cycle. Shows the various transformations of nitrogen in the environment. \textit{Image to be added later depicting the nitrogen cycle with labels for nitrogen fixation, nitrification, denitrification, assimilation, etc.}}
    \end{figure}

    \item \textbf{Phosphorus Cycle}:  The movement of phosphorus through the lithosphere, hydrosphere, and biosphere. Unlike the other cycles, the phosphorus cycle does not have a significant atmospheric component. Phosphorus is crucial for DNA, RNA, and ATP (energy currency of cells).  It is often a limiting nutrient in ecosystems.

    \begin{figure}[h]
        \includegraphics[width=0.7\textwidth]{placeholder-phosphoruscycle.jpg}
        \caption{The Phosphorus Cycle.  Highlights the movement of phosphorus from rocks to living organisms and back. \textit{Image to be added later depicting the phosphorus cycle with labels for weathering, absorption, decomposition, sedimentation, etc.}}
    \end{figure}
\end{itemize}

These nutrient cycles are interconnected and essential for maintaining the health and productivity of ecosystems. Human activities can significantly disrupt these cycles, leading to environmental problems such as pollution and nutrient imbalances.

\begin{stopandthink}
Choose one nutrient cycle (water, carbon, nitrogen, or phosphorus). Explain in your own words how this cycle works and why it is important for ecosystems.
\end{stopandthink}

\begin{investigation}{Building a Terrarium Ecosystem}
\textbf{Objective:} To create a self-sustaining miniature ecosystem and observe nutrient cycling and energy flow.

\textbf{Materials:}
\begin{itemize}
    \item Clear plastic or glass container with a lid (e.g., a large jar or plastic bottle cut in half)
    \item Gravel or small stones
    \item Activated charcoal (optional, helps with drainage and odour control)
    \item Potting soil or garden soil (ensure it is pesticide-free)
    \item Small plants suitable for terrariums (e.g., ferns, mosses, small succulents)
    \item Small invertebrates (optional, e.g., springtails, small snails, earthworms – ensure they are appropriate for a closed terrarium and ethically sourced)
    \item Water spray bottle
\end{itemize}

\textbf{Procedure:}
\begin{enumerate}
    \item Layer the bottom of the container with gravel or small stones for drainage (about 2-3 cm deep). Add a thin layer of activated charcoal if using.
    \item Add a layer of potting soil or garden soil on top of the gravel/charcoal (about 5-7 cm deep).
    \item Carefully plant your chosen plants in the soil, arranging them attractively.
    \item Gently water the terrarium using a spray bottle until the soil is moist but not waterlogged.
    \item If adding invertebrates, introduce them to the terrarium.
    \item Seal the terrarium with a lid. If using a jar, place the lid on tightly. If using a cut plastic bottle, tape the two halves together securely.
    \item Place the terrarium in a location with indirect sunlight. Avoid direct sunlight, which can overheat the terrarium.
    \item Observe your terrarium ecosystem over several weeks. Note any changes you observe in the plants, soil moisture, and any invertebrates (if present). You should observe condensation forming on the inside of the container, demonstrating the water cycle.
\end{enumerate}

\textbf{Analysis:}
\begin{enumerate}
    \item Describe the water cycle you observe in your terrarium. Where does the water come from? Where does it go?
    \item How does energy flow in your terrarium ecosystem? Where is the source of energy? How is it transferred through the system?
    \item What evidence of nutrient cycling can you observe in your terrarium?
    \item Is your terrarium a closed or open system in terms of matter and energy? Explain.
    \item What are the limitations of a terrarium as a model ecosystem? How does it differ from a natural ecosystem?
\end{enumerate}

\textbf{Extension:} Research different types of terrariums (e.g., open terrariums, desert terrariums, rainforest terrariums).  Design and build a different type of terrarium and compare its characteristics to your closed terrarium.
\end{investigation}


\begin{tieredquestions}{Section 3: Components and Functioning of Ecosystems}
\textbf{Basic:}
\begin{enumerate}
    \item What is the primary source of energy for most ecosystems?
    \item Explain the difference between a food chain and a food web.
    \item Name one type of ecological pyramid and describe what it represents.
    \item List two key processes involved in the water cycle.
\end{enumerate}

\textbf{Intermediate:}
\begin{enumerate}
    \item Describe how energy flows through a food chain, explaining the concept of trophic levels.
    \item Explain why ecological pyramids of energy typically have a pyramid shape (narrowing towards the top).
    \item Choose one nutrient cycle (carbon or nitrogen) and describe its main processes and importance in ecosystems.
    \item How do decomposers contribute to nutrient cycling in ecosystems?
\end{enumerate}

\textbf{Advanced:}
\begin{enumerate}
    \item  "Disruptions to nutrient cycles can have significant environmental consequences." Discuss this statement, providing examples of how human activities can disrupt nutrient cycles and the resulting impacts.
    \item  Compare and contrast the flow of energy and the cycling of nutrients in ecosystems. Why is energy flow described as "one-way" while nutrient cycling is "circular"?
    \item  Design an experiment to investigate the effect of a specific abiotic factor (e.g., light intensity, water availability) on the growth of plants in a terrarium ecosystem. Describe your hypothesis, methodology, and expected results.
    \item  Research and discuss the concept of \keyword{biomagnification} in food chains. How can certain pollutants become more concentrated at higher trophic levels, and what are the implications for ecosystems and human health?
\end{enumerate}
\end{tieredquestions}


\section{Environmental Science: Human Impact and Sustainability}

\keyword{Environmental science} is an interdisciplinary field that studies the interactions between humans and the environment. It encompasses aspects of biology, chemistry, physics, geography, and social sciences to understand environmental issues and develop solutions for a sustainable future.  As we have seen, ecosystems are complex and vital for life. However, human activities are increasingly impacting these ecosystems, leading to a range of environmental problems.

\subsection{Key Environmental Issues}

Many environmental issues threaten the health of ecosystems and the well-being of humans. Some of the most pressing include:

\begin{itemize}
    \item \textbf{Pollution}:  The contamination of the environment with harmful substances. Pollution can take many forms:
        \begin{itemize}
            \item \keyword{Air Pollution}:  Introduction of pollutants into the atmosphere, such as smog, particulate matter, and greenhouse gases. Sources include burning fossil fuels, industrial emissions, and vehicle exhaust.  Air pollution contributes to respiratory problems, acid rain, and climate change.
            \item \keyword{Water Pollution}: Contamination of water bodies (rivers, lakes, oceans, groundwater) with pollutants such as sewage, industrial waste, pesticides, and plastic. Water pollution can harm aquatic life, make water unsafe for drinking, and contribute to diseases.
            \item \keyword{Land Pollution}: Contamination of soil and land with pollutants like pesticides, heavy metals, plastic waste, and industrial waste. Land pollution can reduce soil fertility, contaminate groundwater, and harm terrestrial organisms.
            \item \keyword{Noise Pollution}: Excessive or unwanted sound that can be harmful to humans and wildlife. Sources include traffic, industrial machinery, and construction. Noise pollution can cause stress, hearing problems, and disrupt animal communication.
            \item \keyword{Light Pollution}: Excessive artificial light at night, which can disrupt natural cycles of wildlife, impact human health, and obscure astronomical observations.
        \end{itemize}
    \item \textbf{Climate Change}: Long-term shifts in temperatures and weather patterns, primarily caused by increased concentrations of greenhouse gases in the atmosphere due to human activities, especially the burning of fossil fuels. Climate change leads to rising sea levels, extreme weather events, changes in ecosystems, and threats to biodiversity.

    \begin{marginnote}
    \textit{Greenhouse Effect:}  Greenhouse gases (e.g., carbon dioxide, methane, water vapour) trap heat in the Earth's atmosphere, warming the planet.  This is a natural process, but human activities have enhanced the greenhouse effect, leading to rapid climate change. \mathlink{Greenhouse effect physics}
    \end{marginnote}

    \item \textbf{Deforestation and Habitat Loss}: Clearing forests and other natural habitats for agriculture, urban development, and other purposes. Deforestation reduces biodiversity, contributes to climate change (forests are carbon sinks), and leads to soil erosion and habitat loss for countless species.
    \item \textbf{Biodiversity Loss}: The decline in the variety of life on Earth, at all levels from genes to ecosystems. Biodiversity loss is driven by habitat destruction, pollution, climate change, overexploitation of resources, and invasive species.  Loss of biodiversity weakens ecosystems and reduces their resilience.
    \item \textbf{Resource Depletion}:  The consumption of natural resources at a rate faster than they can be replenished. This includes depletion of fossil fuels, minerals, water resources, and forests. Unsustainable resource use threatens future generations and can lead to environmental degradation.
    \item \textbf{Overpopulation}:  The increasing human population puts pressure on resources and ecosystems.  While population growth is a complex issue, it exacerbates many environmental problems, including resource depletion, pollution, and habitat loss.
\end{itemize}

\begin{stopandthink}
Choose one environmental issue from the list above. Describe its causes and consequences, and think about how it might affect your local community.
\end{stopandthink}

\subsection{Human Impact on Ecosystems}

Human activities are the primary driver of many environmental issues.  Our actions have profound and often negative impacts on ecosystems:

\begin{itemize}
    \item \textbf{Habitat Destruction and Fragmentation}:  Clearing land for agriculture, urban development, and infrastructure directly destroys habitats.  Even when habitats are not completely destroyed, they can be fragmented into smaller, isolated patches, reducing biodiversity and disrupting ecological processes.
    \item \textbf{Introduction of Invasive Species}:  Humans can unintentionally or intentionally introduce species to new ecosystems where they may not naturally occur. Invasive species can outcompete native species, disrupt food webs, and alter ecosystem functions.
    \item \textbf{Overexploitation of Resources}:  Overfishing, overhunting, and unsustainable logging practices can deplete populations of species and disrupt ecosystem balance.
    \item \textbf{Pollution (Chemical and Nutrient)}:  Industrial and agricultural activities release pollutants into the air, water, and soil.  Excess nutrients from fertilisers can cause eutrophication in aquatic ecosystems (excessive algal growth leading to oxygen depletion). Chemical pollutants can be toxic to organisms and accumulate in food chains.
    \item \textbf{Climate Change Impacts}:  Rising temperatures, changing precipitation patterns, and extreme weather events are altering ecosystems worldwide. Species are struggling to adapt, leading to shifts in species distributions, ecosystem disruptions, and increased risk of extinctions.
\end{itemize}

\begin{figure}[h]
    \includegraphics[width=0.8\textwidth]{placeholder-humanimpact.jpg}
    \caption{Human impacts on the environment.  Illustrates various ways human activities affect ecosystems. \textit{Image to be added later depicting examples of deforestation, pollution, urban sprawl, etc., showing human impact on the environment.}}
\end{figure}

\begin{stopandthink}
Think about the products you use in your daily life (food, clothes, electronics, etc.).  Trace back the resources needed to produce these products.  How might their production and consumption contribute to environmental issues?
\end{stopandthink}

\subsection{Sustainability and Conservation}

\keyword{Sustainability} is the ability to meet the needs of the present without compromising the ability of future generations to meet their own needs.  Environmental science plays a crucial role in promoting sustainability by:

\begin{itemize}
    \item \textbf{Understanding Ecosystem Functioning}:  Providing knowledge about how ecosystems work and how they respond to human impacts.
    \item \textbf{Identifying Environmental Problems}:  Monitoring and assessing environmental issues, identifying their causes and consequences.
    \item \textbf{Developing Solutions and Strategies}:  Developing technologies, policies, and practices to mitigate environmental problems and promote sustainable resource management.
    \item \textbf{Raising Awareness and Education}:  Educating the public and policymakers about environmental issues and the importance of sustainability.
\end{itemize}

\keyword{Conservation} is the protection, preservation, management, and restoration of natural environments and wildlife.  Conservation efforts are essential for maintaining biodiversity, protecting ecosystems, and ensuring the long-term availability of natural resources.  Conservation strategies include:

\begin{itemize}
    \item \textbf{Establishing Protected Areas}:  Creating national parks, nature reserves, and marine protected areas to safeguard habitats and species.
    \item \textbf{Sustainable Resource Management}:  Implementing practices that allow for the use of resources in a way that does not deplete them for future generations (e.g., sustainable forestry, fisheries management, water conservation).
    \item \textbf{Pollution Reduction and Prevention}:  Developing and implementing technologies and policies to reduce pollution from various sources.
    \item \textbf{Habitat Restoration and Rehabilitation}:  Restoring degraded ecosystems and habitats to improve their ecological function and biodiversity.
    \item \textbf{Species Recovery Programs}:  Implementing targeted programs to protect and recover endangered species.
    \item \textbf{Promoting Sustainable Lifestyles}:  Encouraging individuals and communities to adopt more environmentally friendly behaviours and consumption patterns.
\end{itemize}

\begin{marginnote}
\challenge{Sustainable Development Goals (SDGs):} The United Nations Sustainable Development Goals are a set of 17 global goals to address global challenges, including environmental sustainability, poverty, inequality, and climate change.  Research the SDGs and their relevance to environmental science. \historylink{UN Sustainable Development Goals}
\end{marginnote}

\begin{investigation}{Environmental Audit of Your School or Home}
\textbf{Objective:} To identify areas where your school or home can become more environmentally sustainable.

\textbf{Materials:} Notebook, pen/pencil, checklist (provided below).

\textbf{Procedure:}
\begin{enumerate}
    \item Choose a location to audit: your school or your home.
    \item Use the checklist below to assess environmental practices in your chosen location.  Observe different aspects related to energy use, water consumption, waste management, and resource use.
    \item For each item on the checklist, note down your observations and identify areas for improvement.
    \item Based on your audit, suggest at least three specific actions that could be taken to make your school or home more environmentally sustainable.  Be realistic and consider the feasibility of implementing these actions.
\end{enumerate}

\textbf{Environmental Audit Checklist (Example):}

\textbf{Energy Use:}
\begin{itemize}
    \item Are energy-efficient light bulbs (LEDs) used?
    \item Are lights switched off in unoccupied rooms?
    \item Is heating and cooling used efficiently (e.g., thermostat settings, insulation)?
    \item Are renewable energy sources used (e.g., solar panels)?
    \item Are electronic devices switched off completely when not in use (not just standby)?
\end{itemize}

\textbf{Water Consumption:}
\begin{itemize}
    \item Are there any dripping taps or leaks?
    \item Are water-efficient toilets and showerheads installed?
    \item Is rainwater harvested for gardening or other non-potable uses?
    \item Is water used efficiently in landscaping and gardening?
    \item Are dishwashers and washing machines used efficiently (full loads)?
\end{itemize}

\textbf{Waste Management:}
\begin{itemize}
    \item Is there a recycling system in place? Are materials properly sorted?
    \item Is composting of food waste and garden waste practiced?
    \item Are single-use plastics minimised (e.g., water bottles, plastic bags)?
    \item Is paper used efficiently (double-sided printing, reducing unnecessary printing)?
    \item Are reusable containers and shopping bags used?
\end{itemize}

\textbf{Resource Use:}
\begin{itemize}
    \item Are sustainable and ethically sourced products purchased where possible?
    \item Is paper and wood sourced from sustainable sources (FSC certified)?
    \item Are resources shared and reused where possible?
    \item Is transport used sustainably (walking, cycling, public transport, carpooling)?
    \item Is food waste minimised through meal planning and proper storage?
\end{itemize}

\textbf{Analysis:}
\begin{enumerate}
    \item What were the main areas where your school or home performed well in terms of environmental sustainability?
    \item What were the main areas identified for improvement?
    \item Which of your suggested actions for improvement do you think would have the biggest positive impact? Why?
    \item What are some potential barriers to implementing these sustainable actions?
    \item How can individuals and communities contribute to broader environmental sustainability efforts?
\end{enumerate}

\textbf{Extension:}  Develop a detailed action plan for implementing one or more of your suggested sustainable actions in your school or home.  Present your action plan to school authorities or family members and try to implement it.
\end{investigation}


\begin{tieredquestions}{Section 4: Environmental Science: Human Impact and Sustainability}
\textbf{Basic:}
\begin{enumerate}
    \item Define "environmental science".
    \item List three major types of pollution.
    \item What is climate change and what is its primary cause?
    \item What is meant by "sustainability"?
\end{enumerate}

\textbf{Intermediate:}
\begin{enumerate}
    \item Explain how human activities contribute to air pollution and water pollution.
    \item Describe the main causes and consequences of deforestation and habitat loss.
    \item Discuss the concept of biodiversity loss and its potential impacts on ecosystems.
    \item Explain how sustainable resource management can help address environmental issues.
\end{enumerate}

\textbf{Advanced:}
\begin{enumerate}
    \item  Critically analyse the statement: "Technological solutions alone are sufficient to solve environmental problems."  Discuss the role of technology and other approaches (e.g., policy, behavioural change) in achieving environmental sustainability.
    \item  Research and discuss the concept of the "ecological footprint." How is it calculated, and what does it measure? What are the implications of humanity's current ecological footprint?
    \item  Debate the ethical considerations of environmental conservation.  Should we prioritise the protection of all species, or are some species more important to conserve than others?  Consider different perspectives (e.g., anthropocentric, biocentric, ecocentric views).
    \item  Design a community-based project to address a specific environmental issue in your local area (e.g., reducing plastic waste, improving local biodiversity, promoting energy conservation).  Outline the project goals, activities, stakeholders, and expected outcomes.
\end{enumerate}
\end{tieredquestions}


\section{Interactions within Ecosystems: Living Together}

Ecosystems are not just a collection of species living in the same place; they are dynamic communities where organisms interact in complex ways. These \keyword{ecological interactions} shape the structure and function of ecosystems.

\subsection{Types of Ecological Interactions}

Organisms within an ecosystem interact with each other in various ways. These interactions can be broadly categorised as:

\begin{itemize}
    \item \textbf{Competition}:  Occurs when two or more organisms or species compete for the same limited resources, such as food, water, shelter, mates, or sunlight. Competition can be:
        \begin{itemize}
            \item \textit{Intraspecific competition}: Competition between individuals of the same species.
            \item \textit{Interspecific competition}: Competition between individuals of different species.
        \end{itemize}
        Competition can limit population growth and influence species distribution.

    \item \textbf{Predation}:  An interaction where one organism (the \keyword{predator}) kills and consumes another organism (the \keyword{prey}). Predation is a key factor in regulating prey populations and shaping community structure. Predators and prey often co-evolve, with predators developing hunting adaptations and prey developing defensive mechanisms.

    \item \textbf{Symbiosis}:  Close and long-term biological interactions between two different biological species.  Symbiotic relationships can be:
        \begin{itemize}
            \item \textit{Mutualism}:  A symbiotic relationship where both species benefit from the interaction.  \begin{example} Example:  Bees pollinating flowers. Bees get nectar and pollen for food, and flowers get pollinated, enabling reproduction.\end{example}
            \item \textit{Commensalism}: A symbiotic relationship where one species benefits, and the other species is neither harmed nor helped. \begin{example} Example: Barnacles attaching to whales. Barnacles get a mobile habitat and access to food in flowing water, while the whale is generally unaffected.\end{example}
            \item \textit{Parasitism}: A symbiotic relationship where one species (the \keyword{parasite}) benefits at the expense of the other species (the \keyword{host}), which is harmed. \begin{example} Example:  Ticks feeding on mammals. Ticks benefit by getting blood, while the mammal host can suffer from blood loss, irritation, and disease transmission.\end{example}
        \end{itemize}
\end{itemize}

\begin{figure}[h]
    \includegraphics[width=0.7\textwidth]{placeholder-ecologicalinteractions.jpg}
    \caption{Examples of ecological interactions: competition, predation, mutualism, commensalism, and parasitism. \textit{Image to be added later depicting visual examples of each type of ecological interaction.}}
\end{figure}

\begin{stopandthink}
Think of a pet you might have or an animal you are familiar with.  Describe one example of predation or competition that this animal might be involved in within its ecosystem.
\end{stopandthink}

\subsection{Ecological Niche and Keystone Species}

\begin{itemize}
    \item \textbf{Ecological Niche}:  The role and position a species has in its environment; how it meets its needs for food and shelter, how it survives, and reproduces. The niche includes all the interactions a species has with the biotic and abiotic factors of its environment.  It is more than just the habitat; it's the "job" or "profession" of a species in the ecosystem.  No two species can occupy exactly the same niche in the same ecosystem for long due to competitive exclusion (one species will eventually outcompete the other).

    \begin{marginnote}
    \textit{Competitive Exclusion Principle:}  Also known as Gause's law, it states that two species competing for the same limiting resource cannot coexist at constant population values. One species will be better adapted and eventually exclude the other. \historylink{G.F. Gause}
    \end{marginnote}

    \item \textbf{Keystone Species}:  A species that has a disproportionately large impact on its ecosystem relative to its abundance. Keystone species play a crucial role in maintaining ecosystem structure and function. Their removal can lead to dramatic changes in the ecosystem, often resulting in biodiversity loss.
    \begin{example} Example: Sea otters in kelp forests. Sea otters prey on sea urchins, which are herbivores that graze on kelp. By controlling sea urchin populations, sea otters prevent overgrazing and maintain healthy kelp forests, which provide habitat for many other species.  If sea otters are removed, sea urchin populations can explode, leading to kelp forest destruction.\end{example}
\end{itemize}

\begin{figure}[h]
    \includegraphics[width=0.6\textwidth]{placeholder-keystonespecies.jpg}
    \caption{Sea otters as a keystone species in kelp forests.  Illustrates the impact of sea otters on the kelp forest ecosystem. \textit{Image to be added later depicting sea otters in a kelp forest and the effect of their presence/absence on the ecosystem.}}
\end{figure}

\begin{stopandthink}
Why is the concept of the ecological niche important for understanding species interactions and ecosystem stability?
\end{stopandthink}

\subsection{Ecological Succession}

Ecosystems are not static; they change over time