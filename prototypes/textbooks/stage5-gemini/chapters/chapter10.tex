```latex
\chapter{Energy Conservation and Electricity}

\begin{marginfigure}
\includegraphics[width=0.9\textwidth]{./figures/lightbulb.jpg}
\caption*{\textit{Energy powers our world. From the simple lightbulb to complex technologies, understanding energy and electricity is crucial in the 21st century.}}
\label{fig:lightbulb}
\end{marginfigure}

\section{Introduction: The Power of Energy}

Have you ever stopped to think about \keyword{energy}? We use this word all the time – we talk about having lots of energy after a good night's sleep, or that a car needs energy to move. But what exactly is energy in a scientific sense?  Energy is fundamental to everything that happens in the universe. It's the ability to do \keyword{work}, to cause change.  Without energy, there would be no life, no movement, no light, and no sound.

This chapter will explore the fascinating world of energy, focusing on its conservation and one of its most useful forms: electricity. We will learn how energy can change forms but is never truly lost, and how we can harness the power of electricity to light our homes, power our devices, and drive technological advancements. Get ready to discover the science behind the power that shapes our modern world!

\begin{marginnote}
\textit{Did you know?} The word 'energy' comes from the Greek word \textit{energeia}, meaning 'activity' or 'operation'.
\end{marginnote}

\section{Energy Conservation: Never Lost, Only Transformed}

\begin{keyconcept}{The Law of Conservation of Energy}
Energy cannot be created or destroyed, but it can be transformed from one form to another or transferred between objects. The total amount of energy in a closed system remains constant.
\end{keyconcept}

One of the most fundamental principles in physics is the \keyword{law of conservation of energy}. This law tells us something incredibly important: energy is not something we can use up and lose forever. Instead, energy is always conserved.  Think of it like this: imagine you have a certain amount of money. You can exchange that money for different things – you can buy food, clothes, or entertainment – but the total value of your money remains the same, even if it's in a different form. Similarly, energy can change forms, but the total amount of energy in a closed system stays constant.

\subsection{Forms of Energy}

Energy exists in many different forms. Some of the most common forms you will encounter are:

\begin{itemize}
    \item \textbf{Kinetic Energy}: The energy of motion. Anything that is moving has kinetic energy. A speeding car, a flowing river, and even tiny atoms vibrating all possess kinetic energy.
    \item \textbf{Potential Energy}: Stored energy. This is energy that has the potential to do work later. There are different types of potential energy, including:
    \begin{itemize}
        \item \textbf{Gravitational Potential Energy}: Energy stored due to an object's height above the ground. A book held above a table has gravitational potential energy. When you drop it, this potential energy is converted into kinetic energy.
        \item \textbf{Elastic Potential Energy}: Energy stored in stretched or compressed objects, like a spring or a rubber band. When released, this potential energy can be converted into kinetic energy.
        \item \textbf{Chemical Potential Energy}: Energy stored in the chemical bonds of molecules. This energy is released during chemical reactions, such as burning fuel or digesting food.
        \item \textbf{Electrical Potential Energy}: Energy stored due to the position of electric charges in an electric field. This is crucial for understanding electricity, as we will see later.
    \end{itemize}
    \item \textbf{Thermal Energy (Heat)}: The total kinetic energy of all the particles within a substance. The hotter an object is, the more thermal energy it has.
    \item \textbf{Light Energy (Radiant Energy)}: Energy that travels in electromagnetic waves, including visible light, ultraviolet light, and infrared radiation. The Sun is a major source of light energy.
    \item \textbf{Sound Energy}: Energy that travels in waves through a medium (like air, water, or solids) due to vibrations.
    \item \textbf{Nuclear Energy}: Energy stored in the nucleus of atoms. This energy is released in nuclear reactions, such as in nuclear power plants or nuclear weapons.
\end{itemize}

\begin{marginnote}
\textbf{Definitions:}
\begin{itemize}
    \item \keyword{Kinetic Energy}: Energy of motion.
    \item \keyword{Potential Energy}: Stored energy.
    \item \keyword{Thermal Energy}: Energy of heat.
\end{itemize}
\end{marginnote}

\subsection{Energy Transformations}

Energy is constantly changing from one form to another. We call these changes \keyword{energy transformations} or \keyword{energy conversions}. Let's look at some examples:

\begin{itemize}
    \item \textbf{Burning Wood}: Chemical potential energy stored in wood is converted into thermal energy (heat) and light energy when it burns.
    \item \textbf{A Car Engine}: Chemical potential energy in petrol is converted into kinetic energy to move the car, along with thermal energy (heat) as a byproduct.
    \item \textbf{A Hydroelectric Dam}: Gravitational potential energy of water stored behind a dam is converted into kinetic energy as the water falls, and then into electrical energy by generators.
    \item \textbf{A Solar Panel}: Light energy from the sun is converted directly into electrical energy.
    \item \textbf{Eating Food}: Chemical potential energy in food is converted into kinetic energy for movement, thermal energy to keep our bodies warm, and other forms of energy to power our bodily functions.
\end{itemize}

In each of these examples, energy is not being created or destroyed. It is simply changing form. Some energy transformations are more efficient than others. For example, a lightbulb converts electrical energy into light energy, but also produces a significant amount of thermal energy as a byproduct, which is often considered wasted energy in this context.

\begin{stopandthink}
Think about riding a bicycle. Describe the energy transformations that occur from when you start pedalling to when you are moving at a constant speed. What forms of energy are involved? Where does the energy come from initially?
\end{stopandthink}

\begin{investigation}{Investigating Energy Transformations with a Rubber Band}
\textbf{Materials:} Rubber band

\textbf{Procedure:}
\begin{enumerate}
    \item Hold a rubber band in your hands.
    \item Stretch the rubber band quickly and then immediately touch it to your forehead. What do you feel?
    \item Now, let the rubber band snap back to its original shape and immediately touch it to your forehead again. What do you feel this time?
\end{enumerate}

\textbf{Observations and Analysis:}
\begin{enumerate}
    \item When you stretch the rubber band, you are doing work on it, and storing \textbf{elastic potential energy} within it. You should feel the rubber band get slightly warmer when you stretch it. This is because some of the energy is transformed into thermal energy due to internal friction within the rubber band material.
    \item When the rubber band snaps back, the stored elastic potential energy is converted into \textbf{kinetic energy} of the rubber band's motion, and also some \textbf{thermal energy} as the molecules in the rubber band vibrate and collide more rapidly as it returns to its original shape. You might feel a slight cooling effect as the rubber band returns to its relaxed state, although this might be subtle.
\end{enumerate}

\textbf{Conclusion:}
This simple investigation demonstrates the transformation of energy.  Work done to stretch the rubber band is stored as potential energy, which is then released as kinetic and thermal energy.  This illustrates the principle of energy conservation – energy is transformed, not lost.
\end{investigation}

\begin{tieredquestions}{Energy Conservation}

\begin{enumerate}
    \item \textbf{Basic:} State the law of conservation of energy in your own words.
    \item \textbf{Basic:} Give three examples of energy transformations that you see in everyday life.
    \item \textbf{Intermediate:} Explain how gravitational potential energy is converted into kinetic energy when a ball is dropped from a height.
    \item \textbf{Intermediate:}  Describe the energy transformations that occur in a wind turbine to generate electricity. Start with the kinetic energy of the wind.
    \item \textbf{Advanced:}  A student claims that a "perpetual motion machine" can be built, which would create energy from nothing and run forever. Explain why this is impossible based on the law of conservation of energy.
    \item \textbf{Advanced:} Discuss the concept of energy efficiency in energy transformations.  Why is it important, and give an example of an inefficient energy transformation and a more efficient alternative.
\end{enumerate}

\end{tieredquestions}

\section{Electricity: The Flow of Charge}

\begin{keyconcept}{Electric Charge}
Electric charge is a fundamental property of matter that can be positive or negative.  Like charges repel, and opposite charges attract.
\end{keyconcept}

Electricity is a form of energy closely related to the movement of \keyword{electric charge}.  To understand electricity, we first need to understand what electric charge is.  At the heart of matter are atoms, and atoms are made up of even smaller particles: protons, neutrons, and electrons.

\begin{itemize}
    \item \textbf{Protons} are found in the nucleus (centre) of the atom and carry a \textbf{positive} electric charge.
    \item \textbf{Neutrons} are also found in the nucleus and have \textbf{no} electric charge (they are neutral).
    \item \textbf{Electrons} orbit the nucleus and carry a \textbf{negative} electric charge.
\end{itemize}

Normally, atoms are electrically neutral because they have an equal number of protons and electrons. However, electrons can sometimes be gained or lost by atoms. If an atom loses electrons, it becomes positively charged (because it has more protons than electrons). If an atom gains electrons, it becomes negatively charged (because it has more electrons than protons).

\begin{marginnote}
\textbf{Historical Note:} The study of electricity dates back to ancient Greece, where they observed static electricity from rubbing amber.  The word "electricity" itself comes from the Greek word for amber, \textit{elektron}. \historylink{Ancient Greeks and static electricity.}
\end{marginnote}

\subsection{Electric Current: The Flow of Charge}

\keyword{Electric current} is the rate of flow of electric charge.  Imagine water flowing through a pipe.  The electric current is like the amount of water flowing past a certain point in the pipe per second.  Current is measured in \keyword{amperes} (A), often called "amps" for short.  One ampere is defined as the flow of one \keyword{coulomb} of charge per second.  The coulomb (C) is the unit of electric charge.

For an electric current to flow, there needs to be a closed path, called a \keyword{circuit}, and a source of energy to push the charges along.  Think again of the water analogy.  The circuit is like a closed loop of pipes, and a pump is needed to push the water around the loop. In electrical circuits, a \keyword{voltage source} (like a battery or a power supply) acts like the pump, providing the energy to move the charges.

\begin{marginnote}
\textbf{Unit of Current:}
1 Ampere (A) = 1 Coulomb per second (C/s)
\end{marginnote}

\subsection{Voltage: Electrical Potential Difference}

\keyword{Voltage}, also known as \keyword{electrical potential difference}, is a measure of the electrical potential energy difference between two points in a circuit.  It's like the "electrical pressure" that pushes the electric charge to move.  Voltage is measured in \keyword{volts} (V).

Think about gravitational potential energy again.  Water at the top of a waterfall has higher gravitational potential energy than water at the bottom.  This difference in potential energy is what causes the water to flow downwards.  Similarly, a voltage difference between two points in a circuit creates an "electrical hill," causing electric charge to flow from the point of higher potential to the point of lower potential.

A battery provides a voltage difference between its terminals.  For example, a 1.5V battery creates a 1.5 volt potential difference between its positive and negative terminals.  This voltage difference drives the current in a circuit when the battery is connected.

\subsection{Resistance: Opposing the Flow}

\keyword{Electrical resistance} is the opposition to the flow of electric current in a material.  All materials resist the flow of electric current to some extent.  Some materials, like metals (e.g., copper, silver, gold), offer very little resistance and are called \keyword{conductors}.  They allow electric current to flow easily.  Other materials, like rubber, glass, and plastic, offer very high resistance and are called \keyword{insulators}.  They prevent electric current from flowing easily.

Resistance is measured in \keyword{ohms} ($\Omega$).  A higher resistance means it is harder for electric current to flow.  The resistance of a wire depends on several factors, including:

\begin{itemize}
    \item \textbf{Material}: Different materials have different inherent resistances.
    \item \textbf{Length}: Longer wires have higher resistance. Imagine a longer pipe – it's harder to push water through it.
    \item \textbf{Cross-sectional Area}: Thicker wires have lower resistance. A wider pipe allows water to flow more easily.
    \item \textbf{Temperature}: For most materials, resistance increases with temperature.
\end{itemize}

\begin{marginnote}
\textbf{Unit of Resistance:}
Ohm ($\Omega$)
\end{marginnote}

\begin{stopandthink}
Imagine you have two wires made of the same material, but one wire is twice as long and half as thick as the other. Which wire will have a higher resistance? Explain your reasoning.
\end{stopandthink}

\begin{tieredquestions}{Electric Charge, Current, Voltage, and Resistance}

\begin{enumerate}
    \item \textbf{Basic:} What are the two types of electric charge? How do they interact with each other?
    \item \textbf{Basic:} Define electric current and state its unit.
    \item \textbf{Basic:} What is voltage, and what does it represent in an electrical circuit? What is its unit?
    \item \textbf{Intermediate:} Explain the difference between conductors and insulators in terms of electrical resistance. Give examples of each.
    \item \textbf{Intermediate:}  Describe the relationship between voltage, current, and resistance in simple terms. (Hint: Think about the water pipe analogy.)
    \item \textbf{Advanced:} Explain how the resistance of a wire changes with its length and cross-sectional area. Why does temperature affect the resistance of most materials?
    \item \textbf{Advanced:}  Research and explain the concept of superconductivity. How does it relate to electrical resistance, and what are some potential applications of superconductors? \challenge{Superconductivity and its applications.}
\end{enumerate}

\end{tieredquestions}

\section{Electric Circuits: Pathways for Current}

\begin{keyconcept}{Electric Circuit}
An electric circuit is a complete, closed path through which electric current can flow.  It typically includes a voltage source, conductive wires, and electrical components.
\end{keyconcept}

An \keyword{electric circuit} is a pathway that allows electric current to flow. For current to flow, the circuit must be complete, meaning there are no breaks in the path.  Think of it like a continuous loop.  A simple electric circuit usually consists of the following components:

\begin{itemize}
    \item \textbf{Voltage Source}: Provides the electrical energy to drive the current. Examples include batteries, power supplies, and generators.
    \item \textbf{Conducting Wires}: Provide a low-resistance path for the current to flow. Typically made of copper or other conductive metals.
    \item \textbf{Load (Component)}:  A device that uses electrical energy to perform a function. Examples include light bulbs, resistors, motors, and electronic devices.
    \item \textbf{Switch (Optional)}: A device used to open or close the circuit, controlling the flow of current.
\end{itemize}

\begin{figure}[h]
    \centering
    \includegraphics[width=0.6\textwidth]{./figures/simple_circuit_diagram.png}
    \caption{A simple electric circuit diagram. It includes a battery (voltage source), connecting wires, a light bulb (load), and a switch.}
    \label{fig:simple_circuit}
\end{figure}

Figure \ref{fig:simple_circuit} shows a diagram of a simple circuit.  Circuit diagrams use symbols to represent different components.  Understanding circuit diagrams is essential for designing and analysing electrical circuits.

\subsection{Series Circuits}

In a \keyword{series circuit}, components are connected one after another along a single path.  There is only one path for the electric current to flow.  Imagine people walking in a single file line – they all follow the same path.

\begin{figure}[h]
    \centering
    \includegraphics[width=0.6\textwidth]{./figures/series_circuit_diagram.png}
    \caption{A series circuit diagram with two resistors connected in series.}
    \label{fig:series_circuit}
\end{figure}

\textbf{Characteristics of Series Circuits:}

\begin{itemize}
    \item \textbf{Current is the same everywhere}: The current is the same at every point in a series circuit.  Since there is only one path, all the charge must flow through each component.
    \item \textbf{Voltage is divided}: The total voltage supplied by the voltage source is divided among the components in the series circuit. The sum of the voltage drops across each component equals the total voltage of the source.
    \item \textbf{Total Resistance is the sum}: The total resistance of a series circuit is the sum of the resistances of all the individual components.  Adding more components in series increases the total resistance of the circuit.
\end{itemize}

\begin{example}
Consider a series circuit with a 9V battery and two resistors, $R_1 = 10\ \Omega$ and $R_2 = 20\ \Omega$, connected in series.

\begin{itemize}
    \item \textbf{Total Resistance ($R_{total}$)}: $R_{total} = R_1 + R_2 = 10\ \Omega + 20\ \Omega = 30\ \Omega$.
    \item \textbf{Current ($I$)}: We can use Ohm's Law (which we will learn more about in the next section) to find the current: $V = IR$, so $I = V/R = 9\ V / 30\ \Omega = 0.3\ A$. The current is 0.3A throughout the entire circuit.
    \item \textbf{Voltage drop across $R_1$ ($V_1$)}: $V_1 = IR_1 = 0.3\ A \times 10\ \Omega = 3\ V$.
    \item \textbf{Voltage drop across $R_2$ ($V_2$)}: $V_2 = IR_2 = 0.3\ A \times 20\ \Omega = 6\ V$.
    \item \textbf{Check Voltage Division}: $V_1 + V_2 = 3\ V + 6\ V = 9\ V$, which is equal to the total voltage of the battery, as expected.
\end{itemize}
\end{example}

\subsection{Parallel Circuits}

In a \keyword{parallel circuit}, components are connected in separate branches. There are multiple paths for the electric current to flow. Imagine a road that splits into multiple lanes – cars can choose different paths.

\begin{figure}[h]
    \centering
    \includegraphics[width=0.6\textwidth]{./figures/parallel_circuit_diagram.png}
    \caption{A parallel circuit diagram with two resistors connected in parallel.}
    \label{fig:parallel_circuit}
\end{figure}

\textbf{Characteristics of Parallel Circuits:}

\begin{itemize}
    \item \textbf{Voltage is the same across branches}: The voltage across each branch of a parallel circuit is the same and equal to the voltage of the source.  Each component is directly connected to the voltage source.
    \item \textbf{Current is divided}: The total current from the voltage source is divided among the different branches of the parallel circuit. The sum of the currents in each branch equals the total current from the source.
    \item \textbf{Total Resistance is less than individual resistances}: The total resistance of a parallel circuit is always less than the resistance of the smallest individual resistor.  Adding more components in parallel decreases the total resistance of the circuit. The reciprocal of the total resistance is the sum of the reciprocals of the individual resistances: $\frac{1}{R_{total}} = \frac{1}{R_1} + \frac{1}{R_2} + \frac{1}{R_3} + ...$ \mathlink{Formula for parallel resistance.}
\end{itemize}

\begin{example}
Consider a parallel circuit with a 9V battery and two resistors, $R_1 = 10\ \Omega$ and $R_2 = 20\ \Omega$, connected in parallel.

\begin{itemize}
    \item \textbf{Voltage across each resistor}: $V_1 = V_2 = 9\ V$ (same as the battery voltage).
    \item \textbf{Current through $R_1$ ($I_1$)}: $I_1 = V_1/R_1 = 9\ V / 10\ \Omega = 0.9\ A$.
    \item \textbf{Current through $R_2$ ($I_2$)}: $I_2 = V_2/R_2 = 9\ V / 20\ \Omega = 0.45\ A$.
    \item \textbf{Total Current ($I_{total}$)}: $I_{total} = I_1 + I_2 = 0.9\ A + 0.45\ A = 1.35\ A$.
    \item \textbf{Total Resistance ($R_{total}$)}: We can calculate the total resistance using the formula for parallel resistors:
    $\frac{1}{R_{total}} = \frac{1}{10\ \Omega} + \frac{1}{20\ \Omega} = \frac{2}{20\ \Omega} + \frac{1}{20\ \Omega} = \frac{3}{20\ \Omega}$.
    Therefore, $R_{total} = \frac{20}{3}\ \Omega \approx 6.67\ \Omega$. Notice that the total resistance is less than both $10\ \Omega$ and $20\ \Omega$.
\end{itemize}
\end{example}

\begin{investigation}{Building Simple Series and Parallel Circuits}
\textbf{Materials:}
\begin{itemize}
    \item Battery (e.g., 1.5V or 3V)
    \item Battery holder
    \item Two light bulbs (low voltage, suitable for the battery)
    \item Connecting wires with alligator clips
    \item Switch (optional)
\end{itemize}

\textbf{Procedure:}
\begin{enumerate}
    \item \textbf{Series Circuit:} Connect the components to create a series circuit as shown in Figure \ref{fig:series_circuit_diagram_actual}.
    \begin{figure}[h]
        \centering
        \includegraphics[width=0.5\textwidth]{./figures/series_circuit_actual.jpg}
        \caption{Series circuit setup for investigation.}
        \label{fig:series_circuit_diagram_actual}
    \end{figure}
    Observe the brightness of the bulbs.  Try unscrewing one bulb. What happens to the other bulb? Explain why.
    \item \textbf{Parallel Circuit:} Reconnect the components to create a parallel circuit as shown in Figure \ref{fig:parallel_circuit_diagram_actual}.
    \begin{figure}[h]
        \centering
        \includegraphics[width=0.5\textwidth]{./figures/parallel_circuit_actual.jpg}
        \caption{Parallel circuit setup for investigation.}
        \label{fig:parallel_circuit_diagram_actual}
    \end{figure}
    Observe the brightness of the bulbs. How does it compare to the brightness in the series circuit?  Try unscrewing one bulb. What happens to the other bulb? Explain why.
\end{enumerate}

\textbf{Observations and Analysis:}
\begin{enumerate}
    \item \textbf{Series Circuit Observations:} In the series circuit, both bulbs should light up, but they might be dimmer than if only one bulb was connected to the battery. When one bulb is unscrewed, the circuit is broken, and the other bulb goes out. This is because there is only one path for the current, and breaking the path at any point stops the current flow throughout the circuit.
    \item \textbf{Parallel Circuit Observations:} In the parallel circuit, the bulbs should be brighter than in the series circuit (assuming identical bulbs and battery). When one bulb is unscrewed, the other bulb remains lit. This is because there are multiple paths for the current. Even if one path is broken, the other path remains complete, allowing current to flow through the remaining bulb.
\end{enumerate}

\textbf{Conclusion:}
This investigation demonstrates the key differences between series and parallel circuits. In series circuits, components are dependent on each other, and the total resistance is higher, leading to potentially lower current and dimmer bulbs. In parallel circuits, components are independent, and the total resistance is lower, potentially leading to higher current and brighter bulbs.
\end{investigation}


\begin{stopandthink}
Think about the wiring in your home. Are the lights in your house connected in series or parallel? What evidence supports your answer? Consider what happens when one light bulb burns out in your home.
\end{stopandthink}

\begin{tieredquestions}{Electric Circuits: Series and Parallel}

\begin{enumerate}
    \item \textbf{Basic:} What is the main difference between a series circuit and a parallel circuit in terms of the path of current flow?
    \item \textbf{Basic:} In a series circuit, what is true about the current at different points in the circuit? What is true about the voltage across different components?
    \item \textbf{Basic:} In a parallel circuit, what is true about the voltage across different branches? What is true about the current in different branches?
    \item \textbf{Intermediate:} Explain why the total resistance of a series circuit is always greater than any individual resistance in the circuit.
    \item \textbf{Intermediate:} Explain why the total resistance of a parallel circuit is always less than the smallest individual resistance in the circuit.
    \item \textbf{Advanced:} You have three resistors with resistances of $5\ \Omega$, $10\ \Omega$, and $15\ \Omega$. Calculate the total resistance if they are connected in series and if they are connected in parallel.
    \item \textbf{Advanced:}  Design a circuit that uses a 6V battery and three identical light bulbs such that each bulb receives the same voltage as if it were connected directly to the battery. Draw a circuit diagram and explain your reasoning. \challenge{Circuit design and analysis.}
\end{enumerate}

\end{tieredquestions}

\section{Ohm's Law: The Relationship Between Voltage, Current, and Resistance}

\begin{keyconcept}{Ohm's Law}
Ohm's Law states that the voltage (V) across a conductor is directly proportional to the current (I) flowing through it, provided the temperature remains constant.  The constant of proportionality is the resistance (R).  Mathematically, Ohm's Law is expressed as: $V = IR$. \mathlink{Ohm's Law formula.}
\end{keyconcept}

\keyword{Ohm's Law} is a fundamental law in electricity that describes the relationship between voltage, current, and resistance in a circuit. It is named after German physicist Georg Ohm, who experimentally discovered this relationship.

Ohm's Law states that the voltage (\textit{V}) across a component is directly proportional to the current (\textit{I}) flowing through it, as long as the resistance (\textit{R}) remains constant (and temperature is constant for many materials). This relationship is expressed mathematically as:

$$V = IR$$

This equation can be rearranged to solve for current or resistance:

$$I = \frac{V}{R} \quad \text{and} \quad R = \frac{V}{I}$$

Ohm's Law is incredibly useful for analysing and designing electrical circuits. It allows us to calculate the voltage, current, or resistance in a circuit if we know the other two quantities.

\begin{marginnote}
\textbf{Georg Ohm (1789-1854):} A German physicist who formulated Ohm's Law. His work was initially met with skepticism but later became fundamental to electrical theory. \historylink{Biography of Georg Ohm.}
\end{marginnote}

\begin{example}
A torch bulb has a resistance of $15\ \Omega$ when lit. If it is connected to a 3V battery, what current will flow through the bulb?

\textbf{Solution:}
We are given:
\begin{itemize}
    \item Voltage ($V$) = 3V
    \item Resistance ($R$) = $15\ \Omega$
\end{itemize}
We want to find the current ($I$). Using Ohm's Law, $I = \frac{V}{R} = \frac{3\ V}{15\ \Omega} = 0.2\ A$.
Therefore, a current of 0.2A will flow through the bulb.
\end{example}

\begin{example}
A resistor in a circuit has a voltage of 6V across it and a current of 0.5A flowing through it. What is the resistance of the resistor?

\textbf{Solution:}
We are given:
\begin{itemize}
    \item Voltage ($V$) = 6V
    \item Current ($I$) = 0.5A
\end{itemize}
We want to find the resistance ($R$). Using Ohm's Law, $R = \frac{V}{I} = \frac{6\ V}{0.5\ A} = 12\ \Omega$.
Therefore, the resistance of the resistor is $12\ \Omega$.
\end{example}

\begin{stopandthink}
If you double the voltage across a resistor while keeping its resistance constant, what happens to the current flowing through it?  If you double the resistance while keeping the voltage constant, what happens to the current? Explain your answers using Ohm's Law.
\end{stopandthink}

\begin{investigation}{Verifying Ohm's Law}
\textbf{Materials:}
\begin{itemize}
    \item Variable voltage power supply (DC)
    \item Resistor (e.g., $100\ \Omega$ or $220\ \Omega$)
    \item Ammeter (to measure current)
    \item Voltmeter (to measure voltage)
    \item Connecting wires
\end{itemize}

\textbf{Procedure:}
\begin{enumerate}
    \item Set up the circuit as shown in Figure \ref{fig:ohms_law_circuit}. The ammeter should be connected in series with the resistor to measure the current flowing through it. The voltmeter should be connected in parallel across the resistor to measure the voltage across it.
    \begin{figure}[h]
        \centering
        \includegraphics[width=0.6\textwidth]{./figures/ohms_law_circuit_diagram.png}
        \caption{Circuit diagram for verifying Ohm's Law.}
        \label{fig:ohms_law_circuit}
    \end{figure}
    \item Start with a low voltage setting on the power supply (e.g., 1V).
    \item Record the voltage reading from the voltmeter and the current reading from the ammeter in a table.
    \item Increase the voltage in small steps (e.g., 1V increments) and record the corresponding voltage and current readings for each step.  Take at least 5-6 readings.
    \item Calculate the resistance ($R = V/I$) for each voltage and current reading.
\end{enumerate}

\textbf{Data Table (Example):}
\begin{center}
\begin{tabular}{|c|c|c|c|}
\hline
Reading & Voltage (V) & Current (A) & Resistance (V/I) ($\Omega$) \\
\hline
1 & 1.0 & 0.010 & 100 \\
2 & 2.0 & 0.020 & 100 \\
3 & 3.0 & 0.030 & 100 \\
4 & 4.0 & 0.040 & 100 \\
5 & 5.0 & 0.050 & 100 \\
\hline
\end{tabular}
\end{center}

\textbf{Analysis and Conclusion:}
\begin{enumerate}
    \item Observe the calculated resistance values in the last column of your table.  They should be approximately constant across different voltage and current readings (within experimental error).
    \item Plot a graph of Voltage (V) on the y-axis against Current (I) on the x-axis. The graph should be a straight line passing through the origin.
    \item The slope of the straight line graph represents the resistance (R) of the resistor.
\end{enumerate}

\textbf{Conclusion:}
This investigation demonstrates Ohm's Law. The constant ratio of voltage to current (V/I = R) confirms that for a resistor at constant temperature, voltage is directly proportional to current. The straight-line graph of V vs. I further visually represents this linear relationship, verifying Ohm's Law.
\end{investigation}

\begin{tieredquestions}{Ohm's Law}

\begin{enumerate}
    \item \textbf{Basic:} State Ohm's Law in words and write down the formula.
    \item \textbf{Basic:} What are the units for voltage, current, and resistance in Ohm's Law?
    \item \textbf{Basic:}  If you know the voltage across a resistor and the current flowing through it, how can you calculate the resistance using Ohm's Law?
    \item \textbf{Intermediate:} A 6V battery is connected to a resistor. If the current flowing through the resistor is 0.25A, what is the resistance of the resistor?
    \item \textbf{Intermediate:} A lamp with a resistance of $240\ \Omega$ is connected to a 240V mains supply. Calculate the current flowing through the lamp.
    \item \textbf{Advanced:} Explain why Ohm's Law is not always obeyed for all materials and components.  Give an example of a component that does not strictly follow Ohm's Law and explain why. \challenge{Limitations of Ohm's Law.}
    \item \textbf{Advanced:} A variable resistor (rheostat) is connected in series with a lamp and a battery. Explain how changing the resistance of the rheostat affects the current in the circuit and the brightness of the lamp, based on Ohm's Law.
\end{enumerate}

\end{tieredquestions}

\section{Electrical Power and Energy}

\begin{keyconcept}{Electrical Power}
Electrical power is the rate at which electrical energy is converted into other forms of energy in a circuit. It is calculated as the product of voltage and current: $P = VI$. The unit of power is the watt (W). \mathlink{Power formula.}
\end{keyconcept}

\keyword{Electrical power} is the rate at which electrical energy is used or converted into other forms of energy (like heat, light, or motion) in a circuit.  Power is measured in \keyword{watts} (W). One watt is defined as one joule of energy per second (1 W = 1 J/s).

The electrical power (\textit{P}) dissipated by a component in a circuit can be calculated using the following formula:

$$P = VI$$

where:
\begin{itemize}
    \item \textit{P} is the power in watts (W)
    \item \textit{V} is the voltage across the component in volts (V)
    \item \textit{I} is the current through the component in amperes (A)
\end{itemize}

Using Ohm's Law ($V=IR$), we can also express power in other forms:

$$P = I^2R \quad \text{and} \quad P = \frac{V^2}{R}$$

These different forms are useful depending on what quantities are known in a particular problem.

\begin{marginnote}
\textbf{Unit of Power:}
Watt (W) = Joule per second (J/s)
\end{marginnote}

\begin{example}
A light bulb is rated at 60W and operates at 240V. Calculate the current flowing through the bulb and its resistance.

\textbf{Solution:}
We are given:
\begin{itemize}
    \item Power ($P$) = 60W
    \item Voltage ($V$) = 240V
\end{itemize}
First, calculate the current ($I$) using $P = VI$:
$I = \frac{P}{V} = \frac{60\ W}{240\ V} = 0.25\ A$.

Next, calculate the resistance ($R$) using $V = IR$ or $P = V^2/R$. Using $V = IR$:
$R = \frac{V}{I} = \frac{240\ V}{0.25\ A} = 960\ \Omega$.

Alternatively, using $P = V^2/R$:
$R = \frac{V^2}{P} = \frac{(240\ V)^2}{60\ W} = \frac{57600}{60}\ \Omega = 960\ \Omega$.
Both methods give the same resistance.
\end{example}

\begin{keyconcept}{Electrical Energy}
Electrical energy is the total amount of energy transferred or used in an electrical circuit over a period of time.  Energy is calculated as the product of power and time: $E = Pt$. The unit of energy is the joule (J), but electrical energy is often measured in kilowatt-hours (kWh) for practical purposes. \mathlink{Energy formula.}
\end{keyconcept}

\keyword{Electrical energy} is the total amount of energy transferred or used by an electrical device over a period of time.  Energy is measured in \keyword{joules} (J).  However, for practical purposes, especially when dealing with electricity consumption in homes and industries, the unit \keyword{kilowatt-hour} (kWh) is commonly used.

The electrical energy (\textit{E}) used by a device is calculated as:

$$E = Pt$$

where:
\begin{itemize}
    \item \textit{E} is the energy in joules (J) or kilowatt-hours (kWh)
    \item \textit{P} is the power in watts (W) or kilowatts (kW)
    \item \textit{t} is the time in seconds (s) or hours (h)
\end{itemize}

To use kilowatt-hours, power should be in kilowatts (kW) and time in hours (h). Remember that 1 kilowatt (kW) = 1000 watts (W).

\begin{marginnote}
\textbf{Units of Energy:}
\begin{itemize}
    \item Joule (J) - SI unit of energy.
    \item Kilowatt-hour (kWh) - Practical unit for electrical energy consumption.
    1 kWh = $3.6 \times 10^6$ J
\end{itemize}
\end{marginnote}

\begin{example}
A 100W light bulb is left on for 5 hours. Calculate the electrical energy consumed in joules and in kilowatt-hours.

\textbf{Solution:}
\textbf{In Joules:}
\begin{itemize}
    \item Power ($P$) = 100W
    \item Time ($t$) = 5 hours = $5 \times 3600\ s = 18000\ s$
\end{itemize}
Energy ($E$) = $Pt = 100\ W \times 18000\ s = 1,800,000\ J = 1.8 \times 10^6\ J$.

\textbf{In Kilowatt-hours:}
\begin{itemize}
    \item Power ($P$) = 100W = $\frac{100}{1000}\ kW = 0.1\ kW$
    \item Time ($t$) = 5 hours
\end{itemize}
Energy ($E$) = $Pt = 0.1\ kW \times 5\ h = 0.5\ kWh$.

Therefore, the light bulb consumes $1.8 \times 10^6\ J$ or 0.5 kWh of electrical energy.
\end{example}

\begin{stopandthink}
Why is electrical energy often measured in kilowatt-hours rather than joules for household electricity bills? Consider the typical amounts of energy consumed by households and the magnitude of the units.
\end{stopandthink}

\begin{tieredquestions}{Electrical Power and Energy}

\begin{enumerate}
    \item \textbf{Basic:} Define electrical power and state its unit. Write down the formula for calculating electrical power.