```latex
\chapter{Introduction: Embarking on Your Stage 5 Science Journey}

\epigraph{The important thing is to never stop questioning.}{Albert Einstein}

\begin{marginfigure}[0pt]
\includegraphics[width=\linewidth]{placeholder_beaker.jpg}
\caption*{}
\textit{Science is about exploring the world around us, from the smallest atom to the vast universe.}
\end{marginfigure}

Welcome to the exciting world of Stage 5 Science!  This textbook is your companion as you embark on a fascinating journey of discovery, exploration, and understanding.  Science is more than just a subject you study in school; it is a way of thinking, a method of investigating, and a lens through which we can view the world around us.  Whether you are naturally curious about how things work, eager to solve problems, or simply fascinated by the wonders of nature, science at Stage 5 is designed to ignite your curiosity and equip you with the skills to explore the universe and your place within it.

In the coming chapters, we will delve into the core scientific disciplines – from the fundamental laws governing motion and energy in Physics, to the intricate world of atoms and molecules in Chemistry, the amazing complexity of life in Biology, and the grand scale of Earth and Space Science.  Stage 5 Science is not just about memorising facts; it's about developing a scientific mindset. You will learn to ask insightful questions, design investigations, analyse evidence, and construct explanations based on what you observe and discover.  This is not just about learning *about* science, but learning to *do* science.

\FloatBarrier

\section{Your Guide to This Textbook}

This book has been carefully designed to support you in your Stage 5 Science journey.  We understand that everyone learns in their own way, and we have incorporated a variety of features to make your learning experience engaging, effective, and enjoyable.  Think of this textbook as your personal science laboratory and field guide, all rolled into one! Let's take a tour of what you will find within these pages.

\subsection{The Main Text: Your Core Knowledge}

The heart of each chapter is the main text.  Here, you will find clear and concise explanations of key scientific concepts, principles, and theories.  We have strived to present complex ideas in an accessible way, breaking them down into manageable chunks and using language that is both precise and easy to understand.  You'll find real-world examples, relatable scenarios, and thought-provoking questions woven throughout the text to help you connect with the material and see its relevance in your everyday life.

\begin{marginnote}
\textit{Key Features to Look Out For:}
\begin{itemize}
    \item \textbf{Clear Explanations:} Complex concepts broken down step-by-step.
    \item \textbf{Real-World Examples:} Connecting science to your daily life.
    \item \textbf{Engaging Language:}  Making learning enjoyable and accessible.
\end{itemize}
\end{marginnote}

We believe that science is best learned through understanding, not just rote memorisation.  Therefore, the main text focuses on building a strong foundation of scientific understanding, encouraging you to think critically and apply your knowledge in different contexts. We will guide you through the essential scientific vocabulary, ensuring you become confident in using the language of science to articulate your ideas and understanding.

\FloatBarrier

\subsection{Margin Notes: Your Sidekick for Deeper Learning}

Look to the margins of each page – here you will find a treasure trove of additional information, designed to enhance your learning experience.  These margin notes are like your science sidekick, providing extra insights, interesting facts, definitions, and connections to further enrich your understanding.

\begin{marginfigure}[0pt]
\includegraphics[width=\linewidth]{placeholder_margin_notes.png}
\caption*{}
\textit{Margin notes provide extra information and context, enriching your learning experience.}
\end{marginfigure}

\begin{marginnote}
\textit{Margin Notes Include:}
\begin{itemize}
    \item \textbf{Definitions:} Quick explanations of key scientific terms.
    \item \textbf{Interesting Facts:}  Intriguing snippets of scientific trivia to spark your curiosity.
    \item \textbf{Further Exploration:}  Suggestions for additional reading or research.
    \item \textbf{Links to Other Concepts:}  Connecting ideas across different topics.
    \item \textbf{Historical Context:}  Brief glimpses into the history of scientific discoveries.
\end{itemize}
\end{marginnote}

Some margin notes will provide quick definitions of important scientific terms, ensuring you always have a handy reference right where you need it.  Others will offer fascinating snippets of scientific trivia or historical context, adding depth and colour to your learning.  You might also find suggestions for further exploration – perhaps a related experiment you could try at home, a documentary to watch, or a website to visit to delve deeper into a particular topic.  These margin notes are designed to be both informative and engaging, encouraging you to explore the world of science beyond the main text.  Don’t skip over them – they are valuable nuggets of knowledge!

\FloatBarrier

\subsection{Investigations: Your Hands-On Science Lab}

Science is fundamentally a practical subject.  It's about doing, experimenting, and investigating.  Throughout this book, you will find numerous \textbf{Investigations}. These are not just optional extras; they are integral to your learning.  Investigations provide you with the opportunity to put scientific principles into practice, to develop your experimental skills, and to experience the thrill of scientific discovery firsthand.

\begin{marginnote}
\textit{Investigations Are Designed To:}
\begin{itemize}
    \item \textbf{Apply Your Knowledge:}  Put scientific concepts into practice.
    \item \textbf{Develop Skills:}  Enhance your experimental and analytical skills.
    \item \textbf{Encourage Inquiry:}  Foster your curiosity and investigative spirit.
    \item \textbf{Make Science Real:}  Connect theory to practical application.
\end{itemize}
\end{marginnote}

\begin{marginfigure}[0pt]
\includegraphics[width=\linewidth]{placeholder_lab_equipment.jpg}
\caption*{}
\textit{Investigations provide hands-on experience, making science come alive.}
\end{marginfigure}

Each investigation is carefully designed to be engaging and achievable, often using readily available materials.  They will guide you through the scientific process, from formulating a question and making predictions (hypotheses), to designing a fair test, collecting and analysing data, and drawing conclusions.  You will learn to work safely in a science setting, to use scientific equipment appropriately, and to record your observations and findings systematically.  Investigations are not just about getting the "right" answer; they are about the process of scientific inquiry, learning from both successes and unexpected results.  Embrace these investigations – they are your chance to be a scientist!

\FloatBarrier

\subsection{Checkpoints and Review Questions: Test Your Understanding}

Learning science is an active process, and it's important to check your understanding as you go along.  At the end of each section and chapter, you will find \textbf{Checkpoints} and \textbf{Review Questions}.  These are designed to help you consolidate your learning and identify areas where you might need to revisit the material.

\begin{marginnote}
\textit{Checkpoints and Review Questions:}
\begin{itemize}
    \item \textbf{Self-Assessment:}  Gauge your understanding of key concepts.
    \item \textbf{Practice Application:}  Apply your knowledge to different scenarios.
    \item \textbf{Identify Gaps:}  Pinpoint areas for further study and revision.
    \item \textbf{Prepare for Assessments:}  Build confidence for tests and exams.
\end{itemize}
\end{marginnote}

Checkpoints are typically short, quick questions that focus on the key ideas from a specific section.  They are perfect for a quick self-test immediately after reading a section.  Review Questions, found at the end of each chapter, are more comprehensive and may require you to integrate knowledge from different parts of the chapter.  They often encourage higher-order thinking skills, such as analysis, evaluation, and application.  Don't view these questions as just tests; see them as learning tools.  Attempting them, even if you are unsure of the answers, is a valuable part of the learning process.  Use them to identify areas where you feel confident and areas where you need to spend more time reviewing.

\FloatBarrier

\subsection{Key Terms and Glossary: Building Your Scientific Vocabulary}

Science has its own language, with specific terms and definitions that are essential for clear communication and understanding.  Throughout the text, important scientific terms will be highlighted in \textbf{bold}.  You will also find these terms defined in the margin notes as they appear, and compiled in a comprehensive \textbf{Glossary} at the back of the book.

\begin{marginfigure}[0pt]
\includegraphics[width=\linewidth]{placeholder_glossary.jpg}
\caption*{}
\textit{The glossary is your go-to resource for understanding scientific vocabulary.}
\end{marginfigure}

\begin{marginnote}
\textit{Utilise the Glossary To:}
\begin{itemize}
    \item \textbf{Understand Definitions:}  Quickly find the meaning of scientific terms.
    \item \textbf{Build Vocabulary:}  Expand your scientific language skills.
    \item \textbf{Improve Communication:}  Use precise language in your own explanations.
    \item \textbf{Enhance Comprehension:}  Grasp scientific texts more effectively.
\end{itemize}
\end{marginnote}

Building a strong scientific vocabulary is crucial for success in Stage 5 Science and beyond.  Make it a habit to pay attention to these key terms, understand their definitions, and use them in your own explanations, both written and spoken.  The Glossary is your go-to resource whenever you encounter an unfamiliar term or need to refresh your memory.  Becoming fluent in the language of science will open up a whole new world of understanding and communication.

\FloatBarrier

\subsection{Chapter Summaries:  Recap and Reinforce}

At the end of each chapter, you will find a concise \textbf{Summary} that recaps the main ideas and key concepts covered.  Think of this as a quick revision tool, providing a bird's-eye view of the chapter's content.

\begin{marginnote}
\textit{Chapter Summaries Help You To:}
\begin{itemize}
    \item \textbf{Review Key Concepts:}  Quickly recap the chapter's main points.
    \item \textbf{Reinforce Learning:}  Solidify your understanding of core ideas.
    \item \textbf{Identify Key Takeaways:}  Pinpoint the most important information.
    \item \textbf{Prepare for Revision:}  Use as a starting point for further study.
\end{itemize}
\end{marginnote}

\begin{marginfigure}[0pt]
\includegraphics[width=\linewidth]{placeholder_summary.jpg}
\caption*{}
\textit{Chapter summaries provide a quick and effective way to review key concepts.}
\end{marginfigure}

Use the chapter summaries as a starting point for your revision.  Read through them carefully, and then go back to the relevant sections of the main text if you need to refresh your understanding of any particular point.  Summaries are also useful for getting a quick overview of a chapter before you dive into the details, or for reminding yourself of the key takeaways after you have completed a chapter.

\FloatBarrier

\section{What You Will Explore in Stage 5 Science}

Stage 5 Science is a journey through the major branches of scientific knowledge, giving you a broad and balanced understanding of the physical, chemical, biological, and Earth and space sciences.  We will explore fascinating topics that are relevant to your life and the world around you.  Here is a glimpse of what awaits you in the chapters ahead.

\subsection{Physics: Understanding the Physical World}

Physics is the study of matter, energy, motion, and forces.  It seeks to understand the fundamental laws that govern the universe, from the smallest subatomic particles to the largest galaxies.  In the Physics sections of this book, you will explore:

\begin{marginnote}
\textit{Physics Topics:}
\begin{itemize}
    \item Forces and Motion
    \item Energy and Work
    \item Heat and Temperature
    \item Light and Sound
    \item Electricity and Magnetism
\end{itemize}
\end{marginnote}

\begin{itemize}
    \item \textbf{Forces and Motion:}  We'll investigate Newton's laws of motion and how forces cause objects to move or change their motion.  You will learn about concepts like gravity, friction, and momentum, and how they affect everything from a falling apple to a speeding car.
    \item \textbf{Energy and Work:}  Energy is the driving force of the universe.  We will explore different forms of energy, such as kinetic, potential, thermal, and chemical energy, and how energy is transferred and transformed.  You will also learn about work, power, and efficiency.
    \item \textbf{Heat and Temperature:}  We will delve into the nature of heat and temperature, exploring concepts like thermal energy, specific heat capacity, and heat transfer through conduction, convection, and radiation.  Understanding these principles is crucial for explaining phenomena from weather patterns to cooking.
    \item \textbf{Light and Sound:}  Light and sound are forms of energy that travel in waves.  We will investigate the properties of light, including reflection, refraction, and diffraction, and explore the nature of sound waves, including pitch, loudness, and the speed of sound.
    \item \textbf{Electricity and Magnetism:}  Electricity and magnetism are fundamental forces closely related to each other.  You will learn about electric charge, current, voltage, and resistance, as well as magnetic fields and electromagnetism.  This knowledge underpins many technologies we use every day, from smartphones to power grids.
\end{itemize}

Physics helps us understand how the world works at a fundamental level.  It provides the basis for many other scientific disciplines and technological advancements.  Get ready to explore the forces that shape our universe!

\FloatBarrier

\subsection{Chemistry: Exploring the World of Matter}

Chemistry is the study of matter and its properties, as well as how matter changes.  It is concerned with the composition, structure, properties, and reactions of substances.  In the Chemistry sections, you will discover:

\begin{marginnote}
\textit{Chemistry Topics:}
\begin{itemize}
    \item The Structure of Matter (Atoms and Molecules)
    \item The Periodic Table and Elements
    \item Chemical Reactions and Equations
    \item Acids, Bases, and Salts
    \item Chemical Reactions in Everyday Life
\end{itemize}
\end{marginnote}

\begin{itemize}
    \item \textbf{The Structure of Matter (Atoms and Molecules):}  Everything around us is made of matter, and matter is made of atoms.  We will explore the structure of atoms, including protons, neutrons, and electrons, and how atoms combine to form molecules and compounds.  You will learn about different states of matter (solid, liquid, gas) and the changes between them.
    \item \textbf{The Periodic Table and Elements:}  The periodic table is a chemist's essential tool, organising all known elements based on their properties.  You will learn about the structure and organisation of the periodic table, and how it can be used to predict the properties of elements and their compounds.
    \item \textbf{Chemical Reactions and Equations:}  Chemical reactions are processes in which substances are transformed into new substances.  We will explore different types of chemical reactions, how to represent them using chemical equations, and factors that affect reaction rates.
    \item \textbf{Acids, Bases, and Salts:}  Acids and bases are important classes of chemical compounds with distinct properties.  You will learn about the pH scale, neutralisation reactions, and the properties of acids, bases, and salts, which are crucial in many chemical processes and biological systems.
    \item \textbf{Chemical Reactions in Everyday Life:} Chemistry is not confined to the laboratory; it is all around us!  We will explore the chemistry behind everyday phenomena, such as cooking, cleaning, digestion, and the materials we use.  You will see how chemical principles explain the world we live in.
\end{itemize}

Chemistry unlocks the secrets of matter and its transformations.  It is a central science that connects to biology, physics, and Earth science, and is essential for understanding the materials and processes that shape our world.

\FloatBarrier

\subsection{Biology: Unveiling the Secrets of Life}

Biology is the study of life – from the smallest microorganisms to the largest ecosystems.  It explores the structure, function, growth, origin, evolution, and distribution of living organisms.  In the Biology sections, you will investigate:

\begin{marginnote}
\textit{Biology Topics:}
\begin{itemize}
    \item Cells: The Basic Units of Life
    \item Organisation of Living Things
    \item Life Processes: Nutrition, Respiration, and Excretion
    \item Reproduction and Inheritance
    \item Ecosystems and the Environment
\end{itemize}
\end{marginnote}

\begin{itemize}
    \item \textbf{Cells: The Basic Units of Life:}  Cells are the fundamental building blocks of all living organisms.  We will explore the structure and function of different types of cells, including plant and animal cells, and learn about the organelles within cells and their roles.  You will understand how cells carry out the essential processes of life.
    \item \textbf{Organisation of Living Things:}  Living organisms are organised in complex hierarchical systems, from cells to tissues, organs, organ systems, and organisms.  We will explore how different levels of organisation work together to maintain life and carry out specific functions.
    \item \textbf{Life Processes: Nutrition, Respiration, and Excretion:}  To stay alive, organisms need to obtain nutrients, release energy through respiration, and remove waste products through excretion.  We will investigate these essential life processes in different types of organisms, including humans, plants, and microorganisms.
    \item \textbf{Reproduction and Inheritance:}  Life continues through reproduction, and offspring inherit traits from their parents.  We will explore different modes of reproduction, including sexual and asexual reproduction, and learn about the mechanisms of inheritance, including genes and chromosomes.
    \item \textbf{Ecosystems and the Environment:}  Living organisms interact with each other and their environment, forming ecosystems.  We will investigate different types of ecosystems, food webs, nutrient cycles, and the impact of human activities on the environment.  Understanding ecosystems is crucial for addressing environmental challenges and promoting sustainability.
\end{itemize}

Biology reveals the incredible diversity and complexity of life on Earth.  It helps us understand ourselves and our place in the natural world, and provides insights into health, disease, and conservation.

\FloatBarrier

\subsection{Earth and Space Science:  Our Planet and Beyond}

Earth and Space Science encompasses the study of our planet Earth, its systems, and its place in the vast universe.  It explores the Earth's structure, processes, history, atmosphere, oceans, and its interactions with the solar system and beyond.  In the Earth and Space Science sections, you will delve into:

\begin{marginnote}
\textit{Earth and Space Science Topics:}
\begin{itemize}
    \item Earth's Structure and Processes
    \item The Earth's Atmosphere and Climate
    \item Earth's Resources and Sustainability
    \item The Solar System and Beyond
    \item Space Exploration
\end{itemize}
\end{marginnote}

\begin{itemize}
    \item \textbf{Earth's Structure and Processes:}  Our planet is dynamic and constantly changing.  We will explore the Earth's layers (crust, mantle, core), plate tectonics, earthquakes, volcanoes, and the rock cycle.  Understanding these processes helps us explain the geological features of our planet and the forces that shape it.
    \item \textbf{The Earth's Atmosphere and Climate:}  The atmosphere is a vital layer protecting and sustaining life on Earth.  We will investigate the composition and structure of the atmosphere, weather patterns, climate, climate change, and the greenhouse effect.  Understanding these topics is crucial for addressing environmental challenges and ensuring a sustainable future.
    \item \textbf{Earth's Resources and Sustainability:}  We rely on Earth's resources for our needs, but these resources are finite.  We will explore different types of Earth resources (minerals, water, energy), their formation, distribution, and sustainable use.  You will learn about the importance of conservation and responsible resource management.
    \item \textbf{The Solar System and Beyond:}  Our solar system is just one small part of the vast universe.  We will explore the planets, moons, asteroids, comets, and other objects in our solar system, as well as stars, galaxies, and the universe as a whole.  You will learn about astronomical phenomena, space exploration, and our place in the cosmos.
    \item \textbf{Space Exploration:}  Humans have always been fascinated by space, and space exploration has led to incredible discoveries and technological advancements.  We will explore the history of space exploration, current space missions, and the challenges and opportunities of venturing beyond Earth.
\end{itemize}

Earth and Space Science provides a grand perspective, connecting us to our planet and the universe beyond.  It fosters a sense of wonder and responsibility for our planet and inspires us to explore the unknown.

\FloatBarrier

\section{Making the Most of This Book: Your Science Toolkit for Success}

This textbook is a powerful tool, but like any tool, it is most effective when used correctly.  Here are some tips to help you make the most of this book and excel in your Stage 5 Science studies.

\subsection{Active Reading Strategies: Engage with the Text}

Reading a science textbook is not like reading a novel.  It requires active engagement and a different approach.  Here are some active reading strategies to try:

\begin{marginnote}
\textit{Active Reading Tips:}
\begin{itemize}
    \item \textbf{Highlight Key Terms:}  Mark important vocabulary.
    \item \textbf{Annotate Margins:}  Write notes, questions, and summaries.
    \item \textbf{Summarise Sections:}  Put concepts in your own words.
    \item \textbf{Ask Questions:}  Identify areas of confusion and seek answers.
    \item \textbf{Connect to Prior Knowledge:}  Link new information to what you already know.
\end{itemize}
\end{marginnote}

\begin{itemize}
    \item \textbf{Highlight and Underline Key Terms and Concepts:}  Use a highlighter or pen to mark important definitions, principles, and examples as you read.  This will help you identify the core ideas and make them easier to locate later for review.
    \item \textbf{Annotate in the Margins (or a Notebook):}  Don't just passively read; actively interact with the text.  Write notes in the margins, summarise paragraphs in your own words, ask questions about things you don't understand, and make connections to other topics you have learned.
    \item \textbf{Summarise Each Section in Your Own Words:}  After reading a section, take a moment to summarise the main points in your own words, either verbally or in writing.  This will help you check your understanding and reinforce your learning.  If you struggle to summarise, it's a sign you need to reread the section.
    \item \textbf{Ask Questions as You Read:}  Be curious!  If something is unclear, or if you wonder "why?" or "how?", write down your questions.  Then, actively seek answers by rereading the text, checking margin notes, asking your teacher, or doing further research.
    \item \textbf{Connect New Information to What You Already Know:}  Try to link new concepts to your existing knowledge and experiences.  This helps you build a deeper understanding and see the relevance of science in your life.  Think about real-world examples and applications of the scientific principles you are learning.
\end{itemize}

Active reading makes learning more effective and engaging. It transforms you from a passive recipient of information to an active participant in the learning process.

\FloatBarrier

\subsection{Effective Study Habits: Plan, Practice, and Review}

Success in science, like any subject, relies on good study habits.  Here are some strategies to help you study effectively:

\begin{marginnote}
\textit{Study Habit Tips:}
\begin{itemize}
    \item \textbf{Plan Study Time:}  Schedule regular study sessions.
    \item \textbf{Spaced Repetition:}  Review material at increasing intervals.
    \item \textbf{Practice Questions Regularly:}  Use checkpoints and review questions.
    \item \textbf{Seek Help When Needed:}  Don't hesitate to ask for assistance.
    \item \textbf{Collaborate with Classmates:}  Learn from and with your peers.
\end{itemize}
\end{marginnote}

\begin{itemize}
    \item \textbf{Plan Regular Study Time:}  Don't wait until the last minute to study for tests or exams.  Set aside regular time each week to review material, work through practice questions, and prepare for upcoming topics.  Consistency is key to effective learning.
    \item \textbf{Use Spaced Repetition for Review:**  Our memories are not perfect, and we tend to forget information over time.  Spaced repetition is a technique where you review material at increasing intervals – perhaps a day after learning it, then a few days later, then a week later, and so on.  This helps to consolidate information in your long-term memory.
    \item \textbf{Practice Questions Regularly:**  Working through checkpoints and review questions is essential for testing your understanding and applying your knowledge.  Don't just read the questions; actively attempt to answer them, even if you are unsure.  Check your answers and identify areas where you need to improve.
    \item \textbf{Don't Be Afraid to Ask for Help:**  If you are struggling with a concept or unsure about something, don't hesitate to ask for help.  Talk to your teacher, classmates, or family members.  Asking questions is a sign of strength, not weakness, and it's a crucial part of the learning process.
    \item \textbf{Collaborate with Classmates:**  Study groups can be a valuable tool for learning.  Working with classmates allows you to discuss concepts, explain ideas to each other, and learn from different perspectives.  However, make sure study groups are focused and productive, and that everyone is actively participating.
\end{itemize}

Developing good study habits will not only help you succeed in Stage 5 Science but will also equip you with valuable skills for lifelong learning.

\FloatBarrier

\subsection{Navigating This Book: Finding Your Way Around}

This textbook is designed to be easy to navigate and use effectively.  Here are some tips to help you find your way around:

\begin{marginnote}
\textit{Navigation Tips:}
\begin{itemize}
    \item \textbf{Use the Table of Contents:}  Quickly find chapters and sections.
    \item \textbf{Refer to the Index:}  Locate specific topics and terms.
    \item \textbf{Utilise Margin Notes:**  Access extra information and definitions easily.
    \item \textbf{Follow Cross-References:**  Connect related concepts across chapters.
\end{itemize}
\end{marginnote}

\begin{itemize}
    \item \textbf{Table of Contents:**  The Table of Contents at the beginning of the book provides a clear overview of the chapters and sections.  Use it to quickly find the chapter or section you need.
    \item \textbf{Index:**  The Index at the back of the book is a comprehensive list of topics, terms, and concepts covered in the book, along with the page numbers where they are discussed.  Use the Index to quickly locate specific information you are looking for.
    \item \textbf{Margin Notes:**  As you have seen, margin notes are packed with useful information.  Use them to quickly access definitions, extra facts, and links to other concepts without having to search through the main text.
    \item \textbf{Cross-References (Where Applicable):**  In some cases, you might find cross-references within the text or margin notes, pointing you to related topics in other chapters or sections.  Follow these references to make connections between different areas of science and build a more holistic understanding.
\end{itemize}

By familiarising yourself with the features of this book and using these navigation tips, you will be able to access the information you need quickly and efficiently, making your learning experience smoother and more productive.

\FloatBarrier

\section{Welcome to the Adventure!}

Science is an adventure – an ongoing quest to understand the universe and our place within it.  Stage 5 Science is your opportunity to join this adventure, to develop your scientific thinking skills, and to explore the wonders of the natural world.  We have designed this textbook to be your trusted guide on this journey, providing you with the knowledge, tools, and encouragement you need to succeed.

\begin{marginfigure}[0pt]
\includegraphics[width=\linewidth]{placeholder_telescope.jpg}
\caption*{}
\textit{The universe is full of mysteries waiting to be explored. Are you ready to discover them?}
\end{marginfigure}

We believe that everyone can succeed in science, and we are committed to making this learning experience accessible, engaging, and rewarding for all students.  Embrace your curiosity, ask questions, explore the investigations, and use the features of this book to their full potential.  We are excited to embark on this Stage 5 Science journey with you.  Let's begin!

\FloatBarrier
```