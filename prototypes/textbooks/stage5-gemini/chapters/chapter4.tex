```latex
\chapter{Human Biology and Disease}

\marginnote{This chapter explores the fascinating world of human biology and how it relates to health and disease. Understanding these concepts is crucial for making informed decisions about your own wellbeing and the health of your community.}

\section{Introduction: The Amazing Human Body}

Have you ever stopped to think about how incredible your body is? It's a complex and finely tuned machine, capable of amazing feats, from running a marathon to composing a symphony.  \keyword{Human biology} is the scientific study of the human body – its structure, function, growth, origin, evolution, and interactions with the environment.

This chapter will delve into the fundamental building blocks of life within us, exploring how our bodies are organised into systems that work together in harmony. We will also investigate what happens when this harmony is disrupted, leading to \keyword{disease}.  Understanding disease processes is not just about knowing what goes wrong; it's about appreciating the resilience and adaptability of our bodies and learning how to maintain optimal health.

\begin{stopandthink}
Think about three things your body does automatically without you consciously controlling them. Why are these processes important for survival?
\end{stopandthink}

We will explore the microscopic world of \keyword{cells}, the fundamental units of life, and how they organise into tissues, organs, and organ systems. We will examine the major organ systems that keep us alive and functioning, such as the respiratory system that allows us to breathe and the digestive system that nourishes us.

Furthermore, we will investigate the concept of \keyword{homeostasis}, the body's ability to maintain a stable internal environment despite external changes. Disease often arises when homeostasis is disrupted. We will explore different types of diseases, from infectious diseases caused by tiny invaders to non-infectious diseases resulting from genetic factors or lifestyle choices. Finally, we will discuss how our \keyword{immune system} defends us against disease and how we can take proactive steps to maintain our health and prevent illness.

\historylink{The study of human biology has ancient roots, with early civilisations developing rudimentary understandings of anatomy and physiology. Ancient Egyptians, for example, had some knowledge of organs through mummification.  However, scientific human biology truly blossomed with the development of microscopy and scientific methods in the 17th century and beyond.}

\section{The Cell: The Fundamental Unit of Life}

\begin{keyconcept}{Cell Theory}
All living things are made up of cells. Cells are the basic structural and functional units of life. All cells come from pre-existing cells.
\end{keyconcept}

Just as bricks are the building blocks of a house, \keyword{cells} are the fundamental building blocks of all living organisms, including humans.  In fact, your body is made up of trillions of cells, all working together in a highly organised manner.

\subsection{Structure of a Cell}

While cells come in various shapes and sizes depending on their function, they share some common features.  Imagine a cell as a tiny, bustling city, with different structures, called \keyword{organelles}, performing specific jobs.

\begin{marginnote}
\textbf{Cell Size:} Most human cells are microscopic, ranging from about 10 to 100 micrometres in diameter.  A micrometre (µm) is one-millionth of a metre!
\end{marginnote}

\begin{itemize}
    \item \textbf{Cell Membrane:}  This is the outer boundary of the cell, like the city walls. It's a thin, flexible layer that controls what enters and leaves the cell.  It is selectively permeable, meaning it allows some substances to pass through while blocking others.  The cell membrane is composed mainly of \keyword{phospholipids} and proteins. \challenge{Research the fluid mosaic model of the cell membrane.}

    \item \textbf{Cytoplasm:} This is the gel-like substance filling the cell, like the city streets and parks.  It contains water, salts, and various organic molecules, and it's where many cellular processes occur.

    \item \textbf{Nucleus:}  Often called the "control centre" of the cell, the nucleus is like the city hall. It contains the cell's genetic material in the form of \keyword{DNA} (deoxyribonucleic acid). DNA carries the instructions for everything the cell does. The nucleus is surrounded by a \keyword{nuclear membrane} with pores that control the passage of substances in and out.

    \item \textbf{Organelles:} These are the "mini-organs" within the cell, each with a specific function.  Some key organelles include:
        \begin{itemize}
            \item \textbf{Mitochondria:}  The "powerhouses" of the cell. They generate energy through cellular respiration, converting glucose and oxygen into usable energy in the form of \keyword{ATP} (adenosine triphosphate), carbon dioxide, and water. \mathlink{Cellular respiration equation: \ce{C6H12O6 + 6O2 -> 6CO2 + 6H2O + ATP}}
            \item \textbf{Ribosomes:}  The "protein factories". Ribosomes are responsible for protein synthesis, using the instructions from DNA. They can be found free in the cytoplasm or attached to the endoplasmic reticulum.
            \item \textbf{Endoplasmic Reticulum (ER):} A network of membranes involved in protein and lipid synthesis and transport. There are two types: \textbf{rough ER} (studded with ribosomes, involved in protein synthesis) and \textbf{smooth ER} (involved in lipid synthesis and detoxification).
            \item \textbf{Golgi Apparatus:} The "packaging and processing centre". It modifies, sorts, and packages proteins and lipids for transport to other organelles or outside the cell.
            \item \textbf{Lysosomes:} The "recycling centres". They contain enzymes that break down waste materials and cellular debris.
            \item \textbf{Vacuoles:} Storage sacs within the cell, storing water, nutrients, or waste products. Plant cells have a large central vacuole, but animal cells have smaller vacuoles.
            \item \textbf{Centrioles:}  Involved in cell division in animal cells. They help organise the spindle fibres that separate chromosomes during cell division.
        \end{itemize}
\end{itemize}

\begin{figure}[htbp]
    \centering
    \includegraphics[width=0.7\textwidth]{placeholder-cell-diagram.png}
    \caption{Diagram of an animal cell showing key organelles. (Figure to be added later)}
    \label{fig:animalcell}
\end{figure}

\subsection{Functions of a Cell}

Cells perform a vast array of functions essential for life. These include:

\begin{itemize}
    \item \textbf{Metabolism:} All the chemical reactions that occur within a cell to maintain life. This includes breaking down nutrients for energy (catabolism) and building complex molecules (anabolism).
    \item \textbf{Growth and Reproduction:} Cells grow in size and number.  Most human cells reproduce through \keyword{cell division}, creating new cells for growth, repair, and replacement.
    \item \textbf{Responsiveness:} Cells can respond to stimuli from their environment, such as changes in temperature, pH, or chemical signals. Nerve cells, for example, are highly specialised for responsiveness and communication.
    \item \textbf{Movement:} Some cells, like muscle cells, are specialised for movement. Even cells that are not primarily for movement exhibit internal movement of organelles and substances within the cytoplasm.
    \item \textbf{Transport:} Cells transport substances across their membranes, bringing in nutrients and removing waste products. This can occur through passive transport (requiring no energy) or active transport (requiring energy). \challenge{Research examples of active and passive transport across cell membranes.}
\end{itemize}

\subsection{Cell Specialisation and Differentiation}

While all cells share basic features, they can become highly specialised to perform specific tasks in the body. This process is called \keyword{cell differentiation}.  During development, cells receive signals that cause them to activate or deactivate specific genes, leading to different cell types with unique structures and functions.

\begin{example}
Think about the difference between a nerve cell and a muscle cell. Nerve cells are long and branched, designed to transmit electrical signals rapidly. Muscle cells are elongated and contain contractile proteins, allowing them to contract and generate force. Both started as undifferentiated cells but became specialised through differentiation.
\end{example}

Examples of specialised cells include:

\begin{itemize}
    \item \textbf{Red blood cells:}  Specialised for oxygen transport, containing haemoglobin.
    \item \textbf{Nerve cells (neurons):}  Specialised for transmitting electrical signals, enabling communication throughout the body.
    \item \textbf{Muscle cells (muscle fibres):} Specialised for contraction, enabling movement.
    \item \textbf{Epithelial cells:}  Form linings and coverings, such as skin and the lining of the digestive tract, providing protection and regulating exchange.
\end{itemize}

\begin{stopandthink}
Why is cell specialisation important for the complexity of multicellular organisms like humans? What would happen if all cells were identical and performed the same functions?
\end{stopandthink}

\begin{investigation}{Observing Cells Under a Microscope (Virtual or Real)}
\textbf{Materials:}
\begin{itemize}
    \item Virtual microscope website or prepared microscope slides of different cell types (e.g., onion cells, cheek cells, blood cells).
    \item Microscope (if using real slides).
\end{itemize}
\textbf{Procedure:}
\begin{enumerate}
    \item \textbf{Virtual Microscope:} Access a virtual microscope website (many free online resources are available). Explore different types of cells at varying magnifications. Observe and sketch different cell structures you can identify (cell membrane, nucleus, cytoplasm, etc.).
    \item \textbf{Real Microscope:}  If using real slides, follow your teacher's instructions on how to use a microscope. Observe prepared slides of different cell types. Start at low magnification and gradually increase magnification to observe details. Draw and label the structures you can identify in each cell type.
\end{enumerate}
\textbf{Observations and Analysis:}
\begin{itemize}
    \item Compare and contrast the structures of different cell types you observed.
    \item How do the structures of these cells relate to their functions?
    \item What are the limitations of using a light microscope to study cells? \challenge{Research electron microscopes and their advantages in studying cell structures.}
\end{itemize}
\end{investigation}


\begin{tieredquestions}{The Cell}
\begin{enumerate}
    \item \textbf{Basic:}
    \begin{enumerate}
        \item What is the basic unit of life?
        \item Name three main parts of an animal cell.
        \item What is the function of mitochondria?
    \end{enumerate}
    \item \textbf{Intermediate:}
    \begin{enumerate}
        \item Explain the role of the cell membrane.
        \item Describe the function of the nucleus in a cell.
        \item How do ribosomes contribute to cell function?
    \end{enumerate}
    \item \textbf{Advanced:}
    \begin{enumerate}
        \item Explain the concept of cell differentiation and its importance.
        \item Compare and contrast the functions of the rough and smooth endoplasmic reticulum.
        \item How does the structure of a red blood cell relate to its function of oxygen transport?
    \end{enumerate}
\end{enumerate}
\end{tieredquestions}


\section{Organ Systems: Working Together for Life}

Cells are organised into \keyword{tissues}, groups of similar cells performing a specific function. Different tissues then combine to form \keyword{organs}, which are structures with specific functions (e.g., heart, lungs, stomach). Organs, in turn, work together in \keyword{organ systems} to carry out complex bodily functions.  Imagine organ systems as different departments in a large company, each with its specialised role, but all collaborating for the overall success of the company (the human body!).

\begin{marginnote}
\textbf{Levels of Organisation:}
Cells $\rightarrow$ Tissues $\rightarrow$ Organs $\rightarrow$ Organ Systems $\rightarrow$ Organism
\end{marginnote}

The human body has several major organ systems, each essential for survival.  Let's explore some of the key systems:

\subsection{Major Organ Systems}

\begin{itemize}
    \item \textbf{Integumentary System:} (Skin, hair, nails) - Provides protection, regulates temperature, and detects sensations.  The skin is the largest organ in the body!

    \item \textbf{Skeletal System:} (Bones, joints, cartilage) - Provides support and structure, protects organs, allows movement (with muscles), and produces blood cells in the bone marrow.  \challenge{Research different types of joints and their range of motion.}

    \item \textbf{Muscular System:} (Skeletal muscles, smooth muscles, cardiac muscle) - Enables movement, maintains posture, and generates heat. Skeletal muscles are responsible for voluntary movements, while smooth and cardiac muscles are involuntary.

    \item \textbf{Nervous System:} (Brain, spinal cord, nerves, sensory organs) -  Controls and coordinates body functions, responds to stimuli, and enables thought, learning, and memory. Communication within the nervous system is both electrical (nerve impulses) and chemical (neurotransmitters).

    \item \textbf{Endocrine System:} (Glands that secrete hormones: pituitary, thyroid, adrenal, pancreas, ovaries, testes) - Regulates body functions through hormones, which are chemical messengers transported in the blood. Hormones control processes like growth, metabolism, reproduction, and mood.

    \item \textbf{Cardiovascular System (Circulatory System):} (Heart, blood vessels: arteries, veins, capillaries, blood) - Transports oxygen, nutrients, hormones, and waste products throughout the body. The heart is the pump that drives blood circulation.

    \item \textbf{Respiratory System:} (Lungs, trachea, bronchi, diaphragm) -  Responsible for gas exchange – taking in oxygen and releasing carbon dioxide.  This process is essential for cellular respiration.

    \item \textbf{Digestive System:} (Mouth, oesophagus, stomach, small intestine, large intestine, liver, pancreas, gallbladder) - Breaks down food into smaller molecules that can be absorbed into the bloodstream and used by cells for energy and building materials.

    \item \textbf{Excretory System (Urinary System):} (Kidneys, ureters, bladder, urethra) - Removes metabolic waste products from the blood, maintains water balance, and regulates blood pressure. The kidneys are the primary organs of excretion.

    \item \textbf{Immune System (Lymphatic System):} (Lymph nodes, lymph vessels, thymus, spleen, bone marrow, white blood cells) - Defends the body against pathogens (bacteria, viruses, fungi, parasites) and abnormal cells (e.g., cancer cells).

    \item \textbf{Reproductive System:} (Ovaries, uterus, testes, etc.) -  Enables reproduction, ensuring the continuation of the species.  It also produces sex hormones that influence development and behaviour.

\end{itemize}

\begin{figure}[htbp]
    \centering
    \includegraphics[width=0.8\textwidth]{placeholder-organ-systems-diagram.png}
    \caption{Diagram showing the major organ systems of the human body. (Figure to be added later)}
    \label{fig:organsystems}
\end{figure}

\subsection{Focus on the Respiratory and Cardiovascular Systems}

Let's take a closer look at two vital systems – the respiratory and cardiovascular systems – and how they work together to sustain life.

\subsubsection{The Respiratory System: Breathing and Gas Exchange}

The primary function of the respiratory system is to facilitate \keyword{gas exchange} – the exchange of oxygen (\ce{O2}) and carbon dioxide (\ce{CO2}) between the body and the environment.

\begin{itemize}
    \item \textbf{Breathing (Ventilation):} Air enters the body through the nose and mouth, travels down the \textbf{trachea} (windpipe), and into the \textbf{bronchi}, which branch into smaller and smaller tubes called \textbf{bronchioles} within the lungs.
    \item \textbf{Lungs:} The main organs of respiration.  Inside the lungs, the bronchioles end in tiny air sacs called \textbf{alveoli}.  The alveoli are surrounded by capillaries (tiny blood vessels).
    \item \textbf{Gas Exchange in Alveoli:} Oxygen diffuses from the air in the alveoli into the blood in the capillaries, while carbon dioxide diffuses from the blood into the alveoli to be exhaled.  This exchange occurs because of differences in the concentration of oxygen and carbon dioxide and the thin walls of the alveoli and capillaries.
    \item \textbf{Diaphragm:} A large muscle below the lungs that contracts and relaxes to control breathing.  When the diaphragm contracts, it flattens, increasing the volume of the chest cavity and drawing air into the lungs (inhalation). When it relaxes, it domes upwards, decreasing the volume and pushing air out (exhalation).
\end{itemize}

\subsubsection{The Cardiovascular System: Transport and Delivery}

The cardiovascular system works closely with the respiratory system to transport oxygen from the lungs to all cells in the body and to carry carbon dioxide back to the lungs for removal.

\begin{itemize}
    \item \textbf{Heart:} The muscular pump that drives blood circulation.  It has four chambers: two atria (receiving chambers) and two ventricles (pumping chambers).
    \item \textbf{Blood Vessels:} A network of tubes that carry blood throughout the body.
        \begin{itemize}
            \item \textbf{Arteries:} Carry oxygenated blood away from the heart to the body's tissues (except for the pulmonary artery, which carries deoxygenated blood to the lungs).
            \item \textbf{Veins:} Carry deoxygenated blood back to the heart (except for the pulmonary vein, which carries oxygenated blood from the lungs to the heart).
            \item \textbf{Capillaries:} Tiny blood vessels that connect arteries and veins.  Gas exchange, nutrient exchange, and waste exchange occur between the blood and tissues across the thin capillary walls.
        \end{itemize}
    \item \textbf{Blood:}  A fluid connective tissue consisting of:
        \begin{itemize}
            \item \textbf{Red blood cells:}  Contain haemoglobin and transport oxygen.
            \item \textbf{White blood cells:} Part of the immune system, defending against infection.
            \item \textbf{Platelets:}  Involved in blood clotting.
            \item \textbf{Plasma:} The liquid part of blood, carrying nutrients, hormones, and waste products.
        \end{itemize}
\end{itemize}

\begin{stopandthink}
How do the respiratory and cardiovascular systems depend on each other? What would happen if one of these systems failed?
\end{stopandthink}

\subsection{Homeostasis: Maintaining Balance}

\keyword{Homeostasis} is the body's ability to maintain a stable internal environment despite changes in the external environment. This is crucial for cell survival and proper functioning of all organ systems.  Homeostasis is maintained by complex feedback mechanisms involving the nervous and endocrine systems.

\begin{marginnote}
\textbf{Examples of Homeostasis:}
\begin{itemize}
    \item Body temperature regulation
    \item Blood glucose level regulation
    \item Blood pressure regulation
    \item Water balance
    \item pH balance
\end{itemize}
\end{marginnote}

\begin{example}
Think about body temperature regulation. If you get too hot, your body sweats to cool you down. If you get too cold, you shiver to generate heat. These are examples of negative feedback loops that help maintain a stable internal temperature.
\end{example}

Organ systems work together to maintain homeostasis. For example:

\begin{itemize}
    \item The \textbf{nervous system} and \textbf{endocrine system} detect changes in the internal environment and send signals to other systems to make adjustments.
    \item The \textbf{cardiovascular system} transports hormones and other regulatory substances throughout the body.
    \item The \textbf{excretory system} removes waste products and helps regulate water and electrolyte balance.
    \item The \textbf{respiratory system} regulates blood pH by controlling carbon dioxide levels.
\end{itemize}

Disruptions to homeostasis can lead to disease.  Understanding how organ systems maintain balance is essential for understanding disease processes.

\begin{investigation}{Designing a Model of an Organ System}
\textbf{Objective:} To create a model of an organ system to demonstrate its structure and function.
\textbf{Materials:}  Various craft materials (cardboard, paper, straws, tubing, balloons, etc.), modelling clay, labels, markers.
\textbf{Procedure:}
\begin{enumerate}
    \item Choose an organ system to model (e.g., respiratory, cardiovascular, digestive).
    \item Research the structure and function of the chosen organ system.
    \item Plan your model, deciding which materials to use to represent different organs and structures within the system.
    \item Construct your model, ensuring it accurately represents the key components and functions of the organ system.
    \item Label the different parts of your model clearly.
    \item Prepare a short presentation to explain your model and how it demonstrates the function of the organ system.
\end{enumerate}
\textbf{Analysis:}
\begin{itemize}
    \item How effectively does your model represent the chosen organ system?
    \item What are the limitations of your model?
    \item How does building the model enhance your understanding of the organ system?
\end{itemize}
\end{investigation}


\begin{tieredquestions}{Organ Systems}
\begin{enumerate}
    \item \textbf{Basic:}
    \begin{enumerate}
        \item Name three major organ systems in the human body.
        \item What is the main function of the respiratory system?
        \item What organ pumps blood around the body?
    \end{enumerate}
    \item \textbf{Intermediate:}
    \begin{enumerate}
        \item Describe the role of the digestive system.
        \item Explain how the nervous system and endocrine system work together to control body functions.
        \item What is homeostasis and why is it important?
    \end{enumerate}
    \item \textbf{Advanced:}
    \begin{enumerate}
        \item Explain the process of gas exchange in the alveoli of the lungs.
        \item Describe the different types of blood vessels and their functions.
        \item Discuss how feedback mechanisms help maintain homeostasis in the body, giving a specific example.
    \end{enumerate}
\end{enumerate}
\end{tieredquestions}

\section{Disease: When Homeostasis is Disrupted}

\keyword{Disease} is any condition that impairs the normal functioning of the body. It represents a disruption of homeostasis, preventing the body from maintaining its stable internal environment.  Diseases can arise from a variety of causes, and understanding these causes is crucial for prevention and treatment.

\subsection{Types of Diseases}

Diseases can be broadly classified into two main categories: \keyword{infectious diseases} and \keyword{non-infectious diseases}.

\subsubsection{Infectious Diseases: Caused by Pathogens}

\keyword{Infectious diseases} are caused by \keyword{pathogens} – microscopic organisms that can invade the body and cause harm.  Pathogens include bacteria, viruses, fungi, and parasites.

\begin{marginnote}
\textbf{Historical Perspective:}  For centuries, the cause of infectious diseases was a mystery.  The germ theory of disease, developed in the 19th century by scientists like Louis Pasteur and Robert Koch, revolutionised our understanding and led to effective treatments and preventative measures. \historylink{Research the contributions of Louis Pasteur and Robert Koch to the germ theory of disease.}
\end{marginnote}

\textbf{Types of Pathogens:}

\begin{itemize}
    \item \textbf{Bacteria:} Single-celled prokaryotic organisms. Some bacteria are beneficial (e.g., gut bacteria aiding digestion), but others are pathogenic and cause diseases like \keyword{strep throat}, \keyword{tuberculosis}, and \keyword{food poisoning}. Bacteria can be treated with \keyword{antibiotics}.
    \item \textbf{Viruses:}  Non-cellular entities consisting of genetic material (DNA or RNA) enclosed in a protein coat. Viruses are obligate intracellular parasites, meaning they can only reproduce inside living cells. They cause diseases like the \keyword{common cold}, \keyword{influenza} (flu), \keyword{measles}, and \keyword{HIV/AIDS}.  Antibiotics are ineffective against viruses; antiviral medications can be used in some cases, and \keyword{vaccination} is a key preventative measure.
    \item \textbf{Fungi:} Eukaryotic organisms, some of which are pathogenic. Fungal infections, also called \keyword{mycoses}, can affect the skin (e.g., athlete's foot, ringworm), lungs (e.g., pneumonia in immunocompromised individuals), or other parts of the body. Antifungal medications are used to treat fungal infections.
    \item \textbf{Parasites:} Eukaryotic organisms that live in or on a host organism and obtain nutrients at the host's expense. Parasites can be single-celled protozoa (e.g., \keyword{malaria}, \keyword{amoebic dysentery}) or multicellular organisms (e.g., worms like tapeworms, roundworms). Antiparasitic medications are used to treat parasitic infections.
\end{itemize}

\textbf{Transmission of Infectious Diseases:}

Pathogens can be transmitted in various ways:

\begin{itemize}
    \item \textbf{Direct Contact:}  Physical contact with an infected person, such as touching, kissing, or sexual contact. Examples: common cold, sexually transmitted infections (STIs).
    \item \textbf{Indirect Contact (Vehicle Transmission):} Contact with contaminated objects (fomites), food, water, or air. Examples: influenza (airborne droplets), food poisoning (contaminated food), cholera (contaminated water).
    \item \textbf{Vector Transmission:** Transmission by an intermediate organism, called a vector, usually an arthropod (e.g., mosquito, tick, flea). Examples: malaria (mosquitoes), Lyme disease (ticks), plague (fleas).
\end{itemize}

\textbf{Prevention of Infectious Diseases:}

\begin{itemize}
    \item \textbf{Hygiene:** Frequent handwashing with soap and water, proper food handling, and sanitation are crucial for preventing the spread of pathogens.
    \item \textbf{Vaccination:**  Vaccines stimulate the immune system to develop immunity to specific pathogens, preventing or reducing the severity of disease. \challenge{Research how vaccines work and different types of vaccines.}
    \item \textbf{Antimicrobial Medications:** Antibiotics (for bacteria), antivirals (for some viruses), antifungals (for fungi), and antiparasitics (for parasites) can treat infections, but overuse and misuse of antibiotics have led to \keyword{antibiotic resistance}, a growing global health threat.
    \item \textbf{Public Health Measures:**  Surveillance, quarantine, isolation of infected individuals, and public health campaigns promote awareness and prevention.
\end{itemize}

\begin{stopandthink}
Why is it important to understand the different types of pathogens and how they are transmitted? How can personal hygiene and public health measures help prevent infectious diseases?
\end{stopandthink}

\subsubsection{Non-Infectious Diseases: Not Caused by Pathogens}

\keyword{Non-infectious diseases} are not caused by pathogens and are not contagious. They can result from genetic factors, lifestyle choices, environmental factors, or a combination of these.

\textbf{Types of Non-Infectious Diseases:}

\begin{itemize}
    \item \textbf{Genetic Diseases:** Caused by abnormalities in genes or chromosomes. These can be inherited from parents or arise spontaneously. Examples: \keyword{cystic fibrosis}, \keyword{Down syndrome}, \keyword{haemophilia}. \historylink{Research the Human Genome Project and its impact on understanding genetic diseases.}
    \item \textbf{Lifestyle Diseases:**  Linked to lifestyle choices such as diet, physical activity, smoking, and alcohol consumption. Examples: \keyword{heart disease}, \keyword{type 2 diabetes}, \keyword{some types of cancer}, \keyword{obesity}.  These diseases are often preventable or manageable through lifestyle modifications.
    \item \textbf{Environmental Diseases:**  Caused by exposure to harmful substances or conditions in the environment. Examples: \keyword{asthma} (can be triggered by air pollution), \keyword{allergies} (to pollen, dust mites, etc.), \keyword{lead poisoning}, \keyword{radiation sickness}.
    \item \textbf{Degenerative Diseases:**  Characterised by the progressive deterioration of tissues or organs over time. Examples: \keyword{Alzheimer's disease}, \keyword{osteoarthritis}, \keyword{Parkinson's disease}.  Age is a major risk factor for many degenerative diseases.
    \item \textbf{Autoimmune Diseases:**  Occur when the immune system mistakenly attacks the body's own tissues. Examples: \keyword{rheumatoid arthritis}, \keyword{multiple sclerosis}, \keyword{type 1 diabetes}.
\end{itemize}

\textbf{Prevention and Management of Non-Infectious Diseases:}

Prevention and management strategies vary depending on the type of non-infectious disease.

\begin{itemize}
    \item \textbf{Genetic Diseases:**  Genetic counselling and screening can help identify individuals at risk. Gene therapy is a developing field aiming to treat genetic diseases by modifying genes.
    \item \textbf{Lifestyle Diseases:**  Healthy lifestyle choices are crucial for prevention. This includes a balanced diet, regular physical activity, avoiding smoking and excessive alcohol, and managing stress. Early detection and management are important for slowing disease progression.
    \item \textbf{Environmental Diseases:**  Reducing exposure to environmental hazards is key. This can involve improving air and water quality, reducing exposure to toxins, and managing allergens.
    \item \textbf{Degenerative Diseases:**  While many degenerative diseases are not preventable, lifestyle factors like exercise and a healthy diet may help slow progression in some cases. Research is ongoing to find effective treatments.
    \item \textbf{Autoimmune Diseases:**  There is no cure for most autoimmune diseases, but medications can help manage symptoms and suppress the immune system to reduce tissue damage.
\end{itemize}

\begin{stopandthink}
What are the key differences between infectious and non-infectious diseases?  Give examples of lifestyle choices that can increase or decrease the risk of non-infectious diseases.
\end{stopandthink}

\begin{investigation}{Researching an Infectious Disease}
\textbf{Objective:} To research a specific infectious disease and present information about its cause, transmission, symptoms, treatment, and prevention.
\textbf{Procedure:}
\begin{enumerate}
    \item Choose an infectious disease to research (e.g., influenza, malaria, tuberculosis, HIV/AIDS, measles).
    \item Use reliable sources (e.g., reputable websites like the World Health Organization (WHO), Centers for Disease Control and Prevention (CDC), scientific articles, textbooks) to gather information about the disease.
    \item Focus on the following aspects:
        \begin{itemize}
            \item \textbf{Causative agent:** What pathogen causes the disease?
            \item \textbf{Transmission:** How is the disease spread?
            \item \textbf{Symptoms:** What are the common signs and symptoms of the disease?
            \item \textbf{Treatment:** How is the disease treated? Are there medications or therapies available?
            \item \textbf{Prevention:** How can the disease be prevented? Are there vaccines or other preventative measures?
            \item \textbf{Global impact:** What is the global burden of the disease? Is it a significant public health problem?
        \end{itemize}
    \item Organise your findings and present them in a format of your choice (e.g., written report, presentation, poster).
\end{enumerate}
\textbf{Analysis:}
\begin{itemize}
    \item What are the most significant challenges in controlling the disease you researched?
    \item How has scientific understanding contributed to the prevention and treatment of this disease?
    \item What further research or public health efforts are needed to address this disease effectively?
\end{itemize}
\end{investigation}


\begin{tieredquestions}{Disease}
\begin{enumerate}
    \item \textbf{Basic:}
    \begin{enumerate}
        \item What is a pathogen?
        \item Name two types of infectious diseases.
        \item Give an example of a non-infectious disease.
    \end{enumerate}
    \item \textbf{Intermediate:}
    \begin{enumerate}
        \item Explain the difference between bacteria and viruses.
        \item Describe three ways infectious diseases can be transmitted.
        \item How can lifestyle choices contribute to non-infectious diseases?
    \end{enumerate}
    \item \textbf{Advanced:}
    \begin{enumerate}
        \item Discuss the concept of antibiotic resistance and why it is a public health concern.
        \item Compare and contrast genetic diseases and autoimmune diseases.
        \item Explain how public health measures can help prevent the spread of infectious diseases and manage non-infectious diseases.
    \end{enumerate}
\end{enumerate}
\end{tieredquestions}


\section{The Immune System: Our Body's Defence Force}

The \keyword{immune system} is a complex network of cells, tissues, and organs that defends the body against pathogens and other harmful substances. It is our body's internal defence force, constantly working to protect us from disease.

\subsection{Innate and Adaptive Immunity}

The immune system has two main branches: \keyword{innate immunity} and \keyword{adaptive immunity}.

\begin{itemize}
    \item \textbf{Innate Immunity (Non-specific Immunity):} This is the body's first line of defence, providing rapid and general protection against a wide range of pathogens. Innate immunity is present from birth and does not require prior exposure to a pathogen.  Components of innate immunity include:
        \begin{itemize}
            \item \textbf{Physical Barriers:} Skin, mucous membranes, cilia in the respiratory tract, stomach acid – these prevent pathogens from entering the body.
            \item \textbf{Inflammation:** A non-specific response to tissue injury or infection.  It involves redness, swelling, heat, and pain, and it helps to contain infection and initiate the healing process.
            \item \textbf{Phagocytes:** White blood cells (e.g., macrophages, neutrophils) that engulf and destroy pathogens and cellular debris through phagocytosis ("cell eating").
            \item \textbf{Natural Killer (NK) Cells:**  A type of lymphocyte that can kill virus-infected cells and cancer cells.
            \item \textbf{Complement System:**  A group of proteins in the blood that can enhance phagocytosis, trigger inflammation, and directly kill pathogens.
        \end{itemize}
    \item \textbf{Adaptive Immunity (Specific Immunity):} This is a slower but more specific and long-lasting defence mechanism. Adaptive immunity develops after exposure to a specific pathogen and involves the production of \keyword{antibodies} and specialised immune cells that target that specific pathogen.  Key features of adaptive immunity are:
        \begin{itemize}
            \item \textbf{Specificity:**  Immune responses are tailored to specific pathogens.
            \item \textbf{Memory:**  The immune system "remembers" previous encounters with pathogens, allowing for a faster and stronger response upon subsequent exposure (immunological memory). This is the basis of vaccination.
            \item \textbf{Systemic:**  Adaptive immunity is not limited to the site of infection; it provides body-wide protection.
        \end{itemize}
\end{itemize}

\begin{figure}[htbp]
    \centering
    \includegraphics[width=0.7\textwidth]{placeholder-immune-system-diagram.png}
    \caption{Diagram illustrating the innate and adaptive immune systems. (Figure to be added later)}
    \label{fig:immunesystem}
\end{figure}

\subsection{Cells of the Immune System: White Blood Cells}

\keyword{White blood cells} (leukocytes) are the key cells of the immune system.  They are produced in the bone marrow and circulate in the blood and lymph.  Different types of white blood cells have specific roles in immune responses.

\begin{itemize}
    \item \textbf{Phagocytes:}
        \begin{itemize}
            \item \textbf{Neutrophils:}  The most abundant type of white blood cell.  They are phagocytic and are the first responders to infection.
            \item \textbf{Macrophages:}  Larger phagocytes that can engulf pathogens and cellular debris. They also play a role in activating adaptive immunity.
        \end{itemize}
    \item \textbf{Lymphocytes:}  A type of white blood cell crucial for adaptive immunity.  There are two main types:
        \begin{itemize}
            \item \textbf{B Lymphocytes (B Cells):}  Produce antibodies.  When activated by an antigen, B cells differentiate into plasma cells, which secrete large amounts of antibodies into the blood and lymph.
            \item \textbf{T Lymphocytes (T Cells):}  Have various roles in adaptive immunity.
                \begin{itemize}
                    \item \textbf{Helper T Cells (Th Cells):}  Help activate B cells and cytotoxic T cells. They are essential for coordinating immune responses.
                    \item \textbf{Cytotoxic T Cells (Tc Cells):}  Kill virus-infected cells and cancer cells by directly attacking them.
                    \item \textbf{Regulatory T Cells (Treg Cells):}  Help suppress the immune system and prevent autoimmune reactions.
                \end{itemize}
        \end{itemize}
\end{itemize}

\subsection{Antibodies and Antigens}

\keyword{Antigens} are any substances that trigger an immune response.  They are often proteins or carbohydrates found on the surface of pathogens, but they can also be toxins, allergens, or even the body's own tissues in autoimmune diseases.

\keyword{Antibodies} (immunoglobulins) are Y-shaped proteins produced by B cells (specifically plasma cells) in response to antigens.  Antibodies bind specifically to antigens, marking them for destruction by other immune cells or neutralising their harmful effects.

\begin{marginnote}
\textbf{Antibody Structure:} Antibodies have a variable region that binds to a specific antigen and a constant region that interacts with other immune components. There are different classes of antibodies (IgG, IgM, IgA, IgD, IgE) with different functions and locations in the body. \challenge{Research the different classes of antibodies and their functions.}
\end{marginnote}

\subsection{Vaccination and Immunity}

\keyword{Vacc