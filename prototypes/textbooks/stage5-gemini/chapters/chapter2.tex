```latex
\chapter{Atoms, Elements and Compounds}

\section{Introducing the World Around Us}

\begin{marginfigure}
\includegraphics[width=\marginparwidth]{placeholder-atom.png}
\caption*{\textit{Visualisation of an atom. Imagine the world around you is made of these tiny particles!}}
\end{marginfigure}

Have you ever stopped to think about what everything around you is made of? From the air you breathe to the chair you are sitting on, everything is composed of matter.  \keyword{Matter} is anything that has mass and takes up space. But what exactly is matter made of?  This chapter will take you on a fascinating journey into the microscopic world to explore the fundamental building blocks of matter: \keyword{atoms}, \keyword{elements}, and \keyword{compounds}.

We will start by uncovering the secrets of atoms, the incredibly tiny particles that make up all matter. Then, we'll explore how atoms combine to form elements, the simplest forms of matter, and compounds, which are formed when elements chemically join together.  Understanding atoms, elements, and compounds is crucial to understanding not just science, but the world itself!

\begin{stopandthink}
Think about some objects around you right now. List three objects and try to imagine what they might be made of at a very, very small scale.
\end{stopandthink}

\section{Atoms: The Tiny Building Blocks}

\begin{marginnote}
\textit{Etymology of Atom:} The word "atom" comes from the Greek word \textit{atomos}, meaning "indivisible" or "uncuttable".  Early philosophers believed atoms were the smallest, fundamental particles of matter.
\end{marginnote}

Imagine taking a piece of gold and dividing it into smaller and smaller pieces.  You could keep cutting it, even with incredibly tiny tools, until you eventually reach a point where you can no longer divide it and still have gold.  This smallest particle of gold that retains the properties of gold is an \keyword{atom}. Atoms are the fundamental building blocks of all matter. They are incredibly small – far too small to see with the naked eye, or even with a regular microscope!

\subsection{A Journey Inside the Atom}

For a long time, scientists thought atoms were the smallest particles, as the name suggests. However, we now know that atoms themselves are made up of even smaller \keyword{subatomic particles}.  The three main subatomic particles are:

\begin{itemize}
    \item \textbf{Protons:} Positively charged particles found in the centre of the atom.
    \item \textbf{Neutrons:}  Neutral particles (having no charge) also found in the centre of the atom.
    \item \textbf{Electrons:} Negatively charged particles that orbit the centre of the atom.
\end{itemize}

\begin{marginnote}
\challenge{Charge and Forces:}  You might remember from physics that opposite charges attract and like charges repel.  This is why the negatively charged electrons are attracted to the positively charged protons in the nucleus, holding the atom together.
\end{marginnote}

The centre of the atom, containing the protons and neutrons, is called the \keyword{nucleus}.  Electrons move around the nucleus in regions called \keyword{electron shells} or energy levels.  Think of the nucleus as the sun in the centre of our solar system, and the electrons as planets orbiting around it, but in specific energy levels rather than fixed orbits.

\begin{marginfigure}
\includegraphics[width=\marginparwidth]{placeholder-atomic-structure.png}
\caption*{\textit{A simplified model of atomic structure showing protons and neutrons in the nucleus and electrons orbiting in shells.}}
\end{marginfigure}

\begin{keyconcept}{Structure of an Atom}
Atoms are composed of a central nucleus containing positively charged protons and neutral neutrons.  Negatively charged electrons orbit the nucleus in electron shells. The number and arrangement of these subatomic particles determine the properties of an atom.
\end{keyconcept}

\subsection{Atomic Number and Mass Number: Identifying Atoms}

Each type of atom is defined by the number of protons it has in its nucleus. This number is called the \keyword{atomic number}.  The atomic number is unique to each element (which we will discuss later).  For example, all hydrogen atoms have 1 proton, so hydrogen's atomic number is 1. All carbon atoms have 6 protons, so carbon's atomic number is 6.

The \keyword{mass number} of an atom is the total number of protons and neutrons in its nucleus.  Since protons and neutrons have approximately the same mass (and electrons are much lighter), the mass number gives us an idea of the atom's relative mass.

\begin{example}
Consider a carbon atom. Carbon has an atomic number of 6, meaning it has 6 protons.  A common form of carbon also has 6 neutrons.  Therefore, its mass number is 6 protons + 6 neutrons = 12.
\end{example}

We can represent atoms using symbols that show both the mass number and the atomic number.  The atomic number is written as a subscript before the element symbol, and the mass number as a superscript before the element symbol. For carbon-12, it would be represented as:  $^{12}_{6}\text{C}$.

\begin{marginnote}
\mathlink{Relative Atomic Mass:}  The mass number is a whole number, but the atomic masses listed on the periodic table are often not. This is because they are average atomic masses, taking into account the different isotopes of an element and their abundance. We will explore this further later.
\end{marginnote}

\subsection{Isotopes: Variations within an Element}

Atoms of the same element always have the same number of protons (same atomic number). However, they can have different numbers of neutrons. Atoms of the same element with different numbers of neutrons are called \keyword{isotopes}.

For example, carbon exists in several isotopic forms.  Carbon-12 ($^{12}_{6}\text{C}$) has 6 protons and 6 neutrons. Carbon-13 ($^{13}_{6}\text{C}$) has 6 protons and 7 neutrons. Carbon-14 ($^{14}_{6}\text{C}$) has 6 protons and 8 neutrons. All of these are isotopes of carbon because they all have 6 protons (atomic number 6), which defines them as carbon.

Most elements have several isotopes, some of which are stable and some of which are radioactive, meaning they spontaneously decay and release energy. Carbon-14, for instance, is a radioactive isotope used in carbon dating to determine the age of ancient organic materials.

\begin{marginnote}
\historylink{Isotope Discovery:} The existence of isotopes was discovered by Frederick Soddy in the early 20th century while studying radioactive decay.  This discovery helped to resolve some inconsistencies in the periodic table and further refine our understanding of atomic structure.
\end{marginnote}

\subsection{Ions: Charged Atoms}

Atoms are electrically neutral because they have an equal number of protons (positive charges) and electrons (negative charges). However, atoms can gain or lose electrons. When an atom gains or loses electrons, it becomes electrically charged and is called an \keyword{ion}.

\begin{itemize}
    \item \textbf{Cations:}  Positively charged ions. These are formed when an atom loses one or more electrons. For example, a sodium atom (Na) can lose one electron to become a sodium ion (\ce{Na+}).
    \item \textbf{Anions:} Negatively charged ions. These are formed when an atom gains one or more electrons. For example, a chlorine atom (Cl) can gain one electron to become a chloride ion (\ce{Cl-}).
\end{itemize}

Ions play a crucial role in many chemical processes and are essential for life.  For example, ions are involved in nerve impulses, muscle contractions, and maintaining fluid balance in our bodies.

\begin{stopandthink}
Consider a magnesium atom (Mg) which has 12 protons and 12 electrons. If it loses 2 electrons, what ion will it form? What will be its charge?
\end{stopandthink}

\begin{investigation}{Building Atomic Models}
\textbf{Materials:}  Different coloured balls or beads (e.g., red for protons, blue for neutrons, yellow for electrons), string or wire, labels.

\textbf{Procedure:}
\begin{enumerate}
    \item Choose an element from the first 20 elements of the periodic table (e.g., oxygen, nitrogen, sodium).
    \item Research the atomic number and mass number of your chosen element.
    \item Determine the number of protons, neutrons, and electrons in a neutral atom of your element.
    \item Use the coloured balls/beads to represent protons, neutrons, and electrons.
    \item Construct a model of the atom, placing protons and neutrons in the centre to represent the nucleus, and arranging electrons in shells around the nucleus (you can represent shells with string or wire rings).  For Stage 5, a simplified shell model is sufficient (e.g., 2 electrons in the first shell, up to 8 in the second shell).
    \item Label each part of your model (nucleus, protons, neutrons, electrons, shells, atomic number, mass number, element name and symbol).
\end{enumerate}

\textbf{Discussion:}
\begin{enumerate}
    \item What are the limitations of your model? (e.g., scale, electron orbits are not actually fixed rings).
    \item How does your model help you understand the structure of an atom?
    \item How would you modify your model to represent an isotope of your chosen element?
    \item How would you modify your model to represent an ion of your chosen element?
\end{enumerate}
\end{investigation}

\begin{tieredquestions}{Atoms}
\textbf{Basic:}
\begin{enumerate}
    \item What is an atom?
    \item Name the three main subatomic particles and their charges.
    \item Where are protons and neutrons located in an atom?
    \item What is the atomic number of an atom?
    \item What is the mass number of an atom?
\end{enumerate}

\textbf{Intermediate:}
\begin{enumerate}
    \item Explain the relationship between protons, neutrons, and electrons in a neutral atom.
    \item What are isotopes? Give an example.
    \item How are cations and anions formed?
    \item If an atom has an atomic number of 8 and a mass number of 16, how many protons, neutrons, and electrons does it have?
    \item Explain why the atomic number is more important than the mass number in identifying an element.
\end{enumerate}

\textbf{Advanced:}
\begin{enumerate}
    \item Describe the limitations of the simple Bohr model of the atom and suggest how our understanding of electron arrangement has evolved.
    \item Research and explain the concept of radioactive isotopes and their applications in different fields (e.g., medicine, archaeology).
    \item Explain how the concept of ions is crucial for understanding chemical bonding and the formation of compounds.
    \item Carbon has three naturally occurring isotopes: $^{12}\text{C}$ (98.9\%), $^{13}\text{C}$ (1.1\%), and $^{14}\text{C}$ (trace amounts).  Explain why the atomic mass of carbon on the periodic table is approximately 12.01 atomic mass units (amu) and not exactly 12 amu.
    \item  Challenge: Research and describe the role of quarks and leptons as even more fundamental particles than protons, neutrons, and electrons.
\end{enumerate}
\end{tieredquestions}


\section{Elements: Organising the Atoms}

\begin{marginnote}
\textit{Pure Substance:} An \keyword{element} is a pure substance because it is made up of only one type of atom.  Unlike mixtures, elements cannot be broken down into simpler substances by physical means.
\end{marginnote}

Now that we know about atoms, we can explore \keyword{elements}. An element is a pure substance that consists of only one type of atom.  For example, gold, oxygen, and nitrogen are all elements.  You cannot break down an element into simpler substances using chemical means. If you try to chemically break down gold, you will still be left with gold atoms.

There are over 100 known elements, each with its own unique properties.  Most of these elements occur naturally on Earth, while some are created artificially in laboratories.  Elements are organised in a chart called the \keyword{periodic table}, which is one of the most important tools in chemistry.

\subsection{The Periodic Table: A Map of the Elements}

\begin{marginfigure}
\includegraphics[width=\marginparwidth]{placeholder-periodic-table.png}
\caption*{\textit{A simplified periodic table of elements.  Notice the organisation into periods and groups.}}
\end{marginfigure}

The periodic table is not just a list of elements; it’s a highly organised chart that reveals relationships between elements based on their properties.  It arranges elements in order of increasing atomic number.  The periodic table is organised into:

\begin{itemize}
    \item \textbf{Periods:}  Horizontal rows in the periodic table are called periods.  Elements in the same period have the same number of electron shells.  As you move across a period from left to right, the atomic number increases, and the properties of elements gradually change.
    \item \textbf{Groups (or Families):} Vertical columns in the periodic table are called groups or families. Elements in the same group have similar chemical properties because they have the same number of electrons in their outermost electron shell (valence electrons).  These valence electrons are responsible for how elements react chemically.
\end{itemize}

\begin{marginnote}
\historylink{Mendeleev's Periodic Table:}  Dmitri Mendeleev, a Russian chemist, is credited with creating the first widely recognised periodic table in 1869.  He arranged elements based on their atomic weights and noticed periodic trends in their properties.  Remarkably, his table even predicted the existence of elements that were not yet discovered!
\end{marginnote}

The periodic table also provides information about whether an element is a \keyword{metal}, \keyword{non-metal}, or \keyword{metalloid} (sometimes called semi-metals).

\begin{itemize}
    \item \textbf{Metals:}  Generally found on the left side of the periodic table. Metals are typically shiny, good conductors of heat and electricity, malleable (can be hammered into shapes), and ductile (can be drawn into wires). Examples include gold, silver, copper, iron, and aluminium.
    \item \textbf{Non-metals:}  Generally found on the right side of the periodic table. Non-metals are often dull, poor conductors of heat and electricity, and brittle (easily broken). Examples include oxygen, nitrogen, chlorine, sulfur, and phosphorus.
    \item \textbf{Metalloids:}  Found along the staircase line separating metals and non-metals. Metalloids have properties of both metals and non-metals.  Their properties are often intermediate between metals and non-metals, and they are often semiconductors, meaning they conduct electricity under certain conditions but not others.  Examples include silicon, germanium, and arsenic.
\end{itemize}

\subsection{Element Symbols and Names: A Universal Language}

Each element is represented by a unique \keyword{element symbol}, usually one or two letters. The first letter is always capitalised, and the second letter (if present) is always lowercase.  These symbols are used internationally, providing a universal language for chemists and scientists worldwide.

\begin{example}
\begin{itemize}
    \item Hydrogen: \ce{H}
    \item Oxygen: \ce{O}
    \item Carbon: \ce{C}
    \item Sodium: \ce{Na} (from Latin \textit{natrium})
    \item Iron: \ce{Fe} (from Latin \textit{ferrum})
    \item Copper: \ce{Cu} (from Latin \textit{cuprum})
\end{itemize}
\end{example}

Many element symbols are derived from their English names, but some come from Latin or Greek names, reflecting their historical discovery and naming.  It’s helpful to become familiar with the symbols of common elements.

\subsection{Properties of Elements: What Makes Each Element Unique?}

Each element has a unique set of \keyword{properties} that distinguish it from other elements. These properties can be classified as physical properties and chemical properties.

\begin{itemize}
    \item \textbf{Physical Properties:} Properties that can be observed or measured without changing the chemical composition of the substance. Examples include:
        \begin{itemize}
            \item \textbf{Melting point} and \textbf{boiling point}: Temperatures at which a substance changes state (solid to liquid, liquid to gas).
            \item \textbf{Density}: Mass per unit volume.
            \item \textbf{Hardness}: Resistance to scratching or indentation.
            \item \textbf{Conductivity}: Ability to conduct heat or electricity.
            \item \textbf{Lustre}: How shiny a substance is.
            \item \textbf{Malleability} and \textbf{ductility}: Ability to be hammered into shapes or drawn into wires.
            \item \textbf{State of matter} at room temperature (solid, liquid, or gas).
            \item \textbf{Colour} and \textbf{odour}.
        \end{itemize}
    \item \textbf{Chemical Properties:} Properties that describe how a substance reacts with other substances or how it changes its chemical composition. Examples include:
        \begin{itemize}
            \item \textbf{Flammability}: Ability to burn.
            \item \textbf{Reactivity with acids, water, or air}: How readily a substance reacts with these.
            \item \textbf{Corrosiveness}: Ability to corrode or wear away other materials.
        \end{itemize}
\end{itemize}

The properties of elements are directly related to their atomic structure, particularly the number and arrangement of electrons.  Elements in the same group often have similar chemical properties because they have the same number of valence electrons.

\begin{stopandthink}
Think about the element oxygen.  List some physical and chemical properties of oxygen that you know or can research.  Consider its state at room temperature, colour, reactivity, and uses.
\end{stopandthink}

\begin{investigation}{Investigating Properties of Elements}
\textbf{Safety Note:}  Always follow your teacher's instructions and wear appropriate safety goggles and gloves when handling chemicals.  Some elements can be hazardous.

\textbf{Materials:} Small samples of different elements (e.g., copper wire, aluminium foil, sulfur powder, graphite rod, silicon wafer – ensure safe and readily available elements are used), conductivity tester, hammer, safety goggles, gloves, magnifying glass.

\textbf{Procedure:}
\begin{enumerate}
    \item Obtain small samples of different elements provided by your teacher.
    \item For each element, carefully observe and record its physical properties:
        \begin{itemize}
            \item State at room temperature (solid, liquid, gas).
            \item Colour and lustre (shiny or dull).
            \item Hardness (try to scratch it gently with another material – with caution and teacher guidance).
            \item Malleability (try to gently bend or hammer a small piece – with caution and teacher guidance, only for appropriate materials).
            \item Ductility (observe if it is in wire form or can be drawn into a wire).
            \item Conductivity (test its electrical conductivity using a conductivity tester – with teacher supervision).
        \end{itemize}
    \item Classify each element as metal, non-metal, or metalloid based on your observations and the periodic table.
    \item Research the chemical properties and common uses of each element.
\end{enumerate}

\textbf{Discussion:}
\begin{enumerate}
    \item How do the physical properties of metals, non-metals, and metalloids differ?
    \item How does the periodic table help predict the properties of elements?
    \item What are some limitations of classifying elements strictly as metals, non-metals, or metalloids? (Consider elements near the metalloid boundary).
    \item  Why is understanding the properties of elements important in everyday life and in different industries?
\end{enumerate}
\end{investigation}


\begin{tieredquestions}{Elements}
\textbf{Basic:}
\begin{enumerate}
    \item What is an element?
    \item What is the periodic table?
    \item What are periods and groups in the periodic table?
    \item Give three examples of metals and three examples of non-metals.
    \item What is an element symbol? Give the symbols for oxygen, sodium, and iron.
\end{enumerate}

\textbf{Intermediate:}
\begin{enumerate}
    \item Explain how the periodic table is organised and what information it provides about elements.
    \item Describe the general properties of metals and non-metals.
    \item What are metalloids? Where are they located in the periodic table?
    \item Explain why elements in the same group of the periodic table have similar chemical properties.
    \item Choose two elements from different groups and periods of the periodic table and compare their properties.
\end{enumerate}

\textbf{Advanced:}
\begin{enumerate}
    \item Discuss the historical development of the periodic table and the contributions of scientists like Mendeleev.
    \item Explain how the electron configuration of atoms relates to their position in the periodic table and their properties.
    \item Research and describe the properties and uses of a specific group of elements (e.g., alkali metals, halogens, noble gases).
    \item  Explain how the properties of elements change across a period and down a group in the periodic table (periodic trends).
    \item Challenge:  Investigate the environmental impact of mining and using different elements, considering sustainability and responsible resource management.
\end{enumerate}
\end{tieredquestions}


\section{Compounds: Elements Combined}

\begin{marginnote}
\textit{Chemical Bond:}  A \keyword{chemical bond} is a strong attractive force that holds atoms together in molecules or ionic compounds.  These bonds involve the interaction of electrons between atoms.
\end{marginnote}

Elements are the simplest forms of matter, but most of the matter around us is made up of \keyword{compounds}. A compound is a substance formed when two or more different elements are chemically bonded together in a fixed ratio.  When elements combine to form compounds, they lose their individual properties and create new substances with different properties.

For example, water (\ce{H2O}) is a compound formed from the elements hydrogen and oxygen.  Hydrogen is a flammable gas, and oxygen is a gas that supports combustion. However, when they combine chemically to form water, the resulting compound is a liquid that is neither flammable nor supports combustion; in fact, it is used to put out fires!

\subsection{Types of Chemical Bonds (Introduction)}

Chemical bonds are the forces that hold atoms together in compounds. At Stage 5, we will briefly introduce two main types of chemical bonds:

\begin{itemize}
    \item \textbf{Ionic Bonds:} Formed when electrons are transferred from one atom to another, creating ions. The oppositely charged ions are then attracted to each other, forming an ionic bond.  Ionic bonds typically occur between metals and non-metals.  \example{Sodium chloride (\ce{NaCl}), common table salt, is an ionic compound.}
    \item \textbf{Covalent Bonds:} Formed when atoms share electrons. Covalent bonds typically occur between non-metals. \example{Water (\ce{H2O}) and carbon dioxide (\ce{CO2}) are covalent compounds.}
\end{itemize}

\begin{marginnote}
\challenge{Bonding and Electrons:}  The type of chemical bond that forms depends on the electron arrangement of the atoms involved, particularly their valence electrons. We will explore this in more detail in later stages.
\end{marginnote}

Metallic bonds are another type of chemical bond, found in metals, but we will focus on ionic and covalent bonds when discussing compounds.

\subsection{Molecular and Ionic Compounds}

Compounds can be broadly classified into two categories based on the type of bonding and their structure:

\begin{itemize}
    \item \textbf{Molecular Compounds (Covalent Compounds):}  Formed by covalent bonds where atoms share electrons.  These compounds are made up of \keyword{molecules}, which are discrete groups of atoms held together by covalent bonds.  Molecular compounds are typically formed between non-metals.  \example{Water (\ce{H2O}), methane (\ce{CH4}), sugar (\ce{C12H22O11}).}
    \item \textbf{Ionic Compounds:} Formed by ionic bonds where electrons are transferred and ions are held together by electrostatic attraction.  Ionic compounds are not made up of discrete molecules; instead, they form a giant lattice structure of ions. Ionic compounds are typically formed between metals and non-metals. \example{Sodium chloride (\ce{NaCl}), magnesium oxide (\ce{MgO}), calcium chloride (\ce{CaCl2}).}
\end{itemize}

\begin{marginfigure}
\includegraphics[width=\marginparwidth]{placeholder-ionic-vs-molecular.png}
\caption*{\textit{Illustrations comparing the structure of a molecular compound (water) and an ionic compound (sodium chloride).}}
\end{marginfigure}


\subsection{Chemical Formulas and Naming Compounds (Simple Examples)}

\keyword{Chemical formulas} are used to represent compounds.  They show the types of atoms present in the compound and the ratio in which they are combined.  Subscripts are used to indicate the number of atoms of each element in a formula.

\begin{example}
\begin{itemize}
    \item \textbf{Water:} \ce{H2O} (2 hydrogen atoms and 1 oxygen atom)
    \item \textbf{Carbon Dioxide:} \ce{CO2} (1 carbon atom and 2 oxygen atoms)
    \item \textbf{Sodium Chloride:} \ce{NaCl} (1 sodium atom and 1 chlorine atom - ratio in the ionic lattice)
    \item \textbf{Methane:} \ce{CH4} (1 carbon atom and 4 hydrogen atoms)
\end{itemize}
\end{example}

\begin{marginnote}
\mathlink{Ratios and Proportions:} The fixed ratio of atoms in a compound is a fundamental concept in chemistry, related to the law of definite proportions.  This means that a specific compound always contains the same elements in the same proportions by mass.
\end{marginnote}

Naming compounds follows certain rules, which we will learn in more detail later.  For simple ionic compounds, we often name the metal first, followed by the non-metal with the suffix "-ide".  For example, \ce{NaCl} is sodium chloride.  For simple covalent compounds, we often use prefixes to indicate the number of atoms of each element, like "di-" for two, "tri-" for three, etc. For example, \ce{CO2} is carbon dioxide.

\subsection{Mixtures vs. Compounds: What's the Difference?}

It is important to distinguish between \keyword{mixtures} and compounds.

\begin{itemize}
    \item \textbf{Mixtures:}  Formed by physically mixing two or more substances together.  In a mixture, the substances are not chemically bonded and retain their individual properties. Mixtures can be separated by physical means (e.g., filtration, evaporation, distillation). \example{Air is a mixture of gases (nitrogen, oxygen, etc.), saltwater is a mixture of salt and water, sand and water is a mixture.}
    \item \textbf{Compounds:} Formed by chemically combining two or more elements. In a compound, the elements are chemically bonded and lose their individual properties. Compounds can only be separated into their constituent elements by chemical reactions. \example{Water (\ce{H2O}), carbon dioxide (\ce{CO2}), sodium chloride (\ce{NaCl}).}
\end{itemize}

\begin{marginfigure}
\includegraphics[width=\marginparwidth]{placeholder-mixture-compound.png}
\caption*{\textit{Diagram illustrating the difference between a mixture and a compound at the particle level.}}
\end{marginfigure}

\begin{stopandthink}
Think about a cake. Is a cake a mixture or a compound? Explain your reasoning based on the definitions of mixtures and compounds.
\end{stopandthink}


\begin{investigation}{Separating Mixtures and Compounds}
\textbf{Safety Note:}  Always follow your teacher's instructions and wear appropriate safety goggles and gloves when handling chemicals and using heat.

\textbf{Materials:}  For mixture separation: Sand and salt mixture, water, beakers, stirring rod, filter paper, filter funnel, evaporating dish, Bunsen burner (or hot plate). For compound separation (demonstration by teacher): Copper oxide powder, dilute sulfuric acid, zinc metal, test tubes, Bunsen burner.

\textbf{Procedure:}
\textbf{Part 1: Separating a Mixture (Sand and Salt)}
\begin{enumerate}
    \item Obtain a mixture of sand and salt.
    \item Add water to the mixture and stir well to dissolve the salt.
    \item Set up a filtration apparatus using a filter funnel and filter paper.
    \item Pour the mixture into the filter funnel to separate the sand (residue) from the salt water (filtrate).
    \item Collect the filtrate (salt water) in a beaker.
    \item Carefully evaporate the water from the salt water by gently heating the filtrate in an evaporating dish using a Bunsen burner or hot plate (under teacher supervision).
    \item Observe and collect the salt crystals that remain in the evaporating dish.
\end{enumerate}

\textbf{Part 2: Separating a Compound (Copper Oxide - Teacher Demonstration)}
\begin{enumerate}
    \item Observe your teacher demonstrating the reaction.  (Teacher will react copper oxide with dilute sulfuric acid to form copper sulfate solution, then react copper sulfate solution with zinc to displace copper).
    \item Observe the colour changes and any precipitates formed during the reactions.
    \item Note that chemical reactions are required to separate the elements in copper oxide.
\end{enumerate}

\textbf{Discussion:}
\begin{enumerate}
    \item In Part 1, what physical separation techniques were used to separate the sand and salt mixture? Explain why these techniques worked.
    \item In Part 2, what evidence shows that chemical reactions are involved in separating the elements from copper oxide?
    \item Explain the key differences between separating mixtures and separating compounds.
    \item Give other examples of mixtures and compounds found in everyday life, and suggest methods to separate the components of those mixtures.
    \item Why is it important to be able to separate mixtures and compounds in science and industry?
\end{enumerate}
\end{investigation}


\begin{tieredquestions}{Compounds}
\textbf{Basic:}
\begin{enumerate}
    \item What is a compound?
    \item What is a chemical bond?
    \item Name two main types of chemical bonds.
    \item Give two examples of molecular compounds and two examples of ionic compounds.
    \item What is the chemical formula for water?
\end{enumerate}

\textbf{Intermediate:}
\begin{enumerate}
    \item Explain the difference between ionic and covalent bonds in simple terms.
    \item Describe the difference between molecular compounds and ionic compounds in terms of their structure.
    \item What is the difference between a mixture and a compound? Give examples of each.
    \item Write the chemical formulas for carbon dioxide, sodium chloride, and methane.
    \item Explain why the properties of a compound are different from the properties of the elements that make it up.
\end{enumerate}

\textbf{Advanced:}
\begin{enumerate}
    \item Describe the process of ionic bond formation and covalent bond formation in more detail, mentioning electron transfer and electron sharing.
    \item Compare and contrast the properties of molecular compounds and ionic compounds (e.g., melting point, boiling point, conductivity).
    \item Explain why mixtures can be separated by physical means, while compounds require chemical reactions for separation.
    \item Research and describe the different types of mixtures (homogeneous and heterogeneous) and give examples of each.
    \item Challenge: Investigate and explain the concept of intermolecular forces and their influence on the properties of molecular compounds. How do intermolecular forces compare to the strength of covalent bonds?
\end{enumerate}
\end{tieredquestions}


\section{Chapter Summary}

In this chapter, we have explored the fundamental building blocks of matter: atoms, elements, and compounds.  Let's recap the key points:

\begin{itemize}
    \item \textbf{Atoms} are the smallest particles of an element that retain the properties of that element. They are made up of protons, neutrons, and electrons.
    \item The \textbf{atomic number} defines an element and is the number of protons in the nucleus. The \textbf{mass number} is the total number of protons and neutrons. \textbf{Isotopes} are atoms of the same element with different numbers of neutrons. \textbf{Ions} are charged atoms formed by gaining or losing electrons.
    \item \textbf{Elements} are pure substances consisting of only one type of atom. They are organised in the \textbf{periodic table} by atomic number, into periods and groups. Elements can be classified as metals, non-metals, or metalloids, each with characteristic properties.
    \item \textbf{Compounds} are substances formed when two or more different elements are chemically bonded together in a fixed ratio.  \textbf{Ionic bonds} involve electron transfer, and \textbf{covalent bonds} involve electron sharing.
    \item \textbf{Molecular compounds} are made of molecules held together by covalent bonds, while \textbf{ionic compounds} form giant lattice structures of ions held together by ionic bonds.
    \item \textbf{Chemical formulas} represent compounds, showing the types and ratios of atoms. \textbf{Mixtures} are formed by physically mixing substances and can be separated by physical means, unlike compounds which require chemical reactions for separation.
\end{itemize}

Understanding atoms, elements, and compounds is the foundation for exploring more complex chemistry and understanding the world around us.  In the chapters to come, we will build upon this knowledge to investigate chemical reactions, the behaviour of matter, and much more!
```