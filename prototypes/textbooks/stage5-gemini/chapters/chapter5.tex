```latex
\chapter{Genetics and Evolution}

\begin{marginfigure}
\includegraphics[width=\linewidth]{placeholder_dna_helix.jpg}
\caption*{\textit{The Double Helix of DNA.}  The molecule of heredity that carries the instructions for life.}
\end{marginfigure}

\section{The Amazing World of Heredity and Change}

Have you ever wondered why you look like your parents? Or why animals come in such a bewildering variety of shapes and sizes? The answers lie in the fascinating fields of \keyword{genetics} and \keyword{evolution}. Genetics is the study of \keyword{heredity}, how traits are passed down from one generation to the next. Evolution, on the other hand, explores how life on Earth has changed over vast stretches of time, driven by genetic changes.

This chapter will take you on a journey into the inner workings of cells, revealing the secrets of DNA and genes. We'll discover how these tiny structures dictate everything from the colour of your eyes to the shape of a bird's beak. We'll also explore how genetic changes, combined with the environment, lead to the incredible diversity of life we see around us and how species adapt and change over time.

\begin{marginnote}
\historylink{The term "genetics" was coined by William Bateson in 1905, though the principles of heredity were being explored much earlier, most famously by Gregor Mendel in the 19th century.}
\end{marginnote}

\begin{stopandthink}
Think about some traits you have inherited from your parents. Can you list three physical characteristics and one behavioural tendency that might be influenced by your genes?
\end{stopandthink}

\section{Cells, Chromosomes, and the Code of Life}

To understand genetics, we need to zoom in to the microscopic world of \keyword{cells}. Cells are the fundamental building blocks of all living organisms.  Within each cell, there's a control centre called the \keyword{nucleus}.  It's in the nucleus where we find the instructions for building and operating an entire organism. These instructions are encoded in a remarkable molecule called \keyword{deoxyribonucleic acid}, or \keyword{DNA} for short.

\subsection{DNA: The Blueprint of Life}

Imagine DNA as a very long instruction manual, written in a special code. This manual is organised into sections called \keyword{genes}. Each gene contains the instructions for making a specific \keyword{protein}. Proteins are the workhorses of the cell, carrying out all sorts of essential tasks.  They build structures, speed up chemical reactions, and fight off diseases, to name just a few examples.

DNA itself has a unique structure, often described as a \keyword{double helix}.  Think of it like a twisted ladder.  The sides of the ladder are made of sugar and phosphate molecules, while the rungs are formed by pairs of chemical bases. There are four types of bases, often abbreviated as A, T, C, and G.  The order of these bases along the DNA molecule is the genetic code.

\begin{marginnote}
\challenge{The human genome, the complete set of genetic instructions in a human, contains about 3 billion base pairs!  If you stretched out all the DNA in just one of your cells, it would be about 2 metres long!}
\end{marginnote}

\begin{keyconcept}{DNA Structure}
DNA is a double helix made of two strands. Each strand is a chain of nucleotides.  Nucleotides are made of a sugar, a phosphate group, and a base (Adenine, Thymine, Cytosine, or Guanine).  A always pairs with T, and C always pairs with G. This pairing is crucial for DNA replication and protein synthesis.
\end{keyconcept}

\begin{figure}[htbp]
\centering
\includegraphics[width=0.7\textwidth]{placeholder_dna_structure.jpg}
\caption{A simplified diagram of DNA structure, showing the double helix and base pairing.}
\end{figure}

\subsection{Chromosomes: Organising the DNA}

To keep this incredibly long DNA molecule organised and manageable within the nucleus, it's packaged into structures called \keyword{chromosomes}. Think of chromosomes like chapters in our instruction manual.  In most cells, DNA is organised into multiple chromosomes.  Human cells, for example, typically have 46 chromosomes arranged in 23 pairs.

These pairs are called \keyword{homologous chromosomes}. You inherit one chromosome of each pair from your mother and the other from your father. Homologous chromosomes carry genes for the same traits, but they might have slightly different versions of those genes. These different versions are called \keyword{alleles}.

For example, a gene for eye colour might be found on a particular chromosome.  One allele of this gene might code for blue eyes, while another allele might code for brown eyes.  You inherit two alleles for each gene, one from each parent.

\begin{marginnote}
\historylink{The discovery of the structure of DNA by James Watson and Francis Crick in 1953, with crucial contributions from Rosalind Franklin and Maurice Wilkins, was a monumental achievement in science. It unlocked the secrets of heredity and paved the way for modern genetics.}
\end{marginnote}

\begin{stopandthink}
If you have 23 pairs of chromosomes in each of your body cells, how many chromosomes did you inherit from each of your parents?
\end{stopandthink}

\begin{tieredquestions}{Section 2.1 & 2.2}

\begin{enumerate}
    \item \textbf{Basic:} What is DNA and where is it found in a cell?
    \item \textbf{Basic:} What is a gene and what does it code for?
    \item \textbf{Intermediate:} Describe the structure of DNA. What are the four bases and how do they pair up?
    \item \textbf{Intermediate:} What are chromosomes and why are they important for organising DNA?
    \item \textbf{Advanced:} Explain the relationship between DNA, genes, and chromosomes using an analogy (e.g., a library, a cookbook, etc.).
    \item \textbf{Advanced:}  If a gene has two alleles, one for tallness (T) and one for shortness (t), and an individual inherits one of each, what are the possible combinations of alleles they could inherit from their parents?
\end{enumerate}

\end{tieredquestions}


\section{From Genes to Traits: Genotype and Phenotype}

We've learned that genes carry the instructions for making proteins, and these proteins play a crucial role in determining our traits.  But how exactly do genes influence what we look like and how we function?  Let's explore the concepts of \keyword{genotype} and \keyword{phenotype}.

\subsection{Genotype: Your Genetic Code}

Your \keyword{genotype} refers to the specific combination of alleles you have for a particular gene or set of genes. It's your genetic makeup, written in the language of DNA.  For example, if we are considering the gene for pea plant height, and 'T' represents the allele for tall and 't' represents the allele for short, then possible genotypes could be TT, Tt, or tt.

Remember, you inherit two alleles for each gene, one from each parent.  If the two alleles are the same (e.g., TT or tt), you are said to be \keyword{homozygous} for that gene. If the two alleles are different (e.g., Tt), you are \keyword{heterozygous}.

\subsection{Phenotype: Expressing Your Genes}

Your \keyword{phenotype} is the observable characteristic or trait that results from your genotype. It's what you actually see – like your eye colour, height, or blood type.  The phenotype is not just determined by your genotype alone; it's also influenced by environmental factors.

\begin{marginnote}
\challenge{Identical twins have the same genotype, but they are not always perfectly identical in phenotype.  Environmental factors during development and throughout life can lead to slight differences.}
\end{marginnote}

For example, let's consider pea plant height again.  If the 'T' allele (tall) is \keyword{dominant} over the 't' allele (short), this means that if a plant has at least one 'T' allele (genotypes TT or Tt), it will be tall.  Only plants with the genotype 'tt' (homozygous recessive) will be short.  In this case, the phenotypes are 'tall' and 'short', and they are determined by the interaction of the alleles in the genotype.

\begin{example}
Let's say brown eyes (B) are dominant over blue eyes (b).

\begin{itemize}
    \item Genotype BB: Phenotype Brown eyes (homozygous dominant)
    \item Genotype Bb: Phenotype Brown eyes (heterozygous)
    \item Genotype bb: Phenotype Blue eyes (homozygous recessive)
\end{itemize}
\end{example}

\begin{stopandthink}
If you know that black fur (B) is dominant over brown fur (b) in mice, what phenotype would a mouse with the genotype Bb have? What phenotype would a mouse with genotype bb have?
\end{stopandthink}

\begin{tieredquestions}{Section 3}

\begin{enumerate}
    \item \textbf{Basic:} What is the difference between genotype and phenotype?
    \item \textbf{Basic:} What does it mean to be homozygous and heterozygous for a gene?
    \item \textbf{Intermediate:} Explain how dominant and recessive alleles determine phenotype in a simple Mendelian trait.
    \item \textbf{Intermediate:} Give an example of how the environment can influence phenotype.
    \item \textbf{Advanced:} Consider a trait with incomplete dominance, where the heterozygous phenotype is intermediate between the two homozygous phenotypes (e.g., flower colour: red (RR), white (rr), pink (Rr)). Describe the phenotypes for each genotype.
    \item \textbf{Advanced:}  Explain why understanding the difference between genotype and phenotype is important in fields like medicine and agriculture.
\end{enumerate}

\end{tieredquestions}


\section{Passing on the Genes: Inheritance}

Heredity is all about how traits are passed from parents to offspring.  To understand this process, we need to delve into \keyword{inheritance}, the mechanisms by which genes are transmitted across generations.  The key to inheritance lies in \keyword{cell division}, specifically a special type of cell division called \keyword{meiosis}.

\subsection{Mitosis and Meiosis: Two Types of Cell Division}

Cells divide for two main reasons: growth and repair.  \keyword{Mitosis} is the type of cell division used for these purposes. In mitosis, a parent cell divides into two identical daughter cells. Each daughter cell has the same number of chromosomes and the same genetic information as the parent cell. Mitosis is essential for development, tissue repair, and asexual reproduction in some organisms.

\begin{marginnote}
\challenge{Some organisms, like bacteria, reproduce entirely through mitosis (binary fission). This results in offspring that are genetically identical clones of the parent.}
\end{marginnote}

\begin{figure}[htbp]
\centering
\includegraphics[width=0.8\textwidth]{placeholder_mitosis_meiosis.jpg}
\caption{Simplified diagrams comparing Mitosis and Meiosis.}
\end{figure}

However, sexual reproduction, which involves the fusion of sperm and egg cells, requires a different type of cell division: \keyword{meiosis}.  Meiosis is a special type of cell division that produces \keyword{gametes} (sperm and egg cells) with only half the number of chromosomes as the parent cell.  This reduction in chromosome number is crucial for maintaining the correct chromosome number in offspring.

\subsection{Meiosis: Creating Genetic Variation}

During meiosis, something remarkable happens that increases genetic variation: \keyword{crossing over}.  Homologous chromosomes exchange segments of DNA, shuffling the alleles. This means that the gametes produced by meiosis are genetically different from each other and from the parent cell.

When a sperm cell fertilises an egg cell, the resulting \keyword{zygote} (fertilised egg) receives half of its chromosomes from each parent.  This combination of genetic material from two parents creates offspring that are genetically unique.  Meiosis and sexual reproduction are the primary sources of genetic variation in populations.

\begin{marginnote}
\historylink{Gregor Mendel, often called the "father of genetics," conducted his groundbreaking experiments on pea plants in the 1860s. He meticulously studied inheritance patterns and laid the foundation for our understanding of genes and alleles, long before DNA and chromosomes were discovered.}
\end{marginnote}

\begin{stopandthink}
Why is it important that gametes (sperm and egg cells) have half the number of chromosomes as body cells? What would happen if they had the same number?
\end{stopandthink}

\begin{tieredquestions}{Section 4}

\begin{enumerate}
    \item \textbf{Basic:} What are the two main types of cell division?
    \item \textbf{Basic:} What is the purpose of mitosis? What is the purpose of meiosis?
    \item \textbf{Intermediate:} Explain how meiosis reduces the chromosome number in gametes.
    \item \textbf{Intermediate:} Describe the process of crossing over and explain its importance for genetic variation.
    \item \textbf{Advanced:} Compare and contrast mitosis and meiosis, highlighting their similarities and differences in terms of chromosome number, daughter cells, and biological function.
    \item \textbf{Advanced:} Explain how meiosis contributes to the genetic diversity of a population, which is essential for evolution.
\end{enumerate}

\end{tieredquestions}


\section{Patterns of Inheritance: Mendel's Laws}

Gregor Mendel's work with pea plants revealed fundamental patterns of inheritance. He formulated laws that describe how traits are passed from one generation to the next.  We will focus on some basic principles of Mendelian inheritance.

\subsection{Dominance and Segregation}

Mendel's experiments led him to the concept of \keyword{dominant} and \keyword{recessive} alleles, which we have already touched upon.  He also proposed the \keyword{law of segregation}. This law states that during gamete formation, the two alleles for each gene separate, so that each gamete carries only one allele for each gene.  During fertilisation, when sperm and egg fuse, the offspring receives one allele from each parent, restoring the pair of alleles.

\begin{marginnote}
\mathlink{Punnett squares are a useful tool for visualising Mendelian inheritance. They help predict the possible genotypes and phenotypes of offspring from a cross between two parents.}
\end{marginnote}

\subsection{Monohybrid Crosses and Punnett Squares}

A \keyword{monohybrid cross} involves studying the inheritance of a single trait.  We can use a \keyword{Punnett square} to predict the outcomes of such crosses.  A Punnett square is a diagram that shows all possible combinations of alleles from the parents.

Let's revisit our pea plant height example (T=tall, t=short, T is dominant).  Imagine we cross two heterozygous tall plants (Tt x Tt).  Here's how we can use a Punnett square:

\begin{center}
\begin{tabular}{c|cc}
 & T & t \\
\hline
T & TT & Tt \\
t & Tt & tt \\
\end{tabular}
\end{center}

From the Punnett square, we can see the possible genotypes of the offspring are TT, Tt, and tt in the ratio 1:2:1.  The phenotypes are tall (TT, Tt) and short (tt) in the ratio 3:1.  This 3:1 phenotypic ratio is a classic result of a monohybrid cross with dominant and recessive alleles.

\begin{investigation}{Monohybrid Cross Simulation}
\textbf{Materials:} Two coins per pair of students.

\textbf{Procedure:}
\begin{enumerate}
    \item  Assign heads on a coin to represent the dominant allele (e.g., T) and tails to represent the recessive allele (e.g., t).
    \item Each student in a pair represents a parent with a heterozygous genotype (Tt).
    \item Each student flips their coin to simulate the random segregation of alleles during gamete formation.
    \item Combine the results of the two coin flips to represent the genotype of the offspring. Record the genotype (e.g., TT, Tt, tt) and phenotype (assuming T is dominant).
    \item Repeat steps 3-4 for 20 trials.
    \item Calculate the observed genotype and phenotype ratios from your data.
    \item Compare your observed ratios to the predicted ratios from the Punnett square (1:2:1 for genotype, 3:1 for phenotype).
\end{enumerate}

\textbf{Analysis:}
\begin{itemize}
    \item How closely do your observed ratios match the predicted ratios?
    \item What factors might cause deviations from the predicted ratios in real-world scenarios?
    \item How does this activity demonstrate the principles of segregation and random fertilisation?
\end{itemize}
\end{investigation}

\begin{stopandthink}
If you cross a homozygous tall pea plant (TT) with a homozygous short pea plant (tt), what will be the genotype and phenotype of all the offspring? Use a Punnett square to help you.
\end{stopandthink}


\begin{tieredquestions}{Section 5}

\begin{enumerate}
    \item \textbf{Basic:} State Mendel's law of segregation in your own words.
    \item \textbf{Basic:} What is a Punnett square used for?
    \item \textbf{Intermediate:} Explain how to perform a monohybrid cross using a Punnett square.
    \item \textbf{Intermediate:} In pea plants, round seeds (R) are dominant over wrinkled seeds (r). If you cross a heterozygous round-seeded plant (Rr) with a homozygous wrinkled-seeded plant (rr), what are the predicted genotype and phenotype ratios of the offspring?
    \item \textbf{Advanced:} Explain why the phenotypic ratio in a monohybrid cross of heterozygotes is typically 3:1 when dealing with complete dominance.
    \item \textbf{Advanced:}  Consider a dihybrid cross (involving two traits). For example, pea plants with yellow and round seeds (YYRR) are crossed with plants with green and wrinkled seeds (yyrr).  Describe how you would use a Punnett square to analyse this cross and predict the offspring ratios, assuming independent assortment of genes. (Hint: You may need a larger Punnett square!)
\end{enumerate}

\end{tieredquestions}


\section{Mutations: Changes in the Genetic Code}

Sometimes, the genetic code can change. These changes are called \keyword{mutations}. Mutations are alterations in the DNA sequence.  They can occur spontaneously during DNA replication or be caused by external factors like radiation or certain chemicals.

\subsection{Types of Mutations}

Mutations can range from changes in a single DNA base pair to large-scale alterations in chromosome structure.  Some common types of mutations include:

\begin{itemize}
    \item \textbf{Point mutations:} Changes in a single base pair. These can be \keyword{substitutions} (one base replaced by another), \keyword{insertions} (adding a base), or \keyword{deletions} (removing a base).
    \item \textbf{Frameshift mutations:} Insertions or deletions that shift the reading frame of the genetic code, often leading to drastically altered proteins.
    \item \textbf{Chromosome mutations:}  Large-scale changes affecting entire chromosomes or segments of chromosomes. These can include \keyword{deletions}, \keyword{duplications}, \keyword{inversions} (segments flipping orientation), and \keyword{translocations} (segments moving to a different chromosome).
\end{itemize}

\begin{marginnote}
\challenge{Not all mutations are harmful. Some mutations are neutral, having no significant effect on the organism.  Rarely, mutations can even be beneficial, providing an advantage in a particular environment.}
\end{marginnote}

\begin{figure}[htbp]
\centering
\includegraphics[width=0.7\textwidth]{placeholder_mutation_types.jpg}
\caption{Illustrations of different types of mutations: point mutations and chromosome mutations.}
\end{figure}


\subsection{Impact of Mutations}

The impact of a mutation depends on several factors, including the type of mutation, where it occurs in the DNA, and the function of the affected gene.

\begin{itemize}
    \item \textbf{Harmful mutations:} Some mutations can disrupt gene function, leading to genetic disorders or diseases. For example, mutations in genes involved in cell cycle control can lead to cancer.
    \item \textbf{Neutral mutations:} Many mutations have no noticeable effect on the phenotype. These might occur in non-coding regions of DNA or result in changes that don't significantly alter protein function.
    \item \textbf{Beneficial mutations:} In rare cases, mutations can be beneficial, providing an organism with a new or improved trait that enhances its survival or reproduction in a particular environment. These beneficial mutations are the raw material for evolution.
\end{itemize}

\begin{stopandthink}
Can you think of an example of a mutation that might be beneficial in a particular environment? (Hint: Consider antibiotic resistance in bacteria.)
\end{stopandthink}


\begin{tieredquestions}{Section 6}

\begin{enumerate}
    \item \textbf{Basic:} What is a mutation?
    \item \textbf{Basic:} Name two types of point mutations.
    \item \textbf{Intermediate:} Explain how a frameshift mutation can have a more significant impact than a substitution mutation.
    \item \textbf{Intermediate:} Describe three possible impacts a mutation can have on an organism's phenotype.
    \item \textbf{Advanced:} Discuss the role of mutations as the source of genetic variation in evolution.
    \item \textbf{Advanced:}  Research and describe a specific human genetic disorder caused by a mutation. Explain the type of mutation involved and its effects on the individual.
\end{enumerate}

\end{tieredquestions}


\section{Evolution by Natural Selection: Darwin's Big Idea}

\keyword{Evolution} is the process of change in the characteristics of a species over many generations.  The driving force behind evolution is \keyword{natural selection}, a concept famously proposed by Charles Darwin.

\subsection{Darwin's Theory of Natural Selection}

Darwin observed that individuals within a population show \keyword{variation} in their traits. Some of this variation is heritable, meaning it can be passed on to offspring. He also noticed that populations tend to produce more offspring than the environment can support, leading to competition for resources.

Darwin's theory of natural selection can be summarised as follows:

\begin{enumerate}
    \item \textbf{Variation:} Individuals within a population vary in their traits.
    \item \textbf{Inheritance:} Some of these traits are heritable.
    \item \textbf{Differential Survival and Reproduction:** In a competitive environment, individuals with certain advantageous traits are more likely to survive and reproduce than individuals with less favourable traits.  This is often referred to as "survival of the fittest."
    \item \textbf{Adaptation:** Over time, the frequency of advantageous traits increases in the population, leading to \keyword{adaptation} – the process by which populations become better suited to their environment.
\end{enumerate}

\begin{marginnote}
\historylink{Charles Darwin published his groundbreaking book "On the Origin of Species" in 1859, outlining his theory of evolution by natural selection.  His ideas revolutionised biology and continue to be the cornerstone of modern evolutionary theory.}
\end{marginnote}

\begin{figure}[htbp]
\centering
\includegraphics[width=0.8\textwidth]{placeholder_darwin_finches.jpg}
\caption{Darwin's finches on the Galapagos Islands, showing variation in beak shape adapted to different food sources. An example of natural selection in action.}
\end{figure}

\subsection{Adaptation and Speciation}

Adaptations can be physical traits (e.g., camouflage colouration, sharp claws) or behavioural traits (e.g., migration patterns, cooperative hunting).  Natural selection acts on existing variation within a population. It does not create new traits; it selects for traits that are already present and provide an advantage in the current environment.

Over very long periods of time, natural selection can lead to significant changes in populations and even the formation of new species, a process called \keyword{speciation}.  Speciation often occurs when populations become isolated from each other, preventing gene flow.  Over time, different populations may adapt to different environments, accumulating genetic differences that eventually lead to reproductive isolation – they can no longer interbreed to produce fertile offspring.

\begin{marginnote}
\challenge{Evolution is not always a slow, gradual process.  In some cases, especially in response to rapid environmental changes, evolution can occur relatively quickly (in ecological timescales), a phenomenon known as rapid evolution.}
\end{marginnote}

\begin{stopandthink}
Think of an animal or plant you are familiar with.  Can you identify some adaptations that help it survive in its environment? How might natural selection have led to these adaptations?
\end{stopandthink}


\begin{tieredquestions}{Section 7}

\begin{enumerate}
    \item \textbf{Basic:} What is evolution?
    \item \textbf{Basic:} Briefly explain Darwin's theory of natural selection.
    \item \textbf{Intermediate:} Describe the four key principles of natural selection: variation, inheritance, differential survival and reproduction, and adaptation.
    \item \textbf{Intermediate:} Explain how natural selection leads to adaptation. Give an example of an adaptation and how it benefits an organism.
    \item \textbf{Advanced:} Discuss the role of genetic variation in natural selection and evolution. How do mutations contribute to this variation?
    \item \textbf{Advanced:} Explain the process of speciation. What factors can contribute to the formation of new species?
\end{enumerate}

\end{tieredquestions}


\section{Evidence for Evolution: A Mountain of Proof}

Evolution is not just a theory; it is supported by a vast amount of evidence from different fields of biology.  Let's explore some of the key lines of evidence that demonstrate the reality of evolution.

\subsection{The Fossil Record}

Fossils are the preserved remains or traces of ancient organisms. The \keyword{fossil record} provides a historical sequence of life on Earth. By studying fossils in different layers of rock (strata), scientists can see how life forms have changed over time.  The fossil record shows a progression from simpler to more complex organisms over geological time scales.  It also reveals \keyword{transitional fossils}, which exhibit features of both ancestral and descendant groups, providing evidence for evolutionary transitions.

\begin{marginnote}
\historylink{The discovery of Archaeopteryx, a fossil with features of both reptiles (like teeth and a bony tail) and birds (like feathers and wings), was a significant piece of early evidence supporting the evolutionary link between reptiles and birds.}
\end{marginnote}

\begin{figure}[htbp]
\centering
\includegraphics[width=0.7\textwidth]{placeholder_fossil_record.jpg}
\caption{Diagram illustrating the fossil record and how it shows changes in life forms over time.}
\end{figure}

\subsection{Comparative Anatomy and Embryology}

\keyword{Comparative anatomy} studies the similarities and differences in the anatomical structures of different species.  \keyword{Homologous structures} are structures in different species that have a common evolutionary origin, even if they have different functions. For example, the bones in the forelimbs of humans, bats, whales, and horses are homologous – they have the same basic skeletal structure, modified for different purposes (grasping, flying, swimming, running). Homologous structures are evidence of common ancestry.

\keyword{Embryology} is the study of the development of embryos.  Comparing the embryonic development of different species reveals striking similarities in early stages, even if the adult forms are very different.  These similarities suggest a shared evolutionary history.

\subsection{Molecular Evidence: DNA and Genes}

Perhaps the most powerful evidence for evolution comes from the study of molecules, particularly DNA and genes.  All living organisms share a universal genetic code (DNA and RNA).  By comparing DNA sequences and gene structures across different species, scientists can determine how closely related they are evolutionarily.  The more similar the DNA sequences, the more closely related the species are.  This molecular evidence strongly supports the idea of common ancestry and evolutionary relationships between all life forms.

\begin{marginnote}
\challenge{The field of phylogenetics uses molecular data and anatomical data to construct "evolutionary trees" or phylogenies, which depict the evolutionary relationships between different species or groups of organisms.}
\end{marginnote}


\begin{stopandthink}
Think about the wings of a bird and the wings of a butterfly. Are these homologous structures? Why or why not? What kind of structures are they? (Hint: Consider their evolutionary origin and function.)
\end{stopandthink}


\begin{tieredquestions}{Section 8}

\begin{enumerate}
    \item \textbf{Basic:} What is a fossil and how does the fossil record provide evidence for evolution?
    \item \textbf{Basic:} What are homologous structures? Give an example.
    \item \textbf{Intermediate:} Explain how comparative anatomy and embryology support the theory of evolution.
    \item \textbf{Intermediate:} How does molecular evidence, such as DNA comparisons, provide evidence for evolution?
    \item \textbf{Advanced:} Discuss the strengths and limitations of different lines of evidence for evolution (fossil record, comparative anatomy, molecular evidence).
    \item \textbf{Advanced:}  Explain how the concept of common ancestry is supported by multiple lines of evidence from different fields of biology.
\end{enumerate}

\end{tieredquestions}


\section{Human Impact on Evolution}

Humans are not just observers of evolution; we are also agents of evolutionary change.  Our activities are having a significant impact on the evolution of other species, and even on our own evolution.

\subsection{Artificial Selection and Selective Breeding}

Humans have been intentionally influencing evolution for thousands of years through \keyword{artificial selection}, also known as \keyword{selective breeding}.  In artificial selection, humans choose individuals with desirable traits to breed, favouring the inheritance of those traits in future generations.  This is how we have domesticated plants and animals, creating breeds of livestock, crops, and pets that are very different from their wild ancestors.

\begin{marginnote}
\challenge{The rapid evolution of antibiotic resistance in bacteria is a serious threat to human health.  It is a direct consequence of natural selection acting on bacterial populations in response to the widespread use of antibiotics.}
\end{marginnote}

\begin{figure}[htbp]
\centering
\includegraphics[width=0.7\textwidth]{placeholder_selective_breeding.jpg}
\caption{Examples of selective breeding: wild mustard has been selectively bred into various vegetables like broccoli, cabbage, and kale.}
\end{figure}

\subsection{Genetic Engineering and Biotechnology}

Modern biotechnology allows us to directly manipulate genes through \keyword{genetic engineering}.  We can insert genes from one species into another, modify existing genes, or even create entirely new genes.  Genetic engineering has enormous potential in medicine, agriculture, and industry.  However, it also raises ethical and environmental concerns that need careful consideration.

\subsection{Conservation and Evolutionary Biology}

Human activities, such as habitat destruction, pollution, and climate change, are causing rapid environmental changes that are putting many species at risk of extinction.  Understanding evolutionary principles is crucial for \keyword{conservation biology}.  By understanding how species adapt and evolve, we can develop more effective strategies for protecting biodiversity and managing ecosystems in a changing world.  Conservation efforts often focus on maintaining genetic diversity within populations, as genetic variation is essential for adaptation and long-term survival.

\begin{stopandthink}
Think about the different ways humans are influencing the evolution of other species, both intentionally (like selective breeding) and unintentionally (like causing antibiotic resistance). What are some potential benefits and risks of these human-induced evolutionary changes?
\end{stopandthink}


\begin{tieredquestions}{Section 9}

\begin{enumerate}
    \item \textbf{Basic:} What is artificial selection? Give an example.
    \item \textbf{Basic:} What is genetic engineering?
    \item \textbf{Intermediate:} Compare and contrast artificial selection and natural selection.
    \item \textbf{Intermediate:} Discuss some ethical concerns associated with genetic engineering.
    \item \textbf{Advanced:} Explain how human activities are contributing to the rapid evolution of antibiotic resistance in bacteria. What are the implications of this for human health?
    \item \textbf{Advanced:}  Discuss the role of evolutionary biology in conservation efforts. How can understanding evolution help us protect biodiversity?
\end{enumerate}

\end{tieredquestions}


\section{Chapter Summary: The Dance of Genes and Time}

In this chapter, we have explored the fascinating fields of genetics and evolution. We have learned that:

\begin{itemize}
    \item \textbf{DNA and Genes:} DNA is the molecule of heredity, organised into genes which code for proteins.
    \item \textbf{Chromosomes and Inheritance:} Genes are carried on chromosomes, which are inherited from parents. Meiosis is crucial for creating genetic variation during sexual reproduction.
    \item \textbf{Genotype and Phenotype:} Genotype is the genetic makeup, phenotype is the observable trait, influenced by genotype and environment.
    \item \textbf{Mendelian Inheritance:} Mendel's laws describe basic patterns of inheritance, including dominance, segregation, and monohybrid crosses.
    \item \textbf{Mutations and Variation:} Mutations are changes in DNA, the source of new genetic variation.
    \item \textbf{Natural Selection and Adaptation:} Natural selection acts on variation, leading to adaptation and evolution.
    \item \textbf{Evidence for Evolution:}  Fossils, comparative anatomy, embryology, and molecular evidence all support the theory of evolution.
    \item \textbf{Human Impact:} Humans influence evolution through selective breeding, genetic engineering, and environmental changes.
\end{itemize}

Genetics and evolution are fundamental concepts in biology.  Understanding them is crucial for comprehending the diversity of life, the processes that shape it, and our own place in the grand tapestry of evolution.  As we continue to explore these fields, we will undoubtedly uncover even more amazing secrets about the dance of genes and time that has shaped life on Earth.

\end{document}
```