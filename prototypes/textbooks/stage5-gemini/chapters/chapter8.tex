```latex
\chapter{Applied Chemistry and Environmental Chemistry}

\marginnote{This chapter explores how chemistry shapes our world, from the products we use daily to the environment we live in.}
\section{Introduction: Chemistry in Action}

Chemistry is not just confined to laboratories; it is a dynamic force shaping the world around us. \keyword{Applied chemistry} takes the principles of chemistry and puts them to practical use, developing new materials, processes, and products that improve our lives.  From the medicines we take to the fuels that power our vehicles, applied chemistry is at work.

\marginnote{\historylink{Ancient civilisations used applied chemistry for metallurgy, dyeing, and making pottery.}}
\marginnote{\keyword{Environmental chemistry} studies the chemical processes in the environment and the impact of human activities on these processes.}
Alongside this, \keyword{environmental chemistry} is crucial for understanding and protecting our planet.  It investigates the chemical processes occurring in the environment, examining how pollutants are formed, transported, and transformed, and how they affect living organisms and ecosystems.  As we face increasing environmental challenges, the role of environmental chemistry in finding sustainable solutions becomes ever more important.

In this chapter, we will delve into both applied and environmental chemistry, exploring how chemical knowledge is used to create innovative technologies and address critical environmental issues. We will examine real-world examples, conduct investigations, and consider the ethical responsibilities that come with applying chemistry in our world.

\section{Applied Chemistry: Shaping Our World with Molecules}

Applied chemistry focuses on using chemical knowledge to solve practical problems and create useful products.  It bridges the gap between fundamental chemical research and its real-world applications. Let's explore some key areas of applied chemistry.

\subsection{Materials Chemistry: Building Blocks of Innovation}

\marginnote{\keyword{Materials chemistry} is the design and synthesis of new materials with specific properties.}
\marginnote{\challenge{Research the development of new materials for renewable energy technologies, such as solar cells or batteries.}}
Materials chemistry is a fascinating field that focuses on the design and creation of new materials with tailored properties.  These materials can range from advanced polymers and composites to ceramics and nanomaterials, each designed for specific applications.

\begin{keyconcept}{Polymers: Long Chains, Endless Possibilities}
\keyword{Polymers} are large molecules made up of repeating units called \keyword{monomers}.  Think of them as long chains, where each link in the chain is a monomer.  Polymers are incredibly versatile and are used in countless applications, from plastics and fabrics to adhesives and coatings.

There are two main types of polymers:

\begin{itemize}
    \item \textbf{Natural Polymers:} These are produced by living organisms. Examples include:
    \begin{itemize}
        \item \textbf{Cellulose:}  The main structural component of plant cell walls, found in wood, cotton, and paper.
        \item \textbf{Starch:}  A carbohydrate used by plants for energy storage, found in potatoes, rice, and corn.
        \item \textbf{Proteins:}  Essential for life, performing a vast array of functions in living organisms, from enzymes to structural components like hair and muscles.
        \item \textbf{Natural Rubber:}  A polymer of isoprene, harvested from rubber trees and known for its elasticity.
    \end{itemize}
    \item \textbf{Synthetic Polymers:} These are created by chemists in laboratories and industrial settings. Examples include:
    \begin{itemize}
        \item \textbf{Polyethylene (PE):}  Used in plastic bags, bottles, and films. It is formed by the addition polymerisation of ethene.
        \item \textbf{Polyvinyl Chloride (PVC):}  Used in pipes, window frames, and flooring.  Formed from vinyl chloride monomers.
        \item \textbf{Polystyrene (PS):}  Used in packaging, insulation, and disposable cups. Made from styrene monomers.
        \item \textbf{Nylon:}  A polyamide used in textiles, ropes, and gears.  Formed by condensation polymerisation.
    \end{itemize}
\end{itemize}

The properties of a polymer depend on the type of monomers it is made from and how these monomers are arranged and linked together. For example, polyethylene can be low-density (LDPE), which is flexible and used for plastic bags, or high-density (HDPE), which is stronger and used for milk bottles.
\end{keyconcept}

\begin{stopandthink}
Think about the plastic items around you. Can you identify which polymers they might be made from based on their properties (e.g., flexible, rigid, transparent)?
\end{stopandthink}

\begin{keyconcept}{Composites: Strength in Combination}
\keyword{Composites} are materials made from two or more distinct components that, when combined, produce a material with enhanced properties compared to the individual components alone.  Typically, a composite consists of a \keyword{matrix} material, which surrounds and binds together a \keyword{reinforcement} material.

Common examples of composites include:

\begin{itemize}
    \item \textbf{Fibre-reinforced plastics:}  These use strong fibres, such as carbon fibres or glass fibres, embedded in a polymer matrix (e.g., epoxy resin).  They are lightweight and strong, used in aircraft, sports equipment, and car parts.
    \item \textbf{Concrete:} A composite material made from cement (matrix), aggregate (sand and gravel - reinforcement), and water.  It is strong in compression and widely used in construction.
    \item \textbf{Wood:}  A natural composite material. Cellulose fibres (reinforcement) are embedded in a lignin matrix. This gives wood both strength and flexibility.
\end{itemize}

The key advantage of composites is that we can combine the desirable properties of different materials. For instance, carbon fibre composites combine the high strength and stiffness of carbon fibres with the lightweight and moldability of polymers, resulting in materials that are stronger and lighter than steel for the same volume.
\end{keyconcept}

\begin{investigation}{Investigating Polymer Properties}
\textbf{Aim:} To compare the properties of different types of polymers.

\textbf{Materials:} Samples of different polymers (e.g., polyethylene bags, PVC pipe, polystyrene foam, nylon fabric), beakers, water, heat source (e.g., hot plate or kettle), tongs, safety glasses.

\textbf{Procedure:}
\begin{enumerate}
    \item \textbf{Observation:** Examine each polymer sample. Note its appearance, flexibility, and texture.
    \item \textbf{Density:** Place small pieces of each polymer in beakers of water. Observe whether they float or sink.  (Density less than water floats, density greater than water sinks.)
    \item \textbf{Heat Resistance:** \textit{(Caution: Adult supervision required for this step. Perform in a well-ventilated area.)} Carefully heat a small piece of each polymer using a heat source and tongs. Observe how each polymer responds to heat. Does it melt, soften, burn, or remain unchanged?
    \item \textbf{Record your observations} in a table, noting the properties of each polymer (appearance, flexibility, density relative to water, response to heat).
\end{enumerate}

\textbf{Discussion:**
\begin{enumerate}
    \item How do the properties of the different polymers compare?
    \item Can you relate the properties you observed to the uses of these polymers in everyday life?
    \item What are some limitations of this investigation? How could you improve it to get more quantitative data?
\end{enumerate}
\end{investigation}

\subsubsection{Tiered Questions}
\begin{tieredquestions}{Basic}
\begin{enumerate}
    \item What is a polymer? Give two examples of natural polymers and two examples of synthetic polymers.
    \item Describe the difference between a monomer and a polymer.
    \item What is a composite material? Give an example of a composite and identify its matrix and reinforcement components.
\end{enumerate}
\end{tieredquestions}

\begin{tieredquestions}{Intermediate}
\begin{enumerate}
    \item Explain how the properties of polymers can be changed by altering their monomer composition or structure. Give an example to support your answer.
    \item Compare and contrast the advantages and disadvantages of using natural polymers versus synthetic polymers.
    \item Explain why composite materials are often preferred over single-component materials in certain applications.
\end{enumerate}
\end{tieredquestions}

\begin{tieredquestions}{Advanced}
\begin{enumerate}
    \item Research and describe a specific example of a smart material or advanced composite material and its applications. Explain the chemical principles behind its unique properties.
    \item Discuss the environmental challenges associated with polymer production and disposal. What are some strategies being developed to address these challenges (e.g., biodegradable polymers, polymer recycling)?
    \item Design an experiment to test the tensile strength or flexibility of different composite materials you could create using readily available materials. Outline your method and identify variables you would need to control.
\end{enumerate}
\end{tieredquestions}

\subsection{Industrial Chemistry: Large-Scale Synthesis}

\marginnote{\keyword{Industrial chemistry} involves the large-scale production of chemicals and materials.}
\marginnote{\mathlink{Stoichiometry is crucial in industrial chemistry to ensure efficient and cost-effective production.}}
Industrial chemistry focuses on the large-scale production of chemicals and materials that are essential for various industries, including agriculture, manufacturing, and pharmaceuticals.  Industrial processes must be efficient, cost-effective, and safe, often involving complex chemical reactions and engineering.

\begin{keyconcept}{The Haber Process: Feeding the World}
The \keyword{Haber process} is a crucial industrial process used to synthesise ammonia (\ce{NH3}) from nitrogen (\ce{N2}) and hydrogen (\ce{H2}). Ammonia is a vital ingredient in fertilisers, which are essential for modern agriculture and food production.

The reaction is:
\[\ce{N2(g) + 3H2(g) <=> 2NH3(g)}\]

This reaction is reversible and exothermic (releases heat). To achieve a commercially viable yield of ammonia, the Haber process uses specific conditions:

\begin{itemize}
    \item \textbf{Temperature:} Moderately high temperature (around 400-500 °C).  While lower temperatures favour the formation of ammonia (exothermic reaction), the reaction rate is too slow. Higher temperatures increase the rate but reduce the equilibrium yield. A compromise temperature is used.
    \item \textbf{Pressure:} High pressure (around 200 atmospheres).  High pressure favours the side with fewer moles of gas (Le Chatelier's principle). In this reaction, there are 4 moles of gas on the reactant side and 2 moles on the product side.
    \item \textbf{Catalyst:}  Iron catalyst.  The catalyst speeds up the rate of reaction without being consumed, allowing the reaction to reach equilibrium faster.
\end{itemize}

The Haber process has had a profound impact on global food production, allowing for the large-scale cultivation of crops and supporting a rapidly growing population. However, it is also energy-intensive and contributes to greenhouse gas emissions.  Current research is focused on developing more sustainable methods for ammonia production.
\end{keyconcept}

\begin{stopandthink}
Why is it important to consider both the rate and equilibrium yield when optimising industrial processes like the Haber process?
\end{stopandthink}

\begin{keyconcept}{Electrolysis: Splitting Compounds with Electricity}
\keyword{Electrolysis} is a process that uses electrical energy to drive non-spontaneous chemical reactions.  It is commonly used in industrial chemistry for the extraction of metals from their ores and the production of various chemicals.

\textbf{Electrolysis of Molten Sodium Chloride (\ce{NaCl}):}

Sodium chloride (table salt) can be electrolysed when molten (melted) to produce sodium metal and chlorine gas.

\begin{itemize}
    \item \textbf{Electrolyte:} Molten sodium chloride (\ce{NaCl(l)}).  Ions must be mobile to conduct electricity.
    \item \textbf{Electrodes:} Inert electrodes (e.g., graphite or platinum) that do not react with the electrolyte or products.
    \item \textbf{Process at the Cathode (Negative Electrode):} Sodium ions (\ce{Na+}) are reduced (gain electrons) to form sodium metal (\ce{Na}).
    \[\ce{Na+ + e- -> Na(l)}\]
    \item \textbf{Process at the Anode (Positive Electrode):} Chloride ions (\ce{Cl-}) are oxidised (lose electrons) to form chlorine gas (\ce{Cl2}).
    \[\ce{2Cl- -> Cl2(g) + 2e-}\]
    \item \textbf{Overall Reaction:}
    \[\ce{2NaCl(l) -> 2Na(l) + Cl2(g)}\]
\end{itemize}

Sodium metal is highly reactive and used in various chemical processes. Chlorine gas is used in water treatment, the production of plastics (PVC), and disinfectants.

Electrolysis is also used to extract other reactive metals like aluminium from aluminium oxide (\ce{Al2O3}) and to produce hydrogen gas from water.  It is an important tool in industrial chemistry, but it is also energy-intensive, and efforts are being made to improve its efficiency and sustainability.
\end{keyconcept}

\begin{investigation}{Electrolysis of Copper(II) Chloride Solution}
\textbf{Aim:} To investigate the electrolysis of copper(II) chloride solution.

\textbf{Materials:} Copper(II) chloride solution, two graphite electrodes, DC power supply, connecting wires, crocodile clips, beaker, safety glasses.

\textbf{Procedure:}
\begin{enumerate}
    \item Set up the electrolysis apparatus as shown in \texttt{[PLACEHOLDER FIGURE: Diagram of electrolysis setup]}.
    \item Pour copper(II) chloride solution into the beaker.
    \item Immerse the graphite electrodes into the solution, ensuring they do not touch each other.
    \item Connect the electrodes to the DC power supply using crocodile clips and wires.
    \item Turn on the power supply and observe what happens at each electrode.
    \item Record your observations, noting any colour changes, gas evolution, or deposition of solids at the electrodes.
    \item After a few minutes, turn off the power supply and carefully examine the electrodes.
\end{enumerate}

\textbf{Discussion:**
\begin{enumerate}
    \item What did you observe at the cathode (negative electrode)? What chemical reaction do you think occurred?
    \item What did you observe at the anode (positive electrode)? What chemical reaction do you think occurred? (Hint: Consider the ions present in copper(II) chloride solution and water).
    \item Write half-equations for the reactions occurring at the cathode and anode.
    \item What are some applications of electrolysis in industry?
\end{enumerate}
\end{investigation}

\subsubsection{Tiered Questions}
\begin{tieredquestions}{Basic}
\begin{enumerate}
    \item What is the Haber process used for? Write the balanced chemical equation for the Haber process.
    \item List three conditions used in the Haber process to achieve a good yield of ammonia.
    \item What is electrolysis? Describe the process of electrolysis for molten sodium chloride.
\end{enumerate}
\end{tieredquestions}

\begin{tieredquestions}{Intermediate}
\begin{enumerate}
    \item Explain why a catalyst is used in the Haber process. How does a catalyst affect the equilibrium position and the rate of reaction?
    \item Describe the role of oxidation and reduction in electrolysis. Identify which electrode (cathode or anode) is associated with oxidation and which is associated with reduction.
    \item Explain why molten sodium chloride is used in the electrolysis of sodium chloride instead of solid sodium chloride.
\end{enumerate}
\end{tieredquestions}

\begin{tieredquestions}{Advanced}
\begin{enumerate}
    \item Discuss the environmental and economic considerations associated with the Haber process. What are some alternative methods for producing ammonia being researched?
    \item Explain the factors that affect the products of electrolysis of aqueous solutions. Consider the electrolysis of aqueous copper(II) sulfate solution and predict the products at the cathode and anode.
    \item Research and describe a specific industrial application of electrolysis, other than metal extraction, and explain the chemical principles involved.
\end{enumerate}
\end{tieredquestions}

\subsection{Chemistry in Everyday Products: From Soaps to Sunscreens}

Chemistry is not confined to laboratories and factories; it is also present in the everyday products we use. Understanding the chemistry behind these products can help us make informed choices about their use and impact.

\begin{keyconcept}{Soaps and Detergents: Cleaning Power}
\keyword{Soaps} and \keyword{detergents} are cleaning agents that help to remove dirt and grease from surfaces. They work because of their unique molecular structure, which has both a \keyword{hydrophilic} (water-loving) end and a \keyword{hydrophobic} (water-fearing, oil-loving) end.

\begin{itemize}
    \item \textbf{Soaps:}  Typically made from natural fats or oils reacted with a strong alkali (e.g., sodium hydroxide or potassium hydroxide) in a process called \keyword{saponification}.  The resulting soap molecules have a long hydrocarbon chain (hydrophobic) and a carboxylate salt group (hydrophilic).
    \item \textbf{Detergents:}  Synthetic cleaning agents made from petrochemicals.  They have a similar structure to soaps, with a hydrophobic hydrocarbon chain and a hydrophilic sulfate or sulfonate group.  Detergents are often more effective than soaps in hard water because they are less likely to form insoluble precipitates (soap scum).
\end{itemize}

\textbf{How Soaps and Detergents Work:}
\begin{enumerate}
    \item The hydrophobic tails of soap or detergent molecules dissolve in grease and oil.
    \item The hydrophilic heads are attracted to water.
    \item This forms \keyword{micelles}, where the hydrophobic tails are clustered inside, surrounding the grease, and the hydrophilic heads are on the outside, interacting with water.  \texttt{[PLACEHOLDER FIGURE: Diagram of micelle formation]}
    \item The micelles, with the grease trapped inside, are dispersed in water and can be washed away.
\end{enumerate}
\end{keyconcept}

\begin{stopandthink}
Why are soaps and detergents more effective at removing oily stains compared to just using water alone?
\end{stopandthink}

\begin{keyconcept}{Sunscreens: Protecting Skin from UV Radiation}
\keyword{Sunscreens} are products designed to protect our skin from the harmful effects of ultraviolet (UV) radiation from the sun.  UV radiation can cause sunburn, premature aging, and increase the risk of skin cancer.

Sunscreens work in two main ways:

\begin{itemize}
    \item \textbf{Chemical Sunscreens (Organic UV filters):}  These contain chemicals that absorb UV radiation and convert it into heat, which is then released from the skin. Examples include \keyword{oxybenzone}, \keyword{avobenzone}, and \keyword{octinoxate}.
    \item \textbf{Physical Sunscreens (Inorganic UV filters):} These contain mineral particles, such as \keyword{zinc oxide} (\ce{ZnO}) and \keyword{titanium dioxide} (\ce{TiO2}), that physically block or reflect UV radiation.  These are often considered more environmentally friendly and less irritating to sensitive skin.
\end{itemize}

Sunscreens are rated by their \keyword{Sun Protection Factor (SPF)}, which indicates how well they protect against UVB radiation (the main cause of sunburn). A higher SPF number means greater protection.  Broad-spectrum sunscreens protect against both UVA and UVB radiation.

Choosing the right sunscreen and applying it correctly is important for protecting skin health.  Consider the SPF, broad-spectrum protection, and ingredients when selecting a sunscreen.
\end{keyconcept}

\begin{investigation}{Testing the Effectiveness of Sunscreens}
\textbf{Aim:} To compare the effectiveness of different sunscreens in blocking UV radiation.

\textbf{Materials:} UV-sensitive beads (or UV detection cards), different types of sunscreens with varying SPF values, clear plastic sheets, sunlight or a UV lamp.

\textbf{Procedure:}
\begin{enumerate}
    \item Place UV-sensitive beads (or detection cards) on a dark surface.  Observe their colour in normal light.
    \item Apply a thin layer of different sunscreens to separate clear plastic sheets. Label each sheet with the SPF of the sunscreen applied. Leave one sheet without sunscreen as a control.
    \item Place the plastic sheets over separate groups of UV-sensitive beads (or detection cards).  Ensure the sunscreen-coated side is facing upwards, towards the light source.
    \item Expose all the beads (or cards), including the control, to sunlight or a UV lamp for a set period (e.g., 30 minutes).
    \item Observe and record the colour change of the beads (or cards) under each plastic sheet and the control.  UV-sensitive beads typically change colour when exposed to UV radiation.
\end{enumerate}

\textbf{Discussion:**
\begin{enumerate}
    \item How did the colour of the UV-sensitive beads (or cards) change under the different sunscreens and the control?
    \item How does the SPF value of the sunscreen seem to correlate with its effectiveness in blocking UV radiation?
    \item What are some limitations of this investigation? How could you make it more quantitative?
    \item Research the potential environmental impacts of chemical sunscreen ingredients. What are some more environmentally friendly sunscreen options?
\end{enumerate}
\end{investigation}

\subsubsection{Tiered Questions}
\begin{tieredquestions}{Basic}
\begin{enumerate}
    \item What are soaps and detergents used for? Explain how they work to remove grease and dirt.
    \item What are the two main types of sunscreens? Briefly describe how each type protects the skin from UV radiation.
    \item What does SPF stand for in sunscreen? What does a higher SPF value indicate?
\end{enumerate}
\end{tieredquestions}

\begin{tieredquestions}{Intermediate}
\begin{enumerate}
    \item Explain the terms 'hydrophilic' and 'hydrophobic' and how these properties are essential for the function of soaps and detergents.
    \item Compare and contrast the advantages and disadvantages of chemical sunscreens and physical sunscreens.
    \item Describe the potential health risks associated with exposure to UV radiation and explain why sunscreen is important.
\end{enumerate}
\end{tieredquestions}

\begin{tieredquestions}{Advanced}
\begin{enumerate}
    \item Explain the saponification reaction for soap making. Write a general equation for the reaction of a fat or oil with sodium hydroxide.
    \item Research and discuss the environmental concerns related to certain ingredients in sunscreens, such as oxybenzone. What are the potential impacts on marine ecosystems?
    \item Design an experiment to investigate the effect of water hardness on the effectiveness of soap and detergent. Outline your method and identify variables you would need to control.
\end{enumerate}
\end{tieredquestions}

\section{Environmental Chemistry: Protecting Our Planet}

Environmental chemistry is concerned with the chemical processes that occur in the environment and the impact of human activities on these processes. It is crucial for understanding and addressing environmental problems such as pollution, climate change, and resource depletion.

\subsection{Air Pollution: Breathing Clean Air}

\marginnote{\keyword{Air pollution} is the contamination of the atmosphere by harmful substances.}
\marginnote{\challenge{Investigate air quality monitoring in your local area and the pollutants that are commonly measured.}}
\keyword{Air pollution} refers to the contamination of the atmosphere with harmful substances that can negatively affect human health, ecosystems, and the environment.  These pollutants can come from various sources, both natural and human-induced.

\begin{keyconcept}{Major Air Pollutants and Their Sources}
\begin{itemize}
    \item \textbf{Particulate Matter (PM):}  Tiny solid particles and liquid droplets suspended in the air.  Sources include combustion processes (e.g., vehicle exhaust, power plants, industrial activities, wood burning), dust storms, and volcanic eruptions.  PM can cause respiratory problems and cardiovascular diseases.
    \item \textbf{Nitrogen Oxides (\ce{NOx}):}  Primarily nitrogen monoxide (\ce{NO}) and nitrogen dioxide (\ce{NO2}).  Formed during high-temperature combustion processes, mainly from vehicles and power plants.  \ce{NOx} contributes to smog, acid rain, and respiratory irritation.
    \item \textbf{Sulfur Dioxide (\ce{SO2}):}  Released primarily from the burning of fossil fuels (especially coal) in power plants and industrial processes, as well as volcanic eruptions.  \ce{SO2} is a respiratory irritant and a major contributor to acid rain.
    \item \textbf{Carbon Monoxide (\ce{CO}):}  A colourless, odourless, and highly toxic gas produced by incomplete combustion of fuels.  Main sources are vehicle exhaust and industrial processes.  \ce{CO} reduces the oxygen-carrying capacity of blood, leading to poisoning.
    \item \textbf{Ozone (\ce{O3}):}  In the troposphere (lower atmosphere), ozone is a secondary pollutant, formed by chemical reactions involving \ce{NOx} and volatile organic compounds (VOCs) in the presence of sunlight.  Ground-level ozone is a major component of smog and a respiratory irritant.  (Note: Stratospheric ozone is beneficial as it protects us from harmful UV radiation.)
    \item \textbf{Volatile Organic Compounds (VOCs):}  Organic chemicals that evaporate easily at room temperature.  Sources include vehicle exhaust, industrial processes, solvents, paints, and natural sources like vegetation.  VOCs contribute to smog formation and can have various health effects.
\end{itemize}
\end{keyconcept}

\begin{stopandthink}
Think about the air quality in your local area. What are the main sources of air pollution in your region?
\end{stopandthink}

\begin{keyconcept}{Effects of Air Pollution}
Air pollution has wide-ranging negative effects on human health and the environment:

\begin{itemize}
    \item \textbf{Human Health:** Respiratory problems (asthma, bronchitis, lung cancer), cardiovascular diseases, eye and throat irritation, reduced lung function, premature mortality.
    \item \textbf{Acid Rain:**  \ce{SO2} and \ce{NOx} react with water vapour in the atmosphere to form sulfuric acid and nitric acid, which fall as acid rain.  Acid rain damages forests, lakes, buildings, and statues.
    \item \textbf{Smog:**  A mixture of air pollutants, including ozone, \ce{NOx}, VOCs, and particulate matter, that reduces visibility and causes respiratory problems.  Photochemical smog is formed under the influence of sunlight.
    \item \textbf{Damage to Ecosystems:** Air pollution can harm plants and animals, disrupt ecosystems, and reduce biodiversity.
    \item \textbf{Climate Change:** Some air pollutants, such as black carbon (a component of particulate matter), and tropospheric ozone, are also short-lived climate pollutants that contribute to global warming.
\end{itemize}
\end{keyconcept}

\begin{investigation}{Investigating Particulate Matter in Air}
\textbf{Aim:} To collect and observe particulate matter from the air in different locations.

\textbf{Materials:} White paper or sticky tape, microscope slides (optional), petroleum jelly (Vaseline), microscope (optional), magnifying glass, different locations (e.g., near a busy road, park, industrial area, residential area).

\textbf{Procedure:}
\begin{enumerate}
    \item \textbf{Method 1 (Paper/Tape):}  In each chosen location, place a piece of white paper or sticky tape (sticky side up) in an open area for a set period (e.g., 24 hours). Ensure it is protected from rain.
    \item \textbf{Method 2 (Petroleum Jelly Slides):}  Smear a thin layer of petroleum jelly on microscope slides. Place the slides in open areas in different locations for the same period as above.
    \item After the exposure period, collect the paper/tape/slides carefully.
    \item Observe the collected samples using a magnifying glass or microscope (if available).  Compare the amount and type of particulate matter collected in each location.
    \item Record your observations, noting the location and the characteristics of the particulate matter (e.g., colour, size, shape).
\end{enumerate}

\textbf{Discussion:**
\begin{enumerate}
    \item How does the amount of particulate matter collected vary between different locations?
    \item Can you identify any potential sources of the particulate matter you collected in each location?
    \item What are some limitations of this investigation method? How could you improve it to get more detailed information about air pollution?
    \item Research air quality monitoring data for your local area. Does it support your findings?
\end{enumerate}
\end{investigation}

\subsubsection{Tiered Questions}
\begin{tieredquestions}{Basic}
\begin{enumerate}
    \item What is air pollution? List three major air pollutants and their sources.
    \item Describe two negative effects of air pollution on human health.
    \item What is acid rain? How is it formed and what are its impacts?
\end{enumerate}
\end{tieredquestions}

\begin{tieredquestions}{Intermediate}
\begin{enumerate}
    \item Explain how nitrogen oxides and volatile organic compounds contribute to the formation of smog.
    \item Compare and contrast the sources and effects of particulate matter and sulfur dioxide pollution.
    \item Describe some strategies that can be used to reduce air pollution from vehicles and industrial sources.
\end{enumerate}
\end{tieredquestions}

\begin{tieredquestions}{Advanced}
\begin{enumerate}
    \item Discuss the chemistry of ozone formation in the troposphere and stratosphere. Explain the difference between 'good' ozone and 'bad' ozone.
    \item Research and describe a specific air pollution episode (e.g., London Smog of 1952) and its causes and consequences. What lessons were learned from this event?
    \item Design an experiment to investigate the effect of different types of vegetation on reducing particulate matter pollution in urban areas. Outline your method and identify variables you would need to control.
\end{enumerate}
\end{tieredquestions}

\subsection{Water Pollution: Protecting Our Water Resources}

\marginnote{\keyword{Water pollution} is the contamination of water bodies by harmful substances.}
\marginnote{\historylink{Historically, water pollution has been a problem since the development of settlements and industry, but its scale and complexity have increased significantly in modern times.}}
\keyword{Water pollution} occurs when harmful substances contaminate water bodies such as rivers, lakes, oceans, and groundwater.  Polluted water can be harmful to human health, aquatic life, and ecosystems.

\begin{keyconcept}{Major Water Pollutants and Their Sources}
\begin{itemize}
    \item \textbf{Sewage and Wastewater:**  Untreated or poorly treated sewage from homes, businesses, and industries. Contains organic matter, pathogens (bacteria, viruses), and nutrients (nitrogen and phosphorus).  Can cause oxygen depletion in water bodies, spread diseases, and lead to eutrophication.
    \item \textbf{Industrial Discharges:**  Wastewater from factories and industrial processes. Can contain a wide range of pollutants, including heavy metals (e.g., mercury, lead, cadmium), toxic chemicals, organic pollutants, and thermal pollution (heated water).
    \item \textbf{Agricultural Runoff:**  Water runoff from agricultural fields.  Can contain fertilisers (nitrates and phosphates), pesticides, herbicides, animal waste, and soil sediment.  Contributes to eutrophication and contamination of water sources.
    \item \textbf{Oil Spills:**  Accidental or deliberate releases of oil into waterways or oceans from tankers, pipelines, or offshore drilling.  Oil is toxic to marine life and can persist in the environment for long periods.
    \item \textbf{Plastic Pollution:**  Plastic waste that enters waterways and oceans. Plastics do not biodegrade and can persist for centuries. They can entangle marine animals, be ingested, and break down into microplastics, which can enter the food chain.
    \item \textbf{Heavy Metals:**  Metals such as mercury, lead, cadmium, and arsenic that are toxic even at low concentrations.  Sources include industrial discharges, mining runoff, and atmospheric deposition.  Heavy metals can accumulate in organisms and cause long-term health problems.
\end{itemize}
\end{keyconcept}

\begin{stopandthink}
Consider the water sources in your local area. What are the potential sources of water pollution that could affect these sources?
\end{stopandthink}

\begin{keyconcept}{Effects of Water Pollution}
Water pollution has severe consequences for human health and the environment:

\begin{itemize}
    \item \textbf{Human Health:** Waterborne diseases (e.g., cholera, typhoid, dysentery) caused by pathogens in contaminated water.  Exposure to toxic chemicals and heavy metals through drinking water or consuming contaminated fish.
    \item \textbf{Eutrophication:**  Excessive nutrients (nitrogen and phosphorus) from sewage and agricultural runoff lead to algal blooms in water bodies.  Algal blooms deplete oxygen when they decompose, killing fish and other aquatic life.
    \item \textbf{Harm to Aquatic Life:**  Pollutants can be toxic to fish, invertebrates, and other aquatic organisms, disrupting food chains and ecosystems.  Oil spills can coat and suffocate marine animals and birds.  Plastics can entangle and be ingested by marine life.
    \item \textbf{Damage to Ecosystems:** Water pollution can degrade aquatic habitats, reduce biodiversity, and disrupt ecosystem services such as water purification and nutrient cycling.
    \item \textbf{Contamination of Drinking Water Sources:** Pollution can make water sources unsafe for drinking, requiring expensive water treatment processes or limiting access to clean water.
\end{itemize}
\end{keyconcept}

\begin{investigation}{Testing Water Quality}
\textbf{Aim:} To test the quality of water samples from different sources using simple chemical tests.

\textbf{Materials:} Water samples from different sources (e.g., tap water, river water, pond water, rainwater), test tubes, universal indicator solution or pH paper, nitrate test strips, phosphate test strips, turbidity meter (optional), beakers, safety glasses.

\textbf{Procedure:}
\begin{enumerate}
    \item \textbf{pH Test:**  Use universal indicator solution or pH paper to measure the pH of each water sample. Record the pH value.
    \item \textbf{Nitrate Test:**  Use nitrate test strips to test for the presence of nitrates in each water sample. Follow the instructions on the test strip packaging. Record the nitrate level.
    \item \textbf{Phosphate Test:**  Use phosphate test strips to test for the presence of phosphates in each water sample. Follow the instructions on the test strip packaging. Record the phosphate level.
    \item \textbf{Turbidity (Optional):}  If a turbidity meter is available, measure the turbidity of each water sample. Record the turbidity value.  (Turbidity is a measure of water clarity; higher turbidity indicates more suspended particles.)  Alternatively, visually assess the clarity of each sample.
    \item \textbf{Record your results} in a table, noting the water source and the results of each test (pH, nitrate level, phosphate level, turbidity/clarity).
\end{enumerate}

\textbf{Discussion:**
\begin{enumerate}
    \item How do the water quality parameters (pH, nitrate, phosphate, turbidity) vary between the different water sources?
    \item Based on your results, which water sources appear to be of higher quality and which appear to be more polluted?
    \item What are the potential sources of nitrates and phosphates in water? Why are high levels of nitrates and phosphates a concern?
    \item What are some limitations of these simple water quality tests? What other tests could be used to get a more comprehensive assessment of water quality?
\end{enumerate}
\end{investigation}

\subsubsection{Tiered Questions}
\begin{tieredquestions}{Basic}
\begin{enumerate}
    \item What is water pollution? List three major types of water pollutants and their sources.
    \item Describe two negative effects of water pollution on human health.
    \item What is eutrophication? How does it occur and what are its consequences?
\end{enumerate}
\end{tieredquestions}

\begin{tieredquestions}{Intermediate}
\begin{enumerate}
    \item Explain how sewage and agricultural runoff contribute to water pollution. What are the key pollutants from these sources?
    \item Compare and contrast the impacts of oil spills and plastic pollution on marine ecosystems.
    \item Describe some methods used for treating wastewater and making it safe to release back into the environment.
\end{enumerate}
\end{tieredquestions}

\begin{tieredquestions}{Advanced}
\begin{enumerate}
    \item Discuss the issue of microplastic pollution in oceans. What are the sources of microplastics and what are their potential impacts on marine life and human health?
    \item Research and describe a specific case of water pollution (e.g., Minamata disease, Flint water crisis) and its causes and consequences. What lessons were learned from this event?
    \item Design an experiment to investigate the effectiveness of different methods for removing pollutants (e.g., filtration, activated carbon adsorption) from contaminated water samples. Outline your method and identify variables you would need to control.
\end{enumerate}
\end{tieredquestions}

\subsection{Climate Change: A Chemical Perspective}

\marginnote{\keyword{Climate change} refers to long-term shifts in temperatures and weather patterns, largely driven by human activities.}
\marginnote{\mathlink{The greenhouse effect is related to the absorption of infrared radiation by molecules, which can be explained using molecular vibrations and quantum mechanics.}}
\keyword{Climate change} is one of the most pressing environmental challenges facing humanity. It refers to long-term shifts in temperatures and weather patterns, primarily driven by human activities that release greenhouse gases into the atmosphere. Chemistry plays a crucial role in understanding the causes and consequences of climate change and in developing solutions.

\begin{keyconcept}{The Greenhouse Effect and Greenhouse Gases}
The \keyword{greenhouse effect} is a natural process that warms the Earth's surface.  When sunlight reaches the Earth, some of it is absorbed by the Earth's surface, and some is radiated back into space as infrared radiation (heat).  \keyword{Greenhouse gases} in the atmosphere, such as carbon dioxide (\ce{CO2}), methane (\ce{CH4}), and water vapour (\ce{H2O}), absorb some of this outgoing infrared radiation and re-emit it in all directions, including back towards the Earth's surface.  This trapping of heat warms the planet.

\begin{itemize}
    \item \textbf{Major Greenhouse Gases:}
    \begin{itemize}
        \item \textbf{Carbon Dioxide (\ce{CO2}):}  The most significant long-lived greenhouse gas.  Main sources are burning fossil fuels (coal, oil, gas) for energy, deforestation, and industrial processes.
        \item \textbf{Methane (\ce{CH4}):}  A potent greenhouse gas, though shorter-lived than \ce{CO2}. Sources include agriculture (livestock, rice cultivation), natural gas leaks, and decomposition of organic waste.
        \item \textbf{Nitrous Oxide (\ce{N2O}):}  Released from agricultural activities (fertilisers), industrial processes, and burning fossil fuels.  A powerful greenhouse gas and also depletes stratospheric ozone.
        \item \textbf{Fluorinated Gases (F-gases):}  Synthetic gases used in refrigerants, aerosols, and industrial processes.  Very potent greenhouse gases with long atmospheric lifetimes.  Examples include hydrofluorocarbons (HFCs), perfluorocarbons (PFCs), and sulfur hexafluoride (\ce{SF6}).
        \item \textbf{Water Vapour (\ce{H2O}):}  The most abundant greenhouse gas, but its concentration in the atmosphere is largely determined by temperature and is considered a feedback rather than a primary driver of climate change.
    \end{itemize}
\end{itemize}

The enhanced