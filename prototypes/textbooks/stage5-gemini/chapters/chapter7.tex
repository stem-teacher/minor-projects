```latex
\chapter{Chemical Reactions and Equations}

\section{The Amazing World of Chemical Change}

\marginnote{
\textbf{Did you know?}
Chemical reactions are happening all around us, and even inside us, every second! From the moment you breathe in oxygen to the digestion of your lunch, it's all chemistry in action.
}

Have you ever watched fireworks explode in a burst of colour, seen iron rust over time, or baked a cake and watched it rise in the oven? These are all examples of \keyword{chemical reactions} – processes where substances are transformed into new substances. Chemistry is not just something that happens in laboratories; it's the science of change, and it's fundamental to understanding the world around us.

In this chapter, we will embark on an exciting journey to explore the fascinating world of chemical reactions. We will learn how to identify them, describe them, and even predict them. Get ready to unlock the secrets of how matter changes and discover the power of chemical equations!

\begin{stopandthink}
Think about three everyday examples of changes you observe. Which of these do you think might be chemical reactions, and why?
\end{stopandthink}

\section{What are Chemical Reactions?}

\marginnote{
\textbf{Definition:}
A \keyword{chemical reaction} is a process that involves the rearrangement of atoms and molecules to form new substances.
}

At the heart of every chemical reaction is the rearrangement of atoms. Atoms are the fundamental building blocks of matter, and they combine to form molecules. In a chemical reaction, the bonds holding atoms together in molecules are broken and new bonds are formed, resulting in the creation of different molecules.

Let's consider a simple example: burning wood. Wood is made up of complex molecules, mainly cellulose. When you light wood on fire, it reacts with oxygen in the air.  This reaction breaks down the cellulose molecules and forms new molecules like carbon dioxide, water vapour, and ash. The original wood is gone, and in its place are entirely new substances.

\begin{keyconcept}{Reactants and Products}
In a chemical reaction, the starting materials are called \keyword{reactants}, and the substances formed are called \keyword{products}.  Think of it like baking a cake: the flour, eggs, and sugar are the reactants, and the cake is the product.
\end{keyconcept}

In the case of burning wood, the reactants are wood and oxygen, and the products are carbon dioxide, water vapour, and ash.

\subsection{Evidence of Chemical Reactions}

\marginnote{
\textbf{Observable Changes:}
While not all changes are visible, common signs of a chemical reaction include:
\begin{itemize}
    \item Colour change
    \item Formation of a precipitate
    \item Gas production (effervescence)
    \item Temperature change
    \item Light emission
\end{itemize}
}

How do we know when a chemical reaction has occurred?  Often, there are observable changes that indicate a transformation has taken place. Here are some common signs:

\begin{itemize}
    \item \textbf{Colour Change:}  A dramatic change in colour can be a strong indicator. For example, when iron rusts, it changes from shiny grey to reddish-brown.
    \item \textbf{Formation of a Precipitate:}  A \keyword{precipitate} is a solid that forms when two solutions are mixed. If a clear solution suddenly becomes cloudy or forms solid particles, a precipitate might have formed, indicating a reaction.
    \item \textbf{Gas Production (Effervescence):}  Bubbles of gas forming when substances are mixed is another common sign. Think of fizzy drinks – the bubbles of carbon dioxide are released due to a change in conditions, but gas production during mixing chemicals often suggests a reaction. This is called \keyword{effervescence}.
    \item \textbf{Temperature Change:}  Chemical reactions can either release or absorb energy, often in the form of heat. If a reaction mixture gets hotter, it's likely an \keyword{exothermic reaction} (releases heat). If it gets colder, it's likely an \keyword{endothermic reaction} (absorbs heat).
    \item \textbf{Light Emission:}  Some reactions produce light, like burning or explosions. This is a clear sign of a chemical transformation.
\end{itemize}

\begin{example}
\textbf{Mixing Vinegar and Baking Soda}

When you mix vinegar (acetic acid) and baking soda (sodium bicarbonate), you observe:

\begin{itemize}
    \item \textbf{Effervescence:} Bubbles of gas (carbon dioxide) are produced.
    \item \textbf{Temperature Change:} The mixture usually gets slightly colder (endothermic).
\end{itemize}
These observations suggest that a chemical reaction has taken place.
\end{example}

It's important to note that not all changes are chemical reactions. For example, melting ice or dissolving sugar in water are \keyword{physical changes}. In physical changes, the substance changes its form or appearance, but its chemical composition remains the same.  In chemical reactions, the chemical composition itself is altered.

\begin{investigation}{Observing Evidence of Chemical Reactions}
\textbf{Materials:}
\begin{itemize}
    \item Vinegar
    \item Baking soda
    \item Steel wool (uncoated)
    \item Copper sulfate solution (dilute)
    \item Iron nail (clean)
    \item Test tubes or beakers
    \item Thermometer (optional)
\end{itemize}

\textbf{Procedure:}
\begin{enumerate}
    \item \textbf{Vinegar and Baking Soda:** In a test tube, mix a small amount of baking soda with vinegar. Observe and record any changes.
    \item \textbf{Rusting Steel Wool:**  Moisten a small piece of steel wool with water and leave it exposed to air for a day or two. Observe and record any changes over time. \challenge{Consider placing one piece of dry steel wool and one moist piece side-by-side to compare the rate of change.}
    \item \textbf{Iron Nail in Copper Sulfate Solution:** Place a clean iron nail in a test tube containing copper sulfate solution. Observe and record any changes after about 15-20 minutes.
    \item \textbf{Temperature Change (Optional):} For the vinegar and baking soda reaction, and the iron nail and copper sulfate reaction, carefully measure the temperature of the reactants before mixing and the temperature of the mixture after the reaction. Record any temperature changes.
\end{enumerate}

\textbf{Observations and Analysis:}
For each experiment, note down your observations, focusing on colour changes, gas production, precipitate formation, and temperature changes.  Explain which observations indicate a chemical reaction has occurred in each case.
\end{investigation}

\begin{tieredquestions}{Section 1}
\begin{itemize}
    \item \textbf{Basic:} What is a chemical reaction in your own words? Give one example of a chemical reaction from everyday life.
    \item \textbf{Intermediate:} Explain the difference between reactants and products. List three pieces of evidence that might suggest a chemical reaction has taken place.
    \item \textbf{Advanced:}  Distinguish between a chemical change and a physical change. Provide examples of each and explain why dissolving salt in water is a physical change, not a chemical change.
\end{itemize}
\end{tieredquestions}

\section{Representing Chemical Reactions: Equations}

\marginnote{
\textbf{Communication in Chemistry:}
Chemists use specific ways to represent chemical reactions, just like mathematicians use symbols and equations. This allows for clear and concise communication about chemical changes.
}

To understand and communicate about chemical reactions effectively, we need ways to represent them in a concise and informative manner. We use different types of equations to describe what happens during a chemical reaction.

\subsection{Word Equations}

\marginnote{
\textbf{Word Equations:}
These are the simplest way to describe a reaction, using the names of the reactants and products.
}

The simplest way to describe a chemical reaction is using a \keyword{word equation}. Word equations use the names of the reactants and products to show what is happening.  An arrow ($\rightarrow$) is used to show the direction of the reaction, reading as "reacts to form" or "produces". Reactants are written on the left side of the arrow, and products on the right.

\begin{example}
\textbf{Word Equation for Burning Methane (Natural Gas)}

Methane reacts with oxygen to produce carbon dioxide and water.

Word Equation:
\texttt{Methane + Oxygen $\rightarrow$ Carbon Dioxide + Water}
\end{example}

Word equations are useful for a general overview of a reaction, but they don't tell us about the specific substances involved at the atomic level. For that, we need to use chemical formulas.

\subsection{Chemical Formulas and Formula Equations}

\marginnote{
\textbf{Chemical Formulas:}
Represent elements and compounds using symbols and subscripts. e.g., Water is $\ce{H2O}$, Carbon Dioxide is $\ce{CO2}$.
}
\marginnote{
\textbf{Formula Equations:}
Use chemical formulas instead of words to represent reactants and products.
}

\keyword{Chemical formulas} are a shorthand way of representing elements and compounds using symbols for atoms and subscripts to indicate the number of each type of atom in a molecule. For example, the chemical formula for water is $\ce{H2O}$, indicating that each molecule of water contains two hydrogen atoms (H) and one oxygen atom (O). Carbon dioxide is $\ce{CO2}$, methane is $\ce{CH4}$, and oxygen gas is $\ce{O2}$.

Using chemical formulas, we can write \keyword{formula equations}, which are more informative than word equations.

\begin{example}
\textbf{Formula Equation for Burning Methane}

Using chemical formulas: Methane is $\ce{CH4}$, Oxygen is $\ce{O2}$, Carbon Dioxide is $\ce{CO2}$, and Water is $\ce{H2O}$.

Formula Equation:
\texttt{$\ce{CH4 + O2 \rightarrow CO2 + H2O}$}
\end{example}

This formula equation tells us exactly which substances are involved in the reaction. However, notice something important: if you count the atoms of each element on both sides of the arrow, they are not equal!  On the left side, we have 1 carbon, 4 hydrogen, and 2 oxygen atoms. On the right side, we have 1 carbon, 2 hydrogen, and 3 oxygen atoms. This violates the \keyword{Law of Conservation of Mass}, which states that matter cannot be created or destroyed in a chemical reaction.  Atoms are just rearranged, so the total number of each type of atom must be the same on both sides of the equation.

\subsection{Balanced Chemical Equations}

\marginnote{
\textbf{Law of Conservation of Mass:}
In a chemical reaction, matter is neither created nor destroyed. The total mass of reactants equals the total mass of products. This means the number of each type of atom must be the same on both sides of a balanced equation.
}
\marginnote{
\textbf{Balanced Equations:}
Ensure that the number of atoms of each element is the same on both sides of the equation by using \keyword{coefficients}.
}

To obey the Law of Conservation of Mass, we need to write \keyword{balanced chemical equations}.  A balanced chemical equation has the same number of atoms of each element on both the reactant and product sides.  We achieve this by adding \keyword{coefficients} in front of the chemical formulas. Coefficients are whole numbers that indicate the number of molecules (or moles, which we'll learn about later) of each substance involved in the reaction.

Let's balance the equation for burning methane: $\ce{CH4 + O2 \rightarrow CO2 + H2O}$.

1. \textbf{Start with Carbon:} Carbon is already balanced (1 atom on each side).
2. \textbf{Balance Hydrogen:}  There are 4 hydrogen atoms on the left ($\ce{CH4}$) and 2 on the right ($\ce{H2O}$). To balance hydrogen, we need to place a coefficient of 2 in front of $\ce{H2O}$:  $\ce{CH4 + O2 \rightarrow CO2 + 2H2O}$. Now we have 4 hydrogen atoms on both sides.
3. \textbf{Balance Oxygen:} Now count oxygen atoms. On the left, we have 2 ($\ce{O2}$). On the right, we have 2 in $\ce{CO2}$ and 2 in $2\ce{H2O}$ (2 x 1 = 2), for a total of 4 oxygen atoms. To balance oxygen, we need to place a coefficient of 2 in front of $\ce{O2}$ on the reactant side: $\ce{CH4 + 2O2 \rightarrow CO2 + 2H2O}$.
4. \textbf{Check Balance:** Recount all atoms:
    * Carbon: 1 on left, 1 on right (balanced)
    * Hydrogen: 4 on left, 4 on right (balanced)
    * Oxygen: 4 on left (2 x 2), 4 on right (2 + 2) (balanced)

The balanced chemical equation for burning methane is:

\begin{example}
\textbf{Balanced Chemical Equation for Burning Methane}
\texttt{$\ce{CH4 + 2O2 \rightarrow CO2 + 2H2O}$}
\end{example}

\historylink{
\textbf{Antoine-Laurent Lavoisier (1743-1794):}
Often called the "father of modern chemistry", Lavoisier was crucial in establishing the Law of Conservation of Mass through careful experiments. His work revolutionised chemistry and moved it from alchemy towards a quantitative science. Sadly, he was executed during the French Revolution.
}

\begin{stopandthink}
Why can we change the coefficients in front of chemical formulas when balancing equations, but we cannot change the subscripts within the formulas themselves?
\end{stopandthink}

\subsection{State Symbols}

\marginnote{
\textbf{State Symbols:}
Indicate the physical state of each reactant and product:
\begin{itemize}
    \item (s) - solid
    \item (l) - liquid
    \item (g) - gas
    \item (aq) - aqueous (dissolved in water)
\end{itemize}
}

To make chemical equations even more informative, we often include \keyword{state symbols} in parentheses after each chemical formula. These symbols indicate the physical state of each reactant and product under the reaction conditions:

\begin{itemize}
    \item \textbf{(s)} for solid
    \item \textbf{(l)} for liquid
    \item \textbf{(g)} for gas
    \item \textbf{(aq)} for aqueous solution (dissolved in water)
\end{itemize}

\begin{example}
\textbf{Balanced Equation with State Symbols for Burning Methane}

Methane gas reacts with oxygen gas to produce carbon dioxide gas and water vapour (gas).

\texttt{$\ce{CH4(g) + 2O2(g) \rightarrow CO2(g) + 2H2O(g)}$}
\end{example}

\begin{example}
\textbf{Reaction of Zinc with Hydrochloric Acid}

Zinc metal (solid) reacts with hydrochloric acid (aqueous solution) to produce hydrogen gas and zinc chloride solution (aqueous).

Word Equation: Zinc + Hydrochloric Acid $\rightarrow$ Hydrogen + Zinc Chloride

Unbalanced Formula Equation: $\ce{Zn(s) + HCl(aq) \rightarrow H2(g) + ZnCl2(aq)}$

Balanced Formula Equation with State Symbols: $\ce{Zn(s) + 2HCl(aq) \rightarrow H2(g) + ZnCl2(aq)}$
\end{example}

\begin{investigation}{Balancing Chemical Equations Practice}
\textbf{Instructions:}
Balance the following chemical equations. Write out the balanced equation with coefficients and state symbols (where states are given or can be reasonably assumed).

\begin{enumerate}
    \item  Hydrogen gas reacts with chlorine gas to form hydrogen chloride gas:  $\ce{H2(g) + Cl2(g) \rightarrow HCl(g)}$
    \item  Sodium metal reacts with water to produce sodium hydroxide solution and hydrogen gas: $\ce{Na(s) + H2O(l) \rightarrow NaOH(aq) + H2(g)}$
    \item  Iron(III) oxide solid reacts with carbon monoxide gas to produce iron metal and carbon dioxide gas: $\ce{Fe2O3(s) + CO(g) \rightarrow Fe(s) + CO2(g)}$
    \item  Propane gas ($\ce{C3H8}$) burns in oxygen gas to produce carbon dioxide gas and water vapour: $\ce{C3H8(g) + O2(g) \rightarrow CO2(g) + H2O(g)}$
    \item  Silver nitrate solution reacts with sodium chloride solution to produce silver chloride precipitate and sodium nitrate solution: $\ce{AgNO3(aq) + NaCl(aq) \rightarrow AgCl(s) + NaNO3(aq)}$
\end{enumerate}

\textbf{Extension:}  For equations 3 and 4, identify the reactants and products in word form and write out the word equation for each reaction.
\end{investigation}

\begin{tieredquestions}{Section 2}
\begin{itemize}
    \item \textbf{Basic:} What is a word equation? Write a word equation for the reaction of hydrogen and oxygen to form water.
    \item \textbf{Intermediate:} Explain the Law of Conservation of Mass and how it relates to balanced chemical equations. Balance the equation: $\ce{Mg + O2 \rightarrow MgO}$.
    \item \textbf{Advanced:} Balance the equation: $\ce{C6H12O6(aq) + O2(g) \rightarrow CO2(g) + H2O(l)}$ (This is the equation for respiration). Explain the steps you took to balance it, and why balancing chemical equations is essential in chemistry. \mathlink{Research and explain how balanced equations can be used in stoichiometry to calculate the amounts of reactants and products in a chemical reaction.}
\end{itemize}
\end{tieredquestions}

\section{Types of Chemical Reactions}

\marginnote{
\textbf{Classifying Reactions:}
Chemists classify reactions into different types to help organise and understand chemical changes. This allows for predictions and generalisations about chemical behaviour.
}

Chemical reactions are incredibly diverse, but many can be grouped into common types based on their patterns of reactant and product formation.  Understanding these types helps us to predict what might happen when different substances are mixed. We will explore some major categories of chemical reactions.

\subsection{Synthesis (Combination) Reactions}

\marginnote{
\textbf{Synthesis Reactions:}
Also called \keyword{combination reactions}. Two or more reactants combine to form a single, more complex product. General form: $\ce{A + B \rightarrow AB}$
}

In a \keyword{synthesis reaction}, also known as a \keyword{combination reaction}, two or more simpler substances combine to form a more complex product. Think of it as building something up from smaller parts.

General form: $\ce{A + B \rightarrow AB}$

\begin{example}
\textbf{Formation of Water}

Hydrogen gas and oxygen gas react to form water.

Word Equation: Hydrogen + Oxygen $\rightarrow$ Water

Balanced Equation: $\ce{2H2(g) + O2(g) \rightarrow 2H2O(l)}$
\end{example}

\begin{example}
\textbf{Formation of Magnesium Oxide}

Magnesium metal burns in oxygen to form magnesium oxide.

Word Equation: Magnesium + Oxygen $\rightarrow$ Magnesium Oxide

Balanced Equation: $\ce{2Mg(s) + O2(g) \rightarrow 2MgO(s)}$
\end{example}

\subsection{Decomposition Reactions}

\marginnote{
\textbf{Decomposition Reactions:}
A single compound breaks down into two or more simpler substances. Often requires energy input (heat, light, electricity). General form: $\ce{AB \rightarrow A + B}$
}

\keyword{Decomposition reactions} are the opposite of synthesis reactions. In a decomposition reaction, a single compound breaks down into two or more simpler substances.  These reactions often require energy input, such as heat, light, or electricity, to initiate the breakdown.

General form: $\ce{AB \rightarrow A + B}$

\begin{example}
\textbf{Decomposition of Hydrogen Peroxide}

Hydrogen peroxide solution slowly decomposes into water and oxygen gas. This process is sped up by catalysts (we'll learn about those later).

Word Equation: Hydrogen Peroxide $\rightarrow$ Water + Oxygen

Balanced Equation: $\ce{2H2O2(aq) \rightarrow 2H2O(l) + O2(g)}$
\end{example}

\begin{example}
\textbf{Thermal Decomposition of Copper(II) Carbonate}

When heated strongly, copper(II) carbonate solid decomposes into copper(II) oxide solid and carbon dioxide gas.

Word Equation: Copper(II) Carbonate $\xrightarrow{Heat}$ Copper(II) Oxide + Carbon Dioxide

Balanced Equation: $\ce{CuCO3(s) \xrightarrow{\Delta} CuO(s) + CO2(g)}$ (The $\Delta$ symbol above the arrow indicates heat is required.)
\end{example}

\subsection{Displacement (Single Replacement) Reactions}

\marginnote{
\textbf{Displacement Reactions:}
Also called \keyword{single replacement} or \keyword{substitution} reactions. One element replaces another element in a compound. General form: $\ce{A + BC \rightarrow AC + B}$ or $\ce{X + YZ \rightarrow YX + Z}$
}

\keyword{Displacement reactions}, also known as \keyword{single replacement} or \keyword{substitution reactions}, involve one element replacing another element in a compound.  Think of it as "cutting in line" – one element pushes another out of its compound.

General form: $\ce{A + BC \rightarrow AC + B}$  (where A is more reactive than B)  or $\ce{X + YZ \rightarrow YX + Z}$ (where X is more reactive than Y – if considering non-metals)

\begin{example}
\textbf{Reaction of Zinc with Copper Sulfate Solution}

When zinc metal is placed in copper sulfate solution, zinc displaces copper, forming zinc sulfate solution and solid copper metal.

Word Equation: Zinc + Copper Sulfate $\rightarrow$ Zinc Sulfate + Copper

Balanced Equation: $\ce{Zn(s) + CuSO4(aq) \rightarrow ZnSO4(aq) + Cu(s)}$

In this reaction, zinc is more reactive than copper and is able to displace it from copper sulfate. We can determine the relative reactivity of metals using a \keyword{reactivity series} (or activity series).
\end{example}

\challenge{
\textbf{Reactivity Series of Metals:}
Metals can be arranged in order of their reactivity. More reactive metals will displace less reactive metals from their compounds.  Research the reactivity series of common metals (e.g., potassium, sodium, calcium, magnesium, aluminium, zinc, iron, tin, lead, copper, silver, gold, platinum).  How can this series help predict displacement reactions?
}

\subsection{Double Displacement (Precipitation) Reactions}

\marginnote{
\textbf{Double Displacement Reactions:}
Also called \keyword{precipitation} or \keyword{metathesis} reactions.  Ions in two aqueous solutions exchange partners. Often results in the formation of a precipitate. General form: $\ce{AB + CD \rightarrow AD + CB}$
}

\keyword{Double displacement reactions}, also called \keyword{precipitation reactions} or \keyword{metathesis reactions}, involve the exchange of ions between two aqueous solutions.  Imagine a "partner swap" between two compounds. Often, one of the products is insoluble and forms a solid precipitate.

General form: $\ce{AB + CD \rightarrow AD + CB}$

\begin{example}
\textbf{Reaction of Silver Nitrate with Sodium Chloride}

When silver nitrate solution is mixed with sodium chloride solution, a white precipitate of silver chloride forms, and sodium nitrate solution remains.

Word Equation: Silver Nitrate + Sodium Chloride $\rightarrow$ Silver Chloride + Sodium Nitrate

Balanced Equation: $\ce{AgNO3(aq) + NaCl(aq) \rightarrow AgCl(s) + NaNO3(aq)}$

Silver chloride ($\ce{AgCl}$) is insoluble in water, so it precipitates out of the solution. We can use \keyword{solubility rules} to predict whether a precipitate will form in a double displacement reaction.
\end{example}

\subsection{Combustion Reactions}

\marginnote{
\textbf{Combustion Reactions:}
Reactions with oxygen that produce heat and light. Often involve burning fuels. Usually exothermic. General form: $\ce{Fuel + O2 \rightarrow CO2 + H2O}$ (for complete combustion of hydrocarbons)
}

\keyword{Combustion reactions} are reactions with oxygen that produce heat and light. These are often called burning. Combustion reactions are usually exothermic and involve fuels reacting with oxygen from the air. Complete combustion of hydrocarbons (compounds containing hydrogen and carbon) typically produces carbon dioxide and water.

General form (complete combustion of hydrocarbons): $\ce{Fuel + O2 \rightarrow CO2 + H2O}$

\begin{example}
\textbf{Combustion of Propane (LPG)}

Propane is a common fuel used in portable gas stoves and heating.

Word Equation: Propane + Oxygen $\rightarrow$ Carbon Dioxide + Water

Balanced Equation: $\ce{C3H8(g) + 5O2(g) \rightarrow 3CO2(g) + 4H2O(g)}$
\end{example}

\subsection{Neutralisation Reactions (Acid-Base Reactions)}

\marginnote{
\textbf{Neutralisation Reactions:}
Reactions between an acid and a base. Produce a salt and water. Exothermic reactions. General form: $\ce{Acid + Base \rightarrow Salt + Water}$
}

\keyword{Neutralisation reactions} are reactions between an acid and a base. They are also known as \keyword{acid-base reactions}.  These reactions typically produce a salt and water. Neutralisation reactions are usually exothermic.

General form: $\ce{Acid + Base \rightarrow Salt + Water}$

\begin{example}
\textbf{Reaction of Hydrochloric Acid with Sodium Hydroxide}

Hydrochloric acid is a strong acid, and sodium hydroxide is a strong base.

Word Equation: Hydrochloric Acid + Sodium Hydroxide $\rightarrow$ Sodium Chloride + Water

Balanced Equation: $\ce{HCl(aq) + NaOH(aq) \rightarrow NaCl(aq) + H2O(l)}$

In this reaction, sodium chloride ($\ce{NaCl}$) is the salt formed.
\end{example}

\subsection{Redox Reactions (brief introduction)}

\marginnote{
\textbf{Redox Reactions:}
Short for \keyword{reduction-oxidation reactions}. Involve the transfer of electrons. Many reaction types are redox reactions (combustion, displacement, rusting). We'll learn more about these in later stages.
}

\keyword{Redox reactions} are a broad category of reactions involving the transfer of electrons between reactants.  The term "redox" is short for \keyword{reduction-oxidation}.  Many of the reaction types we've discussed are actually redox reactions. For example, combustion, displacement reactions, and even rusting are all redox processes.  We will explore redox reactions in more detail in later stages, but it's important to know that electron transfer is a fundamental aspect of many chemical transformations.

\begin{investigation}{Investigating Types of Chemical Reactions}
\textbf{Materials:}
\begin{itemize}
    \item Magnesium ribbon
    \item Copper(II) oxide powder
    \item Dilute hydrochloric acid
    \item Sodium hydroxide solution
    \item Copper sulfate solution
    \item Iron filings
    \item Lead(II) nitrate solution
    \item Potassium iodide solution
    \item Test tubes, beakers, Bunsen burner (optional for copper oxide reduction)
\end{itemize}

\textbf{Procedure:}
\begin{enumerate}
    \item \textbf{Synthesis (Magnesium Burning):}  (Demonstration by teacher or with very careful supervision and safety precautions). Burn a small piece of magnesium ribbon in air. Observe the product. Identify the type of reaction.
    \item \textbf{Decomposition (Copper Carbonate - Teacher demo if Bunsen burner used):} Gently heat a small amount of copper(II) carbonate powder in a test tube. Observe any changes, including gas production (test gas with limewater if possible). Identify the type of reaction.
    \item \textbf{Displacement (Iron and Copper Sulfate):} Add iron filings to copper sulfate solution in a test tube. Observe any changes over time. Identify the type of reaction.
    \item \textbf{Double Displacement (Lead Nitrate and Potassium Iodide):} Mix lead(II) nitrate solution and potassium iodide solution in a test tube. Observe any precipitate formed. Identify the type of reaction.
    \item \textbf{Neutralisation (Acid and Base):} Carefully mix dilute hydrochloric acid and sodium hydroxide solution in a test tube.  Feel the test tube gently to detect any temperature change.  Identify the type of reaction.
\end{enumerate}

\textbf{Observations and Analysis:}
For each experiment, record your observations. Classify each reaction into one of the types discussed (synthesis, decomposition, displacement, double displacement, neutralisation - combustion will be demonstrated). Write word and balanced chemical equations (where possible at this stage) for each reaction.
\end{investigation}

\begin{tieredquestions}{Section 3}
\begin{itemize}
    \item \textbf{Basic:}  Name three types of chemical reactions. Give a simple word equation example for a synthesis reaction.
    \item \textbf{Intermediate:} Explain the difference between a synthesis and a decomposition reaction. Classify the following reaction type and write a balanced chemical equation:  Iron + Chlorine $\rightarrow$ Iron(III) Chloride.
    \item \textbf{Advanced:}  Explain the characteristics of displacement and double displacement reactions. Predict the products and write a balanced chemical equation (including state symbols) for the reaction between magnesium metal and silver nitrate solution. Justify your prediction based on reactivity series principles (if known) or by analogy to similar reactions.
\end{itemize}
\end{tieredquestions}

\section{Energy Changes in Chemical Reactions}

\marginnote{
\textbf{Energy and Reactions:}
Chemical reactions are always accompanied by energy changes. Energy is either released or absorbed during the process. This energy change is crucial in determining the feasibility and applications of chemical reactions.
}

Chemical reactions are not just about rearranging atoms; they are also about energy changes. Every chemical reaction involves a change in energy, usually in the form of heat. Reactions can either release energy into their surroundings or absorb energy from their surroundings.

\subsection{Exothermic Reactions}

\marginnote{
\textbf{Exothermic Reactions:}
Reactions that release heat to the surroundings. The temperature of the surroundings increases. Energy of products is lower than energy of reactants. $\Delta H$ is negative.
}

\keyword{Exothermic reactions} are reactions that release heat energy to the surroundings.  If you carry out an exothermic reaction in a test tube, you will feel the test tube get warmer.  In exothermic reactions, the energy stored in the chemical bonds of the reactants is greater than the energy stored in the bonds of the products. The excess energy is released as heat (and sometimes light).

Examples of exothermic reactions include:

\begin{itemize}
    \item \textbf{Combustion reactions:} Burning fuels like wood, gas, and petrol release a lot of heat and light.
    \item \textbf{Neutralisation reactions:} Reactions between acids and bases release heat.
    \item \textbf{Many synthesis reactions:} For example, the reaction of sodium with chlorine to form sodium chloride is highly exothermic.
\end{itemize}

\begin{example}
\textbf{Combustion of Methane (Exothermic)}

Burning methane releases heat.

$\ce{CH4(g) + 2O2(g) \rightarrow CO2(g) + 2H2O(g) + Heat}$

We can represent the heat released as a product in the equation or indicate that the reaction is exothermic.
\end{example}

\subsection{Endothermic Reactions}

\marginnote{
\textbf{Endothermic Reactions:}
Reactions that absorb heat from the surroundings. The temperature of the surroundings decreases. Energy of products is higher than energy of reactants. $\Delta H$ is positive.
}

\keyword{Endothermic reactions} are reactions that absorb heat energy from the surroundings. If you carry out an endothermic reaction in a test tube, you will feel the test tube get colder. In endothermic reactions, the energy stored in the chemical bonds of the reactants is less than the energy stored in the bonds of the products.  Energy must be absorbed from the surroundings to make the reaction happen.

Examples of endothermic reactions include:

\begin{itemize}
    \item \textbf{Melting ice:**  Melting requires heat energy to be absorbed from the surroundings. Although it's a physical change, it illustrates energy absorption.
    \item \textbf{Photosynthesis:} Plants absorb light energy (a form of energy) to convert carbon dioxide and water into glucose and oxygen.
    \item \textbf{Some decomposition reactions:** Many decomposition reactions require heat energy to break down compounds.
\end{itemize}

\begin{example}
\textbf{Decomposition of Ammonium Nitrate (Endothermic)}

The decomposition of ammonium nitrate absorbs heat.

$\ce{NH4NO3(s) + Heat \rightarrow N2O(g) + 2H2O(g)}$

We represent heat as a reactant in the equation to show it is required for the reaction to occur.
\end{example}

\subsection{Activation Energy}

\marginnote{
\textbf{Activation Energy:}
The minimum energy required to start a chemical reaction.  Like pushing a ball over a hill – you need enough energy to get it to the top before it can roll down.
}

Even exothermic reactions, which release energy overall, usually need a little "push" to get started. This initial energy input required to start a chemical reaction is called \keyword{activation energy}. Think of it like pushing a ball over a small hill. You need to put in some energy to push the ball up to the top of the hill, but once it's over the top, it will roll down on its own, releasing energy as it goes.

For example, wood is flammable, and burning wood is exothermic. However, wood doesn't spontaneously burst into flames in air at room temperature. You need to provide activation energy, for example, by striking a match and applying a flame to the wood. The heat from the match provides the initial energy needed to start the combustion reaction. Once started, the reaction is exothermic and releases enough heat to sustain itself and continue burning.

\subsection{Catalysts}

\marginnote{
\textbf{Catalysts:}
Substances that speed up chemical reactions without being used up themselves. They lower the activation energy of the reaction. Enzymes are biological catalysts.
}

\keyword{Catalysts} are substances that speed up the rate of a chemical reaction without being consumed in the reaction itself. Catalysts work by providing an alternative reaction pathway with a lower activation energy.  Think of a catalyst as creating a tunnel through the hill instead of going over the top – it makes it easier for the reaction to proceed.

\begin{example}
\textbf{Catalytic Decomposition of Hydrogen Peroxide}

Hydrogen peroxide naturally decomposes slowly into water and oxygen.  However, the decomposition is greatly sped up by the presence of a catalyst like manganese dioxide ($\ce{MnO2}$).

$\ce{2H2O2(aq) \xrightarrow{\ce{MnO2}} 2H2O(l) + O2(g)}$

Manganese dioxide is not consumed in the reaction; it simply provides a surface and mechanism for the decomposition to occur more easily.

\keyword{Enzymes} are biological catalysts – proteins that speed up biochemical reactions in living organisms.  Without enzymes, many reactions necessary for life would occur too slowly to sustain life. Catalysts are essential in many industrial processes and biological systems.

\begin{stopandthink}
Explain in your own words why even exothermic reactions often need activation energy to start.
\end{stopandthink}

\begin{investigation}{Investigating Exothermic and Endothermic Reactions}
\textbf{Materials:}
\begin{itemize}
    \item Calcium chloride (anhydrous)
    \item Ammonium nitrate
    \item Water
    \item Two beakers or polystyrene cups
    \item Thermometer
    \item Stirring rod
\end{itemize}

\textbf{Procedure:}
\begin{enumerate}
    \item \textbf{Exothermic Reaction (Calcium Chloride):} Measure about 50 mL of water into a beaker and record its initial temperature. Add about 10 g of calcium chloride to the water and stir gently.  Observe and record the temperature of the mixture after a few minutes.
    \item \textbf{Endothermic Reaction (Ammonium Nitrate):} In a separate beaker, measure about 