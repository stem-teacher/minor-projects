```latex
\chapter{Atomic Structure and the Periodic Table}

\section{The Mysterious World Within: Introducing Atoms}

\marginnote{
\historylink{Ancient Ideas}
The idea of atoms is not new! Ancient Greek philosophers like Democritus and Leucippus proposed that matter was made of indivisible particles, which they called 'atomos', meaning 'uncuttable'.  While their ideas were philosophical and not based on experiments, they were remarkably insightful.
}

Have you ever looked at a towering mountain, a sparkling ocean, or even the air you breathe and wondered, "What is it all made of?"  Everything around us, from the smallest grain of sand to the largest star, is composed of \keyword{matter}. And the fundamental building blocks of matter are incredibly tiny particles called \keyword{atoms}.

Imagine you have a piece of gold. If you could keep dividing it into smaller and smaller pieces, would you ever reach a point where you couldn't divide it anymore?  The answer, according to our current understanding of science, is yes! You would eventually reach the smallest unit of gold that still retains the properties of gold. This smallest unit is a gold atom.

\begin{keyconcept}{Atoms}
Atoms are the smallest units of an element that retain the chemical properties of that element. They are the fundamental building blocks of all matter in the universe.
\end{keyconcept}

Atoms are incredibly small – far too small to see with the naked eye, or even with a regular microscope.  If you lined up about 100 million atoms side by side, they would only stretch about one centimetre!  Despite their tiny size, atoms are complex structures with fascinating internal components.

\begin{stopandthink}
Think about different objects around you – a wooden table, a glass of water, a metal spoon.  Do you think they are made of the same types of atoms? Why or why not?
\end{stopandthink}

\section{Elements, Compounds, and Mixtures: Organising Matter}

\marginnote{
\challenge{Diversity of Elements}
While there are only around 90 naturally occurring elements on Earth, scientists have created heavier, synthetic elements in laboratories. These are often unstable and exist for very short periods.
}

Matter can be classified in various ways. One important way is to distinguish between \keyword{elements}, \keyword{compounds}, and \keyword{mixtures}.

\subsection{Elements: Pure Substances}

\marginnote{
\keyword{Periodic Table}
You will soon learn that elements are organised in a chart called the periodic table. This table is a chemist's best friend!
}

An element is a pure substance that cannot be broken down into simpler substances by chemical means.  Each element is made up of only one type of atom.  Gold, oxygen, and carbon are all examples of elements.  There are over 100 known elements, each with its unique properties.  We use symbols to represent elements, often derived from their names. For example, \keyword{O} is the symbol for oxygen, \keyword{Au} for gold (from the Latin word ‘aurum’), and \keyword{C} for carbon.

\begin{keyconcept}{Elements}
Elements are pure substances consisting of only one type of atom. They cannot be broken down into simpler substances by chemical reactions.
\end{keyconcept}

\subsection{Compounds: Joining Atoms Together}

\marginnote{
\challenge{Molecular vs. Ionic Compounds}
Compounds can be further classified as molecular or ionic.  Molecular compounds, like water (\ce{H2O}), are formed by sharing electrons. Ionic compounds, like salt (\ce{NaCl}), are formed by transferring electrons, creating ions.
}

Most of the matter we encounter in our daily lives is not in the form of pure elements, but rather as \keyword{compounds}.  A compound is a substance formed when two or more different elements are chemically bonded together in a fixed ratio.  Water (\ce{H2O}) is a compound made from hydrogen and oxygen atoms. Carbon dioxide (\ce{CO2}), the gas we breathe out, is a compound made from carbon and oxygen atoms. Table salt, or sodium chloride (\ce{NaCl}), is a compound of sodium and chlorine.

\begin{keyconcept}{Compounds}
Compounds are substances formed when two or more different elements are chemically combined in a fixed ratio.  Their properties are different from the properties of the elements they are made from.
\end{keyconcept}

The properties of a compound are usually very different from the properties of the elements that make it up. For example, sodium is a highly reactive metal, and chlorine is a poisonous green gas.  However, when they combine to form sodium chloride (table salt), the resulting compound is a white, crystalline solid that is essential for life!

\subsection{Mixtures: Blending Substances}

\marginnote{
\keyword{Homogeneous and Heterogeneous Mixtures}
Mixtures can be classified as homogeneous (uniform composition, like saltwater) or heterogeneous (non-uniform composition, like sand and water).
}

Unlike compounds, \keyword{mixtures} are combinations of two or more substances that are physically combined, but not chemically bonded.  The substances in a mixture retain their individual properties and can be separated by physical means, such as filtration, evaporation, or distillation.  Air is a mixture of gases, mainly nitrogen and oxygen.  Seawater is a mixture of water, salt, and other substances.  Sand and water form a mixture.

\begin{keyconcept}{Mixtures}
Mixtures are combinations of two or more substances that are physically combined but not chemically bonded.  The components of a mixture retain their individual properties and can be separated by physical means.
\end{keyconcept}

\begin{investigation}{Separating Mixtures}
\textbf{Aim:} To separate different types of mixtures using physical methods.

\textbf{Materials:}
\begin{itemize}
    \item Sand and water mixture
    \item Salt and water mixture
    \item Iron filings and sand mixture
    \item Filter paper and funnel
    \item Beakers
    \item Magnet
    \item Bunsen burner (or hot plate) and evaporation dish
\end{itemize}

\textbf{Procedure:}
\begin{enumerate}
    \item \textbf{Sand and water:}  Pour the sand and water mixture through filter paper in a funnel. Observe what is collected in the beaker and what remains on the filter paper.
    \item \textbf{Salt and water:} Gently heat the salt and water mixture in an evaporation dish using a Bunsen burner or hot plate (under adult supervision). Observe what happens as the water evaporates and what remains in the dish.
    \item \textbf{Iron filings and sand:}  Hold a magnet wrapped in paper over the iron filings and sand mixture.  Observe what happens.
\end{enumerate}

\textbf{Observations and Analysis:}
\begin{itemize}
    \item What method was used to separate each mixture?
    \item Explain why each method was effective for separating that particular mixture based on the properties of the components.
    \item Identify the physical properties that allowed for the separation in each case (e.g., particle size, boiling point, magnetism).
\end{itemize}

\textbf{Conclusion:}
Write a short conclusion summarising what you have learned about separating mixtures using physical methods.
\end{investigation}

\begin{tieredquestions}{Section 1 & 2}

\textbf{Basic:}
\begin{enumerate}
    \item What is the smallest unit of an element?
    \item Give two examples of elements and two examples of compounds.
    \item How is a mixture different from a compound?
\end{enumerate}

\textbf{Intermediate:}
\begin{enumerate}
    \item Explain why sugar (\ce{C12H22O11}) is considered a compound and not a mixture.
    \item Describe a method you could use to separate a mixture of oil and water. Explain why this method works.
    \item \keyword{Diamond} and \keyword{graphite} are both made of only carbon atoms. Are they the same element? Are they the same compound? Explain.
\end{enumerate}

\textbf{Advanced:}
\begin{enumerate}
    \item  Is it possible for a mixture to have a uniform composition throughout? Explain and give an example.
    \item  Think critically: Could you separate the elements in a compound by simply heating it? Explain your reasoning and give an example to support your answer.
    \item Research: Investigate and describe one real-world application where the separation of mixtures is crucial.
\end{enumerate}

\end{tieredquestions}

\section{Peering Inside the Atom: Subatomic Particles}

\marginnote{
\historylink{Discovery of the Electron}
J.J. Thomson discovered the electron in 1897 through experiments with cathode ray tubes. This was a revolutionary discovery that changed our understanding of the atom.
}

For a long time, atoms were thought to be indivisible – the ultimate, fundamental particles of matter. However, scientists discovered that atoms themselves are made up of even smaller particles, called \keyword{subatomic particles}. The three main subatomic particles are:

\begin{itemize}
    \item \textbf{Protons:} Positively charged particles found in the nucleus of the atom.
    \item \textbf{Neutrons:} Neutral (no charge) particles also found in the nucleus of the atom.
    \item \textbf{Electrons:} Negatively charged particles that orbit the nucleus in regions called electron shells.
\end{itemize}

\begin{keyconcept}{Subatomic Particles}
Atoms are composed of three main subatomic particles: protons (positive charge), neutrons (no charge), and electrons (negative charge). Protons and neutrons are located in the nucleus, while electrons orbit the nucleus.
\end{keyconcept}

\subsection{The Nucleus: The Atom's Core}

\marginnote{
\challenge{Nuclear Forces}
The nucleus is incredibly dense because protons, which are positively charged and repel each other, are packed very closely together with neutrons.  Strong nuclear forces within the nucleus overcome this repulsion and hold it together.
}

The \keyword{nucleus} is the central core of an atom. It is incredibly tiny and dense, containing almost all of the atom's mass. The nucleus is made up of protons and neutrons (except for hydrogen-1, which has only one proton and no neutrons).

* \textbf{Protons} carry a positive electrical charge (+1). The number of protons in an atom's nucleus determines what element it is. For example, all atoms with one proton are hydrogen atoms, all atoms with six protons are carbon atoms, and all atoms with 79 protons are gold atoms. This number of protons is called the \keyword{atomic number}.

* \textbf{Neutrons} are electrically neutral (have no charge). Neutrons contribute to the mass of the atom and help to stabilise the nucleus by reducing the repulsion between protons.

\subsection{Electron Shells: Orbiting the Nucleus}

\marginnote{
\challenge{Quantum Mechanics}
The behaviour of electrons in atoms is governed by the principles of quantum mechanics.  Electrons don't orbit the nucleus in simple circular paths like planets around a star. Instead, they exist in specific energy levels and regions of space called orbitals, which have complex shapes.
}

\keyword{Electrons} are much smaller and lighter than protons and neutrons. They carry a negative electrical charge (-1). Electrons are found orbiting the nucleus in specific energy levels, often visualised as \keyword{electron shells} or energy levels.

* Electrons are arranged in shells around the nucleus. The shells are numbered starting from 1 closest to the nucleus, then 2, 3, and so on, moving outwards.
* Each shell can hold a maximum number of electrons. The first shell (shell 1) can hold a maximum of 2 electrons, the second shell (shell 2) can hold a maximum of 8 electrons, and the third shell (shell 3) can also hold up to 8 electrons in Stage 5 (though it can hold more in higher stages of study).
* Electrons fill the shells starting from the innermost shell (shell 1) and moving outwards.

The arrangement of electrons in shells, known as the \keyword{electron configuration}, is crucial because it determines how an atom will interact with other atoms and form chemical bonds.

\begin{example}{Electron Configuration of Oxygen}
Oxygen has an atomic number of 8. This means it has 8 protons and, in a neutral atom, also 8 electrons.

To determine the electron configuration of oxygen:
\begin{enumerate}
    \item Shell 1 can hold a maximum of 2 electrons. So, we place 2 electrons in the first shell.
    \item We have 8 - 2 = 6 electrons remaining.
    \item Shell 2 can hold a maximum of 8 electrons. We place the remaining 6 electrons in the second shell.
\end{enumerate}
Therefore, the electron configuration of oxygen is 2, 6. This means it has 2 electrons in the first shell and 6 electrons in the second shell.
\end{example}

\begin{stopandthink}
Consider a carbon atom (atomic number 6). How many protons, neutrons, and electrons does it have? What would be its electron configuration? (Assume it is a neutral atom).
\end{stopandthink}

\section{Atomic Number, Mass Number, Isotopes, and Ions}

\subsection{Atomic Number: Identifying the Element}

\marginnote{
\keyword{Symbol Z}
The atomic number is often represented by the symbol \textit{Z}.
}

As mentioned earlier, the \keyword{atomic number} of an element is the number of protons in the nucleus of an atom of that element.  It is a unique identifier for each element.  All atoms of the same element have the same atomic number.

\begin{keyconcept}{Atomic Number}
The atomic number (\textit{Z}) is the number of protons in the nucleus of an atom. It defines the element.
\end{keyconcept}

For example:
\begin{itemize}
    \item Hydrogen (H) has an atomic number of 1 (1 proton).
    \item Carbon (C) has an atomic number of 6 (6 protons).
    \item Oxygen (O) has an atomic number of 8 (8 protons).
    \item Iron (Fe) has an atomic number of 26 (26 protons).
    \item Gold (Au) has an atomic number of 79 (79 protons).
\end{itemize}

\subsection{Mass Number: Counting Nucleons}

\marginnote{
\keyword{Symbol A}
The mass number is often represented by the symbol \textit{A}.
}

The \keyword{mass number} of an atom is the total number of protons and neutrons in its nucleus. Protons and neutrons are collectively called \keyword{nucleons}.

\begin{keyconcept}{Mass Number}
The mass number (\textit{A}) is the total number of protons and neutrons in the nucleus of an atom.
\end{keyconcept}

\mathlink{Mass Number Formula}
Mass Number (A) = Number of Protons + Number of Neutrons

To find the number of neutrons in an atom, you can subtract the atomic number from the mass number:

Number of Neutrons = Mass Number (A) - Atomic Number (Z)

For example, consider an atom of carbon with a mass number of 12. Carbon's atomic number is 6 (always).

* Number of protons = Atomic number = 6
* Number of neutrons = Mass number - Atomic number = 12 - 6 = 6
* Number of electrons (in a neutral atom) = Number of protons = 6

So, this carbon atom has 6 protons, 6 neutrons, and 6 electrons.

\begin{stopandthink}
If an oxygen atom has a mass number of 16 and an atomic number of 8, how many protons, neutrons, and electrons does it have? (Assume it is a neutral atom).
\end{stopandthink}


\subsection{Isotopes: Variations in Neutrons}

\marginnote{
\challenge{Radioactive Isotopes}
Some isotopes are radioactive, meaning their nuclei are unstable and decay, emitting radiation. Radioactive isotopes have important applications in medicine, dating ancient artefacts (carbon-14 dating), and energy production.
}

\keyword{Isotopes} are atoms of the same element that have the same atomic number (same number of protons) but different mass numbers (different number of neutrons).  Isotopes of an element have very similar chemical properties because their chemical behaviour is primarily determined by the number of electrons, which is the same for all isotopes of an element.

For example, carbon exists in several isotopic forms:

\begin{itemize}
    \item \textbf{Carbon-12 ($^{12}$C):}  6 protons, 6 neutrons (most common isotope of carbon)
    \item \textbf{Carbon-13 ($^{13}$C):}  6 protons, 7 neutrons
    \item \textbf{Carbon-14 ($^{14}$C):}  6 protons, 8 neutrons (radioactive isotope used in carbon dating)
\end{itemize}

All three are carbon atoms because they all have 6 protons.  They differ only in the number of neutrons. We represent isotopes using the element symbol, with the mass number as a superscript to the left of the symbol and the atomic number as a subscript to the left (though the subscript is often omitted as it is redundant since the element symbol already defines the atomic number).  For example, carbon-12 is written as $^{12}_{6}\text{C}$ or simply $^{12}\text{C}$.

\begin{keyconcept}{Isotopes}
Isotopes are atoms of the same element with the same atomic number but different mass numbers, due to variations in the number of neutrons.
\end{keyconcept}

\subsection{Ions: Gaining or Losing Electrons}

\marginnote{
\keyword{Cations and Anions}
Positively charged ions are called cations (pronounced CAT-eye-ons), and negatively charged ions are called anions (ANN-eye-ons).
}

Atoms are electrically neutral because they have an equal number of protons (positive charge) and electrons (negative charge). However, atoms can gain or lose electrons to become electrically charged particles called \keyword{ions}.

* \textbf{Ions} are atoms or molecules that have gained or lost electrons and therefore have an electrical charge.

* When an atom \textbf{loses} one or more electrons, it becomes positively charged because it now has more protons than electrons. Positively charged ions are called \keyword{cations}. For example, a sodium atom (Na) can lose one electron to become a sodium ion (\ce{Na+}).

* When an atom \textbf{gains} one or more electrons, it becomes negatively charged because it now has more electrons than protons. Negatively charged ions are called \keyword{anions}. For example, a chlorine atom (Cl) can gain one electron to become a chloride ion (\ce{Cl-}).

\begin{keyconcept}{Ions}
Ions are atoms or molecules that have gained or lost electrons, resulting in an electrical charge.  Cations are positively charged ions (lost electrons), and anions are negatively charged ions (gained electrons).
\end{keyconcept}

The charge of an ion is indicated by a superscript after the element symbol, with the magnitude of the charge and the sign (+ or -). For example, \ce{Mg^{2+}} represents a magnesium ion with a +2 charge (lost two electrons), and \ce{O^{2-}} represents an oxide ion with a -2 charge (gained two electrons).

\begin{tieredquestions}{Section 3 & 4}

\textbf{Basic:}
\begin{enumerate}
    \item Name the three main subatomic particles and their charges.
    \item What is the atomic number? What does it tell you about an atom?
    \item What is the mass number? How is it calculated?
    \item What are isotopes? Give an example.
    \item What is an ion? How are cations and anions formed?
\end{enumerate}

\textbf{Intermediate:}
\begin{enumerate}
    \item  Draw a simple diagram of an atom showing the nucleus and electron shells. Label the locations of protons, neutrons, and electrons.
    \item  An atom of element X has 17 protons and 20 neutrons. What is its atomic number and mass number? What is the element X? (Use a periodic table – if available, or hint: atomic number 17 is chlorine).
    \item Explain why isotopes of the same element have similar chemical properties.
    \item How many electrons does a \ce{Ca^{2+}} ion have if a neutral calcium atom has 20 electrons? How many electrons does a \ce{Cl^{-}} ion have if a neutral chlorine atom has 17 electrons?
    \item Explain the difference between mass number and relative atomic mass. (Hint: Relative atomic mass takes into account the average masses of isotopes and their abundance).
\end{enumerate}

\textbf{Advanced:}
\begin{enumerate}
    \item  Explain why the nucleus of an atom is positively charged, even though it contains neutral particles (neutrons).
    \item  Why is the mass of an electron often considered negligible when calculating the mass number of an atom?
    \item  Carbon-14 dating relies on the radioactive decay of $^{14}\text{C}$. Explain how isotopes are used in this process to determine the age of ancient artefacts. (Research may be needed).
    \item  Consider two isotopes of oxygen: $^{16}\text{O}$ and $^{18}\text{O}$. Compare and contrast their subatomic particle composition.
    \item  Think critically: Can the atomic number of an element change during a chemical reaction? Can the mass number change? Explain your reasoning.
\end{enumerate}

\end{tieredquestions}


\section{The Periodic Table: Organising the Elements}

\marginnote{
\historylink{Mendeleev's Genius}
Dmitri Mendeleev, a Russian chemist, is credited with creating the first widely recognised periodic table in 1869. He arranged elements by increasing atomic mass and noticed repeating patterns in their properties.  Remarkably, he even left gaps for undiscovered elements and predicted their properties!
}

Imagine trying to organise all the different types of atoms – the elements – in a meaningful way.  This is exactly what scientists have done with the \keyword{periodic table of elements}.  The periodic table is a chart that arranges elements in order of increasing atomic number, grouping elements with similar chemical properties together. It is one of the most powerful tools in chemistry, providing a vast amount of information about the elements and their behaviour.

\begin{keyconcept}{Periodic Table}
The periodic table is a chart that organises elements in order of increasing atomic number, grouping elements with similar chemical properties into columns (groups) and rows (periods).
\end{keyconcept}

\subsection{Periods and Groups: Rows and Columns}

\begin{itemize}
    \item \textbf{Periods:} The horizontal rows in the periodic table are called periods. There are 7 periods in the modern periodic table. Elements in the same period have the same number of electron shells. As you move across a period from left to right, the atomic number and atomic mass generally increase.

    \item \textbf{Groups (or Families):} The vertical columns in the periodic table are called groups or families. There are 18 groups in the standard periodic table, although sometimes only groups 1, 2, and 13-18 are numbered as main groups, with groups 3-12 referred to as transition metals. Elements in the same group have similar chemical properties because they have the same number of electrons in their outermost electron shell (valence electrons).
\end{itemize}

\subsection{Key Groups to Know}

\marginnote{
\challenge{Group Names}
Some groups have common names:
Group 1: Alkali metals
Group 2: Alkaline earth metals
Group 17: Halogens
Group 18: Noble gases
These names reflect shared properties of the elements in these groups.
}

While understanding the properties of all groups is valuable in advanced chemistry, for Stage 5, it is important to be familiar with some key groups:

\begin{itemize}
    \item \textbf{Group 1: Alkali Metals} (Lithium, Sodium, Potassium, etc.): These are very reactive metals. They readily lose one electron to form +1 ions. They react vigorously with water to produce hydrogen gas and alkaline solutions.
    \item \textbf{Group 2: Alkaline Earth Metals} (Beryllium, Magnesium, Calcium, etc.): These are also reactive metals, but less reactive than alkali metals. They lose two electrons to form +2 ions.
    \item \textbf{Groups 3-12: Transition Metals} (Iron, Copper, Silver, Gold, etc.): These are metals known for their strength, hardness, and ability to conduct electricity. They often form coloured compounds and can have multiple oxidation states (form ions with different charges).
    \item \textbf{Group 17: Halogens} (Fluorine, Chlorine, Bromine, Iodine, etc.): These are very reactive nonmetals. They readily gain one electron to form -1 ions. They exist as diatomic molecules (\ce{F2}, \ce{Cl2}, etc.) and react with metals to form salts.
    \item \textbf{Group 18: Noble Gases} (Helium, Neon, Argon, Krypton, etc.): These are very unreactive gases, sometimes called inert gases. They have a full outer electron shell, making them very stable and unlikely to form chemical bonds under normal conditions.
\end{itemize}

\subsection{Trends in the Periodic Table}

\marginnote{
\challenge{Periodic Trends and Electron Configuration}
Periodic trends are directly related to the electron configuration of elements.  For example, atomic radius trend is influenced by the increasing number of electron shells down a group and increasing nuclear charge across a period. Ionization energy and electronegativity are related to how strongly the nucleus attracts electrons.
}

The periodic table is not just a list of elements; it also reveals important trends in the properties of elements.  Understanding these trends allows us to predict how elements will behave and react.  Two key trends to know at Stage 5 are:

\begin{itemize}
    \item \textbf{Atomic Radius (Size):}
    \begin{itemize}
        \item \textbf{Down a group:} Atomic radius increases. This is because as you go down a group, you add electron shells, making the atom larger.
        \item \textbf{Across a period (left to right):} Atomic radius generally decreases. This is because as you move across a period, the number of protons in the nucleus increases (atomic number increases), leading to a stronger attraction for the electrons and pulling them closer to the nucleus, making the atom smaller.
    \end{itemize}

    \item \textbf{Reactivity of Metals and Nonmetals:}
    \begin{itemize}
        \item \textbf{Metals:} Reactivity of metals generally increases down a group (especially for Group 1 and 2) and decreases across a period (from left to right). The most reactive metals are found in the lower left of the periodic table.
        \item \textbf{Nonmetals:} Reactivity of nonmetals generally decreases down a group (especially for Group 17) and increases across a period (from left to right). The most reactive nonmetals are found in the upper right of the periodic table (excluding noble gases).
    \end{itemize}
\end{itemize}

\begin{figure}
\centering
\includegraphics[width=0.8\textwidth]{periodic_table_placeholder.png}
\caption{A simplified periodic table of elements. Periods are horizontal rows and groups are vertical columns. Key groups like alkali metals, alkaline earth metals, halogens, and noble gases are highlighted. Trends in atomic radius and reactivity are indicated.}
\label{fig:periodic_table}
\marginnote{**(Figure to be added: A visually clear periodic table with periods, groups, and key groups labelled. Arrows indicating trends in atomic radius and reactivity would be helpful.)**}
\end{figure}


\begin{stopandthink}
Looking at the periodic table, which element do you think is more reactive: Sodium (Na) or Potassium (K)?  Explain your reasoning based on periodic trends. Which is more reactive: Chlorine (Cl) or Iodine (I)? Explain.
\end{stopandthink}


\section{Metals, Nonmetals, and Metalloids: Classifying Elements}

\marginnote{
\challenge{Properties and Applications}
The properties of metals, nonmetals, and metalloids dictate their applications in various industries. For example, metals are used in construction and electronics due to their strength and conductivity. Nonmetals like carbon and oxygen are essential for life and used in fuels and plastics. Metalloids are crucial in semiconductors for electronics.
}

Another way to classify elements is based on their general properties as \keyword{metals}, \keyword{nonmetals}, and \keyword{metalloids} (or semimetals).  The periodic table has a "staircase" line that roughly separates metals from nonmetals. Metals are generally found on the left side of the periodic table, and nonmetals are on the right side. Metalloids are located along the staircase line.

\subsection{Metals: Shiny and Conductive}

\begin{keyconcept}{Metals}
Metals are elements that are typically shiny, malleable, ductile, good conductors of heat and electricity, and tend to lose electrons to form positive ions (cations).
\end{keyconcept}

\textbf{Typical Properties of Metals:}
\begin{itemize}
    \item \textbf{Lustrous (Shiny):} Metals have a characteristic shine or lustre when polished.
    \item \textbf{Malleable:** They can be hammered or rolled into thin sheets without breaking.
    \item \textbf{Ductile:** They can be drawn into wires.
    \item \textbf{Good Conductors of Heat and Electricity:** Metals allow heat and electricity to flow through them easily.
    \item \textbf{Solid at Room Temperature:** Most metals are solid at room temperature (except mercury, which is a liquid).
    \item \textbf{High Melting and Boiling Points:** Metals generally have high melting and boiling points.
    \item \textbf{Tend to Lose Electrons:** Metals tend to lose electrons to form positive ions (cations).
\end{itemize}

Examples of metals include iron, copper, aluminium, gold, silver, and sodium.

\subsection{Nonmetals: Dull and Insulating}

\begin{keyconcept}{Nonmetals}
Nonmetals are elements that are typically dull, brittle (if solid), poor conductors of heat and electricity, and tend to gain electrons to form negative ions (anions).
\end{keyconcept}

\textbf{Typical Properties of Nonmetals:}
\begin{itemize}
    \item \textbf{Dull:** They lack lustre and are not shiny.
    \item \textbf{Brittle:** Solid nonmetals are generally brittle and break easily when hammered or bent.
    \item \textbf{Poor Conductors of Heat and Electricity:** Nonmetals are insulators, meaning they do not conduct heat or electricity well.
    \item \textbf{Can be Solids, Liquids, or Gases at Room Temperature:** Nonmetals exist in all three states of matter at room temperature (e.g., oxygen and nitrogen are gases, bromine is a liquid, sulfur and carbon are solids).
    \item \textbf{Low Melting and Boiling Points:** Nonmetals generally have lower melting and boiling points compared to metals.
    \item \textbf{Tend to Gain Electrons:** Nonmetals tend to gain electrons to form negative ions (anions).
\end{itemize}

Examples of nonmetals include oxygen, nitrogen, sulfur, chlorine, and carbon (as graphite or diamond).

\subsection{Metalloids: In-between Properties}

\begin{keyconcept}{Metalloids}
Metalloids (or semimetals) are elements that have properties intermediate between metals and nonmetals. Their properties can be temperature-dependent.
\end{keyconcept}

\textbf{Typical Properties of Metalloids:}
\begin{itemize}
    \item \textbf{Properties of both Metals and Nonmetals:** Metalloids exhibit some properties of metals and some properties of nonmetals.
    \item \textbf{Semiconductors:** Many metalloids are semiconductors, meaning they conduct electricity under certain conditions but not others. This property is crucial in electronics.
    \item \textbf{Can be Lustrous or Dull:** Some metalloids can appear shiny, while others are dull.
    \item \textbf{Brittle:** Like nonmetals, metalloids are often brittle.
\end{itemize}

Examples of metalloids include silicon, germanium, arsenic, and antimony. Silicon is a very important semiconductor used in computer chips and solar panels.

\begin{tieredquestions}{Section 5 & 6 & 7}

\textbf{Basic:}
\begin{enumerate}
    \item What is the periodic table? How are elements arranged in it?
    \item What are periods and groups in the periodic table?
    \item Name two properties of metals and two properties of nonmetals.
    \item Give one example of an alkali metal, a halogen, and a noble gas.
    \item What is a metalloid? Give an example.
\end{enumerate}

\textbf{Intermediate:}
\begin{enumerate}
    \item Explain why elements in the same group of the periodic table have similar chemical properties.
    \item Describe the trends in atomic radius as you go down a group and across a period. Explain the reasons for these trends.
    \item How does the reactivity of metals change as you move down Group 1 (alkali metals)? Explain why.
    \item Classify the following elements as metals, nonmetals, or metalloids: Sodium (Na), Sulfur (S), Silicon (Si), Iron (Fe), Chlorine (Cl).
    \item Explain why noble gases are generally unreactive.
\end{enumerate}

\textbf{Advanced:}
\begin{enumerate}
    \item  Explain how the periodic table can be used to predict the properties of undiscovered elements.
    \item  Compare and contrast the electron configurations of alkali metals and halogens. How do their electron configurations relate to their reactivity?
    \item  Research and describe one specific application of a metalloid and explain why its metalloid properties are essential for that application (e.g., silicon in semiconductors).
    \item  Consider the trend in reactivity of halogens (Group 17). Explain why fluorine is the most reactive halogen and reactivity decreases down the group. (Hint: think about electron attraction and atomic size).
    \item  Think critically: Is hydrogen a metal or a nonmetal? Justify your answer based on its position in the periodic table and its properties. (Hydrogen is a unique case and often placed in Group 1 but is a nonmetal).
\end{enumerate}

\end{tieredquestions}

\section{Chapter Summary}

In this chapter, we have explored the fascinating world of atoms and the periodic table.  We learned that:

\begin{itemize}
    \item All matter is made up of atoms, the fundamental building blocks of elements.
    \item Atoms consist of subatomic particles: protons, neutrons, and electrons.
    \item The nucleus contains protons and neutrons, while electrons orbit the nucleus in shells.
    \item The atomic number defines an element (number of protons), and the mass number is the sum of protons and neutrons.
    \item Isotopes are atoms of the same element with different numbers of neutrons.
    \item Ions are charged atoms formed by gaining or losing electrons.
    \item The periodic table organises elements by atomic number and groups elements with similar properties.
    \item Periods are horizontal rows, and groups are vertical columns in the periodic table.
    \item Key groups include alkali metals, alkaline earth metals, halogens, and noble gases, each with characteristic properties.
    \item Periodic trends exist in properties like atomic radius and reactivity.
    \item Elements can be classified as metals, nonmetals, or metalloids based on their properties.
\end{itemize}

Understanding atomic structure and the periodic table is fundamental to chemistry and provides a framework for understanding the behaviour of matter and the chemical reactions that shape our world. In the next chapter, we will build upon this knowledge to explore how atoms combine to form molecules and compounds through chemical bonding.
```